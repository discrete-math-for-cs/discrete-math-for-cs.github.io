\documentclass[12pt, oneside]{article}

\usepackage[letterpaper, scale=0.8, centering]{geometry}
\usepackage{fancyhdr}
\setlength{\parindent}{0em}
\setlength{\parskip}{1em}

\pagestyle{fancy}
\fancyhf{}
\renewcommand{\headrulewidth}{0pt}
\rfoot{{\footnotesize Copyright Mia Minnes, 2024, Version \today~(\thepage)}}

\usepackage{titlesec}

\author{CSE20S24}

\newcommand{\instructions}{{\bf For all HW assignments:} 
These homework assignments may be done individually or in groups of up to 3 students.
Please ensure your name(s) and PID(s)
are clearly visible on the first page of your homework
submission, start each question on a new page, and upload the PDF to Gradescope.
If you're working in a group, {\it submit only one submission per group}: one partner uploads the
submission through their Gradescope account and then adds the other group member(s) to the Gradescope submission
by selecting their name(s) in the ``Add Group Members'' dialog box. You will need to re-add your group member(s)
every time you resubmit a new version of your assignment.

Each homework question will be graded either for
{\bf correctness} (including clear and precise explanations and justifications of all answers) or
{\bf fair effort completeness}. You may collaborate on ``graded for correctness''
questions only with CSE 20 students in your group; if your
 group has questions about a problem, you may ask in drop-in help hours or post a private
post (visible only to the Instructors) on Piazza.  
 For ``graded for completeness''
 questions: collaboration is allowed with any CSE 20 students this quarter; 
 if your group has questions about a problem, you may ask in drop-in 
 help hours or post a public post on Piazza.

All submitted homework for this class must be typed. 
You can use a word processing editor if you like (Microsoft Word, Open Office, Notepad, Vim, Google Docs, etc.) 
but you might find it useful to take this opportunity to learn LaTeX. 
LaTeX is a markup language used widely in computer science and mathematics. 
The homework assignments are typed using LaTeX and you can use the source files 
as templates for typesetting your solutions.

{\bf Integrity reminders}
\begin{itemize}
\item Problems should be solved together, not divided up between the partners. The homework is
designed to give you practice with the main concepts and techniques of the course, 
while getting to know and learn from your classmates.
\item You may not collaborate on homework questions graded for correctness with anyone other than your group members.
You may ask questions about the homework in office hours (of the instructor, TAs, and/or tutors) and 
on Piazza (as private notes viewable only to the Instructors).  
You \emph{cannot} use any online resources about the course content other than the class material 
from this quarter -- this is primarily to ensure that we all use consistent notation and
definitions (aligned with the textbook) and also to protect the learning experience you will have when
the `aha' moments of solving the problem authentically happen.
\item Do not share written solutions or partial solutions for homework with 
other students in the class who are not in your group. Doing so would dilute their learning 
experience and detract from their success in the class.
\end{itemize}

}

\newcommand{\gradeCorrect}{({\it Graded for correctness}) }
\newcommand{\gradeCorrectFirst}{\gradeCorrect\footnote{This means your solution 
will be evaluated not only on the correctness of your answers, but on your ability
to present your ideas clearly and logically. You should explain how you 
arrived at your conclusions, using
mathematically sound reasoning. Whether you use formal proof techniques or 
write a more informal argument
for why something is true, your answers should always be well-supported. 
Your goal should be to convince the
reader that your results and methods are sound.} }
\newcommand{\gradeComplete}{({\it Graded for completeness}) }
\newcommand{\gradeCompleteFirst}{\gradeComplete\footnote{This means you will 
get full credit so long as your submission demonstrates honest effort to 
answer the question. You will not be penalized for incorrect answers. 
To demonstrate your honest effort in answering the question, we 
expect you to include your attempt to answer *each* part of the question. 
If you get stuck with your attempt, you can still demonstrate 
your effort by explaining where you got stuck and what 
you did to try to get unstuck.} }

%\usepackage{tikz}
%\usetikzlibrary{circuits.logic.US,circuits.logic.IEC}

\usepackage{amssymb,amsmath,pifont,amsfonts,comment,enumerate,enumitem}
\usepackage{currfile,xstring,hyperref,tabularx,graphicx,wasysym}
\usepackage[labelformat=empty]{caption}
\usepackage{xcolor}
\usepackage{multicol,multirow,array,listings,tabularx,lastpage,textcomp,booktabs}

% NOTE(joe): This environment is credit @pnpo (https://tex.stackexchange.com/a/218450)
\lstnewenvironment{algorithm}[1][] %defines the algorithm listing environment
{   
    \lstset{ %this is the stype
        mathescape=true,
        frame=tB,
        numbers=left, 
        numberstyle=\tiny,
        basicstyle=\rmfamily\scriptsize, 
        keywordstyle=\color{black}\bfseries,
        keywords={,procedure, div, for, to, input, output, return, datatype, function, in, if, else, foreach, while, begin, end, }
        numbers=left,
        xleftmargin=.04\textwidth,
        #1
    }
}
{}
\lstnewenvironment{java}[1][]
{   
    \lstset{
        language=java,
        mathescape=true,
        frame=tB,
        numbers=left, 
        numberstyle=\tiny,
        basicstyle=\ttfamily\scriptsize, 
        keywordstyle=\color{black}\bfseries,
        keywords={, int, double, for, return, if, else, while, }
        numbers=left,
        xleftmargin=.04\textwidth,
        #1
    }
}
{}

\newcommand\abs[1]{\lvert~#1~\rvert}
\newcommand{\st}{\mid}

\newcommand{\A}[0]{\texttt{A}}
\newcommand{\C}[0]{\texttt{C}}
\newcommand{\G}[0]{\texttt{G}}
\newcommand{\U}[0]{\texttt{U}}

\newcommand{\cmark}{\ding{51}}
\newcommand{\xmark}{\ding{55}}


