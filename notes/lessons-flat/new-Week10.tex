\documentclass[12pt, oneside]{article}

\usepackage[letterpaper, scale=0.89, centering]{geometry}
\usepackage{fancyhdr}
\setlength{\parindent}{0em}
\setlength{\parskip}{1em}

\pagestyle{fancy}
\fancyhf{}
\renewcommand{\headrulewidth}{0pt}
\rfoot{\href{https://creativecommons.org/licenses/by-nc-sa/2.0/}{CC BY-NC-SA 2.0} Version \today~(\thepage)}

\usepackage{amssymb,amsmath,pifont,amsfonts,comment,enumerate,enumitem}
\usepackage{currfile,xstring,hyperref,tabularx,graphicx,wasysym}
\usepackage[labelformat=empty]{caption}
\usepackage{xcolor}
\usepackage{multicol,multirow,array,listings,tabularx,lastpage,textcomp,booktabs}

\lstnewenvironment{algorithm}[1][] {   
    \lstset{ mathescape=true,
        frame=tB,
        numbers=left, 
        numberstyle=\tiny,
        basicstyle=\rmfamily\scriptsize, 
        keywordstyle=\color{black}\bfseries,
        keywords={,procedure, div, for, to, input, output, return, datatype, function, in, if, else, foreach, while, begin, end, }
        numbers=left,
        xleftmargin=.04\textwidth,
        #1
    }
}
{}
\lstnewenvironment{java}[1][]
{   
    \lstset{
        language=java,
        mathescape=true,
        frame=tB,
        numbers=left, 
        numberstyle=\tiny,
        basicstyle=\ttfamily\scriptsize, 
        keywordstyle=\color{black}\bfseries,
        keywords={, int, double, for, return, if, else, while, }
        numbers=left,
        xleftmargin=.04\textwidth,
        #1
    }
}
{}

\newcommand\abs[1]{\lvert~#1~\rvert}
\newcommand{\st}{\mid}

\newcommand{\A}[0]{\texttt{A}}
\newcommand{\C}[0]{\texttt{C}}
\newcommand{\G}[0]{\texttt{G}}
\newcommand{\U}[0]{\texttt{U}}

\newcommand{\cmark}{\ding{51}}
\newcommand{\xmark}{\ding{55}}

 
\begin{document}
\begin{flushright}
    \StrBefore{\currfilename}{.}
\end{flushright} 
\subsection*{Week 10 at a glance}

\subsubsection*{We will be learning and practicing to:}
\begin{itemize}

\item Clearly and unambiguously communicate computational ideas using appropriate formalism. Translate across levels of abstraction.
\begin{itemize}
    \item Defining functions, predicates, and binary relations using multiple representations
    \item Determining whether a given binary relation is symmetric, antisymmetric, reflexive, and/or transitive
    \item Determining whether a given binary relation is an equivalence relation and/or a partial order
    \item Drawing graph representations of relations and functions e.g. Hasse diagram and partition diagram
\end{itemize}

\item Know, select and apply appropriate computing knowledge and problem-solving techniques. Reason about computation and systems. Use mathematical techniques to solve problems. Determine appropriate conceptual tools to apply to new situations. Know when tools do not apply and try different approaches. Critically analyze and evaluate candidate solutions.
\begin{itemize}
    \item Using the definitions of the div and mod operators on integers
    \item Using divisibility and primality predicates
    \item Applying the definition of congruence modulo n and modular arithmetic
\end{itemize}

\item Apply proof strategies, including direct proofs and proofs by contradiction, and determine whether a proposed argument is valid or not.
\begin{itemize}
    \item Using proofs as knowledge discovery tools to decide whether a statement is true or false
\end{itemize}
\end{itemize}

\subsubsection*{TODO:}
\begin{list}
   {\itemsep2pt}
   \item Homework assignment 6 (due Thursday June 6, 2024).
   \item Review for Final exam. The test is scheduled for Thursday June 13 11:30a-2:29pm, location TBA (see Schedule of Classes).
\end{list}

\newpage
\section*{Week 10 Monday: Equivalence Relations and Partial Orders}
\subsection*{Definitions and representations}


A relation is an {\bf equivalence relation} means it is reflexive, symmetric, and transitive. 

A relation is a {\bf partial ordering} (or partial order) means 
it is reflexive, antisymmetric, and transitive. 

For a partial ordering, its {\bf Hasse diagram} is a graph representing the 
relationship between elements in the ordering. The nodes (vertices) of the graph 
are the elements of the 
domain of the binary relation. The edges do not have arrow heads. The
directionality of the partial order is indicated by 
the arrangements of the notes. The nodes are arranged so that nodes connected to nodes
above them by edges indicate that the relation holds between the 
lower node and the higher node. 
Moreover, the diagram omits self-loops and 
omits edges that are guaranteed by transitivity.
 \vfill


Draw the Hasse diagram of the partial order on the set $\{a,b,c,d,e,f,g\}$ defined as
\begin{align*}
    \{  &(a,a), (b,b), (c,c), (d,d), (e,e), (f,f), (g,g), \\
        &(a,c), (a,d), (d,g), (a,g), (b,f), (b,e), (e,g), (b,g) \}
\end{align*}

\vspace{100pt}

%
 \vfill
\newpage
\subsection*{Exploring equivalence relations}


A {\bf partition} of a set $A$ is a set of non-empty, disjoint subsets 
$A_1, A_2, \cdots, A_n$ such that 
\[
    A = \bigcup_{i=1}^{n} A_i = \{ x \mid \exists i (x \in A_i) \}
\] 

An {\bf equivalence class} of an element $a \in A$ 
with respect to an equivalence relation $R$ on the set $A$ is the set 
\[
    \{s \in A \mid (a, s) \in R \}
\] 
We write $[a]_R$ for this set, which is the equivalence class of $a$ with respect to $R$. 
{\bf Fact}: When $R$ is an equivalence relation on a nonempty set $A$, 
the collection of equivalence classes of $R$ is a partition of $A$.

Also, given a partition $P$ of $A$, the relation $R_P$ on $A$ given by 
\[
    R_P = \{ (x,y) \in A \times A ~|~ \text{$x$ and $y$ are in the same part of the partition $P$}\}
\]
is an equivalence relation on $A$. \vfill


{\it Recall}: We say $a$ is {\bf congruent to} $b$ \textbf{mod} $n$ 
means $(a, b) \in R_{(\textbf{mod } n)}$. 
A common notation is to write this as $a \equiv b (\textbf{mod } n)$.

We can partition the set of integers using equivalence classes of  $R_{(\textbf{mod } 4)}$

\begin{align*}
    [0]_{R_{(\textbf{mod } 4)}} &= \phantom{ \{ x \in \mathbb{Z} \mid x \equiv 0 ((\textbf{mod } 4)) \} 
    = \{ x \in \mathbb{Z} \mid x \textbf{ mod } 4 = 0 \textbf{ mod } 4 = 0 \} = \{ 4c \mid c \in \mathbb{Z}\} }\\
    [1]_{R_{(\textbf{mod } 4)}} &= \phantom{ \{ x \in \mathbb{Z} \mid x \equiv 1 ((\textbf{mod } 4)) \} 
    = \{ x \in \mathbb{Z} \mid x \textbf{ mod } 4 = 1 \textbf{ mod } 4 = 1 \} = \{ 4c+1 \mid c \in \mathbb{Z}\} }\\
    [2]_{R_{(\textbf{mod } 4)}} &= \phantom{ \{ x \in \mathbb{Z} \mid x \equiv 2 ((\textbf{mod } 4)) \} 
    = \{ x \in \mathbb{Z} \mid x \textbf{ mod } 4 = 2 \textbf{ mod } 4 = 0 \} = \{ 4c+2 \mid c \in \mathbb{Z}\} }\\
    [3]_{R_{(\textbf{mod } 4)}} &= \phantom{ \{ x \in \mathbb{Z} \mid x \equiv 3 ((\textbf{mod } 4)) \} 
    = \{ x \in \mathbb{Z} \mid x \textbf{ mod } 4 = 3 \textbf{ mod } 4 = 3 \} = \{ 4c+3 \mid c \in \mathbb{Z}\} }\\
    [4]_{R_{(\textbf{mod } 4)}} &= \phantom{ \{ x \in \mathbb{Z} \mid x \equiv 4 ((\textbf{mod } 4)) \} 
    = \{ x \in \mathbb{Z} \mid x \textbf{ mod } 4 = 4 \textbf{ mod } 4 = 0 \} = \{ 4c \mid c \in \mathbb{Z}\} }\\
    [5]_{R_{(\textbf{mod } 4)}} &= \phantom{ \{ x \in \mathbb{Z} \mid x \equiv 5 ((\textbf{mod } 4)) \} 
    = \{ x \in \mathbb{Z} \mid x \textbf{ mod } 4 = 5 \textbf{ mod } 4 = 1 \} = \{ 4c+1 \mid c \in \mathbb{Z}\} }\\
    [-1]_{R_{(\textbf{mod } 4)}} &= \phantom{ \{ x \in \mathbb{Z} \mid x \equiv -1 ((\textbf{mod } 4)) \} 
    = \{ x \in \mathbb{Z} \mid x \textbf{ mod } 4 = -1 \textbf{ mod } 4 = 3 \} = \{ 4c+3 \mid c \in \mathbb{Z}\} }
\end{align*}
\[
\mathbb{Z} =  [0]_{R_{(\textbf{mod } 4)}}~ \cup ~[1]_{R_{(\textbf{mod } 4)}} ~\cup~[2]_{R_{(\textbf{mod } 4)}}~\cup~
[3]_{R_{(\textbf{mod } 4)}}
\]





 \vfill
\newpage


Integers are useful because they can be used to encode other objects
and have multiple representations. However, infinite sets are sometimes
expensive to work with computationally. Reducing our attention to 
a {\it partition of the integers} based on congrunce mod $n$, where
each part is represented by a (not too large) integer gives a useful 
compromise where many algebraic properties of the integers are preserved, 
and we also get the benefits of a finite domain. Moreover, modular arithmetic
is well-suited to model any cyclic behavior. 

{\bf Lemma}: For $a, b \in \mathbb{Z}$ 
and positive integer $n$, $(a,b) \in R_{(\textbf{mod } n)}$ if and only if  $n | a-b$.

{\bf Proof}: 

\phantom{Consider arbitrary integers $a,b$ and arbitrary positive integer $n$.}

\phantom{Assume $a \textbf{ mod } n = b \textbf{ mod } n$. Call this 
remainder $r$ and we have integers $q_1, q_2$ such that $a = q_1 n + r$
and $b = q_2 n + r$. Calculating $a-b = (q_1 n + r) - (q_2n +r) = (q_1 - q_2)n$ 
an integer multiple of $n$, as required.}

\phantom{Assume there is integer $c$ with $a-b = cn$. By long division 
there are integers $q$ and $r$ ($0 \leq r < n$) with $b = qn + r$. Then 
$a = b + cn = qn+r + cn = (q+c)n + r$. Since long division gives a unique remainder,
this means $a \textbf{ mod } n = r = b \textbf{ mod } n$, as required.}

\vspace{200pt} \vfill


{\bf Modular arithmetic}: 

{\bf Lemma}: For $a, b, c, d \in \mathbb{Z}$ 
and positive integer $n$, if $a \equiv b ~(\textbf{ mod } n)$ and $c \equiv d ~(\textbf{ mod } n)$ 
then $a+c \equiv b+d ~(\textbf{ mod } n)$ and $ac \equiv bd ~(\textbf{ mod } n)$.
{\bf Informally}: can bring mod ``inside" and do it first, for addition and for multiplication.


$(102 + 48) \textbf{ mod } 10 = \underline{\phantom{\hspace{3in}}} $ 

$(7 \cdot 10) \textbf{ mod } 5 = \underline{\phantom{\hspace{3.3in}}} $ 

$(2^5) \textbf{ mod } 3 =  \underline{\phantom{\hspace{3.45in}}} $ 

\vfill

 \vfill
\newpage


{\bf Application: Cycling}

How many minutes past the hour are we at?  \hfill {\it Model with} $+15 \textbf{ mod } 60$

\begin{tabular}{lccccccccccc}
{\bf Time:} &12:00pm  &12:15pm&12:30pm  &12:45pm&1:00pm  &1:15pm&1:30pm  &1:45pm&2:00pm \\
{\bf ``Minutes past":} &$0$ & $15$ & $30$ & $45$ &$0$ & $15$ & $30$ & $45$ &$0$\\
\end{tabular}
\vfill

Replace each English letter by a letter that's fifteen ahead of it in the alphabet
  \hfill {\it Model with} $+15 \textbf{ mod } 26$

{\tiny
\begin{tabular}{lcccccccccccccccccccccccccc}
{\bf Original index:} & $0$ & $1$
 & $2$ & $3$ &  $4$ & $5$ &  $6$ & $7$ &  $8$ & $9$ & $10$ & $11$ & $12$ & $13$ & $14$ & $15$ & 
  $16$ & $17$ &  $18$ & $19$ &  $20$ & $21$ &  $22$ & $23$ & $24$ & $25$\\
{\bf Original letter:} & A & B& C & D & E & F& G& H & I & J & K & L &M & N& O &P &Q & R & S & T & U & V & W & X & Y & Z \\
{\bf Shifted letter}: &P &Q & R & S & T & U & V & W & X & Y & Z & A & B& C & D & E & F& G& H & I & J & K & L &M & N& O \\
{\bf Shifted index:} &$15$ & 
  $16$ & $17$ &  $18$ & $19$ &  $20$ & $21$ &  $22$ & $23$ & $24$ & $25$ & $0$ & $1$
 & $2$ & $3$ &  $4$ & $5$ &  $6$ & $7$ &  $8$ & $9$ & $10$ & $11$ & $12$ & $13$ & $14$ 
\end{tabular}
}
\vfill \vfill
\newpage
{\bf Application: Cryptography}

{\bf Definition}: Let $a$ be a positive integer and $p$ be a 
large\footnote{We leave the definition of ``large'' vague here, but 
think hundreds of digits for practical applications. In practice, 
we also need a particular relationship between $a$ and $p$ to hold, 
which we leave out here.} prime number, both known to everyone. 
Let $k_1$ be a secret large number known only to person $P_1$ (Alice) 
and $k_2$ be a secret large number known only to person $P_2$ (Bob). 
Let the {\bf Diffie-Helman shared key} for $a, p, k_1, k_2$ be 
$(a^{k_1\cdot k_2} \textbf{ mod } p)$.


{\bf Idea}: $P_1$ can quickly compute the Diffie-Helman shared key 
knowing only $a, p, k_1$ and the result of $a^{k_2} \textbf{ mod } p$ 
(that is, $P_1$ can compute the shared key without knowing $k_2$, 
only $a^{k_2} \textbf{ mod } p$). Similarly, $P_2$ can 
quickly compute the Diffie-Helman shared key knowing only 
$a, p, k_2$ and the result of $a^{k_1} \textbf{ mod } p$ 
(that is, $P_2$ can compute the shared key without knowing $k_1$, 
only $a^{k_1} \textbf{ mod } p$). But, any person $P_3$ who 
knows neither $k_1$ nor $k_2$ (but may know any and all of the other values) 
cannot compute the shared secret efficiently.

{\bf Key property for *shared* secret}: 
\[
    \forall a \in \mathbb{Z} \, \forall b \in \mathbb{Z} \, \forall g \in \mathbb{Z}^+ \, 
    \forall n \in \mathbb{Z}^+ ((g^a \textbf{ mod } n)^b, (g^b \textbf{ mod } n)^a) \in R_{(\textbf{mod } n)}
\]

{\bf Key property for shared *secret*}:

There are efficient algorithms to calculate the result of modular exponentiation 
but there are no (known) efficient algorithms to calculate discrete logarithm. \newpage
\section*{Week 10 Wednesday: Applications}



{\it Recall:} 


A relation is an {\bf equivalence relation} means it is reflexive, symmetric, and transitive. 

An {\bf equivalence class} of an element $a \in A$ 
with respect to an equivalence relation $R$ on the set $A$ is the set 
\[
    \{s \in A \mid (a, s) \in R \}
\] 
We write $[a]_R$ for this set, which is the equivalence class of $a$ with respect to $R$. 

A {\bf partition} of a set $A$ is a set of non-empty, disjoint subsets 
$A_1, A_2, \cdots, A_n$ such that 
\[
    A = \bigcup_{i=1}^{n} A_i = \{ x \mid \exists i (x \in A_i) \}
\] 
{\bf Claim}: For each  $a \in U$, $[a]_{E}  \neq  \emptyset$.

{\bf Proof}: Towards a $\underline{\phantom{\hspace{1.3in}}}$ 
consider an arbitrary element $a$ in $U$. 
We will work to show that $[a]_E \neq \emptyset$, namely that $\exists x \in [a]_E$.
By definition of equivalence classes, we can rewrite this goal as 
$$\exists x \in U ~( ~(a,x) \in E~)$$ 
Towards a $\underline{\phantom{\hspace{1.3in}}}$, consider $x = a$, 
an element of $U$ by definition. By $\underline{\phantom{\hspace{1.3in}}}$ of $E$, 
we know that $(a,a) \in E$  and thus the existential quantification has been proved.\\


{\bf Claim}: For each $a \in U$, there is some $b \in U$  such that $a \in [b]_{E}$.

Towards a $\underline{\phantom{\hspace{1.3in}}}$ 
consider an arbitrary element $a$ in $U$. By definition of equivalence classes, 
we can rewrite the goal as 
$$\exists b \in U ~( ~(b,a) \in E~)$$
Towards a $\underline{\phantom{\hspace{1.3in}}}$, consider $b = a$, 
an element of $U$ by definition. By $\underline{\phantom{\hspace{1.3in}}}$  of $E$, 
we know that $(a,a) \in E$  and thus the existential quantification has been proved. \\
 
{\bf Claim}: For each  $a,b  \in U$ , $(~(a,b)  \in  E ~\to ~ [a]_{E}  = [b]_{E}~)$
and  $(~(a,b)  \notin  E ~\to ~ [a]_{E} \cap[b]_{E} = \emptyset~)$

\phantom{Let $a,b$ be arbitrary. For first goal, assume towards
direct proof that $(a,b) \in E$. To show $[a]_E = [b]_E$ 
first consider arbitrary element $x$ in $[a]_E$. By definition
$(a,x) \in E$ so by symmetry, $(x,a) \in E$ and by 
transitivity with assumption, $(x,b)\in E$. Thus, by definition
of equivalence class, $x \in [b]_E$. Similarly, take arbitrary
element $y \in [b]_E$. By definition $(y,b) \in E$ 
and applying symmetry to assumption, we have $(b,a) \in E$
so by transitivity $(y,a) \in E$ and $y \in[a]_E$, as required
to complete the proof of set equality. For the second goal, 
assume (towards a proof by contrapositive) that $[a]_E \cap [b]_E 
\neq \emptyset$. Then there is a witness $w \in [a]_E \cap [b]_E$.
By definition of equivalence classes $(a,w) \in E$ and $(b,w) \in E$.
By symmetry, we have $(w,b) \in E$ so by transitivity $(a,b) \in E$,
as required.}


\vspace{200pt}

{\bf Corollary}: Given an equivalence relation $E$ on set $U$,  
$\{ [x]_{E} \mid x \in U  \}$ is a partition of $U$.
 
Last time, we saw that partitions associated to equivalence relations
were useful in the context of modular arithmetic.
Today we'll look at a different application.


Recall that 
in a movie recommendation system, each 
user's ratings of movies is represented as a $n$-tuple 
(with the positive integer $n$ 
being the number of movies in the database), 
and each component of 
the $n$-tuple is an element of the collection $\{-1,0,1\}$. 

We call $Rt_5$ the set of all ratings $5$-tuples.

Define $d: Rt_5 \times Rt_5 \to \mathbb{N}$ by
\[
    d (~(~ (x_1, x_2, x_3, x_4, x_5), (y_1, y_2, y_3, y_4, y_5) ~) ~) = \sum_{i=1}^5 |x_i - y_i|
\]

Consider the following binary relations on $Rt_5$.
\[
    E_{proj} =  \{ ( ~(x_1, x_2, x_3, x_4, x_5), (y_1, y_2, y_3, y_4, y_5)~) \in
         Rt_5 \times Rt_5 ~\mid~(x_1 = y_1) \land  (x_2 = y_2) \land (x_3 = y_3) \}
\]

Example ordered pair in $E_{proj}$: 

\vspace{20pt}

Reflexive? Symmetric? Transitive? Antisymmetric?

\vspace{120pt}



\[
    E_{dist} =  \{ (u,v) \in Rt_5 \times Rt_5 ~\mid~ d( ~(u,v)~ ) \leq 2 \}
\]
Example ordered pair in $E_{dist}$: 

\vspace{20pt}

Reflexive? Symmetric? Transitive? Antisymmetric?

\vspace{120pt}


\[
E_{circ} =  \{ (u,v) \in Rt_5 \times Rt_5 ~\mid~ d(~ ( ~(0,0,0,0,0)~, u)~ ) =  d( ~(~(0,0,0,0,0),v~)~) \}
\]
Example ordered pair in $E_{circ}$: 

\vspace{20pt}

Reflexive? Symmetric? Transitive? Antisymmetric?

\vspace{120pt}

The partition of $Rt_5$ defined by $\underline{\phantom{E_{proj}}}$ is

\resizebox{0.9\hsize}{!}{
\begin{math}
\begin{aligned}
    \{ ~~ \{~ &(-1,-1,-1,-1,-1), (-1,-1,-1,-1,0), (-1,-1,-1,-1,1),
     (-1,-1,-1,0,-1), (-1,-1,-1,0,0), (-1,-1,-1,0,1), 
     (-1,-1,-1,1,-1), (-1,-1,-1,1,0), (-1,-1,-1,1,1) ~\},\\
    ~~ \{~ &(-1,-1,0,-1,-1), (-1,-1,0,-1,0), (-1,-1,0,-1,1),
    (-1,-1,0,0,-1), (-1,-1,0,0,0), (-1,-1,0,0,1), 
    (-1,-1,0,1,-1), (-1,-1,0,1,0), (-1,-1,0,1,1) ~\},\\
    ~~ \{~ &(-1,-1,1,-1,-1), (-1,-1,1,-1,0), (-1,-1,1,-1,1),
    (-1,-1,1,0,-1), (-1,-1,1,0,0), (-1,-1,1,0,1), 
    (-1,-1,1,1,-1), (-1,-1,1,1,0), (-1,-1,1,1,1) ~\},\\
    ~~ \{~ &(-1,0,-1,-1,-1), (-1,0,-1,-1,0), (-1,0,-1,-1,1),
    (-1,0,-1,0,-1), (-1,0,-1,0,0), (-1,0,-1,0,1), 
    (-1,0,-1,1,-1), (-1,0,-1,1,0), (-1,0,-1,1,1) ~\},\\
    ~~ \{~ &(-1,0,0,-1,-1), (-1,0,0,-1,0), (-1,0,0,-1,1),
    (-1,0,0,0,-1), (-1,0,0,0,0), (-1,0,0,0,1), 
    (-1,0,0,1,-1), (-1,0,0,1,0), (-1,0,0,1,1) ~\},\\
    ~~ \{~ &(-1,0,1,-1,-1), (-1,0,1,-1,0), (-1,0,1,-1,1),
    (-1,0,1,0,-1), (-1,0,1,0,0), (-1,0,1,0,1), 
    (-1,0,1,1,-1), (-1,0,1,1,0), (-1,0,1,1,1) ~\},\\
    ~~ \{~ &(-1,1,-1,-1,-1), (-1,1,-1,-1,0), (-1,1,-1,-1,1),
    (-1,1,-1,0,-1), (-1,1,-1,0,0), (-1,1,-1,0,1), 
    (-1,1,-1,1,-1), (-1,1,-1,1,0), (-1,1,-1,1,1) ~\},\\
    ~~ \{~ &(-1,1,0,-1,-1), (-1,1,0,-1,0), (-1,1,0,-1,1),
    (-1,1,0,0,-1), (-1,1,0,0,0), (-1,1,0,0,1), 
    (-1,1,0,1,-1), (-1,1,0,1,0), (-1,1,0,1,-1) ~\},\\
    ~~ \{~ &(-1,1,1,-1,-1), (-1,1,1,-1,0), (-1,1,1,-1,1),
    (-1,1,1,0,-1), (-1,1,1,0,0), (-1,1,1,0,1), 
    (-1,1,1,1,-1), (-1,1,1,1,0), (-1,1,1,1,1) ~\},\\
~~ \{~ &(0,-1,-1,-1,-1), (0,-1,-1,-1,0), (0,-1,-1,-1,1),
   (0,-1,-1,0,-1), (0,-1,-1,0,0), (0,-1,-1,0,1), 
   (0,-1,-1,1,-1), (0,-1,-1,1,0), (0,-1,-1,1,1) ~\},\\
  ~~ \{~ &(0,-1,0,-1,-1), (0,-1,0,-1,0), (0,-1,0,-1,1),
  (0,-1,0,0,-1), (0,-1,0,0,0), (0,-1,0,0,1), 
  (0,-1,0,1,-1), (0,-1,0,1,0), (0,-1,0,1,1) ~\},\\
  ~~ \{~ &(0,-1,1,-1,-1), (0,-1,1,-1,0), (0,-1,1,-1,1),
  (0,-1,1,0,-1), (0,-1,1,0,0), (0,-1,1,0,1), 
  (0,-1,1,1,-1), (0,-1,1,1,0), (0,-1,1,1,1) ~\},\\
  ~~ \{~ &(0,0,-1,-1,-1), (0,0,-1,-1,0), (0,0,-1,-1,1),
  (0,0,-1,0,-1), (0,0,-1,0,0), (0,0,-1,0,1), 
  (0,0,-1,1,-1), (0,0,-1,1,0), (0,0,-1,1,1) ~\},\\
  ~~ \{~ &(0,0,0,-1,-1), (0,0,0,-1,0), (0,0,0,-1,1),
  (0,0,0,0,-1),(0,0,0,0,0), (0,0,0,0,1), 
  (0,0,0,1,-1),(0,0,0,1,0), (0,0,0,1,1) ~\},\\
  ~~ \{~ &(0,0,1,-1,-1), (0,0,1,-1,0), (0,0,1,-1,1),
  (0,0,1,0,-1), (0,0,1,0,0), (0,0,1,0,1), 
  (0,0,1,1,-1), (0,0,1,1,0), (0,0,1,1,1) ~\},\\
  ~~ \{~ &(0,1,-1,-1,-1), (0,1,-1,-1,0), (0,1,-1,-1,1),
  (0,1,-1,0,-1), (0,1,-1,0,0), (0,1,-1,0,1), 
  (0,1,-1,1,-1), (0,1,-1,1,0), (0,1,-1,1,1) ~\},\\
  ~~ \{~ &(0,1,0,-1,-1), (0,1,0,-1,0), (0,1,0,-1,1),
  (0,1,0,0,-1), (0,1,0,0,0), (0,1,0,0,1), 
  (0,1,0,1,-1), (0,1,0,1,0), (0,1,0,1,-1) ~\},\\
  ~~ \{~ &(0,1,1,-1,-1), (0,1,1,-1,0), (0,1,1,-1,1),
  (0,1,1,0,-1), (0,1,1,0,0), (0,1,1,0,1), 
  (0,1,1,1,-1), (0,1,1,1,0), (0,1,1,1,1) ~\},\\
~~ \{~ &(1,-1,-1,-1,-1), (1,-1,-1,-1,0), (1,-1,-1,-1,1),
    (1,-1,-1,0,-1),(1,-1,-1,0,0), (1,-1,-1,0,1), 
    (1,-1,-1,1,-1), (1,-1,-1,1,0), (1,-1,-1,1,1) ~\},\\
   ~~ \{~ &(1,-1,0,-1,-1), (1,-1,0,-1,0), (1,-1,0,-1,1),
   (1,-1,0,0,-1), (1,-1,0,0,0), (1,-1,0,0,1), 
   (1,-1,0,1,-1), (1,-1,0,1,0), (1,-1,0,1,1) ~\},\\
   ~~ \{~ &(1,-1,1,-1,-1), (1,-1,1,-1,0), (1,-1,1,-1,1),
   (1,-1,1,0,-1), (1,-1,1,0,0), (1,-1,1,0,1), 
   (1,-1,1,1,-1), (1,-1,1,1,0), (1,-1,1,1,1) ~\},\\
   ~~ \{~ &(1,0,-1,-1,-1), (1,0,-1,-1,0), (1,0,-1,-1,1),
   (1,0,-1,0,-1), (1,0,-1,0,0), (1,0,-1,0,1), 
   (1,0,-1,1,-1), (1,0,-1,1,0), (1,0,-1,1,1) ~\},\\
   ~~ \{~ &(1,0,0,-1,-1), (1,0,0,-1,0), (1,0,0,-1,1),
   (1,0,0,0,-1), (1,0,0,0,0), (1,0,0,0,1), 
   (1,0,0,1,-1), (1,0,0,1,0), (1,0,0,1,1) ~\},\\
   ~~ \{~ &(1,0,1,-1,-1), (1,0,1,-1,0), (1,0,1,-1,1),
   (1,0,1,0,-1), (1,0,1,0,0), (1,0,1,0,1), 
   (1,0,1,1,-1), (1,0,1,1,0), (1,0,1,1,1) ~\},\\
   ~~ \{~ &(1,1,-1,-1,-1), (1,1,-1,-1,0), (1,1,-1,-1,1),
   (1,1,-1,0,-1), (1,1,-1,0,0), (1,1,-1,0,1), 
   (1,1,-1,1,-1), (1,1,-1,1,0), (1,1,-1,1,1) ~\},\\
   ~~ \{~ &(1,1,0,-1,-1), (1,1,0,-1,0), (1,1,0,-1,1),
   (1,1,0,0,-1), (1,1,0,0,0), (1,1,0,0,1), 
   (1,1,0,1,-1), (1,1,0,1,0), (1,1,0,1,-1) ~\},\\
   ~~ \{~ &(1,1,1,-1,-1), (1,1,1,-1,0), (1,1,1,-1,1),
   (1,1,1,0,-1), (1,1,1,0,0), (1,1,1,0,1), 
   (1,1,1,1,-1), (1,1,1,1,0), (1,1,1,1,1) ~\}\\
\} \qquad & \\
\end{aligned}
\end{math}
}

The partition of $Rt_5$ defined by $E = \underline{\phantom{E_{circ}}}$ is

\begin{math}
\begin{aligned}
    \{ ~~  & \\
    &[ ~(0,0,0,0,0)~ ]_E   \\
    &, [ ~(0,0,0,0,1)~ ]_E \\
    &, [ ~(0,0,0,1,1)~ ]_E \\
    &, [ ~(0,0,1,1,1)~ ]_E \\
    &, [ ~(0,1,1,1,1)~ ]_E \\
    &, [ ~(1,1,1,1,1)~ ]_E \\
    \qquad\}  & \\
\end{aligned}
\end{math}

How many elements are in each part of the partition? 

{\bf Scenario}: Good morning! You're a user experience engineer at Netflix. A
product goal is to design customized home pages for groups of users who have
similar interests. Your manager tasks you with designing an algorithm for
producing a clustering of users based on their movie interests,
so that customized homepages can be engineered for each group.



Your idea: equivalence relations! 


\[
    E_{id} = \{ ( ~(x_1, x_2, x_3, x_4, x_5), (x_1, x_2, x_3, x_4, x_5)~) \mid 
    (x_1, x_2, x_3, x_4, x_5) \in Rt_5  \}
\]

{\it Describe how each homepage should be designed \ldots }

\vspace{100pt}



\[
    E_{proj} =  \{ ( ~(x_1, x_2, x_3, x_4, x_5), (y_1, y_2, y_3, y_4, y_5)~) \in
         Rt_5 \times Rt_5 ~\mid~(x_1 = y_1) \land  (x_2 = y_2) \land (x_3 = y_3) \}
\]


{\it Describe how each homepage should be designed \ldots }

\vspace{100pt}

\[
E_{circ} =  \{ (u,v) \in Rt_5 \times Rt_5 ~\mid~ d(~ ( ~(0,0,0,0,0)~, u)~ ) =  d( ~(~(0,0,0,0,0),v~)~) \}
\]

{\it Describe how each homepage should be designed \ldots }


\vspace{100pt}































%
 

\newpage
\section*{Week 10 Friday: Review and Advice}


Convert $(2A)_{16}$ to 
\begin{itemize}
\item binary (base \underline{\phantom{~~~2~~}})

\vspace{50pt}

\item decimal (base \underline{\phantom{~~10~~}})

\vspace{50pt}

\item octal (base \underline{\phantom{~~~8~~}})

\vspace{50pt}

\item ternary (base \underline{\phantom{~~~3~~}})

\vspace{50pt}

\end{itemize} \newpage


The bases of RNA strands are elements of the set $B = \{\A, \C, \G, \U \}$. 
The set of RNA strands $S$ is defined (recursively) by:
\[
\begin{array}{ll}
\textrm{Basis Step: } & \A \in S, \C \in S, \U \in S, \G \in S \\
\textrm{Recursive Step: } & \textrm{If } s \in S\textrm{ and }b \in B \textrm{, then }sb \in S
\end{array}
\]
where $sb$ is string concatenation.

Each of the sets below is described using set builder notation. Rewrite them using the roster method. 
\begin{itemize}
\item $\{s \in S ~|~ \text{the leftmost base in $s$ is the same as the rightmost base in $s$ and 
$s$ has length $3$} \}$ 

\vspace{50pt}

\item $\{s \in S ~|~ \text{there are twice as many $\A$s as $\C$s in $s$ and $s$ has length $1$} \}$ 

\vspace{50pt}

\end{itemize}

Certain 
 sequences of bases serve important biological functions in translating RNA to proteins. The following
 recursive definition gives a special set of RNA strands: The set of RNA strands $\hat{S}$ is defined (recursively)
 by 
 
 \begin{alignat*}{2}
\text{Basis step:} & & \A\U\G \in \hat{S}\\
\text{Recursive step:} & \qquad& \text{If } s \in \hat{S} \text{ and } x \in R \text{, then } sx\in \hat{S}\\
 \end{alignat*}
 where $R = \{ \U\U\U, \C\U\C, \A\U\C, \A\U\G, \G\U\U, \C\C\U, \G\C\U, \U\G\G, \G\G\A \}$.

Each of the sets below is described using set builder notation. Rewrite them using the roster method. 
\begin{itemize}
\item $\{s \in \hat{S} ~|~ s \text{ has length less than or equal to $5$} \}$ 

\vspace{50pt}


\item $\{s \in S ~|~ \text{there are twice as many $\C$s as $\A$s in $s$ and $s$ has length $6$} \}$ 

\vspace{50pt}

\end{itemize} \newpage


Let $W = \mathcal{P}( \{ 1,2,3,4,5\})$. Consider the statement 
\[
\forall A \in W~ \forall B \in W ~ \forall C \in W~ ((A \cap B = A \cap C) \to (B=C) )
\]
Translate the statement to English.
Negate the statement 
and translate this negation to English.
Decide whether the original statement or its negation is true
and justify your decision.
 \newpage


The set of linked lists of natural numbers $L$ is defined by 
 \begin{alignat*}{2}
\text{Basis step:} & &[] \in L \\
\text{Recursive step:} & \qquad& \text{If } l \in L \text{ and } n \in \mathbb{N} \text{, then } (n,l) \in L\\
 \end{alignat*}
 The function $length: L \to \mathbb{N}$ that computes the length of a list is
  \begin{alignat*}{2}
\text{Basis step:} & &length([]) = 0\\
\text{Recursive step:} & \qquad& \text{If $l \in L$ and $n \in \mathbb{N}$, then } length( ( n,l) ) = 1 + length(l)\\
 \end{alignat*}

Prove or disprove: the function $length$ is onto.

\vfill

Prove or disprove: the function $length$ is one-to-one.

\vfill
 \newpage


Suppose $A$ and $B$ are sets and $A \subseteq B$:

{\bf True or False}?  If $A$ is infinite then $B$ is finite.

\vspace{50pt}

{\bf True or False}?  If $A$ is countable then $B$ is countable.

\vspace{50pt}

{\bf True or False}?  If $B$ is infinite then $A$ is finite.

\vspace{50pt}

{\bf True or False}?  If $B$ is uncountable then $A$ is countable.

\vspace{50pt} \newpage


Compute the last digit of 
\[
    (42)^{2024}
\]

\vfill

{\it Extra} Describe the pattern that helps you perform this computation 
and prove it using mathematical induction. \newpage


\section*{Looking forward}

\subsection*{Tips for future classes from the CSE 20 TAs and tutors}
\begin{itemize}
\item In class
\begin{itemize}
\item Go to class
\item There's usually a space for skateboards/longboards/eboards to go at the front or rear of the lecture hall 
\item If you have a flask water bottle please ensure that its secured during a lecture and it cannot fall - putting on the floor often leads to it falling since people sometimes cross your seats.
\item Take notes - it's much faster and more effective to note-take in class than watch recordings after, particularly if you do so longhand
\item Resist the urge to sit in the back. You will be able to focus much better sitting near the front, where there are fewer screens in front of you to distract from the lecture content
\item If you bring your laptop to class to take notes / access class materials, sit towards the back of the room to minimize distractions for people sitting behind you!
\item Don't be afraid to talk to the people next to you during group discussions. Odds are they're as nervous as you are, and you can all benefit from sharing your thoughts and understanding of the material 
\item Certain classes will podcast the lectures, just like Zoom archives lecture recordings, at podcast.ucsd.edu . If they aren't podcasted, and you want to record lectures, ask your professor for consent first
\end{itemize}
\item Office hours, tutor hours, and the CSE building
\begin{itemize}
\item Office hours are a good place to hang out and get work done while being able to ask questions as they come up 
\item Get to know the CSE building: CSE B260, basement labs, office hours rooms
\end{itemize}
\item Libraries and on-campus resources
\begin{itemize}
\item Look up what library floors are for what, how to book rooms: East wing of Geisel is open 24/7 (they might ask to see an ID if you stay late), East Wing of Geisel has chess boards and jigsaw puzzles, study pods on the 8th floor, 
free computers/wifi
\item Know Biomed exists and is usually less crowded
\item Most libraries allow you to borrow whiteboards and markers (also laptops, tablets, microphones, and other cool stuff) for 24 hours
\item Take advantage of Dine with a prof / Coffee with a prof program. It's legit free food / coffee once per quarter. 
\item When planning out your daily schedule, think about where classes are, how much time will they take, are their places to eat nearby and how you can schedule social time with friends to nearby areas 
\item Take into account the distances between classes if they are back to back
\end{itemize} 
\newpage
\item Final exams
\begin{itemize}
\item Don't forget your university card during exams. Physical version is best for ID checks.
\item Look up seating assignments for exams and go early to make sure you're in the right place (check the exits to make sure you're reading it the right way) 
\item Know where your exam is being held (find it on a map at least a day beforehand). Finals are often in strange places that take a while to find 
\end{itemize}
\end{itemize}

\newpage
\section*{Review Quiz}
\begin{enumerate}
    \item Binary relations. 
        \begin{enumerate}
            \item \hspace{1in}\\ 

Assume there are five movies in the database, so that each user's ratings
can be represented as a $5$-tuple. We call $Rt_5$ the set of all ratings $5$-tuples.
Consider the binary relation on the  set of all 
$5$-tuples where each  component of the $5$-tuple is an element of the collection $\{-1,0,1\}$
\[
G_1 =  \{  (u,v)  \mid 
\text{the ratings of users $u$  and  $v$  agree about the first 
movie in the database} \}
\]
\[
G_2 =  \{  (u,v)  \mid 
\text{the ratings of users $u$  and  $v$  agree about at least two movies} \}
\]


\begin{enumerate}
    \item {\bf True} or {\bf False}: 
    The  relation $G_1$ holds of  $u=(1,1,1,1,1)$ and
    $v=(-1,-1,-1,-1,-1)$
    \item {\bf True} or {\bf False}: 
    The  relation $G_2$ holds of  $u=(1,0,1,0,-1)$ and
    $v=(-1,0,1,-1,-1)$
    \item {\bf True} or {\bf False}: $G_1$ is reflexive; namely, 
    $\forall u  ~(~(u,u) \in G_1~)$
    \item {\bf True} or {\bf False}:  $G_1$ is symmetric; namely, 
    $\forall u ~\forall  v ~(~(u,v) \in G_1 \to  (v,u) \in G_1~)$
    \item {\bf True} or {\bf False}:  $G_1$ is transitive; namely, 
    $\forall u ~\forall  v  ~\forall w (~\left( (u,v) \in G_1 \wedge (v,w)\in G_1\right) \to  (u,w) \in G_1~)$
    \item {\bf True} or {\bf False}:  $G_2$ is reflexive; namely, 
    $\forall u   ~(~(u,u) \in G_2~)$
    \item {\bf True} or {\bf False}:  $G_2$ is symmetric; namely, 
    $\forall u ~\forall  v  ~(~(u,v)\in G_2 \to  (v,u) \in G_2~)$
    \item {\bf True} or {\bf False}:  $G_2$ is transitive; namely, 
    $\forall u~\forall  v  ~\forall w (~\left( (u,v) \in G_2 \wedge (v,w)\in G_2\right) \to  (u,w) \in G_2~)$
\end{enumerate}
             \item \hspace{1in}\\ 

Consider the binary relation on $\mathbb{Z}^+$ defined by $\{(a,b) ~|~ \exists c \in \mathbb{Z} ( b = ac)\}$.
Select all and only the properties that this binary relation has.
\begin{enumerate}
\item It is reflexive.
\item It is symmetric.
\item It is transitive.
\item It is antisymmetric.
\end{enumerate}         \end{enumerate}
    \item Equivalence relations. 
        \begin{enumerate}
            \item \hspace{1in}\\ 

Recall that the binary relation $EQ_{\mathbb{R}}$ on $\mathcal{P}(\mathbb{R})$ is
\[
\{ (X_1, X_2 ) \in\mathcal{P}(\mathbb{R})  \times \mathcal{P}(\mathbb{R}) ~|~ |X_1| = |X_2| \}
\]
and $R_{(\textbf{mod } n)}$ is the set of all pairs of integers $(a, b)$ 
such that $(a \textbf{ mod } n = b \textbf{ mod } n)$.

Select all and only the correct items.
\begin{enumerate}
\item $(\mathbb{Z}, \mathbb{R}) \in EQ_{\mathbb{R}}$
\item $(0,1) \in EQ_{\mathbb{R}}$
\item $(\emptyset, \emptyset) \in EQ_{\mathbb{R}}$
\item $(-1,1) \in R_{(\textbf{mod } 2)}$
\item $(1,-1) \in R_{(\textbf{mod } 3)}$ 
\item $(4, 16, 0) \in R_{\textbf{(mod } 4)}$ 
\end{enumerate}             \item \hspace{1in}\\ 

Fill in the blanks in the following proof that, for any equivalence relation $R$ on a set $A$,
\[
\forall a \in A ~\forall b \in A~\left( (a,b) \in R \leftrightarrow [a]_R\cap [b]_R \neq \emptyset \right)
\]

{\bf Proof}: Towards a  \textbf{(a)}$\underline{\phantom{\hspace{1.3in}}}$, consider arbitrary elements $a$, $b$ in $A$. We will 
prove the biconditional statement by proving each direction of the conditional in turn.

{\bf Goal 1}: we need to show $(a,b) \in R \to [a]_R\cap [b]_R \neq \emptyset$
{\it Proof of Goal 1}: Assume towards a \textbf{(b)}$\underline{\phantom{\hspace{1.3in}}}$ 
that $(a,b) \in R$. We will work to show
that $[a]_R\cap [b]_R \neq \emptyset$. Namely, we need an element that is in both equivalence classes, that is, we
 need to prove the existential claim $\exists x \in A ~(x \in [a]_{R} \land x \in [b]_{R})$. 
 Towards a \textbf{(c)}$\underline{\phantom{\hspace{1.3in}}}$, consider $x = b$, 
 an element of $A$ by definition. By \textbf{(d)}$\underline{\phantom{\hspace{1.3in}}}$  of $R$, we know that $(b,b) \in R$ 
 and thus, $b \in [b]_{R}$.
 By assumption in this proof, we have that $(a,b) \in R$, and so by  definition of equivalence classes, $b \in [a]_R$.
 Thus, we have proved both conjuncts and this part of the proof is complete.
 
{\bf Goal 2}: we need to show $[a]_R\cap [b]_R \neq \emptyset \to (a,b) \in R $
{\it Proof of Goal 2}: Assume towards a \textbf{(e)}$\underline{\phantom{\hspace{1.3in}}}$ 
that $[a]_R\cap [b]_R \neq \emptyset $. We will work to show
that $(a,b) \in R$. By our assumption, the existential claim $\exists x \in A ~(x \in [a]_{R} \land x \in [b]_{R})$
is true. Call $w$ a witness; thus, $w \in [a]_R$ and $w \in [b]_R$. 
By  definition of equivalence classes, $w \in [a]_R$ means $(a,w) \in R$ and $w \in [b]_R$ means $(b,w) \in R$.
By \textbf{(f)}$\underline{\phantom{\hspace{1.3in}}}$  of $R$, $(w,b) \in R$. By 
\textbf{(g)}$\underline{\phantom{\hspace{1.3in}}}$ of $R$, since $(a,w) \in R$ and $(w,b) \in R$, we have that
$(a,b) \in R$, as required for  this part of the proof.
 
Consider the following expressions as options to fill in the two proofs above. Give your answer as one of the numbers below for each blank a-c. You may use some numbers for more than one blank, but each letter only uses one of the expressions below.

\begin{multicols}{2}
\begin{enumerate}[label=\roman*]
\item exhaustive proof
\item proof by universal generalization
\item proof of existential using a witness
\item proof by cases
\item direct proof
\item proof by contrapositive
\item proof by contradiction
\item reflexivity
\item symmetry
\item transitivity
\end{enumerate}
\end{multicols}         \end{enumerate}
    \item Partial orders. \hspace{1in}\\ 

\begin{enumerate}
\item Consider the partial order on the set $\mathcal{P}(\{1,2,3\})$ given by the binary relation 
    $\{ (X,Y) ~|~X \subseteq Y \}$
    \begin{enumerate}
    \item How many nodes are in the Hasse diagram of this partial order?
    \item How many edges are in the Hasse diagram of this partial order?
    \end{enumerate}
\item Consider the binary relation on $\{1,2,4,5,10,20\}$ 
defined by $\{(a,b) ~|~ \exists c \in \mathbb{Z} ( b = ac)\}$.
    \begin{enumerate}
    \item How many nodes are in the Hasse diagram of this partial order?
    \item How many edges are in the Hasse diagram of this partial order?
    \end{enumerate}
\end{enumerate}     \item Equivalence classes and partitions. 
        \begin{enumerate}
            \item \hspace{1in}\\ 

Select all and only the correct statements about an equivalence relation $E$ on 
a set $U$:
\begin{enumerate}
\item $E \in U \times U$
\item $E = U \times U$
\item $E \subseteq U \times U$
\item $\forall x \in U ~([x]_E \in U)$
\item $\forall x \in U ~([x]_E \subseteq U)$
\item $\forall x \in U ~([x]_E \in \mathcal{P}(U))$
\item $\forall x \in U ~([x]_E \subseteq \mathcal{P}(U))$
\end{enumerate}             \item Select all and only the partitions of $\{1,2,3,4,5\}$ from the sets below.
            \begin{enumerate}
            \item $\{1,2,3,4,5\}$
            \item $\{\{1,2,3,4,5\}\}$
            \item $\{\{1\},\{2\},\{3\},\{4\},\{5\}\}$
            \item $\{ \{1\}, \{2,3\}, \{4\} \}$
            \item $\{ \{\emptyset, 1, 2\}, \{3,4,5\}\}$
            \end{enumerate}
        \end{enumerate}
    \item Modular exponentiation. \hspace{1in}\\ 

Modular exponentiation is required to carry out the Diffie-Helman protocol for 
computing a shared secret over an unsecure channel.

Consider the following algorithm for fast exponentiation (based on binary 
expansion of the exponent).

  

\begin{algorithm}[caption={Modular Exponentation}]
    procedure $modular~exponentiation$($b$: integer; 
                 $n = (a_{k-1}a_{k-2} \ldots a_1 a_0)_2$, $m$: positive integers)
    $x$ := $1$
    $power$ := $b$ mod $m$
    for $i$:= $0$ to $k-1$
      if $a_i = 1$ then $x$:= $(x \cdot power)$ mod $m$
      $power$ := $(power \cdot power)$ mod $m$
    return $x$ $\{x~\textrm{equals}~b^n \textbf{ mod } m\} $
\end{algorithm}     
    \begin{enumerate}
        \item If we wanted to calculate $3^8 \textbf{ mod } 7$ 
        using the modular exponentation algorithm above, what are the values of 
        the parameters $b$, $n$, and $m$?  (Write these values in usual, 
        decimal-like, mathematical notation.)
        \item Give the output of the $modular~exponentiation$ algorithm 
        with these parameters, i.e.\ calculate $3^8 \textbf{ mod } 7$.
        (Write these values in usual, 
        decimal-like, mathematical notation.)
    \end{enumerate} \end{enumerate}
\end{document}