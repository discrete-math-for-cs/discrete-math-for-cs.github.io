%! app: TODOapp
%! outcome: TODOoutcome

For $b$ an integer between $2$ and $9$ (inclusive), we define the set of possible coefficients 
in a base $b$ expansion as  $C_b = \{ x \in \mathbb{N} \mid 0 \leq x < b \}$.
\begin{enumerate}
\item  ({\it Graded for fair effort completeness}) Fill in the recursive definition of the set of all strings that form base $b$ expansions, $S_b$: 

{\bf Definition} The set $S_b$ is defined (recursively) by:
\[
\begin{array}{ll}
\textrm{Basis Step: } & \textrm{If } x \in \underline{\phantom{\hspace{2in}}} \textrm{, then } x \in S_b\\
\textrm{Recursive Step: } & \textrm{If } s \in S_b \textrm{ and } x \in C_b \textrm{, then }
\underline{\phantom{\hspace{2in}}} \in S_b
\end{array}
\]

{\it Hint:} Leave the integer $b$ as a parameter so that your definition works no matter which integer
between $2$ and $9$ (inclusive) is plugged in for $b$.
\item For each integer $b > 1$, we define the function $v_b: S_b \to \mathbb{Z}^+$ defined recursively by 


Basis Step: If $x \in C_b$ and $x \neq 0$, then $v_b (x) = x$.

Recursive Step: If $s \in S_b$ and $x \in C_b$, then $v_b (sx) = bv_b(s) + x$, where $b v_b (s)$
is the result of integer multiplication.

\begin{enumerate}
\item For $b = 3$, calculate $v_3(111)$, including all steps in your calculation and justifying them by the definitions
in the previous parts of this question, or explain why this is impossible.
\item For $b = 8$, calculate $v_8(0)$, including all steps in your calculation and justifying them by the definitions
in the previous parts of this question, or explain why this is impossible.
\end{enumerate}


\item  ({\it Graded for fair effort completeness}) Describe in English the rule defining the output of applying $v_b$. Try not to make a literal translation.
\end{enumerate}

