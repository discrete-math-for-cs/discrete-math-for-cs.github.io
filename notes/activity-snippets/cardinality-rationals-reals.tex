%! app: TODOapp
%! outcome: functions for cardinality, classify cardinality, important sets

{\bf Comparing $\mathbb{Q}$ and $\mathbb{R}$} 


Both $\mathbb{Q}$ and $\mathbb{R}$ have no greatest element.

Both $\mathbb{Q}$ and $\mathbb{R}$ have no least element.

The quantified statement 
\[
    \forall x \forall y (x < y \to \exists z ( x < z < y) )
\]
is true about both $\mathbb{Q}$ and $\mathbb{R}$.

Both $\mathbb{Q}$ and $\mathbb{R}$ are infinite. But, $\mathbb{Q}$ is countably infinite
whereas $\mathbb{R}$ is uncountable.\\


{\bf The set of real numbers}

$\mathbb{Z} \subsetneq \mathbb{Q} \subsetneq \mathbb{R}$


{\bf  Order axioms} (Rosen Appendix 1): 

\begin{center}
\begin{tabular}{p{1.2in}p{4in}}
Reflexivity &  $\forall a \in  \mathbb{R} (a \leq a)$\\
Antisymmetry &  $\forall a \in  \mathbb{R}~\forall b \in \mathbb{R}~(~(a \leq b~ \wedge ~b \leq a) \to (a=b)~)$\\
Transitivity &  $\forall a \in  \mathbb{R}~\forall b \in \mathbb{R}~\forall c \in \mathbb{R}~
(~(a \leq b \wedge b \leq c) ~\to  ~(a \leq c)~)$ \\
Trichotomy & 
$\forall a \in \mathbb{R}~\forall b \in \mathbb{R}~ ( ~(a=b ~\vee~ b > a ~\vee~ a  < b)  $
\end{tabular}
\end{center}


{\bf  Completeness axioms} (Rosen Appendix 1): 


\begin{center}
\begin{tabular}{p{1.4in}p{6in}}
Least upper bound &  Every nonempty set of real numbers that 
is bounded  above has  a  least upper bound  
%\phantom{~$\forall X \in \mathcal{P}(\mathbb{R}) ( ~(\exists x (x \in X)  \wedge \exists  y ~\forall x ( x \in X ~\to~ x \leq y) )~\to ~\exists  y_0 \left( ~\forall x ( x \in X ~\to~ x \leq y) \wedge 
%\forall z ( \forall  x (x \in X ~\to~ x \leq z) ~\to ~ y  \leq  z)\right) )$}\\
\\
Nested intervals &  For each sequence  of intervals  $[a_n , b_n]$
where, for each $n$, $a_n < a_{n+1} < b_{n+1} < b_n$, there
is at least one  real number $x$ such that, for all $n$, 
$a_n \leq x \leq b_n$.\\
\end{tabular}
\end{center}

Each real  number $r  \in  \mathbb{R}$ is described by a function to give better and better approximations
\[
x_r: \mathbb{Z}^+ \to \{0,1\}  \qquad  \text{where  $x_r(n ) =  n^{th} $ bit in  binary expansion of $r$}
\]
\begin{center}
\begin{tabular}{|c|c|p{3.9in}|}
\hline
$r$ & Binary expansion & $x_r$ \\
\hline
%$0.5$ & $0.10000\ldots$ &  $x_{0.5}(n) = \begin{cases} 1 &\text{if $n = 1$} \\ 0&\text{otherwise} \end{cases}$\\
%& & \\
%\hline
$0.1$ & $0.00011001 \ldots$ &  $x_{0.1}(n) = \begin{cases} 0&\text{if $n > 1$ and $(n~\text{\bf mod}~4) =2$} \\
0&\text{if $n=1$ or if $n > 1$ and $(n~\text{\bf mod}~4) =3$} \\1&\text{if $n > 1$ and $(n~\text{\bf mod}~4) =0$} \\
1&\text{if $n > 1$ and $(n~\text{\bf mod}~4) =1$} \end{cases}$  \\
&&  \\
\hline
$\sqrt{2} - 1 = 0.4142135 \ldots$  &$0.01101010\ldots$& Use linear approximations
(tangent lines from calculus) to get algorithm for bounding error of successive operations. Define 
$x_{\sqrt{2}-1}(n)$ to be  $n^{th}$ bit in approximation  that has error less than  $2^{-(n+1)}$.
\\
&& \\
\hline
\end{tabular}
\end{center}

\newpage 

{\bf Claim}: $\{  r \in \mathbb{R} ~\mid~ 0 \leq r ~\wedge~ r \leq 1 \}$ is uncountable.

{\it Approach 1}: Mimic proof that $\mathcal{P}(\mathbb{Z}^+)$ is uncountable.


{\bf Proof}:  By definition of countable, since $\{  r \in \mathbb{R} ~\mid~ 0 \leq r ~\wedge~ r \leq 1 \}$
is not finite, {\bf to show} is $|\mathbb{N}| \neq  |\{  r \in \mathbb{R} ~\mid~ 0 \leq r ~\wedge~ r \leq 1 \}|$ .


{\bf To show} is
$\forall f : \mathbb{Z}^+ \to \{  r \in \mathbb{R} ~\mid~ 0 \leq r ~\wedge~ r \leq 1 \}  ~~(f \text{ is not a bijection})~~$.
Towards a proof by  universal generalization, consider  an arbitrary function 
$f:  \mathbb{Z}^+ \to \{  r \in \mathbb{R} ~\mid~ 0 \leq r ~\wedge~ r \leq 1 \}$.
{\bf To show}: $f$ is not a bijection.  It's enough to show that $f$ is not onto.
Rewriting using the definition of  onto, {\bf to show}:
\[
\exists x \in \{  r \in \mathbb{R} ~\mid~ 0 \leq r ~\wedge~ r \leq 1 \} ~\forall a \in \mathbb{N}  ~(~f(a) \neq  x~)
\]
In search of a witness, define the following  real number by defining its binary expansion
\[
d_f = 0.b_1 b_2 b_3 \cdots
\]
where $b_i = 1-b_{ii}$ where $b_{jk}$ is the coefficient of $2^{-k}$ in the binary expansion of $f(j)$.
Since\footnote{There's a subtle imprecision in this part of the proof as presented, but it can be fixed.} $d_f \neq f(a)$ for any positive integer $a$, $f$ is not onto.
%$x_r: \mathbb{Z}^+ \to  \{0,1\}$ where  $x_r(n) = n^{th}$ bit in binary expansion  of $r$
%\qquad 
%$X_r =  \{ n \in \mathbb{Z}^+ ~\mid~ x_r(n) = 1 \}$

{\it Approach 2}: Nested closed interval property

{\bf To show} $f: \mathbb{N} \to \{  r \in \mathbb{R} ~\mid~ 0  \leq r ~\wedge~ r \leq 1 \}$ is not onto. 
{\bf  Strategy}: Build
a sequence of nested closed intervals that each avoid some $f(n)$.   Then  the real
number that is in all of the intervals  can't be $f(n)$ for any $n$. Hence,  $f$ is not  onto.

Consider the function $f: \mathbb{N} \to \{  r \in \mathbb{R} ~\mid~ 0 \leq r ~\wedge~ r \leq 1 \}$ with  $f(n) = \frac{1+\sin(n)}{2}$

\begin{center}
\begin{tabular}{c||p{1.65in} || p{3in} }
$n$ &  $f(n)$& Interval that avoids $f(n)$ \\
\hline
$0$ & $0.5$ &  \\
$1$ &$0.920735\ldots$  &  \\
$2$ &$0.954649\ldots$ &  \\
$3$ &$0.570560\ldots$ & \\
$4$ &$0.121599\ldots $&  \\
\vdots &  &\\
\end{tabular}
\end{center}
 