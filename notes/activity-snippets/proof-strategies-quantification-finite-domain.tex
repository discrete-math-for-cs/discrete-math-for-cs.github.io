%! app: TODOapp
%! outcome: TODOoutcome

When a predicate $P(x)$ is over a {\bf finite} domain:
\begin{itemize}
\item To show that $\forall x  P(x)$ is true: check that $P(x)$ evaluates to $T$ at each domain element by evaluating over and over. 
This is called ``Proof of universal by {\bf exhaustion}".
\item To show that $\forall x  P(x)$ is false: find a {\bf counterexample}, a domain element where $P(x)$~evaluates~to~$F$.
\item To show that $\exists x  P(x)$ is true: find a {\bf witness}, a domain element where $P(x)$ evaluates to $T$.
\item To show that $\exists x  P(x)$ is false: check that $P(x)$ evaluates to $F$ at each domain element by evaluating over and over.
DeMorgan's Law gives that $\lnot \exists x P(x) ~~\equiv~~ \forall x \lnot P(x)$ so this amounts to a proof of universal by exhaustion.
\end{itemize}