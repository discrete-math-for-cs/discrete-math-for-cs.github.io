%! app: TODOapp
%! outcome: TODOoutcome

For many applications in cryptography and random number generation,
dividing very large integers efficiently is critical.  Recall the definitions
known as {\bf The Division Algorithm}:
Let $n$ be an integer 
and $d$ a positive integer. There are unique integers $q$ and $r$, with $0 \leq r < d$, such that 
$n = dq + r$. In this case, $d$ is called the divisor, $n$ is called the dividend, $q$ is called the quotient, 
and $r$ is called the remainder. We write $q=n \textbf{ div } d$ and $r=n \textbf{ mod } d$.

One application of the Division Algorithm is in computing the integer part of the logarithm.
When we discuss algorithms in this class, we will usually write them in 
pseudocode or English. Sometimes we will find it useful to relate the pseudocode to
runnable code in a programming language. We will typically use Java or python for this.

\begin{multicols}{2}
\begin{algorithm}[caption={Calculating log in pseudocode}]
procedure $\textit{log}$($n$: a positive integer)
$r$ := $0$
while $n$ > $1$
  $r$ := $r + 1$
  $n$ := $n$ div $2$
return $r$ $\{ r~\textrm{holds the result of the}~\log~\textrm{operation}\} $
\end{algorithm}
\columnbreak
\begin{java}[caption={Calculating log in Java}]
int log(int n) {
  if (n < 1) { 
    throw new IllegalArgumentException(); 
  }
  int result = 0;
  while(n > 1) {
    result = result + 1;
    n = n / 2;
  }
  return result;
}
\end{java}
\end{multicols}


\begin{enumerate}
\item Calculate $2024 \textbf{ div } 20$.  {\it You may use a calculator if you like.}
\item Calculate $2024 \textbf{ mod } 20$.  {\it You may use a calculator if you like.}
\item How many different possible values of $r$ (results of taking $n \textbf{ mod } d$) are there when 
we consider positive integer values of $n$ and $d$ is $20$?
\item What is the smallest positive integer $n$ which can be written as $16q+7$ for $q$ an integer?
\item What is the return value of \textit{log$(457)$}?
{\it You can run the Java version in order to calculate it.}
\end{enumerate}

