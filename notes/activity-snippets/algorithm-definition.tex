%! app: TODOapp
%! outcome: trace algorithms

\fbox{\parbox{\textwidth}{%
{\bf New!} An algorithm is a finite sequence of precise instructions for solving a problem.
\hfill
}}

Algorithms can be expressed in English or in more formalized descriptions like pseudocode or fully executable programs.


Sometimes, we can define algorithms whose output matches the 
rule for a function we already care about. Consider the (integer) logarithm function
\[
logb  : \{b \in \mathbb{Z} \mid b >1 \}  \times \mathbb{Z}^+ ~~\to~~ \mathbb{N}
\]
defined by 
\[
logb (~ (b,n)~) =  \text{greatest integer } y \text{ so that } b^y  \text{ is less than or equal to } n 
\]

%! app: Numbers
%! outcome: number representation

\begin{algorithm}[caption={Calculating integer part of base $b$ logarithm}]
   procedure $logb$($b$,$n$: positive integers with $b > 1$)
   $i$ := $0$
   while $n$ > $b-1$
     $i$ := $i + 1$
     $n$ := $n$ div $b$
   return $i$ $\{ i$ holds the integer part of the base $b$ logarithm of $n\}$
\end{algorithm}

Trace this algorithm with inputs $b=3$ and $n=17$

\vfill
  \begin{tabular}{l|c|c|c|c}
  & $b$ & $n$ & $i$  & $n > b-1$?\\
  \hline 
  Initial value & ~$3$~ & $17$ & \phantom{$0$} & \phantom{~Yes~}\\
  After 1 iteration  & \phantom{$3$} & \phantom{$5$} & \phantom{$1$} & \phantom{T}\\
  After 2 iterations & \phantom{$3$} & \phantom{$1$} & \phantom{$2$} & \phantom{F}\\
  After 3 iterations &  &  & & \\
  &&&&\\
  \end{tabular}

\vfill

\vfill

Compare: does the output match the rule for the (integer) logarithm function?


\newpage