%! app: TODOapp
%! outcome: functions for cardinality, classify cardinality, important sets


The set of positive integers $\mathbb{Z}^{+}$ is countably infinite.

The set of integers $\mathbb{Z}$ is countably infinite and is a proper superset of $\mathbb{Z}^{+}$. 
In fact, the set difference 
$$\mathbb{Z} \setminus \mathbb{Z}^{+} = \{ x \in \mathbb{Z} \mid x \notin \mathbb{Z}^+\} 
= \{ x \in \mathbb{Z} \mid x \leq 0 \}$$ is countably infinite.

The set of rationals 
$\mathbb{Q} = \left\{ \frac{p}{q} \mid p \in \mathbb{Z}  \text{ and  } 
q  \in \mathbb{Z} \text{ and } q \neq  0 \right\}$ is countably infinite.


The set of real numbers $\mathbb{R}$ is uncountable. In fact, the closed 
interval $\{x \in \mathbb{R} ~|~ 0 \leq x \leq 1\}$, 
any other nonempty closed interval of real numbers whose endpoints are 
unequal, as well as the related intervals that exclude one or both of the endpoints are
each uncountable.
The set of {\bf irrational} numbers $\overline{\mathbb{Q}} = \mathbb{R} - \mathbb{Q}
= \{ x \in \mathbb{R} \mid x \notin \mathbb{Q} \}$ is uncountable.

\vfill

We can classify any set as 
\begin{itemize}
\item {\bf Finite} size. Fact: For each positive number $n$, for any sets $X$ and $Y$ each 
size $n$, there is a bijection between $X$ and $Y$.
\item {\bf Countably Infinite}. Fact: for any countably infinite sets $X$ and $Y$, there is a bijection
between $X$ and $Y$.
\item {\bf Uncountable}. Examples: $\mathcal{P}(\mathbb{N})$, the power set of any infinite set, 
the set of real numbers, any nonempty interval of real numbers. Fact: there are (many) examples 
of uncountable sets that do not have a bijection between them.
\end{itemize}