%! app: TODOapp
%! outcome: trace algorithms

Real-life representations are often prone to corruption.  Biological codes, like RNA, 
may mutate naturally\footnote{Mutations of specific RNA codons have been linked to many disorders and cancers.}
and during measurement; cosmic radiation and other ambient noise 
can flip bits in computer storage\footnote{This RadioLab podcast episode
goes into more detail on bit flips: \url{https://www.wnycstudios.org/story/bit-flip}}. 
One way to recover from corrupted data is to introduce or exploit redundancy. 

Consider the following algorithm to introduce redundancy in a string of $0$s and $1$s.
\begin{algorithm}[caption={Create redundancy by repeating each bit three times}]
procedure $\textit{redun3}$($a_{k-1} \cdots a_0$: a nonempty bitstring)
for $i$ := $0$ to $k-1$
  $c_{3i}$ := $a_i$
  $c_{3i+1}$ := $a_i$
  $c_{3i+2}$ := $a_i$
return $c_{3k-1} \cdots c_0$
\end{algorithm}

\begin{algorithm}[caption={Decode sequence of bits using majority rule on consecutive three bit sequences}]
procedure $\textit{decode3}$($c_{3k-1} \cdots c_0$: a nonempty bitstring whose length is an integer multiple of $3$)
for $i$ := $0$ to $k-1$
  if exactly two or three of $c_{3i}, c_{3i+1}, c_{3i+2}$ are set to $1$
    $a_i$ := 1
  else 
    $a_i$ := 0
return $a_{k-1} \cdots a_0$
\end{algorithm}

Give a recursive definition of the set of outputs of the $redun3$ procedure, $Out$,

\phantom{{\bf Basis step}: $000 \in Out$ and $111 \in Out$\\ {\bf Recursive step}: If $x \in Out$ then $x000 \in Out$ and $x111 \in Out$ (where $x000$ and $x111$ are the results of string concatenation).}


Consider the message $m = 0001$ so that the sender calculates $redun3(m) = redun3(0001) = 000000000111$.

Introduce $\underline{\phantom{~~4~~}} $
errors into the message so that the signal received by the 
receiver is $\underline{\phantom{010100010101}}$
but the receiver is still able to decode the original message.


{\it Challenge: what is the biggest number of errors you can introduce?} 

Building a circuit for lines 3-6 in $decode$ procedure: given three input bits, we need to determine whether the
majority is a $0$ or a $1$.

\begin{center}
\begin{multicols}{2}\begin{tabular}{ccc|c}
$c_{3i}$ & $c_{3i+1}$ & $c_{3i+2}$ & $a_i$ \\
\hline
$1$ & $1$ & $1$ & $\phantom{1}$ \\
$1$ & $1$ & $0$ & $\phantom{1}$ \\
$1$ & $0$ & $1$ & $\phantom{1}$ \\
$1$ & $0$ & $0$ & $\phantom{0}$ \\
$0$ & $1$ & $1$ & $\phantom{1}$ \\
$0$ & $1$ & $0$ & $\phantom{0}$ \\
$0$ & $0$ & $1$ & $\phantom{0}$ \\
$0$ & $0$ & $0$ & $\phantom{0}$ \\\\
\end{tabular}
\columnbreak

Circuit 
\end{multicols}
\end{center}