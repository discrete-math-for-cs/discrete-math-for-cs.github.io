\input{../../resources/lesson-head.tex}

\subsection*{Week 7 at a glance}

\subsubsection*{We will be learning and practicing to:}
%data types
%proof signposts
%using proofs to evaluate
%universal generalization
%applying proof strategy
%logical structure to proof strategy
%identifying proof strategy in proof
\begin{itemize}

\item Clearly and unambiguously communicate computational ideas using appropriate formalism. Translate across levels of abstraction.
\begin{itemize}
   \item Translating between symbolic and English versions of statements using precise mathematical language
    \item Using appropriate signpost words to improve readability of proofs, including 'arbitrary' and 'assume'
\end{itemize}

\item Know, select and apply appropriate computing knowledge and problem-solving techniques. Reason about computation and systems. Use mathematical techniques to solve problems. Determine appropriate conceptual tools to apply to new situations. Know when tools do not apply and try different approaches. Critically analyze and evaluate candidate solutions.
\begin{itemize}
    \item Judging logical equivalence of compound propositions using symbolic manipulation with known equivalences, including DeMorgan's Law
    \item Writing the converse, contrapositive, and inverse of a given conditional statement
    \item Determining what evidence is required to establish that a quantified statement is true or false
    \item Evaluating quantified statements about finite and infinite domains
\end{itemize}

\item Apply proof strategies, including direct proofs and proofs by contradiction, and determine whether a proposed argument is valid or not.
\begin{itemize}
    \item Identifying the proof strategies used in a given proof
    \item Identifying which proof strategies are applicable to prove a given compound proposition based on its logical structure
    \item Carrying out a given proof strategy to prove a given statement
    \item Carrying out a universal generalization argument to prove that a universal statement is true
    \item Using proofs as knowledge discovery tools to decide whether a statement is true or false
\end{itemize}
\end{itemize}

\subsubsection*{TODO:}
\begin{list}
   {\itemsep2pt}
   \item Homework assignment 4 (due Tuesday May 14, 2024)
   \item Review quiz based on class material each day (due Friday May 17, 2024).
   \item Homework assignment 5 (due Tuesday May 21, 2024)
\end{list}

\newpage

\section*{Week 7 Monday: Mathematical and Strong Induction}
\subsection*{Visualizing induction}
\input{../activity-snippets/induction-dominos.tex}
\input{../activity-snippets/proof-strategy-mathematical-induction.tex}
\input{../activity-snippets/proof-strategy-strong-induction.tex}
\input{../activity-snippets/binary-expansions-exist-proof.tex}

\subsubsection*{Representing positive integers with primes}
\input{../activity-snippets/fundamental-theorem-proof.tex}
\subsubsection*{Sending old-fashioned mail with postage stamps}
%! app: TODOapp
%! outcome: induction flavors, strong induction proofs, mathematical induction proofs

Suppose we had postage stamps worth $5$ cents and $3$ cents.
Which number of cents can we form using these stamps?
In other words, which postage can we pay?

$11$? 

$15$? 


$4$?



\begin{align*}
    &CanPay(0) \land \lnot CanPay(1) \land \lnot CanPay(2) \land \\
    &CanPay(3) \land \lnot CanPay(4) \land CanPay(5) \land CanPay(6) \\
    &\lnot CanPay(7) \land \forall n \in \mathbb{Z}^{\geq 8} CanPay(n)
\end{align*}

where the predicate $CanPay$ with domain $\mathbb{N}$ is
\[
    CanPay(n) = \exists x \in \mathbb{N} \exists y \in \mathbb{N}  ( 5x+3y = n)
\]


{\bf Proof} (idea): First, explicitly give witnesses or general arguments
for postages between $0$ and $7$. 
To prove the universal claim, we can use mathematical induction or strong induction.

{\it Approach 1, mathematical induction}: if we have
stamps that add up to $n$ cents, need to use them (and others)
to give $n+1$ cents. How do we get $1$ cent with just $3$-cent
and $5$-cent stamps?

\vspace{-10pt}
Either \underline{take away a $5$-cent stamps and add two $3$-cent stamps},

\vspace{-10pt}
or \underline{take away three $3$-cent stamps and add two $5$-cent stamps}.

\vspace{-10pt}
The details of this proof by mathematical induction
are making sure we have enough 
stamps to use one of these approaches.

{\it Approach 2, strong induction}: assuming we know how to make postage
for {\bf all} smaller values (greater than or equal to $8$), when
we need to make $n+1$ cents, \underline{add one $3$ cent stamp to 
however we make $(n+1) - 3$ cents}.

\vspace{-10pt}
The details of this proof by strong induction are making sure we 
stay in the domain of the universal when applying the induction hypothesis.

\newpage
\subsubsection*{Finding a winning strategy for a game}
%! app: TODOapp
%! outcome: induction flavors, strong induction proofs

Consider the following game: two players start with 
two (equal) piles of jellybeans in front of them.
They take turns removing any positive integer number
of jellybeans at a time from one of two piles in 
front of them in turns.

The player who removes the last jellybean wins the game.

Which player (if any) has a strategy to guarantee
to win the game?


Try out some games, starting with $1$ jellybean in each pile,
then $2$ jellybeans in each pile, then $3$ jellybeans in each pile.
Who wins in each game?

\vspace{200pt}


Notice that reasoning about the strategy for the $1$ jellybean 
game is easier than about the strategy for the $2$ jellybean game.

{\it Formulate a winning strategy by working to 
transform the game to a simpler one we know we can win.}

\newpage

{\it Player 2's Strategy}: Take the same number of jellybeans that Player 1 did, 
but from the opposite pile. 


{\it Why is this a good idea}: If Player 2 plays this strategy, at the next turn
Player 1 faces a game with the same setup as the original, just with fewer
jellybeans in the two piles. Then Player 2 can keep playing this strategy to win.

{\bf Claim}: Player 2's strategy guarantees they will win the game.

{\bf Proof}: By strong induction, we will prove that for all positive 
integers $n$, Player 2's strategy guarantees a win in the game that starts with 
$n$ jellybeans in each pile.

{\bf Basis step}: WTS Player 2's strategy guarantees a win 
when each pile starts with $1$ jellybean.

In this case, Player 1 has to take the jellybean from one of the piles
(because they can't take from both piles at once).
Following the strategy, Player 2 takes the jellybean from the 
other pile, and wins because this is the last jellybean.

{\bf Recursive step}: Let $n$ be a positive integer. 
As the strong induction hypothesis, assume that
Player 2's strategy guarantees a win in the games 
where there are $1, 2, \ldots, n$ many jellybeans in each 
pile at the start of the game.

WTS that Player 2's strategy guarantees a win in the game where
there are $n+1$ in the jellybeans in each pile at the start of the game.

In this game, the first move has Player 1 take 
some number, call it $c$ (where $1 \leq c \leq n+1$),
of jellybeans from one of the piles. 
Playing according to their strategy, Player 2 then 
takes the same number of jellybeans from  the other pile.

Notice that $(c = n+1) \lor (c \leq n)$.

{\it Case 1}: Assume $c = n+1$, then in their first move, 
Player 2 wins because they take all of the second pile, which 
includes the last jellybean.

{\it Case 2}: Assume $c \leq n$. Then after Player 2's first move,
the two piles have an equal number of jellybeans. The number
of jellybeans in each pile is 
\[
    (n+1) - c
\]
and, since $1 \leq c \leq n$, this number is between $1$ and $n$.
Thus, at this stage of the game, the game appears identical to a new 
game where the two piles have an equal number of jellybeans between $1$
and $n$. Thus, the strong induction hypothesis applies, and Player 2's
strategy guarantees they win.


\newpage

\section*{Week 7 Wednesday: Recursive Data Structures}
\input{../activity-snippets/linked-lists-definition.tex}
\input{../activity-snippets/linked-lists-examples.tex}
\input{../activity-snippets/linked-list-length-definition.tex}
\vspace{50pt}
\input{../activity-snippets/linked-lists-prepend-definition.tex}
\vspace{50pt}
\input{../activity-snippets/linked-list-append-definition.tex}
\vspace{50pt}
\newpage
\input{../activity-snippets/linked-list-append-length-claim-proof.tex}
\newpage
\input{../activity-snippets/linked-list-example-each-length.tex}
\newpage

\section*{Week 7 Friday: Proof by Contradiction}
\input{../activity-snippets/proof-strategy-proof-by-contradiction.tex}
\subsection*{Least and greatest}
\input{../activity-snippets/least-greatest-proofs.tex}

\input{../activity-snippets/gcd-definition.tex}
\input{../activity-snippets/gcd-examples.tex}
\input{../activity-snippets/gcd-basic-claims.tex}
\input{../activity-snippets/gcd-lemma-relatively-prime.tex}

\newpage
\subsection*{Sets of numbers}

We've seen multiple representations of the set of positive integers
(using base expansions and using prime factorization). Now we're 
going to expand our attention to other sets of numbers as well.
\input{../activity-snippets/rational-numbers-definition.tex}
%! app: Numbers
%! outcome: data types, important sets

We have the following subset relationships between sets of numbers:

\[
    \mathbb{Z}^{+} \subsetneq \mathbb{N} \subsetneq \mathbb{Z} \subsetneq \mathbb{Q} \subsetneq \mathbb{R}
\]


Which of the proper subset inclusions above can you prove?

\vspace{50pt}
\input{../activity-snippets/proof-by-contradiction-irrational.tex}


\newpage

\subsection*{Review Quiz}
\begin{enumerate}
    \item Mathematical and strong induction for properties of numbers
    \begin{enumerate}
        \item \hspace{1in} \\ \input{../activity-snippets/quiz-binary-expansions-exist-invalid-proof.tex}
        \item \hspace{1in} \\ \input{../activity-snippets/quiz-making-change-proof-two-ways.tex}
    \end{enumerate}
    \item Winning strategy. \input{../activity-snippets/quiz-nim.tex}
    \item Linked lists. %! app: TODOapp
%! outcome: TODOoutcome

Recall the definition of linked lists from class.

Consider this (incomplete) definition:

{\bf Definition} The function $\textit{increment} : \underline{\hspace{6em}}$ 
that adds 1 to the data in each node of a linked list is defined by:
\[
\begin{array}{llll}
& & \textit{increment} : \underline{\hspace{3em}} & \to \underline{\hspace{3em}} \\
\textrm{Basis Step:} & & \textit{increment}([]) & = [] \\
\textrm{Recursive Step:} & \textrm{If } l \in L, n \in \mathbb{N} & \textit{increment}((n, l)) & = (1 + n, \textit{increment}(l))
\end{array}
\]

Consider this (incomplete) definition:

{\bf Definition} The function $\textit{sum} : L \to \mathbb{N}$ that adds 
together all the data in nodes of the list is defined by:
\[
\begin{array}{llll}
& & \textit{sum} : L & \to \mathbb{N} \\
\textrm{Basis Step:} & & \textit{sum}([]) & = 0 \\
\textrm{Recursive Step:} & \textrm{If } l \in L, n \in \mathbb{N} & \textit{sum}((n, l)) & = \underline{\hspace{8em}}
\end{array}
\]

You will compute a sample function application and then fill in the 
blanks for the domain and codomain of each of these functions.

\begin{enumerate}
    \item Based on the definition, what is the result of $\textit{increment}((4, (2, (7, []))))$? Write your answer directly with no spaces.
    
    \item Which of the following describes the domain and codomain of \textit{increment}?
    
    \begin{multicols}{2}
    \begin{enumerate}
        \item The domain is $L$ and the codomain is $\mathbb{N}$
        \item The domain is $L$ and the codomain is $\mathbb{N} \times L$
        \item The domain is $L \times \mathbb{N}$ and the codomain is $L$
        \item The domain is $L \times \mathbb{N}$ and the codomain is $\mathbb{N}$
        \item The domain is $L$ and the codomain is $L$
        \item None of the above
    \end{enumerate}
    \end{multicols}
    
    \item Assuming we would like $sum((5, (6, [])))$ to evaluate to $11$ and $sum((3, (1, (8, []))))$ to evaluate to $12$, which of the following could be used to fill in the definition of the recursive case of \textit{sum}?
    
     \begin{multicols}{2}
    \begin{enumerate}
        \item $\begin{cases}
            1 + \textit{sum}(l) & \textrm{when } n \neq 0 \\
            \textit{sum}(l) & \textrm{when } n = 0 \\
        \end{cases}$
        \item $1 + \textit{sum}(l)$
        \item $n + \textit{increment}(l)$
        \item $n + \textit{sum}(l)$
        \item None of the above
    \end{enumerate}
    \end{multicols}
    
%    \newpage
    \item Choose only and all of the following statements that are \textbf{well-defined}; that is, they correctly reflect the domains and codomains of the functions and quantifiers, and respect the notational conventions we use in this class. Note that a well-defined statement may be true or false.

    \begin{multicols}{2}    
    \begin{enumerate}
        \item $\forall l \in L \, (\textit{sum}(l))$
        \item $\exists l \in L \, (\textit{sum}(l) \land \textit{length}(l))$
        \item $\forall l \in L \, (\textit{sum}(\textit{increment}(l)) = 10)$
        \item $\exists l \in L \, (\textit{sum}(\textit{increment}(l)) = 10)$
        \item $\forall l \in L \, \forall n \in \mathbb{N} \, ((n \times l) \subseteq L)$
        \item $\forall l_1 \in L \, \exists l_2 \in L \, (\textit{increment}(\textit{sum}(l_1)) = l_2)$
        \item $\forall l \in L \, (\textit{length}(\textit{increment}(l)) = \textit{length}(l))$
    \end{enumerate}
    \end{multicols}
    
    \item Choose only and all of the statements in the previous part that are both well-defined and true.
\end{enumerate}
    \item Primes and divisors
    \begin{enumerate}
        \item \hspace{1in}\\ \input{../activity-snippets/quiz-prime-factorization.tex}
        \item \hspace{1in}\\ \input{../activity-snippets/quiz-no-greatest-prime.tex}
        \item \hspace{1in}\\ \input{../activity-snippets/quiz-calculating-gcd.tex}
    \end{enumerate}
    \item Proof strategies
    \begin{enumerate}
        \item \hspace{1in}\\ \input{../activity-snippets/quiz-choosing-proof-strategy.tex}
        \item \hspace{1in}\\ \input{../activity-snippets/quiz-odd-even-proofs.tex}
    \end{enumerate}
\end{enumerate}

\end{document}