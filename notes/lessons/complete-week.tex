\input{../../resources/lesson-head.tex}
\section*{Let's get started}

We want you to be successful. 

We will work together to build an 
environment in CSE 20 that supports your learning
in a way that respects your
perspectives, experiences, and identities (including race, ethnicity, heritage, gender, sex, 
class, sexuality, religion, ability, age, educational background, etc.).
Our goal is for you to  engage
with interesting and challenging concepts and 
feel comfortable exploring, asking questions, and thriving.

If you or someone you know is suffering from food and/or housing insecurities 
there are UCSD resources here to help:

Basic Needs Office: \href{https://basicneeds.ucsd.edu/}{https://basicneeds.ucsd.edu/}

Triton Food Pantry (in the old Student Center)
is free and anonymous, and includes produce: 

\href{https://www.facebook.com/tritonfoodpantry/}{https://www.facebook.com/tritonfoodpantry/}

Mutual Aid UCSD: \href{https://mutualaiducsd.wordpress.com/}{https://mutualaiducsd.wordpress.com/}

Financial aid resources, the possibility of emergency grant funding, and off-campus housing referral 
resources are available: see your College Dean of Student Affairs.

If you find yourself in an uncomfortable situation, ask for help. 
We are committed to upholding University policies regarding nondiscrimination, sexual violence and sexual harassment.
Here are some campus contacts that could provide this help: 
Counseling and Psychological Services (CAPS) at 858 534-3755 or \href{http://caps.ucsd.edu}{http://caps.ucsd.edu}; 
OPHD at 858 534-8298 or ophd@ucsd.edu , \href{http://ophd.ucsd.edu}{http://ophd.ucsd.edu};
CARE at Sexual Assault Resource Center at 858 534-5793 or sarc@ucsd.edu , \href{http://care.ucsd.edu}{http://care.ucsd.edu}.


Please reach out (minnes@ucsd.edu) if you need support with extenuating circumstances affecting CSE 20.

\vfill

\section*{Welcome to CSE 20: Discrete Math for CS in Spring 2024!}
Class website: \href{https://canvas.ucsd.edu//}{https://canvas.ucsd.edu/}


Instructor: Prof. Mia Minnes {\tiny{"Minnes" rhymes with Guinness}}, minnes@ucsd.edu, 
\href{http://cseweb.ucsd.edu/~minnes}{http://cseweb.ucsd.edu/~minnes}


Our team: One instructor + two TAs and eleven tutors + all of you

Fill in contact info for students around you, if you'd like:

\vfill


On a typical week in CSE 20: {\bf MWF} Lectures (sometimes with pre-class reading), {\bf W} Discussion,
Review quiz, then {\bf T} Homework due.
Office hours (hosted by instructors and TAs and tutors) where you can come to talk 
about course concepts and ask for help as you work through sample problems 
and Q+A on Piazza available throughout the week. CSE 20 has one project and two tests
this quarter.
Demonstration of class website on \href{https://canvas.ucsd.edu/}{Canvas (https://canvas.ucsd.edu/)}:
\begin{enumerate}
\item Syllabus
\item Notes for class + annotations
\item Assignments (PDF, tex, solutions)
\item Gradescope
\item Piazza
\item Dates
\end{enumerate}

\vfill
\begin{comment}
There are lots of great reasons to have a laptop, tablet, or phone open during class. You might be taking notes, 
getting a photo of an important moment on the board, trying out a construction that we're developing together, working 
on the review quiz, and so on. 
The main issue with screens and technology in the classroom isn't that they might 
distract you, 
it's the distraction of other students. We ask that if you would like to use
a device in class and may have 
have unrelated content open, please sit in one of the back two rows of the 
room so that it's not adversely affecting other students.


\vfill
\end{comment}
{\bf Pro-tip}: you can use MATH109 to replace CSE20 for prerequisites and other requirements.

\vfill

\section*{Themes and applications for CSE 20}
\begin{itemize}
\item {\bf Technical skepticism}: Know, select and apply appropriate computing knowledge and problem-solving techniques. 
Reason about computation and systems. 
Use mathematical techniques to solve problems. 
Determine appropriate conceptual tools to apply to new situations. 
Know when tools do not apply and try different approaches. 
Critically analyze and evaluate candidate solutions.
\item {\bf Multiple representations}: Understand, guide, shape impact of computing on society/the world. 
Connect the role of Theory CS classes to other applications (in undergraduate CS curriculum and beyond). 
Model problems using appropriate mathematical concepts.
Clearly and unambiguously communicate computational ideas using appropriate formalism. 
Translate across levels of abstraction.
\end{itemize}

{\bf Applications}: Numbers (how to represent them and use them in Computer Science), 
Recommendation systems and their roots in machine learning (with applications like Netflix),
``Under the hood" of computers (circuits, pixel color representation, data structures),
Codes and information (secret message sharing and error correction),
Bioinformatics algorithms and genomics (DNA and RNA).

\newpage

\subsection*{Week 1 at a glance}

\subsubsection*{We will be learning and practicing to:}
%data types, translating, important sets, write set definition, function and relation definitions, recursive definitions
\begin{itemize}
\item Model systems with tools from discrete mathematics and reason about implications of modelling choices. Explore applications in CS through multiple perspectives, including software, hardware, and theory.
\begin{itemize}

   \item Selecting and representing appropriate data types and using notation conventions to clearly communicate choices

\end{itemize}

\item Translate between different representations to illustrate a concept.

\begin{itemize}
   \item Translating between symbolic and English versions of statements using precise mathematical language
\end{itemize}


\item Use precise notation to encode meaning and present arguments concisely and clearly
\begin{itemize}
    \item Defining important sets of numbers, e.g. set of integers, set of rational numbers
    \item Precisely describing a set using appropriate notation e.g. roster method, set builder notation, and recursive definitions
    \item Defining functions using multiple representations
\end{itemize}

\item Know, select and apply appropriate computing knowledge and problem-solving techniques. Reason about computation and systems. Use mathematical techniques to solve problems. Determine appropriate conceptual tools to apply to new situations. Know when tools do not apply and try different approaches. Critically analyze and evaluate candidate solutions.
\begin{itemize}
    \item Using a recursive definition to evaluate a function or determine membership in a set
\end{itemize}

\end{itemize}

\subsubsection*{TODO:}
\begin{list}
   {\itemsep2pt}
   \item \#FinAid Assignment on Canvas (complete as soon as possible) 
   \item Review quiz based on class material each day (due Friday April 5, 2024)
   \item Homework assignment 1 (due Tuesday April 9, 2024).
\end{list}

\newpage

\section*{Week 1 Monday: Modeling applications}
%! app: TODOapp
%! outcome: data types

What data should we encode about each Netflix account holder to help us make effective recommendations?

\vfill
\vfill

In machine learning, clustering can be used to group similar data for prediction and recommendation.  For example,
each Netflix user's viewing history can be represented as a $n$-tuple indicating their preferences about
movies in the database, where $n$ is the number of movies in the database.  People with similar tastes in movies can then be clustered to provide recommendations
of movies for one another.  Mathematically, clustering is based on a notion of distance between pairs of $n$-tuples.


\vfill
\subsection*{Data Types: sets, $n$-tuples, and strings}
\documentclass[12pt, oneside]{article}

\usepackage[letterpaper, scale=0.89, centering]{geometry}
\usepackage{fancyhdr}
\setlength{\parindent}{0em}
\setlength{\parskip}{1em}

\pagestyle{fancy}
\fancyhf{}
\renewcommand{\headrulewidth}{0pt}
\rfoot{\href{https://creativecommons.org/licenses/by-nc-sa/2.0/}{CC BY-NC-SA 2.0} Version \today~(\thepage)}

\usepackage{amssymb,amsmath,pifont,amsfonts,comment,enumerate,enumitem}
\usepackage{currfile,xstring,hyperref,tabularx,graphicx,wasysym}
\usepackage[labelformat=empty]{caption}
\usepackage{xcolor}
\usepackage{multicol,multirow,array,listings,tabularx,lastpage,textcomp,booktabs}

\lstnewenvironment{algorithm}[1][] {   
    \lstset{ mathescape=true,
        frame=tB,
        numbers=left, 
        numberstyle=\tiny,
        basicstyle=\rmfamily\scriptsize, 
        keywordstyle=\color{black}\bfseries,
        keywords={,procedure, div, for, to, input, output, return, datatype, function, in, if, else, foreach, while, begin, end, }
        numbers=left,
        xleftmargin=.04\textwidth,
        #1
    }
}
{}
\lstnewenvironment{java}[1][]
{   
    \lstset{
        language=java,
        mathescape=true,
        frame=tB,
        numbers=left, 
        numberstyle=\tiny,
        basicstyle=\ttfamily\scriptsize, 
        keywordstyle=\color{black}\bfseries,
        keywords={, int, double, for, return, if, else, while, }
        numbers=left,
        xleftmargin=.04\textwidth,
        #1
    }
}
{}

\newcommand\abs[1]{\lvert~#1~\rvert}
\newcommand{\st}{\mid}

\newcommand{\A}[0]{\texttt{A}}
\newcommand{\C}[0]{\texttt{C}}
\newcommand{\G}[0]{\texttt{G}}
\newcommand{\U}[0]{\texttt{U}}

\newcommand{\cmark}{\ding{51}}
\newcommand{\xmark}{\ding{55}}

 
\begin{document}
\begin{flushright}
    \StrBefore{\currfilename}{.}
\end{flushright} \section*{Sets equality subset definition}


{\bf Definitions}:

A {\bf set} is an  unordered collection of  elements.
When $A$ and  $B$ are sets,  $A = B$ (set equality) means  
\[
    \forall x  ( x\in A \leftrightarrow x \in B)
\]

When $A$ and  $B$ are sets, $A \subseteq B$ (``$A$ is a {\bf subset} of $B$") means 
\[
    \forall x  (x \in A  \to x  \in B)
\]

When $A$ and  $B$ are sets,  $A \subsetneq B$ (``$A$ is a {\bf proper subset} of $B$") means 
\[
    (A\subseteq B) \wedge  (A \neq B)
\] \vfill
\section*{Predicate definition}


{\bf  Definition}: A  {\bf predicate}  is  a function from a given set (domain) to $\{T,F\}$.

A predicate can be applied, or {\bf evaluated} at, an element of the domain.

Usually, a predicate {\it describes a  property} that domain elements may or may not have.

Two predicates over the same domain are {\bf equivalent} means they evaluate to
the same truth values for all possible assignments of domain elements to the
input. In other words, they are equivalent means that they are equal as functions.

To define a predicate, we must specify its domain and its value at each domain element.
The rule assigning truth values to domain elements can be specified using a formula, 
English description, in a table (if the domain is finite), or recursively (if the domain is recursively
defined). \vfill
\section*{Predicate notation}


{\bf Notation}: for a predicate $P$ with domain $X_1 \times \cdots \times X_n$ and a 
$n$-tuple $(x_1, \ldots, x_n)$ 
with each $x_i \in X$, we 
can write $P(x_1, \ldots, x_n)$ to mean $P( ~(x_1, \ldots, x_n)~)$.
 \vfill
\section*{Defining functions more examples}


Let's practice with functions related to some of our applications so far.

Recall: We model the collection of user ratings of the four movies Dune, Oppenheimer, Barbie, Nimona as the set
$\{-1,0,1\}^4$ . One function that compares pairs of ratings is
$$d_0: \{-1,0,1\}^4 \times \{-1,0,1\}^4 \to \mathbb{R}$$
given by
\[
d_0 (~(~ (x_1, x_2, x_3, x_4), (y_1, y_2, y_3, y_4) ~) ~) = \sqrt{ (x_1 - y_1)^2 + (x_2 - y_2)^2 + (x_3 -y_3)^2 + (x_4 -y_4)^2}
\]

Notice: any ordered pair of ratings is an okay input to $d_0$.

Notice: there are (at most) 
\[
(3 \cdot 3 \cdot 3 \cdot 3)\cdot (3 \cdot 3 \cdot 3 \cdot 3) = 3^8 = 6561
\]
many pairs of ratings. There are therefore lots and lots of real numbers that are not the output of $d_0$.

\vfill

Recall: RNA is made up of strands of four different bases that encode genomic information
in specific ways.\\
The bases are elements of the set 
$B  = \{\A, \C, \U, \G \}$. The set of RNA strands $S$ is defined (recursively) by:
\[
\begin{array}{ll}
\textrm{Basis Step: } & \A \in S, \C \in S, \U \in S, \G \in S \\
\textrm{Recursive Step: } & \textrm{If } s \in S\textrm{ and }b \in B \textrm{, then }sb \in S
\end{array}
\]
where $sb$ is string concatenation.

\vfill
\newpage
{\bf Pro-tip}: informal definitions sometime use $\cdots$ to indicate ``continue the pattern''. Often, 
to make this pattern precise we use recursive definitions.

\vspace{-20pt}

\begin{center}
\begin{tabular}{p{0.65in}ccp{2.4in}p{2.4in}}
{\scriptsize {\bf Name}} & {\scriptsize {\bf  Domain}} & {\scriptsize {\bf Codomain}} & {\scriptsize {\bf Rule}} &{\scriptsize {\bf Example}}\\
\hline 
$rnalen$ & $S$ & $\mathbb{Z}^+$ & 
    {\begin{align*}    
    &\textrm{Basis Step:} \\
    &\textrm{If } b \in B\textrm{ then } \textit{rnalen}(b) = 1 \\
    &\textrm{Recursive Step:}\\
    &\textrm{If } s \in S\textrm{ and } b \in B\textrm{, then  }\\
    &\textit{rnalen}(sb) = 1 + \textit{rnalen}(s)
    \end{align*}} & 
    {\begin{align*}
        rnalen(\A\C) &\overset{\text{rec step}}{=} 1 +rnalen(\A) \\ 
        &\overset{\text{basis step}}{=} 1 + 1 = 2
    \end{align*}}\\
\hline
$basecount$ & $S \times B$ & $\mathbb{N}$ & 
{\begin{align*}    
    &\textrm{Basis Step:} \\
    &\textrm{If } b_1 \in B, b_2 \in B \textrm{ then} \\
    &basecount(~(b_1, b_2)~) = \\
    &\begin{cases}
        1 & \textrm{when } b_1 = b_2 \\
        0 & \textrm{when } b_1 \neq b_2 \\
    \end{cases}\\
    &\textrm{Recursive Step:}\\
    &\textrm{If } s \in S, b_1 \in B, b_2 \in B\\
    &basecount(~(sb_1, b_2)~) = \\
    &\begin{cases}
        1 + \textit{basecount}(~(s, b_2)~) & \textrm{when } b_1 = b_2 \\
        \textit{basecount}(~(s, b_2)~) & \textrm{when } b_1 \neq b_2 \\
    \end{cases}
    \end{align*}} & 
    {\begin{align*}
        basecount(~(\A\C\U, \C)~) = 
    \end{align*}}\\
\hline
``$2$ to the power of''& $\mathbb{N}$ & $\mathbb{N}$ & 
{\begin{align*}    
&\textrm{Basis Step:} \\
&2^0= 1 \\
&\textrm{Recursive Step:}\\
&\textrm{If } n \in \mathbb{N}, 2^{n+1} = \phantom{2 \cdot 2^n}
\end{align*}}\\
\hline
``$b$ to the power of $i$''& $\mathbb{Z}^+ \times \mathbb{N}$ & $\mathbb{N}$ & 
{\begin{align*}    
&\textrm{Basis Step:} \\
&b^0 = 1 \\
&\textrm{Recursive Step:}\\
&\textrm{If } i \in \mathbb{N}, b^{i+1} = b \cdot b^i
\end{align*}}
\end{tabular}
\end{center}

\fbox{\parbox{\textwidth}{
    $2^0 = 1$~~\hfill $2^1=2$~~\hfill $2^2=4$~~\hfill $2^3=8$~~
    \hfill $2^4=16$~~\hfill $2^5=32$~~
    \hfill $2^6=64$~~\hfill $2^7=128$~~
    \hfill $2^8=256$~~\hfill $2^9=512$~~
    \hfill $2^{10}=1024$}}
\newpage \vfill
\section*{Division algorithm}


{\bf Integer division and remainders} (aka The Division Algorithm) Let $n$ be an integer 
and $d$ a positive integer. There are unique integers $q$ and $r$, with $0 \leq r < d$, such that 
$n = dq + r$. In this case, $d$ is called the divisor, $n$ is called the dividend, 
$q$ is called the quotient, 
and $r$ is called the remainder. 

Because these numbers are guaranteed to exist, the following functions are well-defined: 
\begin{itemize}\setlength{\leftmargin}{-0.25in}
\item $\textbf{ div } : \mathbb{Z} \times \mathbb{Z}^+ \to \mathbb{Z}$ given by $\textbf{ div } ( ~(n,d)~)$ 
is the quotient when $n$ is the dividend and $d$ is the divisor.
\item $\textbf{ mod } : \mathbb{Z} \times \mathbb{Z}^+ \to \mathbb{Z}$ given by $\textbf{ mod } ( ~(n,d)~)$ 
is the remainder when $n$ is the dividend and $d$ is the divisor.
\end{itemize}
Because these functions are so important, we sometimes use the notation
$n \textbf{ div } d = \textbf{ div } ( ~(n,d)~)$ and $n \textbf{ mod } d = \textbf{ mod } (~(n,d)~)$.


{\bf Pro-tip}: The functions $\textbf{ div }$ and $\textbf{ mod }$ are similar to (but not exactly the same as) 
the operators $/$ and $\%$ in Java and python.

\vfill

{\it Example calculations}:

$20 \textbf{ div } 4$

\vspace{20pt}

$20 \textbf{ mod } 4$

\vspace{20pt}

$20 \textbf{ div } 3$

\vspace{20pt}

$20 \textbf{ mod } 3$

\vspace{20pt}

$-20 \textbf{ div } 3$

\vspace{20pt}

$-20 \textbf{ mod } 3$

\vfill \vfill
\section*{Netflix intro}


What data should we encode about each Netflix account holder to help us make effective recommendations?

\vfill
\vfill

In machine learning, clustering can be used to group similar data for prediction and recommendation.  For example,
each Netflix user's viewing history can be represented as a $n$-tuple indicating their preferences about
movies in the database, where $n$ is the number of movies in the database.  People with similar tastes in movies can then be clustered to provide recommendations
of movies for one another.  Mathematically, clustering is based on a notion of distance between pairs of $n$-tuples.
 \vfill
\section*{Data types}


\begin{center}
    \begin{tabular}{p{4.6in}p{2.6in}}
    {\bf  Term} & {\bf Examples}:\\
    &  (add additional examples from class)\\
    \hline 
    {\bf set} \newline
    unordered collection of elements & $7 \in \{43, 7, 9 \}$ \qquad $2 \notin \{43, 7, 9 \}$ \\
    {\it repetition doesn't matter} & \\
    {\it Equal sets agree on membership of all elements}& \\
    \hline
    {\bf $n$-tuple} \newline
    ordered sequence of elements with $n$ ``slots" ($n >0$) & \\
    {\it repetition matters, fixed length} &\\
    {\it Equal $n$-tuples have corresponding components equal}& \\
    \hline
    {\bf string} \newline
    ordered finite sequence of elements each from specified
    set (called the alphabet over which the string is defined)& \\
    {\it repetition matters, arbitrary finite length} &\\
    {\it Equal strings have same length and corresponding characters equal}
    \end{tabular}
\end{center}

{\it Special cases}: 

When $n=2$, the 2-tuple is called an {\bf ordered pair}.

A string of length $0$ is called the {\bf empty string} and is denoted $\lambda$.

A set with no elements is called the {\bf empty set} and is denoted $\{\}$ or $\emptyset$. \vfill
\section*{Ratings encoding}


In the table  below,  each row represents a user's ratings of movies: 
\cmark~(check) indicates the person liked the movie, \xmark~(x)
that they didn't, and $\bullet$ (dot) that they didn't rate it one way or 
another (neutral rating or didn't watch). Can encode
these ratings numerically with $1$ for \cmark~(check), $-1$ for \xmark~(x), 
and $0$ for $\bullet$ (dot).

\vfill

\begin{center}
\begin{tabular}{c|cccc||c}
Person & Dune & Oppenheimer & Barbie & Nimona & Ratings written as a $4$-tuple\\
\hline
$P_1$     & \xmark & $\bullet$ & \cmark & \phantom{$(-1, 0, 1, 1)$} \\
&&&& \\
$P_2$     & \cmark & \cmark & \xmark & \phantom{$(1, 1, -1, 1)$} \\
&&&& \\
$P_3$     & \cmark & \cmark & \cmark & \phantom{$(1, 1, 1, 1)$} \\
&&&& \\
$P_4$     & $\bullet$ & \xmark & \cmark &  \\
&&&& \\
$You$     &  &  &  &  \\
&&&& \\
\end{tabular}
\end{center} \vfill
\section*{Definitions set prereqs}


\begin{center}
\begin{tabular}{|llp{9.8cm}|}
\hline
{\bf Term} & {\bf Notation Example(s)} & {\bf We say in English \ldots } \\
\hline
all reals & $\mathbb{R}$ & The (set of all) real numbers (numbers on the number line)\\
all integers & $\mathbb{Z}$ & The (set of all) integers (whole numbers including negatives, zero, and positives) \\
all positive integers & $\mathbb{Z}^+$ & The (set of all) strictly positive integers \\
all natural numbers & $\mathbb{N}$ & The (set of all) natural numbers. {\bf Note}: we use the convention that $0$ is a natural number. \\


\hline
\end{tabular}
\end{center} \vfill
\section*{Defining sets}


{\it To define sets:}

To define a set using {\bf roster method}, explicitly list its elements. That is,
start with $\{$ then list elements of 
the set separated by commas and close with $\}$.

\vfill

To define a set using {\bf set builder definition}, either form 
``The set of all $x$ from the universe $U$ such that $x$ is ..." by writing
\[\{x \in U \mid ...x... \}\]
or form ``the collection of all outputs of some operation when the input ranges over the universe $U$"
by writing
\[\{ ...x... \mid x\in U \}\]

\vfill

We use the symbol $\in$ as ``is an element of'' to indicate membership in a set.\\

\newpage 

{\bf Example sets}: For each of the following, identify whether it's defined using the roster method
or set builder notation and give an example element.

Can we infer the data type of the example element from the notation?

\begin{itemize}
    \item[]$\{ -1, 1\}$
    \vfill
    \item[]$\{0, 0 \}$
    \vfill
    \item[]$\{-1, 0, 1 \}$
    \vfill
    \item[]$\{(x,x,x) \mid x \in \{-1,0,1\} \}$
    \vfill
    \item[]$\{ \}$
    \vfill
    \item[]$\{ x \in \mathbb{Z} \mid x \geq 0 \}$
    \vfill
    \item[]$\{ x \in \mathbb{Z}  \mid x > 0 \}$
    \vfill
    \item[]$\{ \smile, \sun \}$
    \vfill
    \item[]$\{\A,\C,\U,\G\}$
    \vfill
    \item[]$\{\A\U\G, \U\A\G, \U\G\A, \U\A\A \}$
    \vfill
\end{itemize}
 \vfill
\section*{Definitions functions prereqs}


\begin{center}
\begin{tabular}{|p{1.2in}p{2.8in}p{3in}|}
\hline
{\bf Term} & {\bf Notation Example(s)} & {\bf We say in English \ldots } \\
\hline
sequence & $x_1, \ldots, x_n$ & A sequence $x_1$ to $x_n$ \\
summation & $\sum_{i=1}^n x_i$ or $\displaystyle{\sum_{i=1}^n x_i}$ & The sum of the terms of the sequence $x_1$ to $x_n$ \\
&&\\
&&\\
piecewise rule definition & $f(x) = \begin{cases} \text{rule 1 for } x & \text{when~COND 1} \\ \text{rule 2 for } x & \text{when COND 2}\end{cases}$ &
Define $f$ of $x$ to be the result of applying rule 1 to $x$ when condition COND 1 is true and the result of 
applying rule 2 to $x$ when condition COND 2 is true. This can be generalized to having more than two conditions
(or cases).\\
&&\\
function application & $f(7)$ & $f$ of $7$ {\bf or} $f$ applied to $7$ {\bf or} the image of $7$ under $f$\\
                     & $f(z)$ & $f$ of $z$ {\bf or} $f$ applied to $z$ {\bf or} the image of $z$ under $f$\\
                     & $f(g(z))$ & $f$ of $g$ of $z$ {\bf or} $f$ applied to the result of $g$ applied to $z$ \\
&&\\
absolute value & $\lvert -3 \rvert$ & The absolute value of $-3$ \\
square root & $\sqrt{9}$ & The non-negative square root of $9$ \\


\hline
\end{tabular}
\end{center}

{\bf Pro-tip}: the meaning of two vertical lines $| ~~~ |$ depends on the data-types of what's between the lines.
For example, when placed around a number, the two vertical lines represent absolute value.
We've seen a single vertial line $|$ used as part of set builder definitions to represent ``such that''.
Again, this is 
(one of the many reasons) why is it very important to declare the data-type of variables before we use them.
 \vfill
\section*{Defining functions}


\fbox{\parbox{\textwidth}{{\bf New! Defining functions} A function is defined by its (1) domain, 
(2) codomain, and (3) rule assigning each 
element in the domain exactly one element in the codomain.\\

The domain and codomain are nonempty sets.

The rule can be depicted as a table, formula, piecewise definition, or English description.

The notation is 
\begin{center}
    ``Let the function FUNCTION-NAME: DOMAIN $\to$ CODOMAIN be given by \\
FUNCTION-NAME(x) = \ldots for every $x \in DOMAIN$''.
\end{center}

or 
\begin{center}
    ``Consider the function FUNCTION-NAME: DOMAIN $\to$ CODOMAIN defined as  \\
FUNCTION-NAME(x) = \ldots for every $x \in DOMAIN$''.
\end{center}
}}

\vfill
\newpage

Example: The absolute value function 

{\bf Domain}

{\bf Codomain}

{\bf Rule}

\vfill 
 \vfill
\section*{Defining functions ratings}


Recall our representation of Netflix users' ratings of movies as $n$-tuples, where
$n$ is the number of movies in the database. 
Each component of the $n$-tuple is $-1$ (didn't like the movie), $0$ 
(neutral rating or didn't watch the movie), or $1$ (liked the movie).

Consider the ratings $P_1 = (-1, 0, 1, 0)$, $P_2 = (1, 1, -1, 0)$, $P_3 = (1, 1, 1, 0)$,
$P_4 = (0,-1,1, 0)$


Which of $P_1$, $P_2$, $P_3$ has movie preferences most similar to $P_4$?

One approach to answer this question: use {\bf functions} to quantify difference among user preferences.

For example, consider the function 
$d_0: \{-1,0,1\}^4 \times \{-1,0,1\}^4 \to \mathbb{R}$
given by
\[
d_0 (~(~ (x_1, x_2, x_3, x_4), (y_1, y_2, y_3, y_4) ~) ~) = \sqrt{ (x_1 - y_1)^2 + (x_2 - y_2)^2 + (x_3 -y_3)^2 + (x_4 -y_4)^2}
\]


\vfill
\vfill

\begin{comment}
    

{\it Extra example:} A new movie is released, and $P_1$ and $P_2$ watch it before $P_3$, and give it
ratings; $P_1$ gives \cmark~and $P_2$ gives \xmark.
Should this movie be recommended to $P_3$? Why or why not?

{\it Extra example:} Define a new function that could be used to compare the $4$-tuples of ratings encoding
movie preferences now that there are four movies in the database.

\vfill
\end{comment}
\newpage \vfill
\section*{Defining functions recursively}


When the domain of a function is a {\it recursively defined set}, the rule assigning 
images to domain elements (outputs) can also be defined recursively.

Recall: The set of RNA strands $S$ is defined (recursively) by:
\[
\begin{array}{ll}
\textrm{Basis Step: } & \A \in S, \C \in S, \U \in S, \G \in S \\
\textrm{Recursive Step: } & \textrm{If } s \in S\textrm{ and }b \in B \textrm{, then }sb \in S
\end{array}
\]
where $sb$ is string concatenation.

{\bf Definition} (Of a function, recursively) A function \textit{rnalen} that computes the length of RNA strands in $S$ is defined by:
\[
\begin{array}{llll}
& & \textit{rnalen} : S & \to \mathbb{Z}^+ \\
\textrm{Basis Step:} & \textrm{If } b \in B\textrm{ then } & \textit{rnalen}(b) & = 1 \\
\textrm{Recursive Step:} & \textrm{If } s \in S\textrm{ and }b \in B\textrm{, then  } & \textit{rnalen}(sb) & = 1 + \textit{rnalen}(s)
\end{array}
\]

The domain of \textit{rnalen} is \phantom{$S$}\\

The codomain of \textit{rnalen} is \phantom{$\mathbb{Z}^+$}\\

Example function application:
\[
rnalen(\A\C\U) = \phantom{1+ rnalen(\A\C) = 1 + (1 + rnalen(\A) ) = 1 + ( 1 + 1) = 3}
\]

\vfill

{\it Example}: A function \textit{basecount} that computes the number of a given base 
$b$ appearing in a RNA strand $s$ is defined recursively:
    
\[
\begin{array}{llll}
& & \textit{basecount} : S \times B & \to \mathbb{N} \\
\textrm{Basis Step:} &  \textrm{If } b_1 \in B, b_2 \in B & \textit{basecount}(~(b_1, b_2)~) & =
        \begin{cases}
            1 & \textrm{when } b_1 = b_2 \\
            0 & \textrm{when } b_1 \neq b_2 \\
        \end{cases} \\
\textrm{Recursive Step:} & \textrm{If } s \in S, b_1 \in B, b_2 \in B &\textit{basecount}(~(s b_1, b_2)~) & =
        \begin{cases}
            1 + \textit{basecount}(~(s, b_2)~) & \textrm{when } b_1 = b_2 \\
            \textit{basecount}(~(s, b_2)~) & \textrm{when } b_1 \neq b_2 \\
        \end{cases}
\end{array}
\]

\begin{comment}
$basecount(~(\A\C\U,\A)~) = basecount( ~(\A\C, \A)~) = basecount(~(\A, \A)~) = 1$\\


$basecount(~(\A\C\U,\G)~) = basecount( ~(\A\C, \G)~) = basecount(~(\A, \G)~) = 0$\\


\vfill
{\it Extra example}: The function which outputs $2^n$ when given a nonnegative integer $n$ can be defined recursively, 
because its domain is the set of nonnegative integers.

\vfill
\end{comment}
 \vfill
\end{document}
\newpage
%! app: Recommendation Systems
%! outcome: data types

In the table  below,  each row represents a user's ratings of movies: 
\cmark~(check) indicates the person liked the movie, \xmark~(x)
that they didn't, and $\bullet$ (dot) that they didn't rate it one way or 
another (neutral rating or didn't watch). Can encode
these ratings numerically with $1$ for \cmark~(check), $-1$ for \xmark~(x), 
and $0$ for $\bullet$ (dot).

\vfill

\begin{center}
\begin{tabular}{c|cccc||c}
Person & Dune & Oppenheimer & Barbie & Nimona & Ratings written as a $4$-tuple\\
\hline
$P_1$     & \xmark & $\bullet$ & \cmark & \phantom{$(-1, 0, 1, 1)$} \\
&&&& \\
$P_2$     & \cmark & \cmark & \xmark & \phantom{$(1, 1, -1, 1)$} \\
&&&& \\
$P_3$     & \cmark & \cmark & \cmark & \phantom{$(1, 1, 1, 1)$} \\
&&&& \\
$P_4$     & $\bullet$ & \xmark & \cmark &  \\
&&&& \\
$You$     &  &  &  &  \\
&&&& \\
\end{tabular}
\end{center}
\vfill
{\bf Conclusion}: Modeling involves choosing data types to represent and organize data

\newpage
\section*{Week 1 Wednesday: Defining sets}
\subsection*{Notation and prerequisites}
%! app: 
%! outcome: data types, translating, important sets, write set definition, function and relation definitions

\begin{center}
\begin{tabular}{|llp{9.8cm}|}
\hline
{\bf Term} & {\bf Notation Example(s)} & {\bf We say in English \ldots } \\
\hline
%$n$-tuple & $(x_1, x_2, x_3)$ & The 3-tuple of $x_1$, $x_2$, and $x_3$ \\
%          & $(3, 4)$ & The 2-tuple or {\bf ordered pair} of $3$ and $4$ \\
%sequence & $x_1, \ldots, x_n$ & A sequence $x_1$ to $x_n$ \\
%         & $x_1, \ldots, x_n$ where $n = 0$ & An empty sequence \\
%         & $x_1, \ldots, x_n$ where $n = 1$ & A sequence containing just $x_1$ \\
%         & $x_1, \ldots, x_n$ where $n = 2$ & A sequence containing just $x_1$ and $x_2$ in order \\
%         & $x_1, x_2$ & A sequence containing just $x_1$ and $x_2$ in order \\
%summation & $\sum_{i=1}^n x_i$ or $\displaystyle{\sum_{i=1}^n x_i}$ & The sum of the terms of the sequence $x_1$ to $x_n$ \\
%&&\\
%maximum & $\displaystyle \max(x, y)$ & The max of $x$ and $y$, when they are numbers \\ % Note that this is different than summation!
%        & $\displaystyle \max_{1 \leq i \leq n} x_i$ & The max of $x_1$ to $x_n$, when they are numbers \\ % Also different from display
%&&\\
%set & & Unordered collection of objects. The set of \ldots \\
all reals & $\mathbb{R}$ & The (set of all) real numbers (numbers on the number line)\\
all integers & $\mathbb{Z}$ & The (set of all) integers (whole numbers including negatives, zero, and positives) \\
all positive integers & $\mathbb{Z}^+$ & The (set of all) strictly positive integers \\
all natural numbers & $\mathbb{N}$ & The (set of all) natural numbers. {\bf Note}: we use the convention that $0$ is a natural number. \\
%roster method & $\{43, 7, 9\}$ & The set whose elements are $43$, $7$, and $9$\\
%              & $\{9, \mathbb{N}\}$ & The set whose elements are $9$ and $\mathbb{N}$\\
%&&\\
%set builder notation & $\{ x \in \mathbb{Z} \mid x > 0\}$ & The set of all $x$ from the integers such that $x$ is greater than $0$ \\
%                     & $\{ 3x  \mid x \in \mathbb{Z} \}$ & The set of all integer multiples of $3$. {\bf Note}: we use the convention that writing two numbers next to each other means multiplication. \\
%&&\\
%function rule definition & $f(x) = x + 4$ & Define $f$ of $x$ to be $x + 4$ \\
%piecewise rule definition & $f(x) = \begin{cases} x & \text{if~}x \geq 0 \\ -x & \text{if~}x<0\end{cases}$ &
%Define $f$ of $x$ to be $x$ when $x$ is nonnegative and to be $-x$ when $x$ is negative\\
%function application & $f(7)$ & $f$ of $7$ {\bf or} $f$ applied to $7$ {\bf or} the image of $7$ under $f$\\
%                     & $f(z)$ & $f$ of $z$ {\bf or} $f$ applied to $z$ {\bf or} the image of $z$ under $f$\\
%                     & $f(g(z))$ & $f$ of $g$ of $z$ {\bf or} $f$ applied to the result of $g$ applied to $z$ \\
%&&\\
%absolute value & $\lvert -3 \rvert$ & The absolute value of $-3$ \\
%square root & $\sqrt{9}$ & The non-negative square root of $9$ \\
%&&\\
%summation notation & $\displaystyle \sum_{i=1}^n i$ & The sum of the integers from $1$ to $n$, inclusive \\
%                    & $\displaystyle \sum_{i=1}^n i^2 - 1$ & The sum of $i^2 - 1$ ($i$ squared minus $1$) for each $i$ from $1$ to $n$, inclusive \\
%&&\\
%quotient, integer division & $n~\textbf{div}~m$ & The (integer) quotient upon dividing $n$ by $m$; informally: divide and then 
%drop the fractional part\\
%modulo, remainder & $n~\textbf{mod}~m$ & The remainder upon dividing $n$ by $m$ \\

\hline
\end{tabular}
\end{center}
%! app: Numbers, Recommendation Systems, Bioinformatics
%! outcome: data types, write set definition, important sets

{\it To define sets:}

To define a set using {\bf roster method}, explicitly list its elements. That is,
start with $\{$ then list elements of 
the set separated by commas and close with $\}$.

\vfill

To define a set using {\bf set builder definition}, either form 
``The set of all $x$ from the universe $U$ such that $x$ is ..." by writing
\[\{x \in U \mid ...x... \}\]
or form ``the collection of all outputs of some operation when the input ranges over the universe $U$"
by writing
\[\{ ...x... \mid x\in U \}\]

\vfill

We use the symbol $\in$ as ``is an element of'' to indicate membership in a set.\\

\newpage 

{\bf Example sets}: For each of the following, identify whether it's defined using the roster method
or set builder notation and give an example element.

Can we infer the data type of the example element from the notation?

\begin{itemize}
    \item[]$\{ -1, 1\}$
    \vfill
    \item[]$\{0, 0 \}$
    \vfill
    \item[]$\{-1, 0, 1 \}$
    \vfill
    \item[]$\{(x,x,x) \mid x \in \{-1,0,1\} \}$
    \vfill
    \item[]$\{ \}$
    \vfill
    \item[]$\{ x \in \mathbb{Z} \mid x \geq 0 \}$
    \vfill
    \item[]$\{ x \in \mathbb{Z}  \mid x > 0 \}$
    \vfill
    \item[]$\{ \smile, \sun \}$
    \vfill
    \item[]$\{\A,\C,\U,\G\}$
    \vfill
    \item[]$\{\A\U\G, \U\A\G, \U\G\A, \U\A\A \}$
    \vfill
\end{itemize}

\newpage
%! app: Bioinformatics, Numbers
%! outcome: recursive definitions

RNA is made up of strands of four different bases that encode genomic information
in specific ways.\\
The bases are elements of the set 
$B  = \{\A, \C, \U, \G \}$.
Strands are ordered nonempty finite sequences of bases.

Formally, to define the set of all RNA strands, we need more than roster
method or set builder descriptions. 


\input{../activity-snippets/recursive-sets-definition.tex}
%! app: Bioinformatics, Numbers
%! outcome: recursive definitions

{\bf Definition} The set of nonnegative integers $\mathbb{N}$ is defined (recursively) by: 
\[
\begin{array}{ll}
\textrm{Basis Step: } & \phantom{0 \in \mathbb{N}} \\
\textrm{Recursive Step: } & \phantom{\textrm{If } n \in \mathbb{N} \textrm{, then } n+1 \in \mathbb{N}}
\end{array}
\]

Examples: 

{\bf Definition} The set of all integers $\mathbb{Z}$ is defined (recursively) by: 
\[
\begin{array}{ll}
\textrm{Basis Step: } & \phantom{0 \in \mathbb{Z}} \\
\textrm{Recursive Step: } & \phantom{\textrm{If } x \in \mathbb{Z} \textrm{, then } x+1 \in \mathbb{Z}
\textrm{ and } x-1 \in \mathbb{Z}}
\end{array}
\]

Examples: 

\vfill

{\bf Definition} The set of RNA strands $S$ is defined (recursively) by:
\[
\begin{array}{ll}
\textrm{Basis Step: } & \A \in S, \C \in S, \U \in S, \G \in S \\
\textrm{Recursive Step: } & \textrm{If } s \in S\textrm{ and }b \in B \textrm{, then }sb \in S
\end{array}
\]
where $sb$ is string concatenation.

Examples: 

\vfill

{\bf Definition} The set of bitstrings (strings of 0s and 1s) is defined (recursively) by:
\[
\begin{array}{ll}
\textrm{Basis Step: } & \phantom{\lambda \in X} \\
\textrm{Recursive Step: } & \phantom{\textrm{If } s \in X \textrm{, then } s0 \in X \text{ and } s1 \in X}
\end{array}
\]

{\it Notation:} We call the set of bitstrings $\{0,1\}^*$ and we say 
this is the set of all strings over $\{0,1\}$.

Examples: 

\vfill
%! app: TODOapp
%! outcome: write set definition, important sets

\fbox{\parbox{\textwidth}{%
To define a set we can use the roster method, set builder notation, a recursive definition, 
and also we can apply a set operation to other sets. \\

{\bf New! Cartesian product of sets} and {\bf set-wise concatenation of sets of strings}\\


{\bf Definition}: Let $X$ and $Y$ be sets.  The {\bf Cartesian product} of $X$ and $Y$, denoted
$X \times Y$, is the set of all ordered pairs $(x,y)$ where $x \in X$ and $y \in Y$
\[
X \times Y = \{ (x,y) \mid x \in X \text{ and } y \in Y \}
\]

Conventions: (1) Cartesian products can be chained together to result in sets of $n$-tuples and 
(2) When we form the Cartesian product of a set with itself $X \times X$ we can denote that set as 
$X^2$, or $X^n$ for the Cartesian product of a set with itself $n$ times for a positive integer $n$.\\

{\bf Definition}: Let $X$ and $Y$ be sets of strings over the same alphabet. The {\bf set-wise concatenation} 
of $X$ and $Y$, denoted $X \circ Y$, is the set of all results of string concatenation $xy$ where $x \in X$ 
and $y \in Y$
\[
X \circ Y = \{ xy \mid x \in X \text{ and } y \in Y \}
\]
}}

{\bf Pro-tip}: the meaning of writing one element next to another like $xy$ depends on the data-types of $x$ and 
$y$. When $x$ and $y$ are strings, the convention is that $xy$ is the result of string concatenation. 
When $x$ and $y$ are numbers, the convention is that $xy$ is the result of multiplication. This is 
(one of the many reasons) why is it very important to declare the data-type of variables before we use them.

{\it Fill in the missing entries in the table}:

\begin{center}
\begin{tabular}{cc}
{\bf  Set} & {\bf Example elements in this set and their data type}:\\
\hline 
& \\
$B$ &\A \qquad \C \qquad \G \qquad \U \\
& \\
\hline
& \\
\phantom{$B \times B$} & $(\A, \C)$ \qquad $(\U, \U)$\\
& \\
\hline
& \\
$B \times \{-1,0,1\}$ & \\
& \\
\hline
& \\
$\{-1,0,1\} \times B$ & \\
& \\
\hline
& \\
\phantom{$\{-1,0,1\} \times \{-1,0,1\}  \times \{-1,0,1\} $} & \qquad $(0,0,0)$ \\
& \\
\hline
& \\
$ \{\A, \C, \G, \U \} \circ  \{\A, \C, \G, \U \}$& \\
& \\
\hline
& \\
\phantom{$\{G\} \circ \{G\} \circ \{G\}$} & \qquad $\G\G\G\G$ \\
& \\
\hline

\end{tabular}
\end{center}

\vfill

\section*{Week 1 Friday: Defining functions}
%! app: 
%! outcome: data types, translating, important sets, write set definition, function and relation definitions

\begin{center}
\begin{tabular}{|p{1.2in}p{2.8in}p{3in}|}
\hline
{\bf Term} & {\bf Notation Example(s)} & {\bf We say in English \ldots } \\
\hline
%$n$-tuple & $(x_1, x_2, x_3)$ & The 3-tuple of $x_1$, $x_2$, and $x_3$ \\
%          & $(3, 4)$ & The 2-tuple or {\bf ordered pair} of $3$ and $4$ \\
sequence & $x_1, \ldots, x_n$ & A sequence $x_1$ to $x_n$ \\
%         & $x_1, \ldots, x_n$ where $n = 0$ & An empty sequence \\
%         & $x_1, \ldots, x_n$ where $n = 1$ & A sequence containing just $x_1$ \\
%         & $x_1, \ldots, x_n$ where $n = 2$ & A sequence containing just $x_1$ and $x_2$ in order \\
%         & $x_1, x_2$ & A sequence containing just $x_1$ and $x_2$ in order \\
summation & $\sum_{i=1}^n x_i$ or $\displaystyle{\sum_{i=1}^n x_i}$ & The sum of the terms of the sequence $x_1$ to $x_n$ \\
&&\\
%maximum & $\displaystyle \max(x, y)$ & The max of $x$ and $y$, when they are numbers \\ % Note that this is different than summation!
%        & $\displaystyle \max_{1 \leq i \leq n} x_i$ & The max of $x_1$ to $x_n$, when they are numbers \\ % Also different from display
%&&\\
%set & & Unordered collection of objects. The set of \ldots \\
%all reals & $\mathbb{R}$ & The (set of all) real numbers (numbers on the number line)\\
%all integers & $\mathbb{Z}$ & The (set of all) integers (whole numbers including negatives, zero, and positives) \\
%all positive integers & $\mathbb{Z}^+$ & The (set of all) strictly positive integers \\
%all natural numbers & $\mathbb{N}$ & The (set of all) natural numbers. {\bf Note}: we use the convention that $0$ is a natural number. \\
%roster method & $\{43, 7, 9\}$ & The set whose elements are $43$, $7$, and $9$\\
%              & $\{9, \mathbb{N}\}$ & The set whose elements are $9$ and $\mathbb{N}$\\
%&&\\
%set builder notation & $\{ x \in \mathbb{Z} \mid x > 0\}$ & The set of all $x$ from the integers such that $x$ is greater than $0$ \\
%                     & $\{ 3x  \mid x \in \mathbb{Z} \}$ & The set of all integer multiples of $3$. {\bf Note}: we use the convention that writing two numbers next to each other means multiplication. \\
&&\\
%function rule definition & $f(x) = x + 4$ & Define $f$ of $x$ to be $x + 4$ \\
piecewise rule definition & $f(x) = \begin{cases} \text{rule 1 for } x & \text{when~COND 1} \\ \text{rule 2 for } x & \text{when COND 2}\end{cases}$ &
Define $f$ of $x$ to be the result of applying rule 1 to $x$ when condition COND 1 is true and the result of 
applying rule 2 to $x$ when condition COND 2 is true. This can be generalized to having more than two conditions
(or cases).\\
&&\\
function application & $f(7)$ & $f$ of $7$ {\bf or} $f$ applied to $7$ {\bf or} the image of $7$ under $f$\\
                     & $f(z)$ & $f$ of $z$ {\bf or} $f$ applied to $z$ {\bf or} the image of $z$ under $f$\\
                     & $f(g(z))$ & $f$ of $g$ of $z$ {\bf or} $f$ applied to the result of $g$ applied to $z$ \\
&&\\
absolute value & $\lvert -3 \rvert$ & The absolute value of $-3$ \\
square root & $\sqrt{9}$ & The non-negative square root of $9$ \\
%&&\\
%summation notation & $\displaystyle \sum_{i=1}^n i$ & The sum of the integers from $1$ to $n$, inclusive \\
%                    & $\displaystyle \sum_{i=1}^n i^2 - 1$ & The sum of $i^2 - 1$ ($i$ squared minus $1$) for each $i$ from $1$ to $n$, inclusive \\
%&&\\
%quotient, integer division & $n~\textbf{div}~m$ & The (integer) quotient upon dividing $n$ by $m$; informally: divide and then 
%drop the fractional part\\
%modulo, remainder & $n~\textbf{mod}~m$ & The remainder upon dividing $n$ by $m$ \\

\hline
\end{tabular}
\end{center}

{\bf Pro-tip}: the meaning of two vertical lines $| ~~~ |$ depends on the data-types of what's between the lines.
For example, when placed around a number, the two vertical lines represent absolute value.
We've seen a single vertial line $|$ used as part of set builder definitions to represent ``such that''.
Again, this is 
(one of the many reasons) why is it very important to declare the data-type of variables before we use them.

%! app: TODOapp
%! outcome: function and relation definitions, data types

\fbox{\parbox{\textwidth}{%
{\bf New! Defining functions} A function is defined by its (1) domain, 
(2) codomain, and (3) rule assigning each 
element in the domain exactly one element in the codomain.\\

The domain and codomain are nonempty sets.

The rule can be depicted as a table, formula, piecewise definition, or English description.

The notation is 
\begin{center}
    ``Let the function FUNCTION-NAME: DOMAIN $\to$ CODOMAIN be given by \\
FUNCTION-NAME(x) = \ldots for every $x \in DOMAIN$''.
\end{center}

or 
\begin{center}
    ``Consider the function FUNCTION-NAME: DOMAIN $\to$ CODOMAIN defined as  \\
FUNCTION-NAME(x) = \ldots for every $x \in DOMAIN$''.
\end{center}
}}

\vfill
\newpage

Example: The absolute value function 

{\bf Domain}

{\bf Codomain}

{\bf Rule}

\vfill 

%! app: Recommendation Systems
%! outcome: function and relation definitions, data types

Recall our representation of Netflix users' ratings of movies as $n$-tuples, where
$n$ is the number of movies in the database. 
Each component of the $n$-tuple is $-1$ (didn't like the movie), $0$ 
(neutral rating or didn't watch the movie), or $1$ (liked the movie).

Consider the ratings $P_1 = (-1, 0, 1, 0)$, $P_2 = (1, 1, -1, 0)$, $P_3 = (1, 1, 1, 0)$,
$P_4 = (0,-1,1, 0)$


Which of $P_1$, $P_2$, $P_3$ has movie preferences most similar to $P_4$?

One approach to answer this question: use {\bf functions} to quantify difference among user preferences.

For example, consider the function 
$d_0: \{-1,0,1\}^4 \times \{-1,0,1\}^4 \to \mathbb{R}$
given by
\[
d_0 (~(~ (x_1, x_2, x_3, x_4), (y_1, y_2, y_3, y_4) ~) ~) = \sqrt{ (x_1 - y_1)^2 + (x_2 - y_2)^2 + (x_3 -y_3)^2 + (x_4 -y_4)^2}
\]


\vfill
\vfill

\begin{comment}
    

{\it Extra example:} A new movie is released, and $P_1$ and $P_2$ watch it before $P_3$, and give it
ratings; $P_1$ gives \cmark~and $P_2$ gives \xmark.
Should this movie be recommended to $P_3$? Why or why not?

{\it Extra example:} Define a new function that could be used to compare the $4$-tuples of ratings encoding
movie preferences now that there are four movies in the database.

\vfill
\end{comment}
\newpage
%! app: TODOapp
%! outcome: function and relation definitions, data types

When the domain of a function is a {\it recursively defined set}, the rule assigning 
images to domain elements (outputs) can also be defined recursively.

Recall: The set of RNA strands $S$ is defined (recursively) by:
\[
\begin{array}{ll}
\textrm{Basis Step: } & \A \in S, \C \in S, \U \in S, \G \in S \\
\textrm{Recursive Step: } & \textrm{If } s \in S\textrm{ and }b \in B \textrm{, then }sb \in S
\end{array}
\]
where $sb$ is string concatenation.

{\bf Definition} (Of a function, recursively) A function \textit{rnalen} that computes the length of RNA strands in $S$ is defined by:
\[
\begin{array}{llll}
& & \textit{rnalen} : S & \to \mathbb{Z}^+ \\
\textrm{Basis Step:} & \textrm{If } b \in B\textrm{ then } & \textit{rnalen}(b) & = 1 \\
\textrm{Recursive Step:} & \textrm{If } s \in S\textrm{ and }b \in B\textrm{, then  } & \textit{rnalen}(sb) & = 1 + \textit{rnalen}(s)
\end{array}
\]

The domain of \textit{rnalen} is \phantom{$S$}\\

The codomain of \textit{rnalen} is \phantom{$\mathbb{Z}^+$}\\

Example function application:
\[
rnalen(\A\C\U) = \phantom{1+ rnalen(\A\C) = 1 + (1 + rnalen(\A) ) = 1 + ( 1 + 1) = 3}
\]

\vfill

{\it Example}: A function \textit{basecount} that computes the number of a given base 
$b$ appearing in a RNA strand $s$ is defined recursively:
    
\[
\begin{array}{llll}
& & \textit{basecount} : S \times B & \to \mathbb{N} \\
\textrm{Basis Step:} &  \textrm{If } b_1 \in B, b_2 \in B & \textit{basecount}(~(b_1, b_2)~) & =
        \begin{cases}
            1 & \textrm{when } b_1 = b_2 \\
            0 & \textrm{when } b_1 \neq b_2 \\
        \end{cases} \\
\textrm{Recursive Step:} & \textrm{If } s \in S, b_1 \in B, b_2 \in B &\textit{basecount}(~(s b_1, b_2)~) & =
        \begin{cases}
            1 + \textit{basecount}(~(s, b_2)~) & \textrm{when } b_1 = b_2 \\
            \textit{basecount}(~(s, b_2)~) & \textrm{when } b_1 \neq b_2 \\
        \end{cases}
\end{array}
\]

\begin{comment}
$basecount(~(\A\C\U,\A)~) = basecount( ~(\A\C, \A)~) = basecount(~(\A, \A)~) = 1$\\


$basecount(~(\A\C\U,\G)~) = basecount( ~(\A\C, \G)~) = basecount(~(\A, \G)~) = 0$\\


\vfill
{\it Extra example}: The function which outputs $2^n$ when given a nonnegative integer $n$ can be defined recursively, 
because its domain is the set of nonnegative integers.

\vfill
\end{comment}


\newpage
\subsection*{Review Quiz}
\begin{enumerate}
\item Modeling
\begin{enumerate}
    \item {%! app: Recommendation Systems
%! outcome: data types

Using the example movie database from class with the four movies Dune, Oppenheimer, Barbie, and Nimona which of the following is a $4$-tuple that represents the ratings of a user who liked Dune? (Select all and only that apply.)

\begin{enumerate}
\item $1$
\item $(1,0,0)$
\item $[1,1,1,0]$
\item $\{-1, 0, 0, 1\}$
\item $(1,-1,0,1)$
\item $(0,1,1, 1)$
\item $(1,1,1,1)$
\end{enumerate}}
    \item {%! app: Recommendation Systems
%! outcome: data types

Using the example movie database from class with the four movies Dune, Oppenheimer, Barbie, and Nimona how many distinct (different) $4$-tuples of ratings are there? 
}
    \item {%! app: Computers
%! outcome: data types, write set definition, important sets

Colors can be described as amounts of red, green, and blue mixed together
\footnote{This RGB representation is common in web applications.  Many online tools are available to play around with mixing these colors,  e.g. \url{https://www.w3schools.com/colors/colors_rgb.asp}. }
Mathematically, a color can be represented as a $3$-tuple $(r, g, b)$ where $r$ represents the red component, $g$ the green component, $b$ the blue component and where each of $r$, $g$, $b$ must be a value from this collection of numbers:
\begin{quote}
$\{$0, 1, 2, 3, 4, 5, 6, 7, 8, 9, 10, 11, 12, 13, 14, 15, 16, 17, 18, 19, 20, 21, 22, 23, 24, 25, 26, 27, 28, 29, 30, 31, 32, 33, 34, 35, 36, 37, 38, 39, 40, 41, 42, 43, 44, 45, 46, 47, 48, 49, 50, 51, 52, 53, 54, 55, 56, 57, 58, 59, 60, 61, 62, 63, 64, 65, 66, 67, 68, 69, 70, 71, 72, 73, 74, 75, 76, 77, 78, 79, 80, 81, 82, 83, 84, 85, 86, 87, 88, 89, 90, 91, 92, 93, 94, 95, 96, 97, 98, 99, 100, 101, 102, 103, 104, 105, 106, 107, 108, 109, 110, 111, 112, 113, 114, 115, 116, 117, 118, 119, 120, 121, 122, 123, 124, 125, 126, 127, 128, 129, 130, 131, 132, 133, 134, 135, 136, 137, 138, 139, 140, 141, 142, 143, 144, 145, 146, 147, 148, 149, 150, 151, 152, 153, 154, 155, 156, 157, 158, 159, 160, 161, 162, 163, 164, 165, 166, 167, 168, 169, 170, 171, 172, 173, 174, 175, 176, 177, 178, 179, 180, 181, 182, 183, 184, 185, 186, 187, 188, 189, 190, 191, 192, 193, 194, 195, 196, 197, 198, 199, 200, 201, 202, 203, 204, 205, 206, 207, 208, 209, 210, 211, 212, 213, 214, 215, 216, 217, 218, 219, 220, 221, 222, 223, 224, 225, 226, 227, 228, 229, 230, 231, 232, 233, 234, 235, 236, 237, 238, 239, 240, 241, 242, 243, 244, 245, 246, 247, 248, 249, 250, 251, 252, 253, 254, 255$\}$
\end{quote}

Select all and only the true statements below.

\begin{enumerate}
\item $(1, 3, 4)$ fits the definition of a color above.
\item $(1, 100, 200, 0)$ fits the definition of a color above.
\item $(510, 255)$ fits the definition of a color above.
\item There is a color $(r_1, g_1, b_1)$ where $r_1 + g_1 + b_1$ is greater than $765$.
\item There is a color $(r_2, g_2, b_2)$ where $r_2 + g_2 + b_2$ is equal to $1$.
\item Another way to write the collection of allowed values for red, green, and blue components is $$\{x \in \mathbb{N}\mid 0 \leq x \leq 255 \}$$.
\item Another way to write the collection of allowed values for red, green, and blue components is $$\{n \in \mathbb{Z}\mid 0 \leq n \leq 255 \}$$.
\item Another way to write the collection of allowed values for red, green, and blue components is $$\{y \in \mathbb{Z}\mid -1 < y \leq 255 \}$$.
\end{enumerate}
\vfill}
    \item {%! app: Computers
%! outcome: data types, write set definition, important sets


In the definition of colors as amounts of red, green, and blue mixed together, why are $3$-tuples a convenient data structure to use rather than sets or strings?

(Select all and only relevant choices)

\begin{enumerate}
\item Ordering matters in $n$-tuples, so we can use the different components of the $3$-tuple to represent the amounts of specific colors. 
\item There are many possible values for each color amount and we don't have individual characters for each value so a string could get unwieldy.
\item It's possible to have the same value of two or all of the colors, and repetition matters in $n$-tuples.
\end{enumerate}
\vfill}
\end{enumerate}
\item Sets and functions
\begin{enumerate}
    \item {\input{../activity-snippets/quiz-set-membership.tex}}
    \item {%! app: TODOapp
%! outcome: TODOoutcome

RNA is made up of strands of four different bases that encode genomic information
in specific ways. The bases are elements of the set 
$B  = \{\A, \C, \G, \U \}$. The set of RNA strands $S$ is defined (recursively) by:

\[
\begin{array}{ll}
\textrm{Basis Step: } & \A \in S, \C \in S, \U \in S, \G \in S \\
\textrm{Recursive Step: } & \textrm{If } s \in S\textrm{ and }b \in B \textrm{, then }sb \in S
\end{array}
\]

A function \textit{rnalen} that computes the length of RNA strands in $S$ is defined by:
\[
\begin{array}{llll}
& & \textit{rnalen} : S & \to \mathbb{Z}^+ \\
\textrm{Basis Step:} & \textrm{If } b \in B\textrm{ then } & \textit{rnalen}(b) & = 1 \\
\textrm{Recursive Step:} & \textrm{If } s \in S\textrm{ and }b \in B\textrm{, then  } & \textit{rnalen}(sb) & = 1 + \textit{rnalen}(s)
\end{array}
\]

\begin{enumerate}
\item How many distinct elements are in the set described using set builder notation as 
\[
\{ x \in S \mid rnalen(x) = 1\} \qquad ?
\]

\item How many distinct elements are in the set described using set builder notation as 
\[
\{ x \in S \mid rnalen(x) = 2\} \qquad ?
\]

\item How many distinct elements are in the set described using set builder notation as 
\[
\{ rnalen(x) \mid x \in S \text{ and } rnalen(x) = 2\} \qquad ?
\]


\item How many distinct elements are in the set obtained as the result
of the set-wise concatenation $\{ \A\A, \A\C \} \circ \{\U, \A\A \}$?

\item How many distinct elements are in the set obtained as the result
of the Cartesian product $\{ \A\A, \A\C \} \times \{\U, \A\A \}$?

\item {\bf True} or {\bf False}: There is an example of an RNA strand that is both in the set obtained as the result
of the set-wise concatenation $\{ \A\A, \A\C \} \circ \{\U, \A\A \}$ and in the set obtained as the result of the 
Cartesian product $\{ \A\A, \A\C \} \times \{\U\A, \A\A \}$

\end{enumerate}
{\it Bonus - not for credit: Describe each of the sets above using roster method.}
}
    \item {%! app: Recommendation Systems
%! outcome: function and relation definitions, data types

Recall the function which takes an ordered pair of ratings $4$-tuples and returns a measure of the difference between them
$d_0: \{-1,0,1\}^4 \times \{-1,0,1\}^4 \to \mathbb{R}$
given by
\[
d_0 (~(~ (x_1, x_2, x_3, x_4), (y_1, y_2, y_3, y_4) ~) ~) = \sqrt{ (x_1 - y_1)^2 + (x_2 - y_2)^2 + (x_3 -y_3)^2 + (x_4 -y_4)^2}
\]

Consider the function application 
\[
  d_0 (~( ~(-1,1,1, 0), (1, 0, -1, 0)~) ~)
\]
\begin{enumerate}
    \item What is the input? 
    \item What is the output?
\end{enumerate}}
%    \item {\input{../activity-snippets/quiz-recursive-definitions.tex}}
%    \item {\input{../activity-snippets/quiz-defining-functions-recursively.tex}}
\end{enumerate}

\end{enumerate}

\newpage

\subsection*{Week 2 at a glance}

\subsubsection*{We will be learning and practicing to:}
%data types, 
%div and mod
%trace algorithms
%translating
%write set definition
%function and relation definitions
%representing numbers
%applications of number representations
\begin{itemize}
\item Model systems with tools from discrete mathematics and reason about implications of modelling choices. Explore applications in CS through multiple perspectives, including software, hardware, and theory.
\begin{itemize}
   \item Selecting and representing appropriate data types and using notation conventions to clearly communicate choices
   \item Determining the properties of positional number representations, including overflow and bit operations
\end{itemize}

\item Translate between different representations to illustrate a concept.

\begin{itemize}
   \item Translating between symbolic and English versions of statements using precise mathematical language
   \item Tracing algorithms specified in pseudocode
   \item Representing numbers using positional representations, including decimal, binary, hexadecimal, fixed-width representations, and 2s complement
\end{itemize}


\item Use precise notation to encode meaning and present arguments concisely and clearly
\begin{itemize}
    \item Precisely describing a set using appropriate notation e.g. roster method, set builder notation, and recursive definitions
    \item Defining functions using multiple representations
\end{itemize}

\item Know, select and apply appropriate computing knowledge and problem-solving techniques. Reason about computation and systems. Use mathematical techniques to solve problems. Determine appropriate conceptual tools to apply to new situations. Know when tools do not apply and try different approaches. Critically analyze and evaluate candidate solutions.
\begin{itemize}
    \item Using a recursive definition to evaluate a function or determine membership in a set
    \item Using the definitions of the div and mod operators on integers
\end{itemize}

\end{itemize}

\subsubsection*{TODO:}
\begin{list}
   {\itemsep2pt}
   \item \#FinAid Assignment on Canvas (complete as soon as possible) 
   \item Review quiz based on class material each day (due Friday April 12, 2024)
   \item Homework assignment 2 (due Tuesday April 16, 2024).
\end{list}

\newpage

\section*{Week 2 Monday: Sets, functions, and algorithms}
%! app: Bioinformatics, Numbers, Recommendation Systems
%! outcome: function and relation definitions, data types

Let's practice with functions related to some of our applications so far.

Recall: We model user ratings of the collection of the four moviews Dune, Oppenheimer, Barbie, Nimona as the set
$\{-1,0,1\}^4 \times \{-1,0,1\}^4$ . One function that compares pairs of ratings is
$$d_0: \{-1,0,1\}^4 \times \{-1,0,1\}^4 \to \mathbb{R}$$
given by
\[
d_0 (~(~ (x_1, x_2, x_3, x_4), (y_1, y_2, y_3, y_4) ~) ~) = \sqrt{ (x_1 - y_1)^2 + (x_2 - y_2)^2 + (x_3 -y_3)^2 + (x_4 -y_4)^2}
\]

Notice: any ordered pair of ratings is an okay input to $d_0$.

Notice: there are (at most) 
\[
(3 \cdot 3 \cdot 3 \cdot 3)\cdot (3 \cdot 3 \cdot 3 \cdot 3) = 3^8 = 6561
\]
many pairs of ratings. There are therefore lots and lots of real numbers that are not the output of $d_0$.

\vfill

Recall: RNA is made up of strands of four different bases that encode genomic information
in specific ways.\\
The bases are elements of the set 
$B  = \{\A, \C, \U, \G \}$. The set of RNA strands $S$ is defined (recursively) by:
\[
\begin{array}{ll}
\textrm{Basis Step: } & \A \in S, \C \in S, \U \in S, \G \in S \\
\textrm{Recursive Step: } & \textrm{If } s \in S\textrm{ and }b \in B \textrm{, then }sb \in S
\end{array}
\]
where $sb$ is string concatenation.

\vfill
\newpage
{\bf Pro-tip}: informal definitions sometime use $\cdots$ to indicate ``continue the pattern''. Often, 
to make this pattern precise we use recursive definitions.

\vspace{-20pt}

\begin{center}
\begin{tabular}{p{0.65in}ccp{2.4in}p{2.4in}}
{\scriptsize {\bf Name}} & {\scriptsize {\bf  Domain}} & {\scriptsize {\bf Codomain}} & {\scriptsize {\bf Rule}} &{\scriptsize {\bf Example}}\\
\hline 
$rnalen$ & $S$ & $\mathbb{Z}^+$ & 
    {\begin{align*}    
    &\textrm{Basis Step:} \\
    &\textrm{If } b \in B\textrm{ then } \textit{rnalen}(b) = 1 \\
    &\textrm{Recursive Step:}\\
    &\textrm{If } s \in S\textrm{ and } b \in B\textrm{, then  }\\
    &\textit{rnalen}(sb) = 1 + \textit{rnalen}(s)
    \end{align*}} & 
    {\begin{align*}
        rnalen(\A\C) &\overset{\text{rec step}}{=} 1 +rnalen(\A) \\ 
        &\overset{\text{basis step}}{=} 1 + 1 = 2
    \end{align*}}\\
\hline
$basecount$ & $S \times B$ & $\mathbb{N}$ & 
{\begin{align*}    
    &\textrm{Basis Step:} \\
    &\textrm{If } b_1 \in B, b_2 \in B \textrm{ then} \\
    &basecount(~(b_1, b_2)~) = \\
    &\begin{cases}
        1 & \textrm{when } b_1 = b_2 \\
        0 & \textrm{when } b_1 \neq b_2 \\
    \end{cases}\\
    &\textrm{Recursive Step:}\\
    &\textrm{If } s \in S, b_1 \in B, b_2 \in B\\
    &basecount(~(sb_1, b_2)~) = \\
    &\begin{cases}
        1 + \textit{basecount}(~(s, b_2)~) & \textrm{when } b_1 = b_2 \\
        \textit{basecount}(~(s, b_2)~) & \textrm{when } b_1 \neq b_2 \\
    \end{cases}
    \end{align*}} & 
    {\begin{align*}
        basecount(~(\A\C\U, \C)~) = 
    \end{align*}}\\
\hline
``$2$ to the power of''& $\mathbb{N}$ & $\mathbb{N}$ & 
{\begin{align*}    
&\textrm{Basis Step:} \\
&2^0= 1 \\
&\textrm{Recursive Step:}\\
&\textrm{If } n \in \mathbb{N}, 2^{n+1} = \phantom{2 \cdot 2^n}
\end{align*}}\\
\hline
``$b$ to the power of $i$''& $\mathbb{Z}^+ \times \mathbb{N}$ & $\mathbb{N}$ & 
{\begin{align*}    
&\textrm{Basis Step:} \\
&b^0 = 1 \\
&\textrm{Recursive Step:}\\
&\textrm{If } i \in \mathbb{N}, b^{i+1} = b \cdot b^i
\end{align*}}
\end{tabular}
\end{center}

\fbox{\parbox{\textwidth}{
    $2^0 = 1$~~\hfill $2^1=2$~~\hfill $2^2=4$~~\hfill $2^3=8$~~
    \hfill $2^4=16$~~\hfill $2^5=32$~~
    \hfill $2^6=64$~~\hfill $2^7=128$~~
    \hfill $2^8=256$~~\hfill $2^9=512$~~
    \hfill $2^{10}=1024$}}
\newpage
\newpage
%! app: Numbers
%! outcome: div and mod, data types

{\bf Integer division and remainders} (aka The Division Algorithm) Let $n$ be an integer 
and $d$ a positive integer. There are unique integers $q$ and $r$, with $0 \leq r < d$, such that 
$n = dq + r$. In this case, $d$ is called the divisor, $n$ is called the dividend, 
$q$ is called the quotient, 
and $r$ is called the remainder. 

Because these numbers are guaranteed to exist, the following functions are well-defined: 
\begin{itemize}\setlength{\leftmargin}{-0.25in}
\item $\textbf{ div } : \mathbb{Z} \times \mathbb{Z}^+ \to \mathbb{Z}$ given by $\textbf{ div } ( ~(n,d)~)$ 
is the quotient when $n$ is the dividend and $d$ is the divisor.
\item $\textbf{ mod } : \mathbb{Z} \times \mathbb{Z}^+ \to \mathbb{Z}$ given by $\textbf{ mod } ( ~(n,d)~)$ 
is the remainder when $n$ is the dividend and $d$ is the divisor.
\end{itemize}
Because these functions are so important, we sometimes use the notation
$n \textbf{ div } d = \textbf{ div } ( ~(n,d)~)$ and $n \textbf{ mod } d = \textbf{ mod } (~(n,d)~)$.


{\bf Pro-tip}: The functions $\textbf{ div }$ and $\textbf{ mod }$ are similar to (but not exactly the same as) 
the operators $/$ and $\%$ in Java and python.

\vfill

{\it Example calculations}:

$20 \textbf{ div } 4$

\vspace{20pt}

$20 \textbf{ mod } 4$

\vspace{20pt}

$20 \textbf{ div } 3$

\vspace{20pt}

$20 \textbf{ mod } 3$

\vspace{20pt}

$-20 \textbf{ div } 3$

\vspace{20pt}

$-20 \textbf{ mod } 3$

\vfill


\section*{Week 2 Wednesday: Representing numbers}
%! app: Numbers
%! outcome: TODOoutcome

Modeling uses data-types that are encoded in a computer.
The details of the encoding impact the efficiency of algorithms
we use to understand the systems we are modeling and the 
impacts of these algorithms on the people using the systems.
Case study: how to encode numbers?

\phantom{
Positional representation with familiar (decimal) number encodings
\vspace{30pt}
}
\vfill
%! app: Numbers
%! outcome: representing numbers

{\bf Definition} For $b$ an integer greater than $1$ and $n$ a positive integer, 
the {\bf base $b$ expansion of $n$}  is
\[
(a_{k-1} \cdots a_1 a_0)_b
\]
where $k$ is a positive integer, $a_0, a_1, \ldots, a_{k-1}$ 
are (symbols for) nonnegative integers less than $b$, $a_{k-1} \neq  0$, and
\[
n =  \sum_{i=0}^{k-1} a_{i} b^{i}
\]

Notice: {\it The base $b$ expansion of a positive integer $n$ is a string over the alphabet 
$\{x \in \mathbb{N} \st x < b\}$
whose leftmost character is nonzero.}

\begin{center}
\begin{tabular}{|c|c|}
\hline
Base $b$ & Collection of possible coefficients in base $b$ expansion of  a positive integer \\
\hline
& \\
Binary ($b=2$) & $\{0,1\}$ \\
\hline
& \\
Ternary ($b=3$) & $\{0,1, 2\}$ \\
\hline
& \\
Octal ($b=8$) & $\{0,1, 2, 3, 4, 5, 6, 7\}$\\
\hline
& \\
Decimal ($b=10$) & $\{0,1, 2, 3, 4, 5, 6, 7, 8, 9\}$\\
\hline
& \\
Hexadecimal ($b=16$) &  $\{0,1, 2, 3, 4, 5, 6, 7, 8, 9, A, B, C, D, E, F\}$\\
& letter coefficient symbols represent numerical values $(A)_{16} = (10)_{10}$\\
&$(B)_{16} = (11)_{10} ~~(C)_{16} = (12)_{10} ~~
 (D)_{16} = (13)_{10} ~~ (E)_{16} = (14)_{10} ~~ (F)_{16} = (15)_{10} $\\
\hline
\end{tabular}
\end{center}


\vfill
\newpage
%! app: Numbers
%! outcome: representing numbers

%\fbox{\parbox{\textwidth}{
%{\bf Common bases}: \hfill Binary  $b=2$ \qquad Octal $b=8$ \qquad Decimal $b=10$ \qquad Hexadecimal $b=16$
%\hfill }}

{\it Examples}:

$(1401)_{2}$

\vfill

$(1401)_{10}$

\vfill
\vfill
\vfill


$(1401)_{16}$

\vfill
\vfill
\vfill

%! app: TODOapp
%! outcome: trace algorithms

\fbox{\parbox{\textwidth}{%
{\bf New!} An algorithm is a finite sequence of precise instructions for solving a problem.
\hfill
}}

Algorithms can be expressed in English or in more formalized descriptions like pseudocode or fully executable programs.


Sometimes, we can define algorithms whose output matches the 
rule for a function we already care about. Consider the (integer) logarithm function
\[
logb  : \{b \in \mathbb{Z} \mid b >1 \}  \times \mathbb{Z}^+ ~~\to~~ \mathbb{N}
\]
defined by 
\[
logb (~ (b,n)~) =  \text{greatest integer } y \text{ so that } b^y  \text{ is less than or equal to } n 
\]

%! app: Numbers
%! outcome: number representation

\begin{algorithm}[caption={Calculating integer part of base $b$ logarithm}]
   procedure $logb$($b$,$n$: positive integers with $b > 1$)
   $i$ := $0$
   while $n$ > $b-1$
     $i$ := $i + 1$
     $n$ := $n$ div $b$
   return $i$ $\{ i$ holds the integer part of the base $b$ logarithm of $n\}$
\end{algorithm}

Trace this algorithm with inputs $b=3$ and $n=17$

\vfill
  \begin{tabular}{l|c|c|c|c}
  & $b$ & $n$ & $i$  & $n > b-1$?\\
  \hline 
  Initial value & ~$3$~ & $17$ & \phantom{$0$} & \phantom{~Yes~}\\
  After 1 iteration  & \phantom{$3$} & \phantom{$5$} & \phantom{$1$} & \phantom{T}\\
  After 2 iterations & \phantom{$3$} & \phantom{$1$} & \phantom{$2$} & \phantom{F}\\
  After 3 iterations &  &  & & \\
  &&&&\\
  \end{tabular}

\vfill

\vfill

Compare: does the output match the rule for the (integer) logarithm function?


\newpage
%! app: Numbers
%! outcome: trace algorithms

{\bf Two algorithms for constructing base $b$ expansion from decimal representation}

{\bf Most significant first}: Start with left-most coefficient of expansion (highest value)

{\it Informally}: Build up to the value we need to represent in ``greedy'' approach, using 
units determined by base.

\vfill


\begin{algorithm}[caption={Calculating base $b$ expansion, from left}]
procedure $\textit{baseb1}$($n, b$: positive integers with $b > 1$)
$v$ := $n$
$k$ := $1 + $ output of $logb$ algorithm with inputs $b$ and $n$
for $i$ := $1$ to $k$
  $a_{k-i}$ := $0$
  while $v \geq b^{k-i}$
    $a_{k-i}$ := $a_{k-i} + 1$
    $v$ := $v -  b^{k-i}$
return $(a_{k-1}, \ldots, a_0) \{(a_{k-1} \ldots a_0)_b~\textrm{ is the base } b \textrm{ expansion of } n \}$
\end{algorithm}

\vfill
\vfill

\newpage
{\bf Least significant first}: Start with right-most coefficient of expansion (lowest value)

%\begin{multicols}{2}
%  \begin{minipage}{3.2in}
    Idea: {\tiny(when $k > 1$)} 
    \begin{align*}
      n &= a_{k-1} b^{k-1} + \cdots + a_1 b + a_0 \\
        &= b ( a_{k-1} b^{k-2} + \cdots + a_1) + a_0\end{align*}
    so $a_0 = n \textbf{ mod } b$ and $a_{k-1} b^{k-2} + \cdots + a_1 = n \textbf{ div } b$.
%\end{minipage}
%\columnbreak
\begin{algorithm}[caption={Calculating base $b$ expansion, from right}]
procedure $\textit{baseb2}$($n, b$: positive integers with $b > 1$)
$q$ := $n$
$k$ := $0$
while $q  \neq 0$
  $a_{k}$ := $q$ mod $b$
  $q$ := $q$ div $b$
  $k$ := $k+1$
return $(a_{k-1}, \ldots, a_0) \{(a_{k-1} \ldots a_0)_b~\textrm{ is the base } b \textrm{ expansion of } n \}$
\end{algorithm}
%\end{multicols}

\vfill
\vfill
\newpage


\newpage
\section*{Week 2 Friday: Algorithms for numbers}
%! app: Numbers
%! outcome: representing numbers

Find and fix any and all mistakes with the following:
\begin{itemize}
\item[(a)] $(1)_2 = (1)_8$
\item[(b)] $(142)_{10} = (142)_{16}$
\item[(c)] $(20)_{10} = (10100)_2$
\item[(d)] $(35)_8 = (1D)_{16}$
\end{itemize}
%! app: Numbers
%! outcome: trace algorithms

%{\it Recall the definition of base expansion we discussed last time.}
%%! app: Numbers
%! outcome: representing numbers

{\bf Definition} For $b$ an integer greater than $1$ and $n$ a positive integer, 
the {\bf base $b$ expansion of $n$}  is
\[
(a_{k-1} \cdots a_1 a_0)_b
\]
where $k$ is a positive integer, $a_0, a_1, \ldots, a_{k-1}$ 
are (symbols for) nonnegative integers less than $b$, $a_{k-1} \neq  0$, and
\[
n =  \sum_{i=0}^{k-1} a_{i} b^{i}
\]

Notice: {\it The base $b$ expansion of a positive integer $n$ is a string over the alphabet 
$\{x \in \mathbb{N} \st x < b\}$
whose leftmost character is nonzero.}

\begin{center}
\begin{tabular}{|c|c|}
\hline
Base $b$ & Collection of possible coefficients in base $b$ expansion of  a positive integer \\
\hline
& \\
Binary ($b=2$) & $\{0,1\}$ \\
\hline
& \\
Ternary ($b=3$) & $\{0,1, 2\}$ \\
\hline
& \\
Octal ($b=8$) & $\{0,1, 2, 3, 4, 5, 6, 7\}$\\
\hline
& \\
Decimal ($b=10$) & $\{0,1, 2, 3, 4, 5, 6, 7, 8, 9\}$\\
\hline
& \\
Hexadecimal ($b=16$) &  $\{0,1, 2, 3, 4, 5, 6, 7, 8, 9, A, B, C, D, E, F\}$\\
& letter coefficient symbols represent numerical values $(A)_{16} = (10)_{10}$\\
&$(B)_{16} = (11)_{10} ~~(C)_{16} = (12)_{10} ~~
 (D)_{16} = (13)_{10} ~~ (E)_{16} = (14)_{10} ~~ (F)_{16} = (15)_{10} $\\
\hline
\end{tabular}
\end{center}



Practice: write an algorithm for converting from base $b_1$ expansion to base $b_2$ expansion:

\phantom{
Earlier, we saw (two different) algorithms for, given 
a target base $b$, converting from decimal to base $b$ expansions. 
We will use either one of these as a subroutine in this algorithm.\\
Given a base expansion in base $b_1$:\\
Step 1: Use the definition of base expansion to calculate the value of
    this number (in decimal).\\
Step 2: Use the Least Significant First algorithm to write this value in 
    base $b_2$ and output the result.
}
\vspace{200pt}
\input{../activity-snippets/fixed-width-definition.tex}
%! app: TODOapp
%! outcome: TODOoutcome

\begin{center}
    \begin{tabular}{|c|c|c|c|c|}
    \hline
    Decimal &  Binary  & Binary fixed-width $10$& Binary fixed-width $7$ & Binary fixed-width $4$\\
    $b=10$ & $b=2$ & $b=2$, $w =  10$& $b=2$, $w =  7$& $b=2$, $w =  4$ \\
    \hline 
    &&&&  \\
    $(20)_{10}$&\phantom{$(10100)_{2}$\qquad\qquad}&&  &\\
    &&&&  \\
%    &(a)&(b)&(c)&(d)  \\
    \hline
    \end{tabular}
    \end{center}

%! app: TODOapp
%! outcome: TODOoutcome

{\bf Definition} For $b$ an integer greater than $1$, $w$ a positive integer, 
$w'$ a positive  integer, and $x$ a real number the {\bf base $b$ fixed-width 
expansion of $x$ with integer part width $w$  and fractional part width $w'$} is
$(a_{w-1} \cdots a_1 a_0 .  c_{1} \cdots c_{w'})_{b,w,w'}$
where  $a_0, a_1, \ldots, a_{w-1}, c_1, \ldots, c_{w'}$ are nonnegative integers less than $b$ and
$$x \geq \sum_{i=0}^{w-1} a_{i} b^{i} + \sum_{j=1}^{w'}  c_{j} b^{-j} \hfill
\textrm{\qquad and \qquad}
\hfill x < \sum_{i=0}^{w-1} a_{i} b^{i} + \sum_{j=1}^{w'} c_{j} b^{-j} + b^{-w'}$$

\begin{center}
\begin{tabular}{|c|p{5in}|}
\hline
& \\
$3.75$  in fixed-width binary,& \\
integer part width $2$,&\\
 fractional part width $8$ & \\
& \\
& \\
& \\
& \\
\hline
& \\
$0.1$  in fixed-width binary, & \\
integer part width $2$, &\\
 fractional part width $8$ & \\
 & \\
 & \\
 & \\
 & \\
 \hline
\end{tabular}
\end{center}

\vfill

\includegraphics[width=2in]{../../resources/images/ArithmeticDemo.png}

Note: Java uses floating point, not fixed width representation, but similar rounding errors appear in both.

%\input{../activity-snippets/expansion-summary.tex}
\newpage
%! app: TODOapp
%! outcome: TODOoutcome

{\bf Representing negative integers in binary}: Fix a positive integer  width for the representation  $w$, $w >1$.

\begin{tabular}{|cc|p{3.4in}|p{3.7in}|}
\hline
& & To  represent a positive integer $n$ & To represent a negative integer $-n$\\
\hline
&& &  \\
&\parbox[t]{2mm}{\multirow{4}{*}{\rotatebox[origin=c]{90}{Sign-magnitude}}} &
$[ 0a_{w-2} \cdots a_0]_{s,w}$, where $n =  (a_{w-2} \cdots a_0)_{2,w-1}$& 
$[1a_{w-2} \cdots a_0]_{s,w}$
, where $n =  (a_{w-2} \cdots a_0)_{2,w-1}$\\
&& & \\
&& Example $n=17$, $w=7$:  & Example $-n=-17$, $w=7$: \\
&& & \\
&& & \\
&& & \\
&& & \\
&& & \\
&& & \\
&& & \\
\hline
&&  &  \\
&\parbox[t]{2mm}{\multirow{4}{*}{\rotatebox[origin=c]{90}{2s complement}}} &
$[0a_{w-2} \cdots a_0]_{2c,w}$, where $n =  (a_{w-2} \cdots a_0)_{2,w-1}$& $[1a_{w-2} \cdots a_0]_{2c,w}$, where $2^{w-1} - n =  (a_{w-2} \cdots a_0)_{2,w-1}$\\
&& & \\
&& Example $n=17$, $w=7$:  & Example $-n=-17$, $w=7$: \\
&& & \\
&& & \\
&& & \\
&& & \\
&& & \\
&& & \\
&& & \\
\hline
% &&  &  \\
% \parbox[t]{1.5mm}{\multirow{4}{*}{\rotatebox[origin=c]{90}{{\it Extra example:}}}} 
% & \parbox[t]{2mm}{\multirow{4}{*}{\rotatebox[origin=c]{90}{1s complement}}} &
% $[0a_{w-2} \cdots a_0]_{1c,w}$, where $n =  (a_{w-2} \cdots a_0)_{2,w-1}$& $[1\bar{a}_{w-2} \cdots \bar{a}_0]_{1c,w}$, where $n =  (a_{w-2} \cdots a_0)_{2,w-1}$ and we define  $\bar{0} = 1$ and $\bar{1} = 0$.\\
% && & \\
% && Example $n=17$, $w=7$:  & Example $-n=-17$, $w=7$: \\
% && & \\
% && & \\
% && & \\
% && & \\
% && & \\
% \hline
\end{tabular}
\input{../activity-snippets/calculating-2s-complement.tex}
\newpage
%! app: TODOapp
%! outcome: TODOoutcome

{\bf Representing $0$}:

So far, we have representations for
positive and negative integers. What about $0$?

\begin{tabular}{|cc|p{3.4in}|p{3.7in}|}
   \hline
   & & To  represent a {\bf non-negative} integer $n$ & To represent a {\bf non-positive} integer $-n$\\
   \hline
   && &  \\
   &\parbox[t]{2mm}{\multirow{4}{*}{\rotatebox[origin=c]{90}{Sign-magnitude}}} &
   $[ 0a_{w-2} \cdots a_0]_{s,w}$, where $n =  (a_{w-2} \cdots a_0)_{2,w-1}$& 
   $[1a_{w-2} \cdots a_0]_{s,w}$
   , where $n =  (a_{w-2} \cdots a_0)_{2,w-1}$\\
   && & \\
   && Example $n=0$, $w=7$:  & Example $-n=0$, $w=7$: \\
   && & \\
   && & \\
   && & \\
   && & \\
   && & \\
   && & \\
   && & \\
   && & \\
   && & \\
   && & \\
%   && (a) & (b)\\
   \hline
   &&  &  \\
   &\parbox[t]{2mm}{\multirow{4}{*}{\rotatebox[origin=c]{90}{2s complement}}} &
   $[0a_{w-2} \cdots a_0]_{2c,w}$, where $n =  (a_{w-2} \cdots a_0)_{2,w-1}$& $[1a_{w-2} \cdots a_0]_{2c,w}$, where $2^{w-1} - n =  (a_{w-2} \cdots a_0)_{2,w-1}$\\
   && & \\
   && Example $n=0$, $w=7$:  & Example $-n=0$, $w=7$: \\
   && & \\
   && & \\
   && & \\
   && & \\
   && & \\
   && & \\
   && & \\
   && & \\
   && & \\
   && & \\
%   && (c) & (d) \\
   \hline
\end{tabular}
\newpage
\subsection*{Review Quiz}
\begin{enumerate}
\item Functions and algorithms
\begin{enumerate}
    \item {%! app: Numbers
%! outcome: data types

What is a recursive definition of the set
$\mathbb{Z}^+ \times \mathbb{N}$ that is the domain of the 
function ``$b$ to the power of $i$''?
For convenience, we'll refer to this set as $X$ 
in the options below.

\begin{enumerate}[labelindent=-10pt, leftmargin=-10pt]
    \item[] Basis step: $(1,0) \in X$. Recursive step: If $(m,n) \in X$ then $(m+1, n+1) \in X$ too.
    \item[] Basis step: $(1,0) \in X$. Recursive step: If $(m,m-1) \in X$ then $(m+1, m) \in X$ too.
    \item[] Basis step: $(b,0) \in X$ for each $b \in \mathbb{Z}^+$. Recursive step: If $(m,n) \in X$ then $(m+1, n+1) \in X$ too.
    \item[] Basis step: $(b,0) \in X$ for each $b \in \mathbb{Z}^+$. Recursive step: If $(m,n) \in X$ then $(m, n+1) \in X$ too.
    \item[] None of the above.
\end{enumerate}}
    \item {%! app: Numbers
%! outcome: data types, trace algorithms, div and mod

When running the algorithm $logb$ for calculating the 
integer part of base $b$ logarithm with inputs $b=4$ and $n=25$, 
which of the following calculations are helpful? Select all and only 
the calculations that are both relevant to the algorithm trace {\bf and}
are correct.

\begin{enumerate}
    \item[] $25 \textbf{ div } 4 = 5$
    \item[] $25 \textbf{ div } 4 = 6$
    \item[] $25 \textbf{ div } 4 = 1$
    \item[] $4 \textbf{ div } 25 = 0$
    \item[] $4 \textbf{ div } 25 = 5$
    \item[] $4 \textbf{ div } 25 = 1$
    \item[] $6 \textbf{ div } 4= 1$
    \item[] $5 \textbf{ div } 4= 1$
    \item[] $4 \textbf{ div } 4= 1$
\end{enumerate}}
\end{enumerate}
\item Base expansions
\begin{enumerate}
    \item {%! app: Numbers
%! outcome: number representation

Give the value (using usual mathematical conventions) of 
each of the following base expansions.
\begin{enumerate}
    \item $(10)_{2}$
    \item $(10)_{4}$
    \item $(17)_{16}$
    \item $(211)_{3}$
    \item $(3)_{8}$
\end{enumerate}}
    \item {%! app: TODOapp
%! outcome: TODOoutcome

Recall the definitions from class for number representations for {\bf base $b$ expansion of $n$},
{\bf  base $b$ fixed-width $w$ expansion of $n$}, and {\bf base $b$ fixed-width expansion of $x$ 
with integer part width $w$ and fractional part width $w'$}.

For example, the base $2$ (binary) expansion of $4$ is 
$\qquad
(100)_2 \qquad$
and the base $2$ (binary) fixed-width $8$ expansion of $4$ is
$\qquad
(00000100)_{2,8} \qquad$
and the base $2$ (binary) fixed-width expansion of $4$ with integer part width $3$ and fractional
part width $2$ of $4$ is
$\qquad
(100.00)_{2,3,2} \qquad$

Compute the listed expansions.  Enter your number using the notation for base 
expansions with parentheses but without subscripts. For example, 
if your answer were $(100)_{2,3}$
you would type \texttt{(100)2,3} into Gradescope.

\begin{enumerate}
\item Give the binary (base $2$) expansion of the number whose octal (base $8$) expansion is
\[
(371)_8
\]
\item Give the decimal (base $10$) expansion of the number whose octal (base $8$) expansion is
\[
(371)_8
\]
\item Give the octal (base $8$) fixed-width $3$ expansion of $(9)_{10}$ .
\item Give the ternary (base $3$) fixed-width $8$ expansion of $(9)_{10}$ .
\item Give the hexadecimal (base $16$) fixed-width $6$ expansion of
$(16711935)_{10}$ .\footnote{This matches a frequent debugging task --
sometimes a program will show a number formatted as a base $10$
integer that is much better understood with another representation.}
\item Give the hexadecimal (base $16$) fixed-width $4$ expansion of
$$(1011~ 1010 ~ 1001~ 0000 )_2$$
Note: the spaces between each group of 4 bits above are for your convenience only.  How
might they help your calculations?
\item Give the binary fixed width expansion of $0.125$ with integer part width $2$ and 
fractional part width $4$.
\item Give the binary fixed width expansion of $1$ with integer part width $2$ and 
fractional part width $3$.
\end{enumerate}}
    \item {\input{../activity-snippets/quiz-expansions-properties.tex}}
    \item {\input{../activity-snippets/quiz-representing-negatives.tex}}
\end{enumerate}
\item Multiple representations
{%! app: CP
%! outcome: data types, div and mod

We saw last week that, mathematically, a color can be represented as a $3$-tuple $(r, g, b)$ 
where $r$ represents the red component, $g$ the green component, $b$ the blue component and where each of $r$, $g$, $b$ must be from the collection $\{x \in \mathbb{N}\mid 0 \leq x \leq 255 \}$.
As an alternative representation, in this assignment
we'll use base $b$ fixed-width expansions to represent colors
as individual numbers.

{\bf Definition}: A {\bf hex color} is a nonnegative
integer, $n$, that has a base $16$ fixed-width $6$  expansion
$$n = (r_1r_2g_1g_2b_1b_2)_{16,6}$$ 
where $(r_1r_2)_{16,2}$ is the red
component, $(g_1g_2)_{16,2}$ is the green component, 
and $(b_1b_2)_{16,2}$ is the
blue.

\begin{enumerate}
\item What is the hex color corresponding to full black? Namely, this means setting the value in each of the 
    red, green, and blue components to be the minimum $0$.
    \begin{enumerate}
    \item[] $0$
    \item[] $(0,0,0)$
    \item[] $15^{5}+15^4+15^3+15^2+15^1+1$
    \item[] $15\cdot 16^{5}+15\cdot 16^4+15\cdot 16^3+15 \cdot 16^2+15 \cdot 16^1+15$
    \item[] $16^{5}+16^4+16^3+16^2+16^1+1$
    \end{enumerate}
\item What is the hex color corresponding to full white? Namely, this means setting the value in each of the 
red, green, and blue components to be the maximum $255$.
    \begin{enumerate}
    \item[] $0$
    \item[] $(0,0,0)$
    \item[] $15^{5}+15^4+15^3+15^2+15^1+1$
    \item[] $15\cdot 16^{5}+15\cdot 16^4+15\cdot 16^3+15 \cdot 16^2+15 \cdot 16^1+15$
    \item[] $16^{5}+16^4+16^3+16^2+16^1+1$
    \end{enumerate}
\item Select all and only  correct representations of the hex color which is 
full green (so the red and blue components are $0$ and green is set to $255$).
\begin{enumerate}
    \item[] $(00FF00)_{16,6}$
    \item[] $(00FF00)_{16}$
    \item[] $255$
    \item[] $255\cdot 256$
    \item[] $255\cdot 16^2$
    \item[] $65280$
    \end{enumerate}
\item Which of the following is a definition using set builder notation for the set of hex colors. (Select all and only correct choices)
    \begin{enumerate}
    \item[] $\{ x \in \mathbb{Z} \mid 0 \leq x \leq 16777215\}$
    \item[] $\{ x \in \mathbb{Z} \mid 0 \leq x \leq 16^6 -1\}$
    \item[] $\{ x \in \mathbb{N} \mid x \leq 16777215\}$
    \item[] $\{ x \in \mathbb{N} \mid x \leq 16^6 -1\}$
    \end{enumerate}
\end{enumerate}}
\end{enumerate}


\newpage

\subsection*{Week 3 at a glance}

\subsubsection*{We will be learning and practicing to:}
%applications of number representations
%circuits
%evaluating compound propositions
%logical equivalence via laws
%logical equivalence via truth tables
%cnf and dnf
%consistency
%translating
\begin{itemize}
\item Model systems with tools from discrete mathematics and reason about implications of modelling choices. Explore applications in CS through multiple perspectives, including software, hardware, and theory.
\begin{itemize}
    \item Determining the properties of positional number representations, including overflow and bit operations
   \item Connecting logical circuits and compound proposition and tracing to calcluate output values
\end{itemize}

\item Translate between different representations to illustrate a concept.
\begin{itemize}
   \item Translating between symbolic and English versions of statements using precise mathematical language
\end{itemize}

\item Use precise notation to encode meaning and present arguments concisely and clearly
\begin{itemize}
    \item Listing the truth tables of atomic boolean functions (and, or, xor, not, if, iff)
\end{itemize}

\item Know, select and apply appropriate computing knowledge and problem-solving techniques. Reason about computation and systems. Use mathematical techniques to solve problems. Determine appropriate conceptual tools to apply to new situations. Know when tools do not apply and try different approaches. Critically analyze and evaluate candidate solutions.
\begin{itemize}
    \item Evaluating compound propositions
    \item Judging logical equivalence of compound propositions using symbolic manipulation with known equivalences, including DeMorgan's Law
    \item Judging logical equivalence of compound propositions using truth tables
    \item Rewriting compound propositions using normal forms
    \item Judging whether a collection of propositions is consistent
\end{itemize}

\end{itemize}

\subsubsection*{TODO:}
\begin{list}
   {\itemsep2pt}
   \item Review quiz based on class material each day (due Friday April 19, 2024)
   \item Homework assignment 3 (due Tuesday April 23, 2024)
\end{list}

\newpage
\section*{Week 3 Monday: Fixed-width Addition and Circuits}

%! app: Computers
%! outcome: circuits

{\bf Fixed-width addition}: adding one bit at time, using the usual column-by-column and carry arithmetic, and dropping the carry from the leftmost column so the result is the same width as the summands.  {\it Does this give the right value for the sum?}
\begin{multicols}{2}
\begin{align*}
   & [0~ 1~ 0~ 1]_{s,4}\\
+ &  [1~ 1~ 0~ 1]_{s,4}\\
&\overline{\phantom{[0~0~1~0]_{s,4}}}\\
\end{align*}

\begin{align*}
   & [0~ 1~ 0~ 1]_{2c,4}\\
+ &  [1~ 0~ 1~ 1]_{2c,4}\\
&\overline{\phantom{[0~0~0~0]_{2c,4}}}\\
\end{align*}

\end{multicols}

\vfill

\begin{multicols}{3}
   \begin{align*}
      & (1~ 1~ 0~ 1~ 0~ 0)_{2,6}\\
   + & (0~ 0~ 0~ 1~ 0~ 1)_{2,6}\\
   &\overline{\phantom{(1~1~1~0~0~1)_{2,6}}}\\
   \end{align*}
   
   \begin{align*}
      & [1~ 1~ 0~ 1~ 0~ 0]_{s,6}\\
   + & [0~ 0~ 0~ 1~ 0~ 1]_{s,6}\\
   &\overline{\phantom{(1~1~1~0~0~1)_2}}\\
   \end{align*}
   
   \begin{align*}
      & [1~ 1~ 0~ 1~ 0~ 0]_{2c,6}\\
   + & [0~ 0~ 0~ 1~ 0~ 1]_{2c,6}\\
   &\overline{\phantom{(1~1~1~0~0~1)_2}}\\
   \end{align*}
\end{multicols}
\vfill

\newpage
\vfill
\vfill
%! app: Computers
%! outcome: circuits

In a {\bf combinatorial circuit} (also known as
a {\bf logic circuit}), we have {\bf logic gates} 
connected
by {\bf wires}. The inputs to the circuits are the 
values set on the input wires: possible
values are 0 (low) or 1 (high). The values
flow along the wires from left to right.
A wire may be split into two or more wires, 
indicated with a filled-in circle (representing
solder). Values stay the same along a wire. When 
one or more wires flow into a gate, the output 
value of that gate is computed
from the input values based on the gate's definition
table. Outputs of gates may become inputs to other
gates. 

%! app: Computers
%! outcome: circuits

\begin{multicols}{2}
\begin{center}\begin{tabular}{cc|c}
Inputs &  & Output \\
$x$ & $y$ & $x \text{ AND } y$  \\
\hline
$1$ & $1$ & $1$\\
$1$ & $0$ & $0$\\
$0$ & $1$ & $0$\\
$0$ & $0$ & $0$\\
\end{tabular}\end{center}
\columnbreak
\begin{center}\includegraphics[height=0.6in]{../../resources/images/xANDy.png} \end{center}
\end{multicols}

\vfill

\begin{multicols}{2}
\begin{center}\begin{tabular}{cc|c}
Inputs &  & Output \\
$x$ & $y$ & $x \text{ XOR } y$  \\
\hline
$1$ & $1$ & $0$\\
$1$ & $0$ & $1$\\
$0$ & $1$ & $1$\\
$0$ & $0$ & $0$\\
\end{tabular}\end{center}
\columnbreak
\begin{center}\includegraphics[height=0.4in]{../../resources/images/xXORy.png} \end{center}
\end{multicols}

\vfill

\begin{multicols}{2}
\begin{center}\begin{tabular}{c|c}
Input  & Output \\
$x$ & $\text{NOT } x$  \\
\hline
$1$ & $0$\\
$0$ & $1$\\
\end{tabular}\end{center}
\columnbreak
\begin{center}\includegraphics[height=0.5in]{../../resources/images/NOTx.png} \end{center}
\end{multicols}

% \begin{center}
%     \begin{tabular}{p{2in}p{2in}p{2in}}
%     \begin{center}\begin{tabular}{cc|c}
%     Inputs &  & Output \\
%     $x$ & $y$ & $x \text{ AND } y$  \\
%     \hline
%     $1$ & $1$ & $1$\\
%     $1$ & $0$ & $0$\\
%     $0$ & $1$ & $0$\\
%     $0$ & $0$ & $0$\\
%     \end{tabular}\end{center}
%     &
%     \begin{center}\begin{tabular}{cc|c}
%     Inputs &  & Output \\
%     $x$ & $y$ & $x \text{ XOR } y$  \\
%     \hline
%     $1$ & $1$ & $0$\\
%     $1$ & $0$ & $1$\\
%     $0$ & $1$ & $1$\\
%     $0$ & $0$ & $0$\\
%     \end{tabular}\end{center}
%     &
%     \begin{center}\begin{tabular}{c|c}
%     Input  & Output \\
%     $x$ & $\text{NOT } x$  \\
%     \hline
%     $1$ & $0$\\
%     $0$ & $1$\\
%     \end{tabular}\end{center}
%     \\
%     \begin{center}\includegraphics[height=0.6in]{../../resources/images/xANDy.png} \end{center} & 
%     \begin{center}\includegraphics[height=0.4in]{../../resources/images/xXORy.png} \end{center}& 
%     \begin{center}\includegraphics[height=0.5in]{../../resources/images/NOTx.png} \end{center}
%     \end{tabular}
%     \end{center}
\vfill
%! app: Computers
%! outcome: circuits

{\bf Example digital circuit}: 

\begin{multicols}{2}
\begin{center}
   \includegraphics[width=1.2in]{../../resources/images/circuitEx.png} 
\end{center}
\columnbreak
Output when $x=1, y=0, z=0, w = 1$ is \underline{\phantom{$~~~0~~~$}}
Output when $x=1, y=1, z=1, w = 1$ is \underline{\phantom{$~~~0~~~$}}
Output when $x=0, y=0, z=0, w = 1$ is \underline{\phantom{$~~~0~~~$}}
\phantom{Output when $x=0, y=0, z=0, w = 0$ is \underline{\phantom{$~~~0~~~$}}}
\end{multicols}



Draw a logic circuit with inputs $x$ and $y$ whose output  is always $0$.  {\it  Can you use exactly 1 gate?}


\vspace{40pt}
\vfill
\newpage
\section*{Week 3 Wednesday: Propositional Logic}
%! app: Computers, Numbers
%! outcome: circuits, applications of number representations

{\bf Fixed-width addition}: adding one bit at time, using the usual column-by-column and carry arithmetic, and dropping the carry from the leftmost column so the result is the same width as the summands.  In many cases, this gives representation of the correct value for the sum when we interpret the summands
in fixed-width binary or in 2s complement.

For single column:

\begin{multicols}{2}
\begin{center}
\begin{tabular}{cc|cc}
\multicolumn{2}{c|}{Input}  & \multicolumn{2}{|c}{Output}  \\
$x_0$ & $y_0$ & $c_0$ & $s_0$  \\
\hline
$1$ & $1$ & \phantom{$1$} & \phantom{$0$} \\
$1$ & $0$ & \phantom{$0$} & \phantom{$1$}\\
$0$ & $1$ & \phantom{$0$} & \phantom{$1$}\\
$0$ & $0$ & \phantom{$0$} & \phantom{$0$}\\
\end{tabular}
\end{center}
\columnbreak
\begin{center}
\includegraphics[width=1.5in]{../../resources/images/half-adder.png}
\end{center}
\end{multicols}
%! app: Computers, Numbers
%! outcome: circuits, applications of number representations

Draw a logic circuit that implements binary addition of 
two numbers that are each represented in fixed-width binary:
\begin{itemize}
\item Inputs  $x_0, y_0, x_1, y_1$ represent $(x_1  x_0)_{2,2}$ and $(y_1 y_0)_{2,2}$
\item Outputs  $z_0, z_1, z_2$ represent $(z_2  z_1 z_0)_{2,3} = (x_1  x_0)_{2,2} + (y_1 y_0)_{2,2}$ (may require up to width  $3$)
\end{itemize}

{\it First approach}: half-adder for each column, then combine carry from right column with sum of left column


Write expressions for the circuit output values in terms of input values:

$z_0 = \underline{\phantom{x_0 \oplus y_0\hspace{3in}}}$

$z_1 = \underline{\phantom{(x_1 \oplus y_1) \oplus c_0}\hspace{2.5in}}$ \phantom{where $c_0 = x_0 \land y_0$}

$z_2 = \underline{\phantom{(c_0 \land (x_1 \oplus y_1)) \oplus c_1}\hspace{2in}}$ \phantom{where $c_1 = x_1 \land y_1$}\\

\includegraphics[width=1.7in]{../../resources/images/width-2-adder.png}


\vfill

{\it There are other approaches, for example}: for middle column, first add carry from right column to $x_1$, then add result to $y_1$

\begin{comment}
Write expressions for the circuit output values in terms of input values:

$z_0 = \underline{\phantom{x_0 \oplus y_0}\hspace{3in}}$

$z_1 = \underline{ \phantom{(c_0 \oplus x_1) \oplus y_1}\hspace{2.4in}}$ \phantom{where $c_0 = x_0 \land y_0$}

$z_2 = \underline{\phantom{(c_0 \land x_1) \oplus ((c_0 \oplus x_1)\land y_1)}\hspace{1.5in}}$

\vfill

{\it Extra example} Describe how to generalize this addition circuit for larger width inputs.
\end{comment}

\newpage
%! app: Computers
%! outcome: evaluating compound propositions

{\bf Logical operators} aka propositional connectives

\begin{tabular}{lccccp{4in}}
{\bf Conjunction} & AND & $\land$ &\verb|\land| & 2 inputs & Evaluates to $T$ exactly when {\bf both} inputs are $T$\\
{\bf Exclusive or} & XOR & $\oplus$ &\verb|\oplus| & 2 inputs & Evaluates to $T$ exactly when {\bf exactly one} of inputs is $T$\\
{\bf Disjunction} & OR & $\lor$ &\verb|\lor| & 2 inputs & Evaluates to $T$ exactly when {\bf at least one} of inputs is $T$\\
{\bf Negation} & NOT & $\lnot$ &\verb|\lnot| & 1 input & Evaluates to $T$ exactly when its input is $F$\\
\end{tabular}
%! app: Computers
%! outcome: evaluating compound propositions, logical equivalence via laws, circuits, truth table definitions

Truth tables: Input-output tables where we use $T$ for $1$ and $F$ for $0$.

\begin{center}
\begin{tabular}{cc||c|c|c}
\multicolumn{2}{c||}{Input}  & \multicolumn{3}{c}{Output} \\
& & {\bf Conjunction} &  {\bf Exclusive or} & {\bf Disjunction} \\
$p$ & $q$ & $p \land q$ &  $p  \oplus  q$ & $p \lor  q$ \\
\hline
$T$ & $T$ & $T$ & $F$ & $T$\\
$T$ & $F$ & $F$ & $T$ & $T$\\
$F$ & $T$ & $F$ & $T$ & $T$\\
$F$ & $F$ & $F$ & $F$ & $F$\\
\hline
& & \includegraphics[width=0.5in]{../../resources/images/xANDy.png}
&  \includegraphics[width=0.5in]{../../resources/images/xXORy.png}
&  \includegraphics[width=0.5in]{../../resources/images/xORy.png}
\end{tabular}
\qquad \qquad\qquad
\begin{tabular}{c||c}
Input & Output \\
& {\bf Negation} \\
$p$ & $\lnot p$ \\
\hline
$T$ & $F$ \\
$F$ & $T$\\
\hline & \includegraphics[width=0.5in]{../../resources/images/NOTx.png}
\end{tabular}
\end{center}

\vfill
%! app: Computers
%! outcome: evaluating compound propositions, logical equivalence via truth tables

\begin{center}
    \begin{tabular}{ccc||p{3in}|c|c}
    \multicolumn{3}{c||}{Input}  & \multicolumn{3}{c}{Output} \\
    $p$ & $q$ & $r$  &  &  $(p \land q) \oplus (~ ( p \oplus q) \land r~)$ & $(p \land q) \vee (~ ( p \oplus q) \land r~)$ \\
    \hline
    $T$ & $T$  & $T$ &   && \\
    $T$ & $T$  & $F$ &   && \\
    $T$ & $F$  & $T$ &   && \\
    $T$ & $F$  & $F$ &   && \\
    $F$ & $T$  & $T$ &   && \\
    $F$ & $T$  & $F$ &   && \\
    $F$ & $F$  & $T$ &   && \\
    $F$ & $F$  & $F$ &   && \\
    \end{tabular}
\end{center}
    \vfill
\vfill
\newpage
\section*{Week 3 Friday: Logical Equivalence}
%! app: Computers
%! outcome: evaluating compound propositions, cnf and dnf

Given a truth table, how do we find an expression
using the input variables and logical operators that has the 
output values specified in this table?

{\it Application}: design a circuit given a desired input-output relationship.

\begin{center}
\begin{tabular}{cc||cc}
\multicolumn{2}{c||}{Input}  &\multicolumn{2}{c}{Output}\\
$p$ & $q$& $mystery_1$ & $mystery_2$\\
\hline
$T$ & $T$  & $T$ & $F$\\
$T$ & $F$  & $T$ & $F$\\
$F$ & $T$  & $F$ & $F$\\
$F$ & $F$  & $T$ & $T$\\
\end{tabular}
\end{center}


Expressions that have output $mystery_1$ are

\vspace{100pt}

Expressions that have output $mystery_2$ are

\vspace{100pt}

{\it Idea}: To develop an algorithm for translating truth tables to expressions, 
define a convenient {\bf normal form} for expressions.
%! app: Computers
%! outcome: cnf and dnf

{\bf  Definition} An expression built of variables and logical 
operators is in {\bf disjunctive normal form}  (DNF) means
that it is an OR of ANDs of variables and their negations.

{\bf  Definition} An expression built of variables and logical 
operators is in {\bf conjunctive normal form}  (CNF) means
that it is an AND of ORs of variables and their negations.

\newpage
%! app: TODOapp
%! outcome: translating, evaluating compound propositions

{\bf Proposition}: Declarative sentence that is true or false (not both).

{\bf Propositional variable}: Variable that represents a proposition.

{\bf Compound proposition}: New proposition formed from existing propositions (potentially) using logical operators.
{\it Note}: A propositional variable is one example of a compound proposition.

{\bf Truth table}: Table with one row for each of the possible combinations of truth values of the input and 
    an additional column that shows the truth value of the result of the operation corresponding to a particular row.
    

%! app: TODOapp
%! outcome: logical equivalence via laws, logical equivalence via truth tables

{\bf Logical equivalence }: Two compound  propositions are {\bf logically  equivalent} means that  they 
have the  same  truth  values for all settings of truth  values to their propositional  variables.

{\bf Tautology}:  A compound proposition that evaluates to true
for all settings of truth  values to its propositional  variables; it is  abbreviated $T$.

{\bf Contradiction}: A compound proposition that  evaluates  to  false 
for  all settings of truth  values to its propositional  variables; it  is abbreviated $F$.

{\bf Contingency}: A compound proposition that is neither a tautology nor a contradiction.

\vfill
%! app: TODOapp
%! outcome: TODOoutcome

Label each of the following as a tautology, contradiction, or contingency.

$p \land p$

\vfill

$p \oplus p$

\vfill

$p \lor p$

\vfill

$p \lor \lnot p$

\vfill

$p \land \lnot p$

\vfill


\vfill
%! app: TODOapp
%! outcome: logical equivalence via truth tables, logical equivalence via laws

{\it Extra Example}: Which of the  compound propositions in the table below are logically equivalent?
\begin{center}
\begin{tabular}{cc||c|c|c|c|c}
\multicolumn{2}{c||}{Input}  & \multicolumn{5}{c}{Output} \\
$p$ & $q$ & $\lnot (p \land \lnot q)$ & $\lnot (\lnot p  \lor \lnot q)$ &  $(\lnot p \lor  q)$
& $(\lnot q \lor \lnot p)$ & $(p \land q)$  \\
\hline
$T$ & $T$ & &&&&\\
$T$ & $F$ & &&&&\\
$F$ & $T$ & &&&&\\
$F$ & $F$ & &&&&\\
\end{tabular}
\end{center}
\vfill
\newpage
\subsection*{Review Quiz}
\begin{enumerate}
    \item Fixed-width addition: Recall the definitions of signed integer representations from class: 
    sign-magnitude and 2s complement.    
        \begin{enumerate}
            \item {%! app: Computers
%! outcome: circuits, applications of number representations

Recall the definitions of signed integer representations from class: 
sign-magnitude and 2s complement.

\begin{enumerate}
   \item What is the least integer that can be represented in sign-magnitude 
   width $4$?
   \item What is the greatest integer that  can be represented in sign-magnitude 
   width $4$?
   \item What is the least integer that can be represented in 2s complement
   width $4$?
   \item What is the greatest integer that  can be represented in 2s complement
   width $4$?
\end{enumerate}
}
            \item {%! app: Numbers, Computers
%! outcome: applications of number representations

\begin{enumerate}
    \item In binary fixed-width addition (adding one bit at time, using 
    the usual column-by-column and carry arithmetic, and ignoring the carry 
    from the  leftmost column), we get: 
    \begin{align*}
        &1110  \qquad  \text{first summand}\\
        +&0100 \qquad  \text{second summand}\\
        &\overline{0010} \qquad \text{result}
    \end{align*}
    Select all and only the  true  statements below:
    \begin{enumerate}
        \item When interpreting each of the summands and the result in binary fixed-width 4, 
        the result represents the actual value of the sum of the summands.
        \item When interpreting each of the summands and the sum in sign-magnitude width 4, the result  
        represents the actual value of the sum of the summands.
        \item When interpreting each of the summands and the sum in 2s complement width 4, the result 
        represents the actual value of the sum of the summands.
    \end{enumerate}    
    \item In binary fixed-width addition (adding one bit at time, using the 
    usual column-by-column and carry arithmetic, and ignoring the carry from the 
    leftmost column), we get: 
    \begin{align*}
        &0110  \qquad  \text{first summand}\\
        +&0111 \qquad  \text{second summand}\\
        &\overline{1101} \qquad \text{result}
    \end{align*}
    Select all and only the  true  statements below:
    \begin{enumerate}
        \item When interpreting each of the summands and the result in binary fixed-width 4, 
        the result represents the actual value of the sum of the summands.
        \item When interpreting each of the summands and the sum in sign-magnitude width 4, 
        the result  
        represents the actual value of the sum of the summands.
        \item When interpreting each of the summands and the sum in 2s complement width 4, 
        the result 
        represents the actual value of the sum of the summands.
    \end{enumerate}   
\end{enumerate}}
        \end{enumerate}
    \newpage
    \item Circuits
        \begin{enumerate}
            \item {\input{../activity-snippets/quiz-circuit-tracing.tex}}
            \item {\input{../activity-snippets/quiz-circuit-implementing-operation.tex}}
            \item \input{../activity-snippets/quiz-circuit-tracing-with-or.tex}
        \end{enumerate}
    \item Compound Propositions
        \begin{enumerate}
            \item %! app: Computers
%! outcome: cnf and dnf



Recall the definition of DNF (disjunctive normal form) and CNF (conjunctive normal form).
In particular, remember that to build an expression in DNF whose output matches a given 
truth table, we focus on the rows of the truth table that output $T$; 
To build an expression in CNF whose output matches a given 
truth table, we focus on the rows of the truth table that output $F$.
Select all and only true statements about an expression  that has output $?$ in the truth table below:

\begin{tabular}{ccc||c}
    \multicolumn{3}{c||}{Input}  & Output\\
    $p$ & $q$ & $r$  &  ?\\
    \hline
    $T$ & $T$  & $T$ & $T$ \\
    $T$ & $T$  & $F$ & $T$ \\
    $T$ & $F$  & $T$ & $F$ \\
    $T$ & $F$  & $F$ & $T$ \\
    $F$ & $T$  & $T$ & $F$ \\
    $F$ & $T$  & $F$ & $F$ \\
    $F$ & $F$  & $T$ & $T$ \\
    $F$ & $F$  & $F$ & $F$ \\
\end{tabular}

\begin{enumerate}
\item[] An expression in DNF that has output $?$ is 
$$(p \land q \land r) \lor (p \land q \land \lnot r) \lor (p \land \lnot q \land \lnot r) \lor (\lnot p \land \lnot q \land r)$$
\item[] An expression in DNF that has output $?$ is 
$$(\lnot p \land \lnot q \land \lnot r) \lor (\lnot p \land \lnot q \land r) \lor (\lnot p \land q \land r) \lor (p \land q \land \lnot r)$$
\item[] An expression in CNF that has output $?$ is 
$$(p \lor \lnot q \lor r) \land (\lnot p \lor q \lor r) \land (\lnot p \lor  q \lor \lnot r) \land (\lnot p \lor \lnot q \lor \lnot r)$$
\item[] An expression in CNF that has output $?$ is 
$$(\lnot p \lor  q \lor \lnot r) \land ( p \lor \lnot q \lor \lnot r) \land ( p \lor \lnot q \lor r) \land (p \lor q \lor r)$$
\end{enumerate}

            \item \input{../activity-snippets/quiz-cnf-dnf.tex}
        \end{enumerate}
    \item Logical equivalence.
    \input{../activity-snippets/quiz-truth-values-or-and.tex}
\end{enumerate}

\newpage

\subsection*{Week 4 at a glance}

\subsubsection*{We will be learning and practicing to:}
%evaluating compound propositions
%truth table definitions
%consistency
%translating
%logical equivalence via laws
%logical equivalence via truth tables
%variants of conditionals
\begin{itemize}

\item Translate between different representations to illustrate a concept.
\begin{itemize}
   \item Translating between symbolic and English versions of statements using precise mathematical language
   \item Translating between truth tables (tables of values) and compound propositions
\end{itemize}

\item Use precise notation to encode meaning and present arguments concisely and clearly
\begin{itemize}
    \item Listing the truth tables of atomic boolean functions (and, or, xor, not, if, iff)
    \item Defining functions, predicates, and binary relations using multiple representations
\end{itemize}

\item Know, select and apply appropriate computing knowledge and problem-solving techniques. Reason about computation and systems. Use mathematical techniques to solve problems. Determine appropriate conceptual tools to apply to new situations. Know when tools do not apply and try different approaches. Critically analyze and evaluate candidate solutions.
\begin{itemize}
    \item Evaluating compound propositions
    \item Judging logical equivalence of compound propositions using symbolic manipulation with known equivalences, including DeMorgan's Law
    \item Writing the converse, contrapositive, and inverse of a given conditional statement
    \item Determining what evidence is required to establish that a quantified statement is true or false
    \item Evaluating quantified statements about finite and infinite domains
\end{itemize}

\end{itemize}

\subsubsection*{TODO:}
\begin{list}
   {\itemsep2pt}
   \item Review quiz based on class material each day (due Friday April 26, 2024)
   \item Start reviewing for Test 1, in class next week on Friday May 3, 2024.
\end{list}

\newpage

\section*{Week 4 Monday: Conditionals and Logical Equivalence}
%! app: TODOapp
%! outcome: evaluating compound propositions, truth table definitions

\begin{center}
    \begin{tabular}{cc||c|c|c|c|c}
    \multicolumn{2}{c||}{Input}  & \multicolumn{5}{c}{Output} \\
     & & Conjunction &  Exclusive or & Disjunction  &  Conditional & Biconditional  \\
    $p$ & $q$ & $p \wedge q$ &  $p  \oplus  q$ & $p \vee  q$ & $p \to q$ & $p \leftrightarrow q$\\
    \hline
    $T$ & $T$ & $T$ & $F$ & $T$ & $T$& $T$\\
    $T$ & $F$ & $F$ & $T$ & $T$ & $F$& $F$\\
    $F$ & $T$ & $F$ & $T$ & $T$ & $T$& $F$\\
    $F$ & $F$ & $F$ & $F$ & $F$ & $T$& $T$\\
    \hline
    && ``$p$ and $q$'' & ``$p$ xor $q$'' & ``$p$ or $q$'' & ``if $p$ then $q$'' & ``$p$ if and only if $q$''
    \end{tabular}
\end{center}
    
%! app: TODOapp
%! outcome: evaluating compound propositions, translating, variants of conditionals

The only way to make  the conditional statement $p \to q$ false is to \underline{\phantom{\hspace{3in}}}\\

\begin{tabular}{llll}
The {\bf  hypothesis}  of $p \to q$ is  &\underline{\phantom{\hspace{1in}}} &
The {\bf  antecedent}  of $p \to q$ is  &\underline{\phantom{\hspace{1in}}} \\
&&&  \\
The {\bf  conclusion}  of $p \to q$ is & \underline{\phantom{\hspace{1in}}}&
The {\bf  consequent}  of $p \to q$ is & \underline{\phantom{\hspace{1in}}}\\
&&&  \\
\end{tabular}

%! app: TODOapp
%! outcome: translating

The {\bf converse}  of $p \to q$ is \underline{\phantom{ $q \to p$\hspace{1.6in}}}\\

The {\bf inverse}  of $p \to q$ is \underline{\phantom{ $\lnot p \to \lnot q$\hspace{1.6in}}}\\

The {\bf contrapositive}  of $p \to q$ is \underline{\phantom{$\lnot q \to \lnot p$\hspace{1.6in}}} \\

\vfill
%! app: TODOapp
%! outcome: evaluating compound propositions

We can use a recursive definition to describe all 
compound propositions that use propositional variables 
from a specified collection.  Here's the definition
for all compound propositions whose propositional variables 
are in $\{p, q\}$.

\[
\begin{array}{ll}
\textrm{Basis Step: } & p \textrm{ and } q \textrm{ are each a compound proposition} \\
\textrm{Recursive Step: } & \textrm{If } x \textrm{ is a compound proposition then so is } (\lnot x) 
\textrm{ and if } \\
& x \textrm{ and } y \textrm{ are both compound propositions then so is each of }\\
&(x \land y), (x \oplus y), (x \lor y), (x \to y), (x \leftrightarrow y)
\end{array}
\]
%! app: TODOapp
%! outcome: evaluating compound propositions

Order of operations (Precedence) for logical operators: 

Negation, then conjunction / disjunction, then conditional / biconditionals.

Example: $\lnot p \lor \lnot q$ means $(\lnot p) \lor (\lnot q)$.

\newpage
%! app: TODOapp
%! outcome: logical equivalence via laws, logical equivalence via truth tables

{\bf (Some) logical equivalences}

{\it Can replace $p$ and $q$ with any compound proposition}

\begin{tabular}{llp{3in}}
$\lnot ( \lnot p) \equiv p$ & & {\bf Double negation}\\
&& \\
&& \\
$p \lor q \equiv q \lor p$ & $p \land q \equiv q \land p$ & {\bf Commutativity} Ordering of terms\\
&& \\
&& \\
$(p \lor q) \lor r  \equiv p \lor (q \lor r)$ & $(p \land q) \land r  \equiv p \land (q \land r)$ & {\bf Associativity} Grouping of terms\\
&& \\
&& \\
$p \land F \equiv F$ \qquad $p \lor T \equiv T$ & $p \land T \equiv p$ \qquad $p \lor F \equiv p$ & {\bf Domination} aka 
short circuit evaluation\\
&& \\
&& \\
$\lnot (p \land q) \equiv \lnot p \lor \lnot q$ & $\lnot (p \lor q) \equiv \lnot p \land\lnot q$  & {\bf DeMorgan's Laws}\\
&& \\
\end{tabular}
\vfill

\begin{tabular}{llp{3in}}
$p \to q \equiv \lnot p \lor q$ & & \\
&& \\
&& \\
$p \to q \equiv \lnot q \to \lnot p$ & &{\bf Contrapositive} \\
&& \\
&& \\
$\lnot (p \to q) \equiv p\land \lnot q$  & &\\
&& \\
&& \\
$\lnot( p \leftrightarrow q) \equiv p \oplus q$ && \\
&& \\
&& \\
$p \leftrightarrow q \equiv q \leftrightarrow p$ &&\\
&& \\
\end{tabular}

\vfill

{\it Extra examples}:

$p \leftrightarrow q$ is not logically equivalent to $p \land q$ because \underline{\phantom{\hspace{4in}}} 

$p \to q$ is not logically equivalent to $q \to p$ because \underline{\phantom{\hspace{4in}}} 
\vfill

\newpage
%! app: TODOapp
%! outcome: translating

{\bf Common ways to express logical operators in English}:

{\bf Negation} $\lnot p$ can be said in English as 

\vspace{-20pt}
\begin{itemize}
\item Not $p$.
\item It's not the case that $p$.
\item $p$ is false.
\end{itemize}

{\bf Conjunction} $p \land q$ can be said in English as

\vspace{-20pt}
\begin{itemize}
    \item $p$ and $q$.
    \item Both $p$ and $q$ are true.
    \item $p$ but $q$.
\end{itemize}

{\bf Exclusive or} $p \oplus q$ can be said in English as

\vspace{-20pt}
\begin{itemize}
    \item $p$ or $q$, but not both.
    \item Exactly one of $p$ and $q$ is true.
\end{itemize}

{\bf Disjunction} $p \lor q$ can be said in English as

\vspace{-20pt}
\begin{itemize}
    \item $p$ or $q$, or both.
    \item $p$ or $q$ (inclusive).
    \item At least one of $p$ and $q$ is true.
\end{itemize}

{\bf Conditional} $p \to q$ can be said in English as

\begin{multicols}{2}
\begin{itemize}
    \item if $p$, then $q$.
    \item $p$ is sufficient for $q$.
    \item $q$ when $p$.
    \item $q$ whenever $p$.
    \item $p$ implies $q$.
    \item $q$ follows from $p$.
    \item $p$ is sufficient for $q$.
    \item $q$ is necessary for $p$.
    \item $p$ only if $q$.
\end{itemize}
\end{multicols}

{\bf Biconditional}

\vspace{-20pt}
\begin{itemize}
    \item $p$ if and only if $q$.
    \item $p$ iff $q$.
    \item If $p$ then $q$, and conversely.
    \item $p$ is necessary and sufficient for $q$.
\end{itemize}
\newpage
%! app: TODOapp
%! outcome: translating

{\bf Translation}: Express each of the following sentences as compound propositions, using
the given propositions.

\begin{multicols}{2}
``A sufficient condition for the warranty to be good is that you bought the computer less than a year ago"
\columnbreak
\begin{align*}
w &\text{ is  ``the warranty is good"} \\
b &\text{ is  ``you bought the computer less than a year ago"} \\
\end{align*}
\end{multicols}
\vfill

\begin{multicols}{2}
``Whenever the message was sent from an unknown system, it is scanned for viruses."
\columnbreak
\begin{align*}
s &\text{ is  ``The message is scanned for viruses"} \\
u &\text{ is  ``The message was sent from an unknown system"} \\
\end{align*}
\end{multicols}
\vfill

\begin{multicols}{2}
``I will complete my to-do list only if I put a reminder in my calendar"
\columnbreak
\begin{align*}
d &\text{ is  ``I will complete my to-do list"} \\
c &\text{ is  ``I put a reminder in my calendar"} \\
\end{align*}
\end{multicols}
\vfill
\newpage
%! app: TODOapp
%! outcome: consistency

{\bf Definition}: A collection of  compound  propositions
is called {\bf consistent} if  there
is  an assignment  of  truth values
to  the  propositional variables that makes
each of the compound propositions  true.

%! app: 
%! outcome: evaluating compound propositions, consistency

{\bf Consistency}: 
\begin{quote}
Whenever the system software is being upgraded, users cannot access the file system. 
If users can access the file system, then they can save new files. 
If users cannot save new files, then the system software is not being upgraded.
\end{quote}

\begin{enumerate}
\item Translate to symbolic compound propositions
\vfill
\item Look for some truth assignment to the propositional variables for which all the compound propositions output $T$
\vfill
\end{enumerate}
\newpage


\section*{Week 4 Wednesday: Predicates and Quantifiers}
%! app: TODOapp
%! outcome: data types, evidence for quantified statements, function and relation definitions

{\bf  Definition}: A  {\bf predicate}  is  a function from a given set (domain) to $\{T,F\}$.

A predicate can be applied, or {\bf evaluated} at, an element of the domain.

Usually, a predicate {\it describes a  property} that domain elements may or may not have.

Two predicates over the same domain are {\bf equivalent} means they evaluate to
the same truth values for all possible assignments of domain elements to the
input. In other words, they are equivalent means that they are equal as functions.

To define a predicate, we must specify its domain and its value at each domain element.
The rule assigning truth values to domain elements can be specified using a formula, 
English description, in a table (if the domain is finite), or recursively (if the domain is recursively
defined).
%! app: TODOapp
%! outcome: evidence for quantified statements

\begin{center}
    \begin{tabular}{c||c|c|c}
    Input & \multicolumn{3}{c}{Output} \\
    &$V(x)$ & $N(x)$ & $Mystery(x)$\\
    $x$ & $[x]_{2c,3} > 0$ & $[x]_{2c,3} < 0$& \\
    \hline
    $000$  & $F$ & & $T$\\
    $001$  & $T$ & & $T$\\
    $010$  & $T$ & & $T$\\
    $011$  & $T$ & & $F$\\
    $100$  & $F$ & & $F$\\
    $101$  & $F$ & & $T$\\
    $110$  & $F$ & & $F$\\
    $111$  & $F$ & & $T$\\
    \end{tabular}
    \end{center}
    
    The domain for each of the predicates $V(x)$, $N(x)$, $Mystery(x)$ is
    \underline{\phantom{$\{ b_1b_2b_3 ~\mid~ b_i \in \{0,1\} \textrm{ for each } i, 1 \leq i \leq 3 \}$}}.

    Fill in the table of values for the predicate $N(x)$ based on the formula given.
\vfill
%! app: TODOapp
%! outcome: evidence for quantified statements, quantified statement proofs, function and relation definitions

{\bf Definition}: The {\bf truth  set} of a  predicate is the collection of all elements in its
domain where the predicate evaluates to $T$.

Notice that specifying the domain and the truth set is sufficient for defining
a predicate.
%! app: TODOapp
%! outcome: evidence for quantified statements, quantified statement proofs, function and relation definitions

The truth set for the predicate $V(x)$ is $\underline{\phantom{\{ x ~\mid~ V(x) = T\} = \{ 001, 010, 011 \}}}$.

The truth set for the predicate $N(x)$ is $\underline{\phantom{\{ x ~\mid~ N(x) = T\} = \{ 101, 111 \}}}$.

The truth set for the predicate $Mystery(x)$ is $\underline{\phantom{\{ x ~\mid~ Mystery(x) = T\} = \{ 000, 001, 010, 101, 111 \}}}$.


\vfill
\newpage
\input{../activity-snippets/quantification-definition.tex}
\vfill
\input{../activity-snippets/quantification-logical-equivalence.tex}
\vfill
%! app: TODOapp
%! outcome: TODOoutcome

Examples of quantifications using $V(x), N(x), Mystery(x)$:

{\bf True} or {\bf False}: $\exists x~ (~V(x) \land N(x)~)$

\vfill

{\bf True} or {\bf False}: $\forall x~ (~V(x) \to N(x)~)$

\vfill

{\bf True} or {\bf False}: $\exists x~ (~N(x) \leftrightarrow Mystery(x)~)$

\vfill

Rewrite $\lnot \forall x~(~V(x) \oplus Mystery(x)~)$ into a logical equivalent statement.

\vspace{50pt}


Notice that these are examples where the predicates have {\it finite} domain.
How would we evaluate quantifications where the domain may be infinite?

\vfill


\vfill
\newpage
\input{../activity-snippets/rna-rnalen-basecount-definitions.tex}
%! app: TODOapp
%! outcome: evidence for quantified statements

{\bf Example predicates on $S$, the set of RNA strands (an infinite set)}


$H: S \to \{T, F\}$ where $H(s) = T$ for all $s$.

Truth set of $H$ is \underline{\phantom{$S$\hspace{1in}}}

\vfill

$F_{\A}: S \to \{T, F\}$  defined recursively by: 

~~Basis step: $F_{\A}(\A) = T$, $F_{\A}(\C) = F_{\A}(\G) = F_{\A}(\U) = F$

~~Recursive step: If $s \in S$ and $b \in B$, then $F_{\A}(sb) = F_{\A}(s)$.

Example where $F_{\A}$ evaluates to $T$ is \underline{\phantom{$\A\C\G$~\hspace{1in}}}

\vfill

Example where $F_{\A}$ evaluates to $F$ is \underline{\phantom{$\U\A\C\U$\hspace{1in}}}

\vfill
\newpage
%! app: TODOapp
%! outcome: TODOoutcome

{\bf Using functions to define predicates}:

\fbox{\parbox{\textwidth}{
$L$ with domain $S \times \mathbb{Z}^+$ is defined by, for $s \in S$ and $n \in \mathbb{Z}^+$,
\[
L( ~(s, n)~) = \begin{cases}
T &\qquad\text{if $rnalen(s) = n$}\\
F &\qquad\text{otherwise}\\
\end{cases}
\]
In other words, $L(~(s,n)~)$ means $rnalen(s) = n$
}}

\vfill

\fbox{\parbox{\textwidth}{
$BC$ with domain $S \times B \times \mathbb{N}$ is defined by, 
for $s \in S$ and $b \in B$ and $n \in \mathbb{N}$,
\[
BC(~(s, b, n)~) = \begin{cases}
T &\qquad\text{if $basecount(~(s,b)~) = n$}\\
F &\qquad\text{otherwise}\\
\end{cases}
\]
In other words, $BC(~(s,b,n)~)$ means $basecount(~(s,b)~) = n$
}}


\vfill


Example where $L$ evaluates to $T$: $\underline{\phantom{(\A, 1)\hspace{1in}}}$  Why?

\vfill


Example where $BC$ evaluates to $T$: $\underline{\phantom{(\A, \A1)\hspace{1in}}}$  Why?

\vfill


Example where $L$ evaluates to $F$: $\underline{\phantom{(\A, 2)\hspace{1in}}}$ Why?

\vfill


Example where $BC$ evaluates to $F$: $\underline{\phantom{(\A, \C, 1)\hspace{1in}}}$ Why? 

\vfill


\fbox{\parbox{\textwidth}{
\[\exists t ~BC(t) \qquad \qquad 
\exists (s,b,n) \in S \times B \times \mathbb{N}~ (basecount(~(s,b)~) = n)\]

In English: \phantom{There exists an ordered $3$-tuple 
%of strand, base, and nonnegative integer
at which the predicate $BC$ evaluates to $T$.}

\vspace{30pt}

Witness that proves this existential quantification is true:\phantom{$(\G\G, \G, 2)$ or $(\G\A\U\G, \G, 2)$)}
}}

\fbox{\parbox{\textwidth}{
\[\forall t ~BC(t) \qquad \qquad 
\forall(s,b,n) \in S \times B \times \mathbb{N} ~(basecount(~(s,b)~) = n)\]

In English:\phantom{For all ordered $3$-tuples
%of strand, base, and nonnegative integer
the predicate $BC$ evaluates to $T$.}

\vspace{30pt}

Counterexample that proves this universal quantification is false: \phantom{$(\G\G, \A, 2)$ or $(\G\A\U\G, \G, 3)$)}
}}

\newpage

\section*{Week 4 Friday: Evaluating Nested Quantifiers}
%! app: TODOapp
%! outcome: TODOoutcome

{\bf New predicates from old}
\begin{enumerate}
\item Define the {\bf new} predicate with domain $S \times B$ and rule
\[
basecount(~(s,b)~) = 3
\]
Example domain element where predicate is $T$: \phantom{$(\A\U\A\A, \A)$}\\

\vfill

\item Define the {\bf new} predicate with domain $S \times \mathbb{N}$ and rule
\[
basecount(~(s,\A)~) = n
\]
Example domain element where predicate is $T$: \phantom{$(\A\U\A,2)$}\\

\vfill


\item Define the {\bf new} predicate with domain $S \times B$ and rule
\[
\exists n \in \mathbb{N} ~(basecount(~(s,b)~) = n)
\]
Example domain element where predicate is $T$: \phantom{$(\A\U\A,\A)$}\\

\vfill


\item Define the {\bf new} predicate with domain $S$ and rule
\[
\forall b \in B ~(basecount(~(s,b)~) = 1)
\]
Example domain element where predicate is $T$: \phantom{$\A\C\G\U$}\\

\vfill


\end{enumerate}
\vfill
%! app: TODOapp
%! outcome: data types

{\bf Notation}: for a predicate $P$ with domain $X_1 \times \cdots \times X_n$ and a 
$n$-tuple $(x_1, \ldots, x_n)$ 
with each $x_i \in X$, we 
can write $P(x_1, \ldots, x_n)$ to mean $P( ~(x_1, \ldots, x_n)~)$.

\vfill
%! app: TODOapp
%! outcome: TODOoutcome

{\bf Nested quantifiers}

\fbox{\parbox{\textwidth}{
\[
    \forall s \in S ~\forall b \in B ~\forall n \in \mathbb{N} ~(basecount(~(s,b)~) = n)
\]

In English: \phantom{There exists an ordered $3$-tuple 
%of strand, base, and nonnegative integer
at which the predicate $BC$ evaluates to $T$.}

\vspace{30pt}

Counterexample that proves this universal quantification is false:
\phantom{$(\G\G, \G, 3)$ or $(\G\A\U\G, \G, 2)$)}

\vspace{30pt}

}}

\vfill

\fbox{\parbox{\textwidth}{
\[
    ~\forall n \in \mathbb{N} ~\forall s \in S ~\forall b \in B  ~(basecount(~(s,b)~) = n)
\]

In English: \phantom{There exists an ordered $3$-tuple 
%of strand, base, and nonnegative integer
at which the predicate $BC$ evaluates to $T$.}

\vspace{30pt}

Counterexample that proves this universal quantification is false:
\phantom{$(\G\G, \G, 3)$ or $(\G\A\U\G, \G, 2)$)}

\vspace{30pt}

}}

\vfill
\newpage
%! app: Bioinformatics
%! outcome: TODOoutcome

{\bf Alternating nested quantifiers}

\fbox{\parbox{\textwidth}{
$$\forall s \in S ~\exists b\in B ~(~basecount(~(s,b)~) = 3~)$$

In English: For each RNA strand there is a base that occurs 3 times in this strand.\\

Write the negation and use De Morgan's law to find a 
logically equivalent version where the negation is applied only to the 
$BC$ predicate (not next to a quantifier).

\vspace{60pt}


Is the original statement {\bf True} or {\bf False}?

}}

\fbox{\parbox{\textwidth}{

$$\exists s \in S ~\forall b \in B ~\exists n \in \mathbb{N} ~(~basecount(~(s,b)~) = n~)$$

In English: There is an RNA strand so that for each base there is some nonnegative
integer that counts the number of occurrences of that base in this strand.\\

Write the negation and use De Morgan's law to find a 
logically equivalent version where the negation is applied only to the 
$BC$ predicate (not next to a quantifier).

\vspace{60pt}


Is the original statement {\bf True} or {\bf False}?

}}

\newpage

\subsection*{Review Quiz}
\begin{enumerate}
\item Logical equivalence
 \input{../activity-snippets/quiz-truth-values-conditional.tex}
\item Translating propositional logic
    \begin{enumerate}
    \item %! app: TODOapp
%! outcome: TODOoutcome

Express each of the following sentences as compound propositions, using
the given propositions.

\begin{enumerate}
\item ``If you try to run Zoom while your computer is running many applications,
the video is likely to be choppy and laggy." $t$ is ``you run Zoom while your
computer is running many applications'', $c$ is ``the video is likely to be choppy'',
$g$ is ``the video is likely to be laggy''
\begin{multicols}{2}
    \begin{enumerate}
    \item[] $t \to (c \land g)$
    \item[] $(c \land g) \to t$
    \item[] $(c \land g) \leftrightarrow t$
    \item[] $t \oplus (c \land g)$
\end{enumerate}
\end{multicols}
\item ``To connect wirelessly on campus without logging in you need to use
the UCSD-Guest network."  $c$ is ``connect wirelessly 
on campus'', $g$ is ``logging in'', and $u$ is ``use UCSD-Guest network''.
\begin{multicols}{2}
    \begin{enumerate}
    \item[] $c \land \lnot g \land u$
    \item[] $(c \land \lnot g) \lor u$
    \item[] $(c \land \lnot g) \oplus u$
    \item[] $(c \land \lnot g) \to u$
    \item[] $u \to (c \land \lnot g)$
    \item[] $u \leftrightarrow (c \land \lnot g)$
\end{enumerate}
\end{multicols}
\end{enumerate}

    \newpage
    \item %! app: TODOapp
%! outcome: TODOoutcome

For each of  the following  system specifications, 
identify the compound propositions  that give their
translations to logic  and then determine if the
translated collection  of compound
propositions is consistent.

\begin{enumerate}
    \item Specification: If the computer is out of memory, then network connectivity is unreliable. No disk errors can occur when the computer is out of memory. Disk
    errors only occur when network connectivity is unreliable.
    
    Translation: $M =$ ``the computer is  out of memory"; $N = $ ``network connectivity
    is unreliable"; $D = $  ``disk errors  can occur".

    \begin{multicols}{3}
    \begin{enumerate}
        \item[] \begin{align*} &\neg M \to  N  \\ & \neg D \to M \\ & D \to N \end{align*}
        \item[] \begin{align*} &M \to  \neg N  \\ & \neg D \wedge M \\ & N \to D \end{align*}
        \item[] \begin{align*} &M \to  N  \\ &  M \to \neg D \\ & \neg  N \to \neg D \end{align*}
    \end{enumerate}
    \end{multicols}
    
    \item Specification: Whether you think you can, or you think you can't - you're right.
\footnote{Henry Ford}
    
    Translation: $T =$ ``you  think  you  can"; $C = $  ``you  can".
    
    \begin{multicols}{3}
    \begin{enumerate}
        \item[] \begin{align*} &T \to C \\&  \neg T \to \neg C \end{align*}
        \item[] \begin{align*} &T \wedge C \\  & \neg  T \wedge \neg C \end{align*}
        \item[] \begin{align*} &T \to \neg T  \\ & C  \to \neg  C \end{align*}
    \end{enumerate}
    \end{multicols}

    \item Specification: A secure password must be private and complicated. If
    a password is  complicated then  it will be hard to  remember.  People
    write down hard-to-remember passwords. If a password is written down, it's  not private.   The password is secure.

    Translation: $S =$ ``the password is secure"; $P = $ ``the password is private"; 
    $C = $  ``the password is  complicated"; $H = $ ``the password is hard to remember";
    $W =  $ ``the password is written down".
    
    \begin{multicols}{3}
    \begin{enumerate}
        \item[] \begin{align*} &\neg (P \wedge C) \to \neg  S  \\ & C \to H  \\ & W \wedge H \\ & W \to  \neg P \\ & S \end{align*}
        \item[] \begin{align*} &(P \wedge  C)  \to S  \\ &  C \to H\\ & W  \to  H \\  & W \to P \\ & S\end{align*}
        \item[] \begin{align*} & S  \to (P \wedge C)  \\ &  C \to H\\ & H  \to  W \\  & W \to \neg P \\ & S\end{align*}
    \end{enumerate}
    \end{multicols}
\end{enumerate}
    \end{enumerate}
\newpage
\item Evaluating predicates
    \begin{enumerate}
    \item \hspace{1in}\\ \input{../activity-snippets/quiz-predicates-finite-domain.tex}
    \item \hspace{1in}\\ \input{../activity-snippets/quiz-predicates.tex}
    \item \hspace{1in}\\ \input{../activity-snippets/quiz-predicates-rna.tex}
    \end{enumerate}
\newpage
\item Evaluating nested predicates
    \begin{enumerate}
    \item \hspace{1in}\\\input{../activity-snippets/quiz-predicates-alternating-quantifiers-rnalen.tex}
    \item \hspace{1in}\\\input{../activity-snippets/quiz-predicates-alternating-quantifiers-basecount.tex}
    \end{enumerate}
\end{enumerate}

\newpage

\subsection*{Week 5 at a glance}

\subsubsection*{We will be learning and practicing to:}
%data types
%proof signposts
%using proofs to evaluate
%universal generalization
%applying proof strategy
%logical structure to proof strategy
%identifying proof strategy in proof
\begin{itemize}

\item Clearly and unambiguously communicate computational ideas using appropriate formalism. Translate across levels of abstraction.
\begin{itemize}
   \item Translating between symbolic and English versions of statements using precise mathematical language
    \item Using appropriate signpost words to improve readability of proofs, including 'arbitrary' and 'assume'
\end{itemize}

\item Know, select and apply appropriate computing knowledge and problem-solving techniques. Reason about computation and systems. Use mathematical techniques to solve problems. Determine appropriate conceptual tools to apply to new situations. Know when tools do not apply and try different approaches. Critically analyze and evaluate candidate solutions.
\begin{itemize}
    \item Judging logical equivalence of compound propositions using symbolic manipulation with known equivalences, including DeMorgan's Law
    \item Writing the converse, contrapositive, and inverse of a given conditional statement
    \item Determining what evidence is required to establish that a quantified statement is true or false
    \item Evaluating quantified statements about finite and infinite domains
\end{itemize}

\item Apply proof strategies, including direct proofs and proofs by contradiction, and determine whether a proposed argument is valid or not.
\begin{itemize}
    \item Identifying the proof strategies used in a given proof
    \item Identifying which proof strategies are applicable to prove a given compound proposition based on its logical structure
    \item Carrying out a given proof strategy to prove a given statement
    \item Carrying out a universal generalization argument to prove that a universal statement is true
    \item Using proofs as knowledge discovery tools to decide whether a statement is true or false
\end{itemize}
\end{itemize}

\subsubsection*{TODO:}
\begin{list}
   {\itemsep2pt}
   \item Project due May 7, 2024. No review quiz this week.
   \item Test 1, in class this week, on Friday May 3, 2024.
   The test covers material in Weeks 1 through 4 and Monday of Week 5. 
   To study for the exam, we recommend reviewing class notes 
   (e.g. annotations linked on the class website, podcast, supplementary video from the class website), 
   reviewing homework (and its posted sample solutions), and in particular *working examples* 
   (extra examples in lecture notes, review quizzes, discussion examples) and getting feedback (office hours and Piazza).
   Some practice questions (and their solutions) are available on the class website, linked from Week 5 and from the Assignments page.
\end{list}

\newpage

\begin{comment}
Removed definition of insertion, deletion, mutation from Wednesday of Week 4 -- when do we need them?
%! app: TODOapp
%! outcome: trace algorithms

Real-life representations are often prone to corruption.  Biological codes, like RNA, 
may mutate naturally\footnote{Mutations of specific RNA codons have been linked to many disorders and cancers.}
and during measurement; cosmic radiation and other ambient noise 
can flip bits in computer storage\footnote{This RadioLab podcast episode
goes into more detail on bit flips: \url{https://www.wnycstudios.org/story/bit-flip}}. 
One way to recover from corrupted data is to introduce or exploit redundancy. 

Consider the following algorithm to introduce redundancy in a string of $0$s and $1$s.
\begin{algorithm}[caption={Create redundancy by repeating each bit three times}]
procedure $\textit{redun3}$($a_{k-1} \cdots a_0$: a nonempty bitstring)
for $i$ := $0$ to $k-1$
  $c_{3i}$ := $a_i$
  $c_{3i+1}$ := $a_i$
  $c_{3i+2}$ := $a_i$
return $c_{3k-1} \cdots c_0$
\end{algorithm}

\begin{algorithm}[caption={Decode sequence of bits using majority rule on consecutive three bit sequences}]
procedure $\textit{decode3}$($c_{3k-1} \cdots c_0$: a nonempty bitstring whose length is an integer multiple of $3$)
for $i$ := $0$ to $k-1$
  if exactly two or three of $c_{3i}, c_{3i+1}, c_{3i+2}$ are set to $1$
    $a_i$ := 1
  else 
    $a_i$ := 0
return $a_{k-1} \cdots a_0$
\end{algorithm}

Give a recursive definition of the set of outputs of the $redun3$ procedure, $Out$,

\phantom{{\bf Basis step}: $000 \in Out$ and $111 \in Out$\\ {\bf Recursive step}: If $x \in Out$ then $x000 \in Out$ and $x111 \in Out$ (where $x000$ and $x111$ are the results of string concatenation).}


Consider the message $m = 0001$ so that the sender calculates $redun3(m) = redun3(0001) = 000000000111$.

Introduce $\underline{\phantom{~~4~~}} $
errors into the message so that the signal received by the 
receiver is $\underline{\phantom{010100010101}}$
but the receiver is still able to decode the original message.


{\it Challenge: what is the biggest number of errors you can introduce?} 

Building a circuit for lines 3-6 in $decode$ procedure: given three input bits, we need to determine whether the
majority is a $0$ or a $1$.

\begin{center}
\begin{multicols}{2}\begin{tabular}{ccc|c}
$c_{3i}$ & $c_{3i+1}$ & $c_{3i+2}$ & $a_i$ \\
\hline
$1$ & $1$ & $1$ & $\phantom{1}$ \\
$1$ & $1$ & $0$ & $\phantom{1}$ \\
$1$ & $0$ & $1$ & $\phantom{1}$ \\
$1$ & $0$ & $0$ & $\phantom{0}$ \\
$0$ & $1$ & $1$ & $\phantom{1}$ \\
$0$ & $1$ & $0$ & $\phantom{0}$ \\
$0$ & $0$ & $1$ & $\phantom{0}$ \\
$0$ & $0$ & $0$ & $\phantom{0}$ \\\\
\end{tabular}
\columnbreak

Circuit 
\end{multicols}
\end{center}
\newpage
\input{../activity-snippets/cartesian-product-definition.tex}
%! app: Bioinformatics
%! outcome: trace algorithms

Recall that $S$ is defined as the set of all RNA strands, nonempty strings made of the bases in 
$B = \{\A,\U,\G,\C\}$. 
We define the functions 
\[
  \textit{mutation}: S \times \mathbb{Z}^+ \times B \to S
\qquad \qquad
  \textit{insertion}: S \times \mathbb{Z}^+ \times B \to S
\]
\[
  \textit{deletion}: \{ s\in S \mid rnalen(s) > 1\} \times \mathbb{Z}^+ \to S
  \qquad \qquad \textrm{with rules}
\]

\begin{algorithm}
procedure $\textit{mutation}$($b_1\cdots b_n$: $\textrm{a RNA strand}$, $k$: $\textrm{a  positive integer}$, $b$: $\textrm{an  element of } B$)
for $i$ := $1$ to $n$
  if $i$ = $k$
    $c_i$ := $b$
  else
    $c_i$ := $b_i$
return $c_1\cdots c_n$ $\{ \textrm{The return value is a RNA strand made of the } c_i \textrm{ values}\}$
\end{algorithm}

\begin{algorithm}
procedure $\textit{insertion}$($b_1\cdots b_n$: $\textrm{a RNA strand}$, $k$: $\textrm{a  positive integer}$, $b$: $\textrm{an  element of } B$)
if $k > n$
  for $i$ := $1$ to $n$
    $c_i$ := $b_i$
  $c_{n+1}$ := $b$
else 
  for $i$ := $1$ to $k-1$
    $c_i$ := $b_i$
  $c_k$ := $b$
  for $i$ := $k+1$ to $n+1$
    $c_i$ := $b_{i-1}$
return $c_1\cdots c_{n+1}$ $\{ \textrm{The return value is a RNA strand made of the } c_i \textrm{ values}\}$
\end{algorithm}

\begin{algorithm}
procedure $\textit{deletion}$($b_1\cdots b_n$: $\textrm{a RNA strand with } n>1$, $k$: $\textrm{a  positive integer}$)
if $k > n$
  $m$ := $n$
  for $i$ := $1$ to $n$
    $c_i$ := $b_i$
else
  $m$ := $n-1$
  for $i$ := $1$ to $k-1$ 
    $c_i$ := $b_i$
  for $i$ := $k$ to $n-1$
    $c_i$ := $b_{i+1}$
return $c_1\cdots c_m$ $\{ \textrm{The return value is a RNA strand made of the } c_i \textrm{ values}\}$
\end{algorithm}

\input{../activity-snippets/rna-mutation-insertion-deletion-example.tex}
\end{comment}

\section*{Week 5 Monday: Nested Quantifiers}
\input{../activity-snippets/rna-rnalen-basecount-definitions.tex}
%! app: Bioinformatics
%! outcome: TODOoutcome

{\bf Alternating nested quantifiers}



$$\forall s \in S ~\exists n \in \mathbb{N} ~(~basecount(~(s,\U)~) = n~)$$

In English: For each strand, there is a nonnnegative integer that counts the number of occurrences of $\U$ in that 
strand.\\

$$\exists n \in ~\forall s \in S ~\mathbb{N} ~(~basecount(~(s,\U)~) = n~)$$

In English: There is a nonnnegative integer that counts the number of occurrences of $\U$ in every 
strand.\\

\vfill

Are these statements true or false?

\newpage

$$\forall s \in S ~\exists b\in B ~(~basecount(~(s,b)~) = 3~)$$

In English: For each RNA strand there is a base that occurs 3 times in this strand.\\

Write the negation and use De Morgan's law to find a 
logically equivalent version where the negation is applied only to the 
$BC$ predicate (not next to a quantifier).

\vspace{60pt}


Is the original statement {\bf True} or {\bf False}?

\vfill

\subsection*{Proof strategies}
%! app: TODOapp
%! outcome: TODOoutcome

When a predicate $P(x)$ is over a {\bf finite} domain:
\begin{itemize}
\item To show that $\forall x  P(x)$ is true: check that $P(x)$ evaluates to $T$ at each domain element by evaluating over and over. 
This is called ``Proof of universal by {\bf exhaustion}".
\item To show that $\forall x  P(x)$ is false: find a {\bf counterexample}, a domain element where $P(x)$~evaluates~to~$F$.
\item To show that $\exists x  P(x)$ is true: find a {\bf witness}, a domain element where $P(x)$ evaluates to $T$.
\item To show that $\exists x  P(x)$ is false: check that $P(x)$ evaluates to $F$ at each domain element by evaluating over and over.
DeMorgan's Law gives that $\lnot \exists x P(x) ~~\equiv~~ \forall x \lnot P(x)$ so this amounts to a proof of universal by exhaustion.
\end{itemize}
%%! app: TODOapp
%! outcome: TODOoutcome

\fbox{\parbox{\linewidth}{%
{\bf Proof of universal by exhaustion}: To prove that $\forall x \, P(x)$
is true when $P$ has a finite domain, evaluate the predicate at {\bf each} domain element to confirm that it is always T.
}}
\input{../activity-snippets/proof-strategy-universal-generalization.tex}
\newpage
%! app: TODOapp
%! outcome: TODOoutcome

Suppose $P(x)$ is  a predicate over a domain $D$.
\begin{enumerate}
    \item True or False: To translate the statement
    ``There are at least two  elements in $D$
    where the predicate $P$ evaluates to true", we
    could  write
    \[
    \exists  x_1 \in D \, \exists x_2 \in D  \, (P(x_1) \wedge P(x_2))
    \]
    \vfill
    \item True or False: To translate the statement
    ``There are at most two  elements in $D$
    where the predicate $P$ evaluates to true", we
    could write
    \[
    \forall  x_1 \in D \, \forall x_2 \in D \, \forall x_3 \in  D \, \left(~ (~P(x_1) \wedge P(x_2)  \wedge P(x_3) ~) \to (~ x_1 = x_2 \vee x_2 = x_3 \vee x_1 = x_3~)~\right)
    \]
    \vfill
\end{enumerate}

\newpage
\section*{Week 5 Wednesday: Proof Strategies and Sets}
%! app: TODOapp
%! outcome: data types

{\bf Definitions}:

A {\bf set} is an  unordered collection of  elements.
When $A$ and  $B$ are sets,  $A = B$ (set equality) means  
\[
    \forall x  ( x\in A \leftrightarrow x \in B)
\]

When $A$ and  $B$ are sets, $A \subseteq B$ (``$A$ is a {\bf subset} of $B$") means 
\[
    \forall x  (x \in A  \to x  \in B)
\]

When $A$ and  $B$ are sets,  $A \subsetneq B$ (``$A$ is a {\bf proper subset} of $B$") means 
\[
    (A\subseteq B) \wedge  (A \neq B)
\]
\input{../activity-snippets/proof-strategies-conditionals.tex}
\input{../activity-snippets/proof-strategies-proof-by-cases.tex}
\input{../activity-snippets/proof-strategies-ands.tex}
\input{../activity-snippets/sets-proof-strategies.tex}
\newpage
\input{../activity-snippets/sets-equality-example.tex}
\input{../activity-snippets/sets-basic-proofs.tex}
\vfill
\input{../activity-snippets/proofs-signposting.tex}
\newpage
\input{../activity-snippets/set-operations-union-intersection-powerset.tex}

\newpage

\subsection*{Week 6 at a glance}

\subsubsection*{We will be learning and practicing to:}
%data types
%proof signposts
%using proofs to evaluate
%universal generalization
%applying proof strategy
%logical structure to proof strategy
%identifying proof strategy in proof
\begin{itemize}

\item Clearly and unambiguously communicate computational ideas using appropriate formalism. Translate across levels of abstraction.
\begin{itemize}
   \item Translating between symbolic and English versions of statements using precise mathematical language
    \item Using appropriate signpost words to improve readability of proofs, including 'arbitrary' and 'assume'
\end{itemize}

\item Know, select and apply appropriate computing knowledge and problem-solving techniques. Reason about computation and systems. Use mathematical techniques to solve problems. Determine appropriate conceptual tools to apply to new situations. Know when tools do not apply and try different approaches. Critically analyze and evaluate candidate solutions.
\begin{itemize}
    \item Judging logical equivalence of compound propositions using symbolic manipulation with known equivalences, including DeMorgan's Law
    \item Writing the converse, contrapositive, and inverse of a given conditional statement
    \item Determining what evidence is required to establish that a quantified statement is true or false
    \item Evaluating quantified statements about finite and infinite domains
\end{itemize}

\item Apply proof strategies, including direct proofs and proofs by contradiction, and determine whether a proposed argument is valid or not.
\begin{itemize}
    \item Identifying the proof strategies used in a given proof
    \item Identifying which proof strategies are applicable to prove a given compound proposition based on its logical structure
    \item Carrying out a given proof strategy to prove a given statement
    \item Carrying out a universal generalization argument to prove that a universal statement is true
    \item Using proofs as knowledge discovery tools to decide whether a statement is true or false
\end{itemize}
\end{itemize}

\subsubsection*{TODO:}
\begin{list}
   {\itemsep2pt}
   \item Project due this week: May 8, 2024. 
   \item Review quiz based on class material each day (due Friday May 10, 2024).
\end{list}

\newpage

\section*{Week 6 Monday: Proofs for properties of sets and numbers}
%% Review sets and proof strategies by starting with this example:
\input{../activity-snippets/sets-proof-strategies.tex}
%! app: TODOapp
%! outcome: TODOoutcome

Let $W =  \mathcal{P}(  \{ 1,2,3,4,5\} )$

Example elements in $W$ are:
\vspace{20pt}


{\bf Prove} or {\bf  disprove}:  $\forall  A \in W\,  \forall B \in W\,  \left( A \subseteq B
~\to ~ \mathcal{P}(A) \subseteq \mathcal{P}(B) \right)$

\vfill
\vfill
\vfill

{\it Extra example}: {\bf Prove} or {\bf  disprove}:  $\forall  A \in W\,  \forall B \in W\,  \left( \mathcal{P}(A)  =\mathcal{P}(B)
~\to ~ A = B \right)$

\vspace{20pt}

{\it Extra example}: {\bf Prove} or {\bf  disprove}:  $\forall  A \in W\,  \forall B \in W\, \forall C  \in W\,  \left( A\cup B  = A \cup  C
~\to ~ B = C \right)$

\vspace{20pt}


\subsection*{Facts about numbers}
\input{../activity-snippets/proof-strategies-road-map.tex}
\input{../activity-snippets/numbers-facts.tex}
\newpage
\subsection*{Factoring}
%! app: TODOapp
%! outcome: TODOoutcome

{\bf Definition}: When $a$ and $b$ are integers and $a$ is nonzero, 
{\bf $a$ divides $b$} means there is an integer $c$ such that $b = ac$ . 


Symbolically, $F(~(a,b)~) = \phantom{\exists c\in \mathbb{Z}~(b=ac)}$
and is  a predicate over the domain \underline{\phantom{$\mathbb{Z}^{\neq 0} \times \mathbb{Z}$}}


Other (synonymous) ways to say that $F(~(a,b)~)$ is true: 
\begin{center}
$a$ is a {\bf factor} of $b$
\qquad 
$a$ is a {\bf divisor} of $b$
\qquad  $b$ is a {\bf multiple} of $a$
\qquad
$a | b$
\end{center}

When $a$ is a positive integer and $b$ is any integer, $a | b$
exactly when $b \textbf{ mod } a = 0$

When $a$ is a positive integer and $b$ is any integer, $a | b$
exactly when $b = a \cdot (b \textbf{ div } a)$
\input{../activity-snippets/factoring-translation-examples.tex}
\input{../activity-snippets/factoring-basic-claims.tex}
\input{../activity-snippets/factoring-basic-claims-continued.tex}
\input{../activity-snippets/factoring-even-odd.tex}
\input{../activity-snippets/prime-number-definition.tex}
\input{../activity-snippets/primes-basic-claims.tex}
\newpage


\section*{Week 6 Wednesday: Structural Induction}
\input{../activity-snippets/rna-rnalen-basecount-definitions.tex}

At this point, we've seen the proof strategies
\begin{multicols}{2}
    \begin{itemize}
        \item A {\bf counterexample} to prove that  $\forall x P(x)$ is {\bf false}.
        \item  A {\bf witness} to prove that  $\exists x P(x)$ is {\bf true}.
        \item {\bf Proof of universal by exhaustion} to prove that $\forall x \, P(x)$
    is true when $P$ has a finite domain
        \item  {\bf Proof by universal generalization} to prove that $\forall x \, P(x)$
    is true using an arbitrary element of the domain.
        \item To  prove  that $\exists x P(x)$ is {\bf false}, write the universal statement that is 
        logically equivalent to its negation and then prove it true using universal generalization.
        \item To prove that $p \land q$ is true, have two subgoals: 
        subgoal (1) prove $p$ is  true; and, subgoal (2) prove $q$ is true. To prove that $p \land q$ is false, it's enough to prove that $p$ is false.
     To prove that $p \land q$ is false, it's enough to prove that $q$ is false.
        \item Proof of conditional by {\bf direct proof}
        \item Proof of conditional by {\bf contrapositive proof}
        \item Proof of disjuction using equivalent conditional: To prove that the 
        disjunction $p \lor q$ is true, we can rewrite it equivalently as $\lnot p \to q$ and
        then use direct proof or contrapositive proof.
        \item {\bf Proof by cases}.
    \end{itemize}
\end{multicols}
\newpage
%! app: Bioinformatics
%! outcome: TODOoutcome

Which proof strategies could be used to prove each of the following statements?

{\it Hint: first translate the statements to English and identify the main logical structure.}

$\forall s \in S~(~rnalen(s) > 0~)$

\vspace{100pt}

$\forall b \in B~\exists s \in S~(~basecount(~(s,b)~)~ > 0~)$

\vspace{100pt}

$\forall s \in S ~\exists b\in B ~(~basecount(~(s,b)~) > 0~)$

\vspace{100pt}

$\exists s \in S \, (\textit{rnalen(s)} = \textit{basecount}(~(s, \A)~)$

\vspace{100pt}

$\forall s \in S \, (\textit{rnalen(s)} \geq \textit{basecount}(~(s, \A)~))$

\vspace{100pt}


\newpage
\input{../activity-snippets/structural-induction-motivating-example-rna.tex}
\input{../activity-snippets/proof-strategies-structural-induction.tex}
\newpage
\input{../activity-snippets/structural-induction-example-rnalen-basecount.tex}
\newpage

\section*{Week 6 Friday: Structural and Mathematical Induction}
\input{../activity-snippets/proofs-signposting-kinds-of-claims.tex}
%! app: TODOapp
%! outcome: induction flavors

\begin{center}
    \includegraphics[width=3in]{../../resources/images/robot-grid.png}
\end{center}
    
{\bf Theorem}: A robot on an infinite 2-dimensional integer grid starts at $(0,0)$ and at each step moves
to diagonally adjacent grid point. This robot can / cannot {\footnotesize({\it circle one})} reach $(1,0)$.


{\bf Definition} The set of positions the robot can visit  $Pos$ is defined by:
\[
\begin{array}{ll}
    \textrm{Basis Step: } & (0,0) \in Pos \\
    \textrm{Recursive Step: } & \textrm{If } (x,y) \in Pos \textrm{, then } \\
    &\phantom{(x+1, y+1), (x+1, y-1), (x-1, y-1), (x-1, y+1)} \textrm{ are also in } Pos
\end{array}
\]

{\it Example elements of $Pos$ are}:
\vspace{20pt}

{\bf Lemma}: $\forall (x,y) \in Pos~~( x+y \textrm{ is an even integer}~)$

{\it Why are we calling this a lemma?}


Proof of theorem using lemma: To show is $(1,0) \notin Pos$. Rewriting the lemma to explicitly 
restrict the domain of the universal, 
we have $\forall (x,y) ~(~ (x,y) \in Pos~~  \to ~~(x+y \textrm{ is an even integer})~)$.  Since
the universal is true, 
$ (~ (1,0) \in Pos~~ \to ~~(1+0 \textrm{ is an even integer})~)$ is a true statement.
Evaluating the conclusion of this conditional statement: 
By definition of long division, since $1 = 0 \cdot 2 + 1$ (where $0 \in \mathbb{Z}$ and 
$1 \in \mathbb{Z}$ and $0 \leq 1 < 2$ mean that $0$ is the quotient and $1$ is the remainder), $1 ~\textrm{\bf mod}~ 2 = 1$ which is not $0$ 
so the conclusion is false.  A true conditional with a false conclusion must have a false hypothesis: $(1,0) \notin Pos$, QED. $\square$

\vspace{20pt}

Proof of lemma by structural induction:

{\bf Basis Step}:

\vspace{100pt}


{\bf Recursive Step}:  Consider arbitrary $(x,y) \in Pos$.  To show is:
\[
(x+y \text{ is an even integer}) \to (\text{sum of coordinates of next position is even integer})
\]
Assume {\bf as the induction hypothesis, IH} that: 


\vspace{400pt}
\newpage
\input{../activity-snippets/structural-induction-example-sum-of-powers.tex}
\vfill
\input{../activity-snippets/proof-strategy-mathematical-induction.tex}
\newpage


\subsection*{Review Quiz}
\begin{enumerate}
\item Set properties
\begin{enumerate}
    \item \hspace{1in}\\ \input{../activity-snippets/quiz-sets-claims.tex}
    \item \hspace{1in}\\ \input{../activity-snippets/quiz-sets-proof-strategies.tex}
    \item \hspace{1in}\\ \input{../activity-snippets/quiz-sets-claims-subset-equality.tex}
\end{enumerate}
\item Number properties\begin{enumerate}
    \item \hspace{1in}\\ \input{../activity-snippets/quiz-factoring-quantifiers.tex}
    \item \hspace{1in}\\ \input{../activity-snippets/quiz-prime-formalizing-definition.tex}
\end{enumerate}
\item Structural induction
\begin{enumerate}
    \item \input{../activity-snippets/quiz-basecount-rnalen-induction.tex}
    \item \hspace{1in}\\ %! app: TODOapp
%! outcome: TODOoutcome

Recall the set $Pos$ defined by the recursive definition
\[
\begin{array}{ll}
    \textrm{Basis Step: } & (0,0) \in Pos\\
     \textrm{Recursive Step: } & \textrm{If } (x,y) \in Pos \textrm{ then } 
     (x+1, y+1) \in Pos \textrm{ and } (x+1, y-1) \in Pos \textrm{ and }\\ 
     & (x-1,y-1) \in Pos 
     \textrm{ and } (x-1, y+1) \in Pos
\end{array}
\]
\begin{enumerate}
\item Select all and only the ordered pairs below that are elements of $Pos$
\begin{enumerate}
\item $(0,0)$
\item $(4,0)$
\item $(1,1)$
\item $(1.5,2.5)$
\item $(0, -2)$
\end{enumerate}
\item What is another description of the set $Pos$ ? (Select all and only the true descriptions.)
\begin{enumerate}
\item $\mathbb{Z} \times \mathbb{Z}$
\item $\{ (n,n) ~|~ n \in \mathbb{Z} \}$
\item $\{ (a,b) \in \mathbb{Z} \times \mathbb{Z} ~|~ (a+b) \textbf{ mod } 2 =0 \}$
\end{enumerate}
\end{enumerate}
\end{enumerate}
\item Mathematical induction
\begin{enumerate}
    \item \hspace{1in}\\ \input{../activity-snippets/quiz-comparing-structural-mathematical-induction.tex}
    \item \hspace{1in}\\ \input{../activity-snippets/quiz-exponential-factorial.tex}
\end{enumerate}
\item Midquarter feedback
%% Put midquarter feedback in Review quiz for Week 6%% TODO
\end{enumerate}


\newpage

\subsection*{Week 7 at a glance}

\subsubsection*{We will be learning and practicing to:}
%data types
%proof signposts
%using proofs to evaluate
%universal generalization
%applying proof strategy
%logical structure to proof strategy
%identifying proof strategy in proof
\begin{itemize}

\item Clearly and unambiguously communicate computational ideas using appropriate formalism. Translate across levels of abstraction.
\begin{itemize}
   \item Translating between symbolic and English versions of statements using precise mathematical language
    \item Using appropriate signpost words to improve readability of proofs, including 'arbitrary' and 'assume'
\end{itemize}

\item Know, select and apply appropriate computing knowledge and problem-solving techniques. Reason about computation and systems. Use mathematical techniques to solve problems. Determine appropriate conceptual tools to apply to new situations. Know when tools do not apply and try different approaches. Critically analyze and evaluate candidate solutions.
\begin{itemize}
    \item Judging logical equivalence of compound propositions using symbolic manipulation with known equivalences, including DeMorgan's Law
    \item Writing the converse, contrapositive, and inverse of a given conditional statement
    \item Determining what evidence is required to establish that a quantified statement is true or false
    \item Evaluating quantified statements about finite and infinite domains
\end{itemize}

\item Apply proof strategies, including direct proofs and proofs by contradiction, and determine whether a proposed argument is valid or not.
\begin{itemize}
    \item Identifying the proof strategies used in a given proof
    \item Identifying which proof strategies are applicable to prove a given compound proposition based on its logical structure
    \item Carrying out a given proof strategy to prove a given statement
    \item Carrying out a universal generalization argument to prove that a universal statement is true
    \item Using proofs as knowledge discovery tools to decide whether a statement is true or false
\end{itemize}
\end{itemize}

\subsubsection*{TODO:}
\begin{list}
   {\itemsep2pt}
   \item Homework assignment 4 (due Tuesday May 14, 2024)
   \item Review quiz based on class material each day (due Friday May 17, 2024).
   \item Homework assignment 5 (due Tuesday May 21, 2024)
\end{list}

\newpage

\section*{Week 7 Monday: Mathematical and Strong Induction}
\subsection*{Visualizing induction}
\input{../activity-snippets/induction-dominos.tex}
\input{../activity-snippets/proof-strategy-mathematical-induction.tex}
\input{../activity-snippets/proof-strategy-strong-induction.tex}
\input{../activity-snippets/binary-expansions-exist-proof.tex}

\subsubsection*{Representing positive integers with primes}
\input{../activity-snippets/fundamental-theorem-proof.tex}
\subsubsection*{Sending old-fashioned mail with postage stamps}
%! app: TODOapp
%! outcome: induction flavors, strong induction proofs, mathematical induction proofs

Suppose we had postage stamps worth $5$ cents and $3$ cents.
Which number of cents can we form using these stamps?
In other words, which postage can we pay?

$11$? 

$15$? 


$4$?



\begin{align*}
    &CanPay(0) \land \lnot CanPay(1) \land \lnot CanPay(2) \land \\
    &CanPay(3) \land \lnot CanPay(4) \land CanPay(5) \land CanPay(6) \\
    &\lnot CanPay(7) \land \forall n \in \mathbb{Z}^{\geq 8} CanPay(n)
\end{align*}

where the predicate $CanPay$ with domain $\mathbb{N}$ is
\[
    CanPay(n) = \exists x \in \mathbb{N} \exists y \in \mathbb{N}  ( 5x+3y = n)
\]


{\bf Proof} (idea): First, explicitly give witnesses or general arguments
for postages between $0$ and $7$. 
To prove the universal claim, we can use mathematical induction or strong induction.

{\it Approach 1, mathematical induction}: if we have
stamps that add up to $n$ cents, need to use them (and others)
to give $n+1$ cents. How do we get $1$ cent with just $3$-cent
and $5$-cent stamps?

\vspace{-10pt}
Either \underline{take away a $5$-cent stamps and add two $3$-cent stamps},

\vspace{-10pt}
or \underline{take away three $3$-cent stamps and add two $5$-cent stamps}.

\vspace{-10pt}
The details of this proof by mathematical induction
are making sure we have enough 
stamps to use one of these approaches.

{\it Approach 2, strong induction}: assuming we know how to make postage
for {\bf all} smaller values (greater than or equal to $8$), when
we need to make $n+1$ cents, \underline{add one $3$ cent stamp to 
however we make $(n+1) - 3$ cents}.

\vspace{-10pt}
The details of this proof by strong induction are making sure we 
stay in the domain of the universal when applying the induction hypothesis.

\newpage
\subsubsection*{Finding a winning strategy for a game}
%! app: TODOapp
%! outcome: induction flavors, strong induction proofs

Consider the following game: two players start with 
two (equal) piles of jellybeans in front of them.
They take turns removing any positive integer number
of jellybeans at a time from one of two piles in 
front of them in turns.

The player who removes the last jellybean wins the game.

Which player (if any) has a strategy to guarantee
to win the game?


Try out some games, starting with $1$ jellybean in each pile,
then $2$ jellybeans in each pile, then $3$ jellybeans in each pile.
Who wins in each game?

\vspace{200pt}


Notice that reasoning about the strategy for the $1$ jellybean 
game is easier than about the strategy for the $2$ jellybean game.

{\it Formulate a winning strategy by working to 
transform the game to a simpler one we know we can win.}

\newpage

{\it Player 2's Strategy}: Take the same number of jellybeans that Player 1 did, 
but from the opposite pile. 


{\it Why is this a good idea}: If Player 2 plays this strategy, at the next turn
Player 1 faces a game with the same setup as the original, just with fewer
jellybeans in the two piles. Then Player 2 can keep playing this strategy to win.

{\bf Claim}: Player 2's strategy guarantees they will win the game.

{\bf Proof}: By strong induction, we will prove that for all positive 
integers $n$, Player 2's strategy guarantees a win in the game that starts with 
$n$ jellybeans in each pile.

{\bf Basis step}: WTS Player 2's strategy guarantees a win 
when each pile starts with $1$ jellybean.

In this case, Player 1 has to take the jellybean from one of the piles
(because they can't take from both piles at once).
Following the strategy, Player 2 takes the jellybean from the 
other pile, and wins because this is the last jellybean.

{\bf Recursive step}: Let $n$ be a positive integer. 
As the strong induction hypothesis, assume that
Player 2's strategy guarantees a win in the games 
where there are $1, 2, \ldots, n$ many jellybeans in each 
pile at the start of the game.

WTS that Player 2's strategy guarantees a win in the game where
there are $n+1$ in the jellybeans in each pile at the start of the game.

In this game, the first move has Player 1 take 
some number, call it $c$ (where $1 \leq c \leq n+1$),
of jellybeans from one of the piles. 
Playing according to their strategy, Player 2 then 
takes the same number of jellybeans from  the other pile.

Notice that $(c = n+1) \lor (c \leq n)$.

{\it Case 1}: Assume $c = n+1$, then in their first move, 
Player 2 wins because they take all of the second pile, which 
includes the last jellybean.

{\it Case 2}: Assume $c \leq n$. Then after Player 2's first move,
the two piles have an equal number of jellybeans. The number
of jellybeans in each pile is 
\[
    (n+1) - c
\]
and, since $1 \leq c \leq n$, this number is between $1$ and $n$.
Thus, at this stage of the game, the game appears identical to a new 
game where the two piles have an equal number of jellybeans between $1$
and $n$. Thus, the strong induction hypothesis applies, and Player 2's
strategy guarantees they win.


\newpage

\section*{Week 7 Wednesday: Recursive Data Structures}
\input{../activity-snippets/linked-lists-definition.tex}
\input{../activity-snippets/linked-lists-examples.tex}
\input{../activity-snippets/linked-list-length-definition.tex}
\vspace{50pt}
\input{../activity-snippets/linked-lists-prepend-definition.tex}
\vspace{50pt}
\input{../activity-snippets/linked-list-append-definition.tex}
\vspace{50pt}
\newpage
\input{../activity-snippets/linked-list-append-length-claim-proof.tex}
\newpage
\input{../activity-snippets/linked-list-example-each-length.tex}
\newpage

\section*{Week 7 Friday: Proof by Contradiction}
\input{../activity-snippets/proof-strategy-proof-by-contradiction.tex}
\subsection*{Least and greatest}
\input{../activity-snippets/least-greatest-proofs.tex}

\input{../activity-snippets/gcd-definition.tex}
\input{../activity-snippets/gcd-examples.tex}
\input{../activity-snippets/gcd-basic-claims.tex}
\input{../activity-snippets/gcd-lemma-relatively-prime.tex}

\newpage
\subsection*{Sets of numbers}

We've seen multiple representations of the set of positive integers
(using base expansions and using prime factorization). Now we're 
going to expand our attention to other sets of numbers as well.
\input{../activity-snippets/rational-numbers-definition.tex}
%! app: Numbers
%! outcome: data types, important sets

We have the following subset relationships between sets of numbers:

\[
    \mathbb{Z}^{+} \subsetneq \mathbb{N} \subsetneq \mathbb{Z} \subsetneq \mathbb{Q} \subsetneq \mathbb{R}
\]


Which of the proper subset inclusions above can you prove?

\vspace{50pt}
\input{../activity-snippets/proof-by-contradiction-irrational.tex}


\newpage

\subsection*{Review Quiz}
\begin{enumerate}
    \item Mathematical and strong induction for properties of numbers
    \begin{enumerate}
        \item \hspace{1in} \\ \input{../activity-snippets/quiz-binary-expansions-exist-invalid-proof.tex}
        \item \hspace{1in} \\ \input{../activity-snippets/quiz-making-change-proof-two-ways.tex}
    \end{enumerate}
    \item Winning strategy. \input{../activity-snippets/quiz-nim.tex}
    \item Linked lists. %! app: TODOapp
%! outcome: TODOoutcome

Recall the definition of linked lists from class.

Consider this (incomplete) definition:

{\bf Definition} The function $\textit{increment} : \underline{\hspace{6em}}$ 
that adds 1 to the data in each node of a linked list is defined by:
\[
\begin{array}{llll}
& & \textit{increment} : \underline{\hspace{3em}} & \to \underline{\hspace{3em}} \\
\textrm{Basis Step:} & & \textit{increment}([]) & = [] \\
\textrm{Recursive Step:} & \textrm{If } l \in L, n \in \mathbb{N} & \textit{increment}((n, l)) & = (1 + n, \textit{increment}(l))
\end{array}
\]

Consider this (incomplete) definition:

{\bf Definition} The function $\textit{sum} : L \to \mathbb{N}$ that adds 
together all the data in nodes of the list is defined by:
\[
\begin{array}{llll}
& & \textit{sum} : L & \to \mathbb{N} \\
\textrm{Basis Step:} & & \textit{sum}([]) & = 0 \\
\textrm{Recursive Step:} & \textrm{If } l \in L, n \in \mathbb{N} & \textit{sum}((n, l)) & = \underline{\hspace{8em}}
\end{array}
\]

You will compute a sample function application and then fill in the 
blanks for the domain and codomain of each of these functions.

\begin{enumerate}
    \item Based on the definition, what is the result of $\textit{increment}((4, (2, (7, []))))$? Write your answer directly with no spaces.
    
    \item Which of the following describes the domain and codomain of \textit{increment}?
    
    \begin{multicols}{2}
    \begin{enumerate}
        \item The domain is $L$ and the codomain is $\mathbb{N}$
        \item The domain is $L$ and the codomain is $\mathbb{N} \times L$
        \item The domain is $L \times \mathbb{N}$ and the codomain is $L$
        \item The domain is $L \times \mathbb{N}$ and the codomain is $\mathbb{N}$
        \item The domain is $L$ and the codomain is $L$
        \item None of the above
    \end{enumerate}
    \end{multicols}
    
    \item Assuming we would like $sum((5, (6, [])))$ to evaluate to $11$ and $sum((3, (1, (8, []))))$ to evaluate to $12$, which of the following could be used to fill in the definition of the recursive case of \textit{sum}?
    
     \begin{multicols}{2}
    \begin{enumerate}
        \item $\begin{cases}
            1 + \textit{sum}(l) & \textrm{when } n \neq 0 \\
            \textit{sum}(l) & \textrm{when } n = 0 \\
        \end{cases}$
        \item $1 + \textit{sum}(l)$
        \item $n + \textit{increment}(l)$
        \item $n + \textit{sum}(l)$
        \item None of the above
    \end{enumerate}
    \end{multicols}
    
%    \newpage
    \item Choose only and all of the following statements that are \textbf{well-defined}; that is, they correctly reflect the domains and codomains of the functions and quantifiers, and respect the notational conventions we use in this class. Note that a well-defined statement may be true or false.

    \begin{multicols}{2}    
    \begin{enumerate}
        \item $\forall l \in L \, (\textit{sum}(l))$
        \item $\exists l \in L \, (\textit{sum}(l) \land \textit{length}(l))$
        \item $\forall l \in L \, (\textit{sum}(\textit{increment}(l)) = 10)$
        \item $\exists l \in L \, (\textit{sum}(\textit{increment}(l)) = 10)$
        \item $\forall l \in L \, \forall n \in \mathbb{N} \, ((n \times l) \subseteq L)$
        \item $\forall l_1 \in L \, \exists l_2 \in L \, (\textit{increment}(\textit{sum}(l_1)) = l_2)$
        \item $\forall l \in L \, (\textit{length}(\textit{increment}(l)) = \textit{length}(l))$
    \end{enumerate}
    \end{multicols}
    
    \item Choose only and all of the statements in the previous part that are both well-defined and true.
\end{enumerate}
    \item Primes and divisors
    \begin{enumerate}
        \item \hspace{1in}\\ \input{../activity-snippets/quiz-prime-factorization.tex}
        \item \hspace{1in}\\ \input{../activity-snippets/quiz-no-greatest-prime.tex}
        \item \hspace{1in}\\ \input{../activity-snippets/quiz-calculating-gcd.tex}
    \end{enumerate}
    \item Proof strategies
    \begin{enumerate}
        \item \hspace{1in}\\ \input{../activity-snippets/quiz-choosing-proof-strategy.tex}
        \item \hspace{1in}\\ \input{../activity-snippets/quiz-odd-even-proofs.tex}
    \end{enumerate}
\end{enumerate}

\newpage

\subsection*{Week 8 at a glance}

\subsubsection*{We will be learning and practicing to:}
%classify cardinality
%important sets
%function and relation definitions
%functions for cardinality
%contradiction proofs
\begin{itemize}

\item Clearly and unambiguously communicate computational ideas using appropriate formalism. Translate across levels of abstraction.
\begin{itemize}
    \item Defining important sets of numbers, e.g. set of integers, set of rational numbers
    \item Defining functions using multiple representations
    \item Classifying sets into: finite sets, countably infinite sets, uncountable sets
    \item Using functions to compare cardinality of sets
\end{itemize}

\item Know, select and apply appropriate computing knowledge and problem-solving techniques. Reason about computation and systems. Use mathematical techniques to solve problems. Determine appropriate conceptual tools to apply to new situations. Know when tools do not apply and try different approaches. Critically analyze and evaluate candidate solutions.
\begin{itemize}
    \item Determining what evidence is required to establish that a quantified statement is true or false
    \item Evaluating quantified statements about finite and infinite domains
\end{itemize}

\item Apply proof strategies, including direct proofs and proofs by contradiction, and determine whether a proposed argument is valid or not.
\begin{itemize}
    \item Tracing and/or modifying a proof by contradiction
    \item Using proofs as knowledge discovery tools to decide whether a statement is true or false
\end{itemize}
\end{itemize}

\subsubsection*{TODO:}
\begin{list}
   {\itemsep2pt}
   \item Homework assignment 5 (due Tuesday May 21, 2024)
   \item Review quiz based on class material each day (due Friday May 24, 2024).
   \item Start reviewing for Test 2. The test is in class next week on Friday May 31, 2024.

\end{list}

\newpage

\section*{Week 8 Monday: Cardinality of Sets}

%%! app: Numbers
%! outcome: data types, important sets

We have the following subset relationships between sets of numbers:

\[
    \mathbb{Z}^{+} \subsetneq \mathbb{N} \subsetneq \mathbb{Z} \subsetneq \mathbb{Q} \subsetneq \mathbb{R}
\]


Which of the proper subset inclusions above can you prove?

\vspace{50pt}
%! app: TODOapp
%! outcome: classify cardinality

{\bf Definition}: A {\bf finite} set is one whose distinct elements can be counted by a natural number.

%! app: TODOapp
%! outcome: classify cardinality

{\bf Motivating question}: when can we say one set is {\it bigger than} another?

Which is bigger? 
\begin{itemize}
    \item The set $\{1,2,3\}$ or the set $\{0,1,2,3\}$?
    \item The set $\{0, \pi, \sqrt{2} \}$ or the set $\{\mathbb{N}, \mathbb{R}, \emptyset\}$?
    \item The set $\mathbb{N}$ or the set $\mathbb{R}^+$?
\end{itemize}

{\it Which of the sets above are finite? infinite?}
%! app: TODOapp
%! outcome: classify cardinality, functions for cardinality

{\bf Key idea for cardinality}: Counting 
distinct elements is a way of labelling elements
with natural numbers. This is a function!
In general, functions let us 
associate elements of one set with those
of another. If the association is ``{\it good}", 
we get a correspondence between the elements of the subsets
which can relate the sizes of the sets.
%! app: TODOapp
%! outcome: functions for cardinality

{\it Analogy}: Musical chairs

\begin{multicols}{2}
\includegraphics[width=1.8in]{../../resources/images/musicalchairs.png}
\columnbreak

People try to sit down when the music stops

Person\sun~ sits in Chair 1,
Person\smiley~ sits in Chair 2,

Person\frownie~  is left standing!
\end{multicols}
What does this say about the number of chairs and the number of people?

\vspace{100pt}
\newpage
%! app: TODOapp
%! outcome: function and relation definitions

Recall that a function is defined by its (1) domain, (2) codomain, and (3) rule assigning each 
element in the domain exactly one element in the codomain. 
The domain and codomain are nonempty sets.
The rule can be depicted as a table, formula, English description, etc.

A function can {\it fail to be well-defined} if there is some 
domain element where the function rule doesn't give a
unique codomain element. For example, the function rule might lead to 
more than one potential image, or to an image outside of the codomain.


{\it Example}: $f_A: \mathbb{R}^+ \to \mathbb{Q}$ with $f_A(x) = x$ is {\bf not} a well-defined function because

\vspace{100pt}


{\it Example}: $f_B: \mathbb{Q} \to \mathbb{Z}$ with $f_B\left(\frac{p}{q}\right) = p+q$ is {\bf not} a well-defined function because

\vspace{100pt}


{\it Example}: $f_C: \mathbb{Z} \to \mathbb{R}$ with $f_C(x) = \frac{x}{|x|}$ is {\bf not} a well-defined function because

\vspace{100pt}

\newpage
%! app: TODOapp
%! outcome: function and relation definitions

{\bf Definition} : A function $f: D  \to C$ is {\bf one-to-one} (or  injective) 
means for every $a,b$ in the domain $D$, 
if $f(a) = f(b)$ then  $a=b$.

Formally, $f: D  \to  C$ is  one-to-one  means $\underline{\phantom{\forall a \in D \forall b \in D ~(f(a) = f(b) \to a = b)}}$.

%! app: TODOapp
%! outcome: functions for cardinality, function and relation definitions

Informally, a function being one-to-one means ``no duplicate images''.

\phantom{Draw finite domain, finite codomain picture with duplicate image.}
\vspace{50pt}
%! app: TODOapp
%! outcome: functions for cardinality

{\bf Definition}:  For nonempty sets $A, B$, we say that {\bf the  cardinality of $A$ is  no  bigger than the cardinality of $B$}, 
and write $|A| \leq |B|$, to mean there is a  one-to-one function  with domain $A$  and codomain $B$.
Also, we define $|\emptyset| \leq |B|$ for all sets $B$.
%! app: TODOapp
%! outcome: functions for cardinality

{\it In the analogy}: The function $sitter: \{ Chair1, Chair2\} \to \{ Person\sun, Person\smiley, Person\frownie \}$ given
by $sitter(Chair1) = Person\sun$,  $sitter(Chair2) = Person\smiley$, is one-to-one and witnesses that 
\[
| \{ Chair1, Chair2\} | \leq |\{ Person\sun, Person\smiley, Person\frownie \}|
\]
%\input{../activity-snippets/rna-injective-cardinality.tex}
\newpage
%! app: TODOapp
%! outcome: function and relation definitions

{\bf Definition}: A function $f: D  \to C$ is {\bf onto} (or  surjective) means for every $b$ in the codomain, 
there  is an element $a$ in the domain with  $f(a) = b$.

Formally, $f: D  \to  C$ is  onto  means $\underline{\phantom{\forall b \in C  \exists a \in D ( f(a) = b)}}$.

%! app: TODOapp
%! outcome: function and relation definitions

Informally, a function being onto means ``every potential image is an actual image''.

\phantom{Draw finite domain, finite codomain picture with duplicate image.}
\vspace{50pt}
%! app: TODOapp
%! outcome: functions for cardinality

{\bf Definition}:  For nonempty sets $A, B$, we say that {\bf the  cardinality of $A$ is  no  smaller than 
the cardinality of  $B$}, and 
write $|A| \geq |B|$, to mean there is an onto function  with domain $A$  and codomain $B$.
Also, we define $|A| \geq |\emptyset|$ for all sets $A$.

%! app: TODOapp
%! outcome: functions for cardinality

{\it In the analogy}: The function $triedToSit: \{ Person\sun, Person\smiley, Person\frownie \} \to  \{ Chair1, Chair2\} $ given
by $triedToSit(Person\sun) = Chair1$,  $triedToSit(Person\smiley) = Chair2$, 
$triedToSit(Person\frownie) = Chair2$, is onto and witnesses that 
\[
 |\{ Person\sun, Person\smiley, Person\frownie \}| \geq | \{ Chair1, Chair2\} |
\]
%! app: TODOapp
%! outcome: function and relation definitions

{\bf Definition} : A function $f: D  \to C$ is a {\bf bijection} means that it is both 
one-to-one  and onto. The {\bf inverse} of a  bijection $f: D  \to  C$ is 
the function $g: C  \to  D$  such that $g(b) = a$ iff  $f(a) =  b$.


%\input{../activity-snippets/rna-surjective-cardinality.tex}

\newpage

\section*{Week 8 Wednesday and Friday: Finite, countably infinite, and uncountable sets}
\subsection*{Cardinality of sets}
\input{../activity-snippets/cardinality-definition.tex}
\input{../activity-snippets/cardinality-caution.tex}
%! app: TODOapp
%! outcome: functions for cardinality

{\bf Properties of cardinality}
\begin{align*}
&\forall A ~ (~  |A| = |A| ~)\\
&\forall A ~ \forall B ~(~ |A| = |B|  ~\to ~ |B| = |A|~)\\
&\forall A ~ \forall B ~ \forall C~ (~ (|A| = |B| ~\wedge~ |B| = |C|) ~\to ~ |A| = |C|~)
\end{align*}

{\it Extra practice with proofs:} Use the definitions of bijections to prove these properties.
\input{../activity-snippets/cantor-schroder-bernstein-theorem.tex}
\newpage
\input{../activity-snippets/countably-infinite-definition.tex}
\input{../activity-snippets/countably-infinite-examples-sets-of-numbers.tex}
\input{../activity-snippets/countably-infinite-examples-other-sets.tex}
\newpage
\subsection*{Cardinality categories}
\input{../activity-snippets/cardinality-categories.tex}
\input{../activity-snippets/cardinality-countability-lemmas.tex}

\subsection*{Are there always *bigger* sets?}
%! app: TODOapp
%! outcome: functions for cardinality, classify cardinality

{\it Recall}: When $U$ is a set, $\mathcal{P}(U) = \{ X \mid X \subseteq U\}$

{\it Key idea}: For finite sets, the power set of a set has strictly greater size than the set itself.
Does this extend to infinite sets?

{\bf Definition}: For two sets $A, B$, we use the notation $|A| < |B|$ to denote
$(~|A| \leq |B| ~) \land \lnot (~|A| = |B|)$.

\begin{alignat*}{4}
    &\emptyset = \{ \} \qquad &&\mathcal{P}(\emptyset) = \{ \emptyset \} \qquad &&|\emptyset| < |\mathcal{P}(\emptyset)| \\
    &\{1 \} \qquad &&\mathcal{P}(\{1\}) = \{ \emptyset, \{1\} \} \qquad &&|\{1\}| < |\mathcal{P}(\{1\})| \\
    &\{1,2 \} \qquad &&\mathcal{P}(\{1,2\}) = \{ \emptyset, \{1\}, \{2\}, \{1,2\} \} \qquad &&|\{1,2\}| < |\mathcal{P}(\{1,2\})| \\
\end{alignat*}

{\bf $\mathbb{N}$ and its power set}

Example elements of $\mathbb{N}$ 

\vspace{20pt}

Example elements of $\mathcal{P}(\mathbb{N})$

\vspace{20pt}

{\bf Claim}: $| \mathbb{N} | \leq |\mathcal{P} ( \mathbb{N} ) |$

\vspace{100pt}
\newpage
{\bf Claim}: There is an uncountable set.  Example: $\underline{\phantom{~~~\mathcal{P}(\mathbb{N})~~~}}$

{\bf Proof}:  By definition of countable, since $\underline{\phantom{~~~\mathcal{P}(\mathbb{N})~~~}}$
is not finite, {\bf to show} is $|\mathbb{N}| \neq  |\mathcal{P}(\mathbb{N})|$ .

Rewriting using  the definition of  cardinality, {\bf to show} is

\phantom{$\neg \exists f : \mathbb{N} \to \mathcal{P}(\mathbb{N})  ~~(f \text{ is a bijection})~~$}

\phantom{or equivalently $\forall f : \mathbb{N} \to \mathcal{P}(\mathbb{N})  ~~(f \text{ is not a bijection})~~$}


Towards a proof by  universal generalization,  consider  an arbitrary function $f:  \mathbb{N} \to\mathcal{P}(\mathbb{N})$.

{\bf To show}: $f$ is not a bijection.  It's enough to show that $f$ is not onto.

Rewriting using the definition of  onto, {\bf to show}:
\[
\neg  \forall  B \in  \mathcal{P}(\mathbb{N}) ~\exists a \in \mathbb{N}  ~(~f(a) =  B~)
\]
. By logical  equivalence, we can write this as an existential statement:
\[
\underline{\phantom{\qquad\qquad\exists B \in  \mathcal{P}(\mathbb{N}) ~\forall a \in \mathbb{N}  ~(~f(a) \neq  B~)\qquad\qquad}}
\]
In search of a witness, define the following  collection of nonnegative integers:
\[
D_f = \{ n \in \mathbb{N}  ~\mid~  n \notin f(n)  \}
\]
. By  definition  of power  set, since  all elements  of  $D_f$ are  in  $\mathbb{N}$,   $D_f \in \mathcal{P}(\mathbb{N})$.  It's enough to prove the following Lemma: 

{\bf Lemma}: $\forall a \in \mathbb{N}  ~(~f(a) \neq  D_f~)$.


{\bf Proof  of lemma}: \phantom{Towards universal  generalization, consider an arbitrary  $a \in \mathbb{N}$.
By definition  of set equality, {\bf to show} is  $\exists  x ( \neg  (x \in f(a)~  \leftrightarrow  ~x \in D_f))$.
For a witness, consider $x = a$.  There are two cases:  $a \in  f(a)~\vee~a \notin f(a)$. By definition 
of $D_f$, each guarantees that $f(a) \neq  D_f$.}\\

\vspace{50pt}

By  the Lemma, we  have proved that $f$ is not onto, and since $f$ was arbitrary, there are no onto
functions from $\mathbb{N}$ to $\mathcal{P}(\mathbb{N})$. QED


{\bf Where does $D_f$ come from?} The idea is to build a set that would ``disagree" with 
each of the images of $f$ about some element. 

\begin{center}
\begin{tabular}{c|c|ccccccc}
$n \in \mathbb{N}$ & $f(n) = X_n$ &  Is $0   \in X_n$?   & Is $1 \in X_n$?  &  Is $2 \in X_n$?  &  Is $3 \in X_n$?  &
 Is $4 \in X_n$?  &  \ldots & Is $n \in D_f$?\\
\hline
$0$ & $f(0) = X_0$ & {\bf  Y~/~N}  & Y~/~N & Y~/~N & Y~/~N &Y~/~N & \ldots & {\bf  N~/~Y }\\
$1$ & $f(1) = X_1$ & Y~/~N  & {\bf  Y~/~N} & Y~/~N & Y~/~N & Y~/~N & \ldots & {\bf  N~/~Y }\\
$2$ & $f(2) = X_2$ & Y~/~N  & Y~/~N & {\bf  Y~/~N} & Y~/~N &Y~/~N & \ldots & {\bf  N~/~Y }\\
$3$ & $f(3) = X_3$ & Y~/~N  & Y~/~N & Y~/~N & {\bf  Y~/~N} & Y~/~N & \ldots & {\bf  N~/~Y }\\
$4$ & $f(4) = X_4$ & Y~/~N  & Y~/~N & Y~/~N & Y~/~N &{\bf  Y~/~N} & \ldots & {\bf  N~/~Y }\\
\vdots
\end{tabular}
\end{center}
\newpage
\subsection*{Countable vs.\ uncountable: sets of numbers}
%! app: TODOapp
%! outcome: functions for cardinality, classify cardinality, important sets

{\bf Comparing $\mathbb{Q}$ and $\mathbb{R}$} 


Both $\mathbb{Q}$ and $\mathbb{R}$ have no greatest element.

Both $\mathbb{Q}$ and $\mathbb{R}$ have no least element.

The quantified statement 
\[
    \forall x \forall y (x < y \to \exists z ( x < z < y) )
\]
is true about both $\mathbb{Q}$ and $\mathbb{R}$.

Both $\mathbb{Q}$ and $\mathbb{R}$ are infinite. But, $\mathbb{Q}$ is countably infinite
whereas $\mathbb{R}$ is uncountable.\\


{\bf The set of real numbers}

$\mathbb{Z} \subsetneq \mathbb{Q} \subsetneq \mathbb{R}$


{\bf  Order axioms} (Rosen Appendix 1): 

\begin{center}
\begin{tabular}{p{1.2in}p{4in}}
Reflexivity &  $\forall a \in  \mathbb{R} (a \leq a)$\\
Antisymmetry &  $\forall a \in  \mathbb{R}~\forall b \in \mathbb{R}~(~(a \leq b~ \wedge ~b \leq a) \to (a=b)~)$\\
Transitivity &  $\forall a \in  \mathbb{R}~\forall b \in \mathbb{R}~\forall c \in \mathbb{R}~
(~(a \leq b \wedge b \leq c) ~\to  ~(a \leq c)~)$ \\
Trichotomy & 
$\forall a \in \mathbb{R}~\forall b \in \mathbb{R}~ ( ~(a=b ~\vee~ b > a ~\vee~ a  < b)  $
\end{tabular}
\end{center}


{\bf  Completeness axioms} (Rosen Appendix 1): 


\begin{center}
\begin{tabular}{p{1.4in}p{6in}}
Least upper bound &  Every nonempty set of real numbers that 
is bounded  above has  a  least upper bound  
%\phantom{~$\forall X \in \mathcal{P}(\mathbb{R}) ( ~(\exists x (x \in X)  \wedge \exists  y ~\forall x ( x \in X ~\to~ x \leq y) )~\to ~\exists  y_0 \left( ~\forall x ( x \in X ~\to~ x \leq y) \wedge 
%\forall z ( \forall  x (x \in X ~\to~ x \leq z) ~\to ~ y  \leq  z)\right) )$}\\
\\
Nested intervals &  For each sequence  of intervals  $[a_n , b_n]$
where, for each $n$, $a_n < a_{n+1} < b_{n+1} < b_n$, there
is at least one  real number $x$ such that, for all $n$, 
$a_n \leq x \leq b_n$.\\
\end{tabular}
\end{center}

Each real  number $r  \in  \mathbb{R}$ is described by a function to give better and better approximations
\[
x_r: \mathbb{Z}^+ \to \{0,1\}  \qquad  \text{where  $x_r(n ) =  n^{th} $ bit in  binary expansion of $r$}
\]
\begin{center}
\begin{tabular}{|c|c|p{3.9in}|}
\hline
$r$ & Binary expansion & $x_r$ \\
\hline
%$0.5$ & $0.10000\ldots$ &  $x_{0.5}(n) = \begin{cases} 1 &\text{if $n = 1$} \\ 0&\text{otherwise} \end{cases}$\\
%& & \\
%\hline
$0.1$ & $0.00011001 \ldots$ &  $x_{0.1}(n) = \begin{cases} 0&\text{if $n > 1$ and $(n~\text{\bf mod}~4) =2$} \\
0&\text{if $n=1$ or if $n > 1$ and $(n~\text{\bf mod}~4) =3$} \\1&\text{if $n > 1$ and $(n~\text{\bf mod}~4) =0$} \\
1&\text{if $n > 1$ and $(n~\text{\bf mod}~4) =1$} \end{cases}$  \\
&&  \\
\hline
$\sqrt{2} - 1 = 0.4142135 \ldots$  &$0.01101010\ldots$& Use linear approximations
(tangent lines from calculus) to get algorithm for bounding error of successive operations. Define 
$x_{\sqrt{2}-1}(n)$ to be  $n^{th}$ bit in approximation  that has error less than  $2^{-(n+1)}$.
\\
&& \\
\hline
\end{tabular}
\end{center}

\newpage 

{\bf Claim}: $\{  r \in \mathbb{R} ~\mid~ 0 \leq r ~\wedge~ r \leq 1 \}$ is uncountable.

{\it Approach 1}: Mimic proof that $\mathcal{P}(\mathbb{Z}^+)$ is uncountable.


{\bf Proof}:  By definition of countable, since $\{  r \in \mathbb{R} ~\mid~ 0 \leq r ~\wedge~ r \leq 1 \}$
is not finite, {\bf to show} is $|\mathbb{N}| \neq  |\{  r \in \mathbb{R} ~\mid~ 0 \leq r ~\wedge~ r \leq 1 \}|$ .


{\bf To show} is
$\forall f : \mathbb{Z}^+ \to \{  r \in \mathbb{R} ~\mid~ 0 \leq r ~\wedge~ r \leq 1 \}  ~~(f \text{ is not a bijection})~~$.
Towards a proof by  universal generalization, consider  an arbitrary function 
$f:  \mathbb{Z}^+ \to \{  r \in \mathbb{R} ~\mid~ 0 \leq r ~\wedge~ r \leq 1 \}$.
{\bf To show}: $f$ is not a bijection.  It's enough to show that $f$ is not onto.
Rewriting using the definition of  onto, {\bf to show}:
\[
\exists x \in \{  r \in \mathbb{R} ~\mid~ 0 \leq r ~\wedge~ r \leq 1 \} ~\forall a \in \mathbb{N}  ~(~f(a) \neq  x~)
\]
In search of a witness, define the following  real number by defining its binary expansion
\[
d_f = 0.b_1 b_2 b_3 \cdots
\]
where $b_i = 1-b_{ii}$ where $b_{jk}$ is the coefficient of $2^{-k}$ in the binary expansion of $f(j)$.
Since\footnote{There's a subtle imprecision in this part of the proof as presented, but it can be fixed.} $d_f \neq f(a)$ for any positive integer $a$, $f$ is not onto.
%$x_r: \mathbb{Z}^+ \to  \{0,1\}$ where  $x_r(n) = n^{th}$ bit in binary expansion  of $r$
%\qquad 
%$X_r =  \{ n \in \mathbb{Z}^+ ~\mid~ x_r(n) = 1 \}$

{\it Approach 2}: Nested closed interval property

{\bf To show} $f: \mathbb{N} \to \{  r \in \mathbb{R} ~\mid~ 0  \leq r ~\wedge~ r \leq 1 \}$ is not onto. 
{\bf  Strategy}: Build
a sequence of nested closed intervals that each avoid some $f(n)$.   Then  the real
number that is in all of the intervals  can't be $f(n)$ for any $n$. Hence,  $f$ is not  onto.

Consider the function $f: \mathbb{N} \to \{  r \in \mathbb{R} ~\mid~ 0 \leq r ~\wedge~ r \leq 1 \}$ with  $f(n) = \frac{1+\sin(n)}{2}$

\begin{center}
\begin{tabular}{c||p{1.65in} || p{3in} }
$n$ &  $f(n)$& Interval that avoids $f(n)$ \\
\hline
$0$ & $0.5$ &  \\
$1$ &$0.920735\ldots$  &  \\
$2$ &$0.954649\ldots$ &  \\
$3$ &$0.570560\ldots$ & \\
$4$ &$0.121599\ldots $&  \\
\vdots &  &\\
\end{tabular}
\end{center}
 
\subsection*{Other examples of uncountable sets}
%! app: TODOapp
%! outcome: classify cardinality

\begin{itemize}
    \item The power set of any countably infinite set is uncountable. For example:
    \[
        \mathcal{P}(\mathbb{N}), \mathcal{P}(\mathbb{Z^+}), \mathcal{P}(\mathbb{Z})
    \]
    are each uncountable.
    \item The closed interval $\{x \in \mathbb{R} ~|~ 0 \leq x \leq 1\}$, any other nonempty closed interval of real numbers whose endpoints are 
    unequal, as well as the related intervals that exclude one or both of the endpoints.
    \item The set of all real numbers $\mathbb{R}$ is uncountable and the set of irrational
    real numbers $\overline{\mathbb{Q}}$ is uncountable.
\end{itemize}
\newpage

\section*{Review Quiz}
\begin{enumerate}
    \item Sets of numbers. \hspace{1in}\\ \input{../activity-snippets/quiz-rationals-proofs.tex}
    \item Finite vs. infinite. \hspace{1in}\\ \input{../activity-snippets/quiz-finite-sets.tex}
    \item Functions. \hspace{1in}\\ \input{../activity-snippets/quiz-injective-surjective.tex}
    \item Functions. \hspace{1in}\\ \input{../activity-snippets/quiz-cardinality-witnessing-functions-q1.tex} 
    \item Functions. \hspace{1in}\\ \input{../activity-snippets/quiz-cardinality-witnessing-functions-q2.tex} 
    \item Diagonalization. \hspace{1in}\\ \input{../activity-snippets/quiz-diagonalization.tex}
    \item Classifying cardinality. \hspace{1in}\\ \input{../activity-snippets/quiz-cardinality-classifying.tex}
\end{enumerate}
\newpage
\end{document}