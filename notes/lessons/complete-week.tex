\input{../../resources/lesson-head.tex}
\section*{Let's get started}

We want you to be successful. 

We will work together to build an 
environment in CSE 20 that supports your learning
in a way that respects your
perspectives, experiences, and identities (including race, ethnicity, heritage, gender, sex, 
class, sexuality, religion, ability, age, educational background, etc.).
Our goal is for you to  engage
with interesting and challenging concepts and 
feel comfortable exploring, asking questions, and thriving.

If you or someone you know is suffering from food and/or housing insecurities 
there are UCSD resources here to help:

Basic Needs Office: \href{https://basicneeds.ucsd.edu/}{https://basicneeds.ucsd.edu/}

Triton Food Pantry (in the old Student Center)
is free and anonymous, and includes produce: 

\href{https://www.facebook.com/tritonfoodpantry/}{https://www.facebook.com/tritonfoodpantry/}

Mutual Aid UCSD: \href{https://mutualaiducsd.wordpress.com/}{https://mutualaiducsd.wordpress.com/}

Financial aid resources, the possibility of emergency grant funding, and off-campus housing referral 
resources are available: see your College Dean of Student Affairs.

If you find yourself in an uncomfortable situation, ask for help. 
We are committed to upholding University policies regarding nondiscrimination, sexual violence and sexual harassment.
Here are some campus contacts that could provide this help: 
Counseling and Psychological Services (CAPS) at 858 534-3755 or \href{http://caps.ucsd.edu}{http://caps.ucsd.edu}; 
OPHD at 858 534-8298 or ophd@ucsd.edu , \href{http://ophd.ucsd.edu}{http://ophd.ucsd.edu};
CARE at Sexual Assault Resource Center at 858 534-5793 or sarc@ucsd.edu , \href{http://care.ucsd.edu}{http://care.ucsd.edu}.


Please reach out (minnes@ucsd.edu) if you need support with extenuating circumstances affecting CSE 20.

\vfill

\section*{Welcome to CSE 20: Discrete Math for CS in Spring 2024!}
Class website: \href{https://canvas.ucsd.edu//}{https://canvas.ucsd.edu/}


Instructor: Prof. Mia Minnes {\tiny{"Minnes" rhymes with Guinness}}, minnes@ucsd.edu, 
\href{http://cseweb.ucsd.edu/~minnes}{http://cseweb.ucsd.edu/~minnes}


Our team: One instructor + two TAs and eleven tutors + all of you

Fill in contact info for students around you, if you'd like:

\vfill


On a typical week in CSE 20: {\bf MWF} Lectures (sometimes with pre-class reading), {\bf W} Discussion,
Review quiz, then {\bf T} Homework due.
Office hours (hosted by instructors and TAs and tutors) where you can come to talk 
about course concepts and ask for help as you work through sample problems 
and Q+A on Piazza available throughout the week. CSE 20 has one project and two tests
this quarter.
Demonstration of class website on \href{https://canvas.ucsd.edu/}{Canvas (https://canvas.ucsd.edu/)}:
\begin{enumerate}
\item Syllabus
\item Notes for class + annotations
\item Assignments (PDF, tex, solutions)
\item Gradescope
\item Piazza
\item Dates
\end{enumerate}

\vfill
\begin{comment}
There are lots of great reasons to have a laptop, tablet, or phone open during class. You might be taking notes, 
getting a photo of an important moment on the board, trying out a construction that we're developing together, working 
on the review quiz, and so on. 
The main issue with screens and technology in the classroom isn't that they might 
distract you, 
it's the distraction of other students. We ask that if you would like to use
a device in class and may have 
have unrelated content open, please sit in one of the back two rows of the 
room so that it's not adversely affecting other students.


\vfill
\end{comment}
{\bf Pro-tip}: you can use MATH109 to replace CSE20 for prerequisites and other requirements.

\vfill

\section*{Themes and applications for CSE 20}
\begin{itemize}
\item {\bf Technical skepticism}: Know, select and apply appropriate computing knowledge and problem-solving techniques. 
Reason about computation and systems. 
Use mathematical techniques to solve problems. 
Determine appropriate conceptual tools to apply to new situations. 
Know when tools do not apply and try different approaches. 
Critically analyze and evaluate candidate solutions.
\item {\bf Multiple representations}: Understand, guide, shape impact of computing on society/the world. 
Connect the role of Theory CS classes to other applications (in undergraduate CS curriculum and beyond). 
Model problems using appropriate mathematical concepts.
Clearly and unambiguously communicate computational ideas using appropriate formalism. 
Translate across levels of abstraction.
\end{itemize}

{\bf Applications}: Numbers (how to represent them and use them in Computer Science), 
Recommendation systems and their roots in machine learning (with applications like Netflix),
``Under the hood" of computers (circuits, pixel color representation, data structures),
Codes and information (secret message sharing and error correction),
Bioinformatics algorithms and genomics (DNA and RNA).

\newpage

\subsection*{Week 1 at a glance}

\subsubsection*{We will be learning and practicing to:}
%data types, translating, important sets, write set definition, function and relation definitions, recursive definitions
\begin{itemize}
\item Model systems with tools from discrete mathematics and reason about implications of modelling choices. Explore applications in CS through multiple perspectives, including software, hardware, and theory.
\begin{itemize}

   \item Selecting and representing appropriate data types and using notation conventions to clearly communicate choices

\end{itemize}

\item Translate between different representations to illustrate a concept.

\begin{itemize}
   \item Translating between symbolic and English versions of statements using precise mathematical language
\end{itemize}


\item Use precise notation to encode meaning and present arguments concisely and clearly
\begin{itemize}
    \item Defining important sets of numbers, e.g. set of integers, set of rational numbers
    \item Precisely describing a set using appropriate notation e.g. roster method, set builder notation, and recursive definitions
    \item Defining functions using multiple representations
\end{itemize}

\item Know, select and apply appropriate computing knowledge and problem-solving techniques. Reason about computation and systems. Use mathematical techniques to solve problems. Determine appropriate conceptual tools to apply to new situations. Know when tools do not apply and try different approaches. Critically analyze and evaluate candidate solutions.
\begin{itemize}
    \item Using a recursive definition to evaluate a function or determine membership in a set
\end{itemize}

\end{itemize}

\subsubsection*{TODO:}
\begin{list}
   {\itemsep2pt}
   \item \#FinAid Assignment on Canvas (complete as soon as possible) 
   \item Review quiz based on class material each day (due Friday April 5, 2024)
   \item Homework assignment 1 (due Tuesday April 9, 2024).
\end{list}

\newpage

\section*{Week 1 Monday: Modeling applications}
\input{../activity-snippets/netflix-intro.tex}

\vfill
\subsection*{Data Types: sets, $n$-tuples, and strings}
\input{../../resources/lesson-head.tex}
\section*{Sets equality subset definition}
\input{../activity-snippets/sets-equality-subset-definition.tex}
\vfill
\section*{Predicate definition}
\input{../activity-snippets/predicate-definition.tex}
\vfill
\section*{Predicate notation}
\input{../activity-snippets/predicate-notation.tex}
\vfill
\section*{Sets numbers subsets}
\input{../activity-snippets/sets-numbers-subsets.tex}
\vfill
\section*{Defining functions more examples}
\input{../activity-snippets/defining-functions-more-examples.tex}
\vfill
\section*{Division algorithm}
\input{../activity-snippets/division-algorithm.tex}
\vfill
\section*{Netflix intro}
\input{../activity-snippets/netflix-intro.tex}
\vfill
\section*{Data types}
\input{../../resources/lesson-head.tex}
\section*{Sets equality subset definition}
\input{../activity-snippets/sets-equality-subset-definition.tex}
\vfill
\section*{Predicate definition}
\input{../activity-snippets/predicate-definition.tex}
\vfill
\section*{Predicate notation}
\input{../activity-snippets/predicate-notation.tex}
\vfill
\section*{Sets numbers subsets}
\input{../activity-snippets/sets-numbers-subsets.tex}
\vfill
\section*{Defining functions more examples}
\input{../activity-snippets/defining-functions-more-examples.tex}
\vfill
\section*{Division algorithm}
\input{../activity-snippets/division-algorithm.tex}
\vfill
\section*{Netflix intro}
\input{../activity-snippets/netflix-intro.tex}
\vfill
\section*{Data types}
\input{../../resources/lesson-head.tex}
\section*{Sets equality subset definition}
\input{../activity-snippets/sets-equality-subset-definition.tex}
\vfill
\section*{Predicate definition}
\input{../activity-snippets/predicate-definition.tex}
\vfill
\section*{Predicate notation}
\input{../activity-snippets/predicate-notation.tex}
\vfill
\section*{Sets numbers subsets}
\input{../activity-snippets/sets-numbers-subsets.tex}
\vfill
\section*{Defining functions more examples}
\input{../activity-snippets/defining-functions-more-examples.tex}
\vfill
\section*{Division algorithm}
\input{../activity-snippets/division-algorithm.tex}
\vfill
\section*{Netflix intro}
\input{../activity-snippets/netflix-intro.tex}
\vfill
\section*{Data types}
\input{../activity-snippets/data-types.tex}
\vfill
\section*{Ratings encoding}
\input{../activity-snippets/ratings-encoding.tex}
\vfill
\section*{Definitions set prereqs}
\input{../activity-snippets/definitions-set-prereqs.tex}
\vfill
\section*{Defining sets}
\input{../activity-snippets/defining-sets.tex}
\vfill
\section*{Definitions functions prereqs}
\input{../activity-snippets/definitions-functions-prereqs.tex}
\vfill
\section*{Defining functions}
\input{../activity-snippets/defining-functions.tex}
\vfill
\section*{Defining functions ratings}
\input{../activity-snippets/defining-functions-ratings.tex}
\vfill
\section*{Defining functions recursively}
\input{../activity-snippets/defining-functions-recursively.tex}
\vfill
\end{document}
\vfill
\section*{Ratings encoding}
\input{../activity-snippets/ratings-encoding.tex}
\vfill
\section*{Definitions set prereqs}
\input{../activity-snippets/definitions-set-prereqs.tex}
\vfill
\section*{Defining sets}
\input{../activity-snippets/defining-sets.tex}
\vfill
\section*{Definitions functions prereqs}
\input{../activity-snippets/definitions-functions-prereqs.tex}
\vfill
\section*{Defining functions}
\input{../activity-snippets/defining-functions.tex}
\vfill
\section*{Defining functions ratings}
\input{../activity-snippets/defining-functions-ratings.tex}
\vfill
\section*{Defining functions recursively}
\input{../activity-snippets/defining-functions-recursively.tex}
\vfill
\end{document}
\vfill
\section*{Ratings encoding}
\input{../activity-snippets/ratings-encoding.tex}
\vfill
\section*{Definitions set prereqs}
\input{../activity-snippets/definitions-set-prereqs.tex}
\vfill
\section*{Defining sets}
\input{../activity-snippets/defining-sets.tex}
\vfill
\section*{Definitions functions prereqs}
\input{../activity-snippets/definitions-functions-prereqs.tex}
\vfill
\section*{Defining functions}
\input{../activity-snippets/defining-functions.tex}
\vfill
\section*{Defining functions ratings}
\input{../activity-snippets/defining-functions-ratings.tex}
\vfill
\section*{Defining functions recursively}
\input{../activity-snippets/defining-functions-recursively.tex}
\vfill
\end{document}
\newpage
\input{../activity-snippets/ratings-encoding.tex}
\vfill
{\bf Conclusion}: Modeling involves choosing data types to represent and organize data

\newpage
\section*{Week 1 Wednesday: Defining sets}
\subsection*{Notation and prerequisites}
\input{../activity-snippets/definitions-set-prereqs.tex}
\input{../activity-snippets/defining-sets.tex}
\newpage
\input{../activity-snippets/rna-motivation.tex}
\input{../activity-snippets/recursive-sets-definition.tex}
\input{../activity-snippets/set-recursive-examples.tex}
\input{../activity-snippets/set-operations.tex}

\section*{Week 1 Friday: Defining functions}
\input{../activity-snippets/definitions-functions-prereqs.tex}
\input{../activity-snippets/defining-functions.tex}
\input{../activity-snippets/defining-functions-ratings.tex}
\input{../activity-snippets/defining-functions-recursively.tex}

\newpage
\subsection*{Review Quiz}
\begin{enumerate}
\item Modeling
\begin{enumerate}
    \item {\input{../activity-snippets/quiz-ratings-tuples.tex}}
    \item {\input{../activity-snippets/quiz-ratings-count.tex}}
    \item {\input{../activity-snippets/quiz-color-rgb-definitions.tex}}
    \item {\input{../activity-snippets/quiz-color-rgb-definitions-data-types.tex}}
\end{enumerate}
\item Sets and functions
\begin{enumerate}
    \item {\input{../activity-snippets/quiz-set-membership.tex}}
    \item {\input{../activity-snippets/quiz-strands-set-operations.tex}}
    \item {\input{../activity-snippets/quiz-defining-functions-domain-codomain.tex}}
%    \item {\input{../activity-snippets/quiz-recursive-definitions.tex}}
%    \item {\input{../activity-snippets/quiz-defining-functions-recursively.tex}}
\end{enumerate}

\end{enumerate}

\newpage

\subsection*{Week 10 at a glance}

\subsubsection*{We will be learning and practicing to:}
%classify cardinality
%important sets
%function and relation definitions
%functions for cardinality
%div and mod
%divisibility and primes
%congruence mod n
%binary relations properties
%special binary relations
%graph representations of relations
\begin{itemize}

\item Clearly and unambiguously communicate computational ideas using appropriate formalism. Translate across levels of abstraction.
\begin{itemize}
    \item Defining functions, predicates, and binary relations using multiple representations
    \item Determining whether a given binary relation is symmetric, antisymmetric, reflexive, and/or transitive
    \item Determining whether a given binary relation is an equivalence relation and/or a partial order
    \item Drawing graph representations of relations and functions e.g. Hasse diagram and partition diagram
\end{itemize}

\item Know, select and apply appropriate computing knowledge and problem-solving techniques. Reason about computation and systems. Use mathematical techniques to solve problems. Determine appropriate conceptual tools to apply to new situations. Know when tools do not apply and try different approaches. Critically analyze and evaluate candidate solutions.
\begin{itemize}
    \item Using the definitions of the div and mod operators on integers
    \item Using divisibility and primality predicates
    \item Applying the definition of congruence modulo n and modular arithmetic
\end{itemize}

\item Apply proof strategies, including direct proofs and proofs by contradiction, and determine whether a proposed argument is valid or not.
\begin{itemize}
    \item Using proofs as knowledge discovery tools to decide whether a statement is true or false
\end{itemize}
\end{itemize}

\subsubsection*{TODO:}
\begin{list}
   {\itemsep2pt}
   \item Homework assignment 6 (due Thursday June 6, 2024).
   \item Review quiz based on class material each day (due Friday June 7, 2024).
   \item Review for Final exam. The test is scheduled for Thursday June 13 11:30a-2:29pm, location REC GYM.
\end{list}

\newpage
\section*{Week 10 Monday: Equivalence Relations and Partial Orders}
\subsection*{Definitions and representations}
\input{../activity-snippets/equivalence-relation-definition.tex}
\input{../activity-snippets/partial-order-definition.tex}
%! app: TODOapp
%! outcome: TODOoutcome

For a partial ordering, its {\bf Hasse diagram} is a graph representing the 
relationship between elements in the ordering. The nodes (vertices) of the graph 
are the elements of the 
domain of the binary relation. The edges do not have arrow heads. The
directionality of the partial order is indicated by 
the arrangements of the nodes. The nodes are arranged so that nodes connected to nodes
above them by edges indicate that the relation holds between the 
lower node and the higher node. 
Moreover, the diagram omits self-loops and 
omits edges that are guaranteed by transitivity.

\vfill
\input{../activity-snippets/hasse-diagram-example.tex}
\vfill
\newpage
\subsection*{Exploring equivalence relations}
\input{../activity-snippets/partition-definition.tex}
\input{../activity-snippets/equivalence-class-definition.tex}
\input{../activity-snippets/partitions-equivalence-classes.tex}
\vfill
\input{../activity-snippets/congruence-classes-mod-four.tex}
\vfill
\newpage
\input{../activity-snippets/modular-arithmetic-motivation.tex}
\input{../activity-snippets/congruence-mod-n-lemma.tex}
\vfill
\input{../activity-snippets/modular-arithmetic.tex}
\vfill
\newpage
\input{../activity-snippets/modular-arithmetic-cycling-examples.tex}
\vfill
\newpage
\input{../activity-snippets/diffie-helman.tex}
\newpage
\section*{Week 10 Wednesday: Applications}

\input{../activity-snippets/equivalence-relations-partitions.tex}

Last time, we saw that partitions associated to equivalence relations
were useful in the context of modular arithmetic.
Today we'll look at a different application.
\input{../activity-snippets/equivalence-relations-examples-ratings.tex}
\input{../activity-snippets/netflix-clustering-scenario.tex}


\newpage
\section*{Week 10 Friday: Review and Advice}
\input{../activity-snippets/base-expansion-final-review.tex}
\newpage
\input{../activity-snippets/set-construction-final-review.tex}
\newpage
\input{../activity-snippets/set-operations-final-review.tex}
\newpage
\input{../activity-snippets/function-properties-final-review.tex}
\newpage
\input{../activity-snippets/cardinality-final-review.tex}
\newpage
\input{../activity-snippets/modular-arithmetic-final-review.tex}
\newpage


\section*{Looking forward}

\subsection*{Tips for future classes from the CSE 20 TAs and tutors}
\begin{itemize}
\item In class
\begin{itemize}
\item Go to class
%\item Show up to class early because sometimes seats get taken/ the classroom gets full and then you have to sit on the floor
\item There's usually a space for skateboards/longboards/eboards to go at the front or rear of the lecture hall 
\item If you have a flask water bottle please ensure that its secured during a lecture and it cannot fall - putting on the floor often leads to it falling since people sometimes cross your seats.
\item Take notes - it's much faster and more effective to note-take in class than watch recordings after, particularly if you do so longhand
\item Resist the urge to sit in the back. You will be able to focus much better sitting near the front, where there are fewer screens in front of you to distract from the lecture content
\item If you bring your laptop to class to take notes / access class materials, sit towards the back of the room to minimize distractions for people sitting behind you!
%\item On zoom it's easy to just type a question out in chat, it might be a little more awkward to do so in person, but it is definitely worth it. Don't feel like you should already know what's being covered
%\item Always check you have your iclicker\footnote{iclickers are used in many classes to encourage active participation in class. They're remotes that allow you to respond to multiple choice questions and the instructor can show a histogram of responses in real time.} on you. Just keep it in your backpack permanently. That way you can never forget it. 
\item Don't be afraid to talk to the people next to you during group discussions. Odds are they're as nervous as you are, and you can all benefit from sharing your thoughts and understanding of the material 
\item Certain classes will podcast the lectures, just like Zoom archives lecture recordings, at podcast.ucsd.edu . If they aren't podcasted, and you want to record lectures, ask your professor for consent first
\end{itemize}
\item Office hours, tutor hours, and the CSE building
\begin{itemize}
\item Office hours are a good place to hang out and get work done while being able to ask questions as they come up 
%\item Office hour attendance is typically much busier in person (and confined to the space in the room)
\item Get to know the CSE building: CSE B260, basement labs, office hours rooms
%\item Know how to get in to the building after-hours
\end{itemize}
\item Libraries and on-campus resources
\begin{itemize}
\item Look up what library floors are for what, how to book rooms: East wing of Geisel is open 24/7 (they might ask to see an ID if you stay late), East Wing of Geisel has chess boards and jigsaw puzzles, study pods on the 8th floor, 
free computers/wifi
\item Know Biomed exists and is usually less crowded
\item Most libraries allow you to borrow whiteboards and markers (also laptops, tablets, microphones, and other cool stuff) for 24 hours
\item Take advantage of Dine with a prof / Coffee with a prof program. It's legit free food / coffee once per quarter. 
\item When planning out your daily schedule, think about where classes are, how much time will they take, are their places to eat nearby and how you can schedule social time with friends to nearby areas 
\item Take into account the distances between classes if they are back to back
\end{itemize} 
\newpage
\item Final exams
\begin{itemize}
%\item What are 8am finals? Basically in-person exams are different
\item Don't forget your university card during exams. Physical version is best for ID checks.
%\item Some classes required blue books for exams (what they are, where to get them) 
\item Look up seating assignments for exams and go early to make sure you're in the right place (check the exits to make sure you're reading it the right way) 
\item Know where your exam is being held (find it on a map at least a day beforehand). Finals are often in strange places that take a while to find 
\end{itemize}
\end{itemize}

\newpage
\section*{Review Quiz}
\begin{enumerate}
    \item Binary relations. 
        \begin{enumerate}
            \item \hspace{1in}\\ \input{../activity-snippets/quiz-binary-relation-ratings.tex}
            \item \hspace{1in}\\ \input{../activity-snippets/quiz-binary-relation-divisibility.tex}
        \end{enumerate}
    \item Equivalence relations. 
        \begin{enumerate}
            \item \hspace{1in}\\ \input{../activity-snippets/quiz-binary-relation-same-size.tex} 
            \item \hspace{1in}\\ \input{../activity-snippets/quiz-equivalence-relation-properties-proof.tex}
        \end{enumerate}
    \item Partial orders. \hspace{1in}\\ \input{../activity-snippets/quiz-partial-order-hasse.tex}
    \item Equivalence classes and partitions. 
        \begin{enumerate}
            \item \hspace{1in}\\ \input{../activity-snippets/quiz-equivalence-class-types.tex}
            \item Select all and only the partitions of $\{1,2,3,4,5\}$ from the sets below.
            \begin{enumerate}
            \item $\{1,2,3,4,5\}$
            \item $\{\{1,2,3,4,5\}\}$
            \item $\{\{1\},\{2\},\{3\},\{4\},\{5\}\}$
            \item $\{ \{1\}, \{2,3\}, \{4\} \}$
            \item $\{ \{\emptyset, 1, 2\}, \{3,4,5\}\}$
            \end{enumerate}
        \end{enumerate}
    \item Modular exponentiation. \hspace{1in}\\ \input{../activity-snippets/quiz-modular-exponentiation.tex}
%    \item \hspace{1in}\\ \input{../activity-snippets/quiz-clustering-ratings.tex} 
\end{enumerate}
\newpage

\subsection*{Week 2 at a glance}

\subsubsection*{We will be learning and practicing to:}
%data types, 
%div and mod
%trace algorithms
%translating
%write set definition
%function and relation definitions
%representing numbers
%applications of number representations
\begin{itemize}
\item Model systems with tools from discrete mathematics and reason about implications of modelling choices. Explore applications in CS through multiple perspectives, including software, hardware, and theory.
\begin{itemize}
   \item Selecting and representing appropriate data types and using notation conventions to clearly communicate choices
   \item Determining the properties of positional number representations, including overflow and bit operations
\end{itemize}

\item Translate between different representations to illustrate a concept.

\begin{itemize}
   \item Translating between symbolic and English versions of statements using precise mathematical language
   \item Tracing algorithms specified in pseudocode
   \item Representing numbers using positional representations, including decimal, binary, hexadecimal, fixed-width representations, and 2s complement
\end{itemize}


\item Use precise notation to encode meaning and present arguments concisely and clearly
\begin{itemize}
    \item Precisely describing a set using appropriate notation e.g. roster method, set builder notation, and recursive definitions
    \item Defining functions using multiple representations
\end{itemize}

\item Know, select and apply appropriate computing knowledge and problem-solving techniques. Reason about computation and systems. Use mathematical techniques to solve problems. Determine appropriate conceptual tools to apply to new situations. Know when tools do not apply and try different approaches. Critically analyze and evaluate candidate solutions.
\begin{itemize}
    \item Using a recursive definition to evaluate a function or determine membership in a set
    \item Using the definitions of the div and mod operators on integers
\end{itemize}

\end{itemize}

\subsubsection*{TODO:}
\begin{list}
   {\itemsep2pt}
   \item \#FinAid Assignment on Canvas (complete as soon as possible) 
   \item Review quiz based on class material each day (due Friday April 12, 2024)
   \item Homework assignment 2 (due Tuesday April 16, 2024).
\end{list}

\newpage

\section*{Week 2 Monday: Sets, functions, and algorithms}
\input{../activity-snippets/defining-functions-more-examples.tex}
\newpage
\input{../activity-snippets/division-algorithm.tex}


\section*{Week 2 Wednesday: Representing numbers}
\input{../activity-snippets/why-represent-numbers.tex}
\input{../activity-snippets/base-expansion-definition.tex}
\vfill
\newpage
\input{../activity-snippets/base-expansion-examples.tex}
\input{../activity-snippets/algorithm-definition.tex}
\input{../activity-snippets/base-expansion-algorithms.tex}


\newpage
\section*{Week 2 Friday: Algorithms for numbers}
\input{../activity-snippets/base-expansion-review.tex}
\input{../activity-snippets/base-conversion-algorithm.tex}
\input{../activity-snippets/fixed-width-definition.tex}
\input{../activity-snippets/fixed-width-example.tex}
\input{../activity-snippets/fixed-width-fractional-definition.tex}
%\input{../activity-snippets/expansion-summary.tex}
\newpage
\input{../activity-snippets/negative-int-expansions.tex}
\input{../activity-snippets/calculating-2s-complement.tex}
\newpage
\input{../activity-snippets/representing-zero.tex}
\newpage
\subsection*{Review Quiz}
\begin{enumerate}
\item Functions and algorithms
\begin{enumerate}
    \item {\input{../activity-snippets/quiz-power-function-recursive.tex}}
    \item {\input{../activity-snippets/quiz-tracing-log.tex}}
\end{enumerate}
\item Base expansions
\begin{enumerate}
    \item {\input{../activity-snippets/quiz-base-expansion-calculations.tex}}
    \item {\input{../activity-snippets/quiz-fixed-width-expansions.tex}}
    \item {\input{../activity-snippets/quiz-expansions-properties.tex}}
    \item {\input{../activity-snippets/quiz-representing-negatives.tex}}
\end{enumerate}
\item Multiple representations
{\input{../activity-snippets/quiz-color-hexcolor-intro-question.tex}}
\end{enumerate}


\newpage

\subsection*{Week 3 at a glance}

\subsubsection*{We will be learning and practicing to:}
%applications of number representations
%circuits
%evaluating compound propositions
%logical equivalence via laws
%logical equivalence via truth tables
%cnf and dnf
%consistency
%translating
\begin{itemize}
\item Model systems with tools from discrete mathematics and reason about implications of modelling choices. Explore applications in CS through multiple perspectives, including software, hardware, and theory.
\begin{itemize}
    \item Determining the properties of positional number representations, including overflow and bit operations
   \item Connecting logical circuits and compound proposition and tracing to calcluate output values
\end{itemize}

\item Translate between different representations to illustrate a concept.
\begin{itemize}
   \item Translating between symbolic and English versions of statements using precise mathematical language
\end{itemize}

\item Use precise notation to encode meaning and present arguments concisely and clearly
\begin{itemize}
    \item Listing the truth tables of atomic boolean functions (and, or, xor, not, if, iff)
\end{itemize}

\item Know, select and apply appropriate computing knowledge and problem-solving techniques. Reason about computation and systems. Use mathematical techniques to solve problems. Determine appropriate conceptual tools to apply to new situations. Know when tools do not apply and try different approaches. Critically analyze and evaluate candidate solutions.
\begin{itemize}
    \item Evaluating compound propositions
    \item Judging logical equivalence of compound propositions using symbolic manipulation with known equivalences, including DeMorgan's Law
    \item Judging logical equivalence of compound propositions using truth tables
    \item Rewriting compound propositions using normal forms
    \item Judging whether a collection of propositions is consistent
\end{itemize}

\end{itemize}

\subsubsection*{TODO:}
\begin{list}
   {\itemsep2pt}
   \item Review quiz based on class material each day (due Friday April 19, 2024)
   \item Homework assignment 3 (due Tuesday April 23, 2024)
\end{list}

\newpage
\section*{Week 3 Monday: Fixed-width Addition and Circuits}

\input{../activity-snippets/fixed-width-addition.tex}
\vfill
\vfill
\input{../activity-snippets/circuits-basics.tex}

\input{../activity-snippets/logic-gates-definitions.tex}
\vfill
\input{../activity-snippets/digital-circuits-basic-examples.tex}
\vfill
\newpage
\section*{Week 3 Wednesday: Propositional Logic}
\input{../activity-snippets/half-adder-circuit.tex}
\input{../activity-snippets/two-bit-adder-circuit.tex}
\newpage
\input{../activity-snippets/logical-operators.tex}
\input{../activity-snippets/logical-operators-truth-tables.tex}
\vfill
\input{../activity-snippets/logical-operators-example-truth-table.tex}
\vfill
\newpage
\section*{Week 3 Friday: Logical Equivalence}
\input{../activity-snippets/truth-table-to-compound-proposition.tex}
\input{../activity-snippets/dnf-cnf-definition.tex}
\newpage
\input{../activity-snippets/compound-proposition-definitions.tex}
\input{../activity-snippets/logical-equivalence.tex}
\vfill
\input{../activity-snippets/tautology-contradiction-contingency-examples.tex}
\vfill
\input{../activity-snippets/logical-equivalence-extra-example.tex}
\vfill
\newpage
\subsection*{Review Quiz}
\begin{enumerate}
    \item Fixed-width addition: Recall the definitions of signed integer representations from class: 
    sign-magnitude and 2s complement.    
        \begin{enumerate}
            \item {\input{../activity-snippets/quiz-fixed-width-bounds.tex}}
            \item {\input{../activity-snippets/quiz-fixed-width-addition.tex}}
        \end{enumerate}
    \newpage
    \item Circuits
        \begin{enumerate}
            \item {\input{../activity-snippets/quiz-circuit-tracing.tex}}
            \item {\input{../activity-snippets/quiz-circuit-implementing-operation.tex}}
            \item \input{../activity-snippets/quiz-circuit-tracing-with-or.tex}
        \end{enumerate}
    \item Compound Propositions
        \begin{enumerate}
            \item \input{../activity-snippets/quiz-dnf-cnf-example.tex}
            \item \input{../activity-snippets/quiz-cnf-dnf.tex}
        \end{enumerate}
    \item Logical equivalence.
    \input{../activity-snippets/quiz-truth-values-or-and.tex}
\end{enumerate}

\newpage

\subsection*{Week 4 at a glance}

\subsubsection*{We will be learning and practicing to:}
%evaluating compound propositions
%truth table definitions
%consistency
%translating
%logical equivalence via laws
%logical equivalence via truth tables
%variants of conditionals
\begin{itemize}

\item Translate between different representations to illustrate a concept.
\begin{itemize}
   \item Translating between symbolic and English versions of statements using precise mathematical language
   \item Translating between truth tables (tables of values) and compound propositions
\end{itemize}

\item Use precise notation to encode meaning and present arguments concisely and clearly
\begin{itemize}
    \item Listing the truth tables of atomic boolean functions (and, or, xor, not, if, iff)
    \item Defining functions, predicates, and binary relations using multiple representations
\end{itemize}

\item Know, select and apply appropriate computing knowledge and problem-solving techniques. Reason about computation and systems. Use mathematical techniques to solve problems. Determine appropriate conceptual tools to apply to new situations. Know when tools do not apply and try different approaches. Critically analyze and evaluate candidate solutions.
\begin{itemize}
    \item Evaluating compound propositions
    \item Judging logical equivalence of compound propositions using symbolic manipulation with known equivalences, including DeMorgan's Law
    \item Writing the converse, contrapositive, and inverse of a given conditional statement
    \item Determining what evidence is required to establish that a quantified statement is true or false
    \item Evaluating quantified statements about finite and infinite domains
\end{itemize}

\end{itemize}

\subsubsection*{TODO:}
\begin{list}
   {\itemsep2pt}
   \item Review quiz based on class material each day (due Friday April 26, 2024)
   \item Start reviewing for Test 1, in class next week on Friday May 3, 2024.
\end{list}

\newpage

\section*{Week 4 Monday: Conditionals and Logical Equivalence}
\input{../activity-snippets/logical-operators-full-truth-table.tex}
\input{../activity-snippets/hypothesis-conclusion.tex}
\input{../activity-snippets/converse-inverse-contrapositive.tex}
\vfill
\input{../activity-snippets/compound-propositions-recursive-definition.tex}
\input{../activity-snippets/compound-propositions-precedence.tex}
\newpage
\input{../activity-snippets/logical-equivalence-identities.tex}
\newpage
\input{../activity-snippets/logical-operators-english-synonyms.tex}
\newpage
\input{../activity-snippets/compound-propositions-translation.tex}
\newpage
\input{../activity-snippets/consistency-def.tex}
\input{../activity-snippets/consistency-example.tex}
\newpage


\section*{Week 4 Wednesday: Predicates and Quantifiers}
\input{../activity-snippets/predicate-definition.tex}
\input{../activity-snippets/predicate-examples-finite-domain.tex}
\vfill
\input{../activity-snippets/predicate-truth-set-definition.tex}
\input{../activity-snippets/predicate-truth-set-example.tex}
\newpage
\input{../activity-snippets/quantification-definition.tex}
\vfill
\input{../activity-snippets/quantification-logical-equivalence.tex}
\vfill
\input{../activity-snippets/quantification-examples-finite-domain.tex}
\vfill
\newpage
\input{../activity-snippets/rna-rnalen-basecount-definitions.tex}
\input{../activity-snippets/predicate-rna-example.tex}
\newpage
\input{../activity-snippets/predicates-example-rnalen-basecount.tex}
\newpage

\section*{Week 4 Friday: Evaluating Nested Quantifiers}
\input{../activity-snippets/predicates-projecting-example-rna-basecount.tex}
\vfill
\input{../activity-snippets/predicate-notation.tex}
\vfill
\input{../activity-snippets/nested-quantifiers.tex}
\newpage
\input{../activity-snippets/alternating-quantifiers-strategies-rna-examples.tex}
\newpage

\subsection*{Review Quiz}
\begin{enumerate}
\item Logical equivalence
 \input{../activity-snippets/quiz-truth-values-conditional.tex}
\item Translating propositional logic
    \begin{enumerate}
    \item \input{../activity-snippets/quiz-compound-propositions-translation.tex}
    \newpage
    \item \input{../activity-snippets/quiz-consistency.tex}
    \end{enumerate}
\newpage
\item Evaluating predicates
    \begin{enumerate}
    \item \hspace{1in}\\ \input{../activity-snippets/quiz-predicates-finite-domain.tex}
    \item \hspace{1in}\\ \input{../activity-snippets/quiz-predicates.tex}
    \item \hspace{1in}\\ \input{../activity-snippets/quiz-predicates-rna.tex}
    \end{enumerate}
\newpage
\item Evaluating nested predicates
    \begin{enumerate}
    \item \hspace{1in}\\\input{../activity-snippets/quiz-predicates-alternating-quantifiers-rnalen.tex}
    \item \hspace{1in}\\\input{../activity-snippets/quiz-predicates-alternating-quantifiers-basecount.tex}
    \end{enumerate}
\end{enumerate}

\newpage

\subsection*{Week 5 at a glance}

\subsubsection*{We will be learning and practicing to:}
%data types
%proof signposts
%using proofs to evaluate
%universal generalization
%applying proof strategy
%logical structure to proof strategy
%identifying proof strategy in proof
\begin{itemize}

\item Clearly and unambiguously communicate computational ideas using appropriate formalism. Translate across levels of abstraction.
\begin{itemize}
   \item Translating between symbolic and English versions of statements using precise mathematical language
    \item Using appropriate signpost words to improve readability of proofs, including 'arbitrary' and 'assume'
\end{itemize}

\item Know, select and apply appropriate computing knowledge and problem-solving techniques. Reason about computation and systems. Use mathematical techniques to solve problems. Determine appropriate conceptual tools to apply to new situations. Know when tools do not apply and try different approaches. Critically analyze and evaluate candidate solutions.
\begin{itemize}
    \item Judging logical equivalence of compound propositions using symbolic manipulation with known equivalences, including DeMorgan's Law
    \item Writing the converse, contrapositive, and inverse of a given conditional statement
    \item Determining what evidence is required to establish that a quantified statement is true or false
    \item Evaluating quantified statements about finite and infinite domains
\end{itemize}

\item Apply proof strategies, including direct proofs and proofs by contradiction, and determine whether a proposed argument is valid or not.
\begin{itemize}
    \item Identifying the proof strategies used in a given proof
    \item Identifying which proof strategies are applicable to prove a given compound proposition based on its logical structure
    \item Carrying out a given proof strategy to prove a given statement
    \item Carrying out a universal generalization argument to prove that a universal statement is true
    \item Using proofs as knowledge discovery tools to decide whether a statement is true or false
\end{itemize}
\end{itemize}

\subsubsection*{TODO:}
\begin{list}
   {\itemsep2pt}
   \item Project due May 7, 2024. No review quiz this week.
   \item Test 1, in class this week, on Friday May 3, 2024.
   The test covers material in Weeks 1 through 4 and Monday of Week 5. 
   To study for the exam, we recommend reviewing class notes 
   (e.g. annotations linked on the class website, podcast, supplementary video from the class website), 
   reviewing homework (and its posted sample solutions), and in particular *working examples* 
   (extra examples in lecture notes, review quizzes, discussion examples) and getting feedback (office hours and Piazza).
   Some practice questions (and their solutions) are available on the class website, linked from Week 5 and from the Assignments page.
\end{list}

\newpage

\begin{comment}
Removed definition of insertion, deletion, mutation from Wednesday of Week 4 -- when do we need them?
\input{../activity-snippets/algorithm-redundancy.tex}
\newpage
\input{../activity-snippets/cartesian-product-definition.tex}
\input{../activity-snippets/algorithm-rna-mutation-insertion-deletion.tex}
\input{../activity-snippets/rna-mutation-insertion-deletion-example.tex}
\end{comment}

\section*{Week 5 Monday: Nested Quantifiers}
\input{../activity-snippets/rna-rnalen-basecount-definitions.tex}
%! app: Bioinformatics
%! outcome: TODOoutcome

{\bf Alternating nested quantifiers}



$$\forall s \in S ~\exists n \in \mathbb{N} ~(~basecount(~(s,\U)~) = n~)$$

In English: For each strand, there is a nonnnegative integer that counts the number of occurrences of $\U$ in that 
strand.\\

$$\exists n \in \mathbb{N} ~\forall s \in S ~(~basecount(~(s,\U)~) = n~)$$

In English: There is a nonnnegative integer that counts the number of occurrences of $\U$ in every 
strand.\\

\vfill

Are these statements true or false?

\newpage

$$\forall s \in S ~\exists b\in B ~(~basecount(~(s,b)~) = 3~)$$

In English: For each RNA strand there is a base that occurs 3 times in this strand.\\

Write the negation and use De Morgan's law to find a 
logically equivalent version where the negation is applied only to the 
$BC$ predicate (not next to a quantifier).

\vspace{60pt}


Is the original statement {\bf True} or {\bf False}?

\vfill

\subsection*{Proof strategies}
%! app: TODOapp
%! outcome: TODOoutcome

When a predicate $P(x)$ is over a {\bf finite} domain:
\begin{itemize}
\item To show that $\forall x  P(x)$ is true: check that $P(x)$ evaluates to $T$ at each domain element by evaluating over and over. 
This is called ``Proof of universal by {\bf exhaustion}".
\item To show that $\forall x  P(x)$ is false: find a {\bf counterexample}, a domain element where $P(x)$~evaluates~to~$F$.
\item To show that $\exists x  P(x)$ is true: find a {\bf witness}, a domain element where $P(x)$ evaluates to $T$.
\item To show that $\exists x  P(x)$ is false: check that $P(x)$ evaluates to $F$ at each domain element by evaluating over and over.
DeMorgan's Law gives that $\lnot \exists x P(x) ~~\equiv~~ \forall x \lnot P(x)$ so this amounts to a proof of universal by exhaustion.
\end{itemize}
%\input{../activity-snippets/proof-strategy-universal-exhaustion.tex}
\input{../activity-snippets/proof-strategy-universal-generalization.tex}
\newpage
\input{../activity-snippets/quiz-translating-counting-quantifiers.tex}

\newpage
\section*{Week 5 Wednesday: Proof Strategies and Sets}
\input{../activity-snippets/sets-equality-subset-definition.tex}
\input{../activity-snippets/proof-strategies-conditionals.tex}
\input{../activity-snippets/proof-strategies-proof-by-cases.tex}
\input{../activity-snippets/proof-strategies-ands.tex}
\input{../activity-snippets/sets-proof-strategies.tex}
\newpage
\input{../activity-snippets/sets-equality-example.tex}
\input{../activity-snippets/sets-basic-proofs.tex}
\vfill
\input{../activity-snippets/proofs-signposting.tex}
\newpage
\input{../activity-snippets/set-operations-union-intersection-powerset.tex}

\newpage

\subsection*{Week 6 at a glance}

\subsubsection*{We will be learning and practicing to:}
%data types
%proof signposts
%using proofs to evaluate
%universal generalization
%applying proof strategy
%logical structure to proof strategy
%identifying proof strategy in proof
\begin{itemize}

\item Clearly and unambiguously communicate computational ideas using appropriate formalism. Translate across levels of abstraction.
\begin{itemize}
   \item Translating between symbolic and English versions of statements using precise mathematical language
    \item Using appropriate signpost words to improve readability of proofs, including 'arbitrary' and 'assume'
\end{itemize}

\item Know, select and apply appropriate computing knowledge and problem-solving techniques. Reason about computation and systems. Use mathematical techniques to solve problems. Determine appropriate conceptual tools to apply to new situations. Know when tools do not apply and try different approaches. Critically analyze and evaluate candidate solutions.
\begin{itemize}
    \item Judging logical equivalence of compound propositions using symbolic manipulation with known equivalences, including DeMorgan's Law
    \item Writing the converse, contrapositive, and inverse of a given conditional statement
    \item Determining what evidence is required to establish that a quantified statement is true or false
    \item Evaluating quantified statements about finite and infinite domains
\end{itemize}

\item Apply proof strategies, including direct proofs and proofs by contradiction, and determine whether a proposed argument is valid or not.
\begin{itemize}
    \item Identifying the proof strategies used in a given proof
    \item Identifying which proof strategies are applicable to prove a given compound proposition based on its logical structure
    \item Carrying out a given proof strategy to prove a given statement
    \item Carrying out a universal generalization argument to prove that a universal statement is true
    \item Using proofs as knowledge discovery tools to decide whether a statement is true or false
\end{itemize}
\end{itemize}

\subsubsection*{TODO:}
\begin{list}
   {\itemsep2pt}
   \item Project due this week: May 8, 2024. 
   \item Review quiz based on class material each day (due Friday May 10, 2024).
\end{list}

\newpage

\section*{Week 6 Monday: Proofs for properties of sets and numbers}
%% Review sets and proof strategies by starting with this example:
\input{../activity-snippets/sets-proof-strategies.tex}
\input{../activity-snippets/sets-basic-proofs-operations.tex}

\subsection*{Facts about numbers}
\input{../activity-snippets/proof-strategies-road-map.tex}
\input{../activity-snippets/numbers-facts.tex}
\newpage
\subsection*{Factoring}
\input{../activity-snippets/factoring-definition.tex}
\input{../activity-snippets/factoring-translation-examples.tex}
\input{../activity-snippets/factoring-basic-claims.tex}
\input{../activity-snippets/factoring-basic-claims-continued.tex}
\input{../activity-snippets/factoring-even-odd.tex}
\input{../activity-snippets/prime-number-definition.tex}
\input{../activity-snippets/primes-basic-claims.tex}
\newpage


\section*{Week 6 Wednesday: Structural Induction}
\input{../activity-snippets/rna-rnalen-basecount-definitions.tex}

At this point, we've seen the proof strategies
\begin{multicols}{2}
    \begin{itemize}
        \item A {\bf counterexample} to prove that  $\forall x P(x)$ is {\bf false}.
        \item  A {\bf witness} to prove that  $\exists x P(x)$ is {\bf true}.
        \item {\bf Proof of universal by exhaustion} to prove that $\forall x \, P(x)$
    is true when $P$ has a finite domain
        \item  {\bf Proof by universal generalization} to prove that $\forall x \, P(x)$
    is true using an arbitrary element of the domain.
        \item To  prove  that $\exists x P(x)$ is {\bf false}, write the universal statement that is 
        logically equivalent to its negation and then prove it true using universal generalization.
        \item To prove that $p \land q$ is true, have two subgoals: 
        subgoal (1) prove $p$ is  true; and, subgoal (2) prove $q$ is true. To prove that $p \land q$ is false, it's enough to prove that $p$ is false.
     To prove that $p \land q$ is false, it's enough to prove that $q$ is false.
        \item Proof of conditional by {\bf direct proof}
        \item Proof of conditional by {\bf contrapositive proof}
        \item Proof of disjuction using equivalent conditional: To prove that the 
        disjunction $p \lor q$ is true, we can rewrite it equivalently as $\lnot p \to q$ and
        then use direct proof or contrapositive proof.
        \item {\bf Proof by cases}.
    \end{itemize}
\end{multicols}
\newpage
\input{../activity-snippets/alternating-quantifiers-proofs-rna-examples.tex}
\newpage
\input{../activity-snippets/structural-induction-motivating-example-rna.tex}
\input{../activity-snippets/proof-strategies-structural-induction.tex}
\newpage
\input{../activity-snippets/structural-induction-example-rnalen-basecount.tex}
\newpage

\section*{Week 6 Friday: Structural and Mathematical Induction}
\input{../activity-snippets/proofs-signposting-kinds-of-claims.tex}
%! app: TODOapp
%! outcome: induction flavors

\begin{center}
    \includegraphics[width=3in]{../../resources/images/robot-grid.png}
\end{center}
    
{\bf Theorem}: A robot on an infinite 2-dimensional integer grid starts at $(0,0)$ and at each step moves
to diagonally adjacent grid point. This robot can / cannot {\footnotesize({\it circle one})} reach $(1,0)$.


{\bf Definition} The set of positions the robot can visit  $Pos$ is defined by:
\[
\begin{array}{ll}
    \textrm{Basis Step: } & (0,0) \in Pos \\
    \textrm{Recursive Step: } & \textrm{If } (x,y) \in Pos \textrm{, then } \\
    &\phantom{(x+1, y+1), (x+1, y-1), (x-1, y-1), (x-1, y+1)} \textrm{ are also in } Pos
\end{array}
\]

{\it Example elements of $Pos$ are}:
\vspace{20pt}

{\bf Lemma}: $\forall (x,y) \in Pos~~( x+y \textrm{ is an even integer}~)$

{\it Why are we calling this a lemma?}


Proof of theorem using lemma: To show is $(1,0) \notin Pos$. Rewriting the lemma to explicitly 
restrict the domain of the universal, 
we have $\forall (x,y) ~(~ (x,y) \in Pos~~  \to ~~(x+y \textrm{ is an even integer})~)$.  Since
the universal is true, 
$ (~ (1,0) \in Pos~~ \to ~~(1+0 \textrm{ is an even integer})~)$ is a true statement.
Evaluating the conclusion of this conditional statement: 
By definition of long division, since $1 = 0 \cdot 2 + 1$ (where $0 \in \mathbb{Z}$ and 
$1 \in \mathbb{Z}$ and $0 \leq 1 < 2$ mean that $0$ is the quotient and $1$ is the remainder), $1 ~\textrm{\bf mod}~ 2 = 1$ which is not $0$ 
so the conclusion is false.  A true conditional with a false conclusion must have a false hypothesis: $(1,0) \notin Pos$, QED. $\square$

\vspace{20pt}

Proof of lemma by structural induction:

{\bf Basis Step}:

\vspace{100pt}


{\bf Recursive Step}:  Consider arbitrary $(x,y) \in Pos$.  To show is:
\[
(x+y \text{ is an even integer}) \to (\text{sum of coordinates of next position is even integer})
\]
Assume {\bf as the induction hypothesis, IH} that: 


\vspace{400pt}
\newpage
\input{../activity-snippets/structural-induction-example-sum-of-powers.tex}
\vfill
\input{../activity-snippets/proof-strategy-mathematical-induction.tex}
\newpage


\subsection*{Review Quiz}
\begin{enumerate}
\item Set properties
\begin{enumerate}
    \item \hspace{1in}\\ \input{../activity-snippets/quiz-sets-claims.tex}
    \item \hspace{1in}\\ \input{../activity-snippets/quiz-sets-proof-strategies.tex}
    \item \hspace{1in}\\ \input{../activity-snippets/quiz-sets-claims-subset-equality.tex}
\end{enumerate}
\item Number properties\begin{enumerate}
    \item \hspace{1in}\\ \input{../activity-snippets/quiz-factoring-quantifiers.tex}
    \item \hspace{1in}\\ \input{../activity-snippets/quiz-prime-formalizing-definition.tex}
\end{enumerate}
\item Structural induction
\begin{enumerate}
    \item \input{../activity-snippets/quiz-basecount-rnalen-induction.tex}
    \item \hspace{1in}\\ \input{../activity-snippets/quiz-robot-grid.tex}
\end{enumerate}
\item Mathematical induction
\begin{enumerate}
    \item \hspace{1in}\\ \input{../activity-snippets/quiz-comparing-structural-mathematical-induction.tex}
    \item \hspace{1in}\\ \input{../activity-snippets/quiz-exponential-factorial.tex}
\end{enumerate}
\item Midquarter feedback
%% Put midquarter feedback in Review quiz for Week 6%% TODO
\end{enumerate}


\newpage

\subsection*{Week 7 at a glance}

\subsubsection*{We will be learning and practicing to:}
%data types
%proof signposts
%using proofs to evaluate
%universal generalization
%applying proof strategy
%logical structure to proof strategy
%identifying proof strategy in proof
\begin{itemize}

\item Clearly and unambiguously communicate computational ideas using appropriate formalism. Translate across levels of abstraction.
\begin{itemize}
   \item Translating between symbolic and English versions of statements using precise mathematical language
    \item Using appropriate signpost words to improve readability of proofs, including 'arbitrary' and 'assume'
\end{itemize}

\item Know, select and apply appropriate computing knowledge and problem-solving techniques. Reason about computation and systems. Use mathematical techniques to solve problems. Determine appropriate conceptual tools to apply to new situations. Know when tools do not apply and try different approaches. Critically analyze and evaluate candidate solutions.
\begin{itemize}
    \item Judging logical equivalence of compound propositions using symbolic manipulation with known equivalences, including DeMorgan's Law
    \item Writing the converse, contrapositive, and inverse of a given conditional statement
    \item Determining what evidence is required to establish that a quantified statement is true or false
    \item Evaluating quantified statements about finite and infinite domains
\end{itemize}

\item Apply proof strategies, including direct proofs and proofs by contradiction, and determine whether a proposed argument is valid or not.
\begin{itemize}
    \item Identifying the proof strategies used in a given proof
    \item Identifying which proof strategies are applicable to prove a given compound proposition based on its logical structure
    \item Carrying out a given proof strategy to prove a given statement
    \item Carrying out a universal generalization argument to prove that a universal statement is true
    \item Using proofs as knowledge discovery tools to decide whether a statement is true or false
\end{itemize}
\end{itemize}

\subsubsection*{TODO:}
\begin{list}
   {\itemsep2pt}
   \item Homework assignment 4 (due Tuesday May 14, 2024)
   \item Review quiz based on class material each day (due Friday May 17, 2024).
   \item Homework assignment 5 (due Tuesday May 21, 2024)
\end{list}

\newpage

\section*{Week 7 Monday: Mathematical and Strong Induction}
\subsection*{Visualizing induction}
\input{../activity-snippets/induction-dominos.tex}
\input{../activity-snippets/proof-strategy-mathematical-induction.tex}
\input{../activity-snippets/proof-strategy-strong-induction.tex}
\input{../activity-snippets/binary-expansions-exist-proof.tex}

\subsubsection*{Representing positive integers with primes}
\input{../activity-snippets/fundamental-theorem-proof.tex}
\subsubsection*{Sending old-fashioned mail with postage stamps}
\input{../activity-snippets/strong-induction-making-change-proof-idea.tex}
\newpage
\subsubsection*{Finding a winning strategy for a game}
\input{../activity-snippets/strong-induction-nim.tex}
\newpage

\section*{Week 7 Wednesday: Recursive Data Structures}
\input{../activity-snippets/linked-lists-definition.tex}
\input{../activity-snippets/linked-lists-examples.tex}
\input{../activity-snippets/linked-list-length-definition.tex}
\vspace{50pt}
\input{../activity-snippets/linked-lists-prepend-definition.tex}
\vspace{50pt}
\input{../activity-snippets/linked-list-append-definition.tex}
\vspace{50pt}
\newpage
\input{../activity-snippets/linked-list-append-length-claim-proof.tex}
\newpage
\input{../activity-snippets/linked-list-example-each-length.tex}
\newpage

\section*{Week 7 Friday: Proof by Contradiction}
\input{../activity-snippets/proof-strategy-proof-by-contradiction.tex}
\subsection*{Least and greatest}
\input{../activity-snippets/least-greatest-proofs.tex}

\input{../activity-snippets/gcd-definition.tex}
\input{../activity-snippets/gcd-examples.tex}
\input{../activity-snippets/gcd-basic-claims.tex}
\input{../activity-snippets/gcd-lemma-relatively-prime.tex}

\newpage
\subsection*{Sets of numbers}

We've seen multiple representations of the set of positive integers
(using base expansions and using prime factorization). Now we're 
going to expand our attention to other sets of numbers as well.
\input{../activity-snippets/rational-numbers-definition.tex}
\input{../activity-snippets/sets-numbers-subsets.tex}
\input{../activity-snippets/proof-by-contradiction-irrational.tex}


\newpage

\subsection*{Review Quiz}
\begin{enumerate}
    \item Mathematical and strong induction for properties of numbers
    \begin{enumerate}
        \item \hspace{1in} \\ \input{../activity-snippets/quiz-binary-expansions-exist-invalid-proof.tex}
        \item \hspace{1in} \\ \input{../activity-snippets/quiz-making-change-proof-two-ways.tex}
    \end{enumerate}
    \item Winning strategy. \input{../activity-snippets/quiz-nim.tex}
    \item Linked lists. \input{../activity-snippets/quiz-linked-list-definitions.tex}
    \item Primes and divisors
    \begin{enumerate}
        \item \hspace{1in}\\ \input{../activity-snippets/quiz-prime-factorization.tex}
        \item \hspace{1in}\\ \input{../activity-snippets/quiz-no-greatest-prime.tex}
        \item \hspace{1in}\\ \input{../activity-snippets/quiz-calculating-gcd.tex}
    \end{enumerate}
    \item Proof strategies
    \begin{enumerate}
        \item \hspace{1in}\\ \input{../activity-snippets/quiz-choosing-proof-strategy.tex}
        \item \hspace{1in}\\ \input{../activity-snippets/quiz-odd-even-proofs.tex}
    \end{enumerate}
\end{enumerate}

\newpage

\subsection*{Week 8 at a glance}

\subsubsection*{We will be learning and practicing to:}
%classify cardinality
%important sets
%function and relation definitions
%functions for cardinality
%contradiction proofs
\begin{itemize}

\item Clearly and unambiguously communicate computational ideas using appropriate formalism. Translate across levels of abstraction.
\begin{itemize}
    \item Defining important sets of numbers, e.g. set of integers, set of rational numbers
    \item Defining functions using multiple representations
    \item Classifying sets into: finite sets, countably infinite sets, uncountable sets
    \item Using functions to compare cardinality of sets
\end{itemize}

\item Know, select and apply appropriate computing knowledge and problem-solving techniques. Reason about computation and systems. Use mathematical techniques to solve problems. Determine appropriate conceptual tools to apply to new situations. Know when tools do not apply and try different approaches. Critically analyze and evaluate candidate solutions.
\begin{itemize}
    \item Determining what evidence is required to establish that a quantified statement is true or false
    \item Evaluating quantified statements about finite and infinite domains
\end{itemize}

\item Apply proof strategies, including direct proofs and proofs by contradiction, and determine whether a proposed argument is valid or not.
\begin{itemize}
    \item Tracing and/or modifying a proof by contradiction
    \item Using proofs as knowledge discovery tools to decide whether a statement is true or false
\end{itemize}
\end{itemize}

\subsubsection*{TODO:}
\begin{list}
   {\itemsep2pt}
   \item Homework assignment 5 (due Tuesday May 21, 2024)
   \item Review quiz based on class material each day (due Friday May 24, 2024).
   \item Start reviewing for Test 2. The test is in class next week on Friday May 31, 2024.

\end{list}

\newpage

\section*{Week 8 Monday: Cardinality of Sets}

%\input{../activity-snippets/sets-numbers-subsets.tex}
\input{../activity-snippets/finite-sets-definition.tex}
\input{../activity-snippets/cardinality-motivation.tex}
\input{../activity-snippets/cardinality-rationale-for-functions.tex}
\input{../activity-snippets/musical-chairs-analogy.tex}
\newpage
\input{../activity-snippets/well-defined-functions.tex}
\newpage
\input{../activity-snippets/injective-function-definition.tex}
%! app: TODOapp
%! outcome: functions for cardinality, function and relation definitions

Informally, a function being one-to-one means ``no duplicate images''.

\phantom{Draw finite domain, finite codomain picture with duplicate image.}
\vspace{50pt}
\input{../activity-snippets/cardinality-lower-bound-definition.tex}
\input{../activity-snippets/injective-cardinality-musical-chairs.tex}
%\input{../activity-snippets/rna-injective-cardinality.tex}
\newpage
\input{../activity-snippets/surjective-function-definition.tex}
\input{../activity-snippets/surjective-functions-visually.tex}
\input{../activity-snippets/cardinality-upper-bound-definition.tex}
\input{../activity-snippets/surjective-cardinality-musical-chairs.tex}
\input{../activity-snippets/bijection-definition.tex}

%\input{../activity-snippets/rna-surjective-cardinality.tex}

\newpage

\section*{Week 8 Wednesday and Friday: Finite, countably infinite, and uncountable sets}
\subsection*{Cardinality of sets}
\input{../activity-snippets/cardinality-definition.tex}
\input{../activity-snippets/cardinality-caution.tex}
\input{../activity-snippets/cardinality-properties.tex}
\input{../activity-snippets/cantor-schroder-bernstein-theorem.tex}
\newpage
\input{../activity-snippets/countably-infinite-definition.tex}
\input{../activity-snippets/countably-infinite-examples-sets-of-numbers.tex}
\input{../activity-snippets/countably-infinite-examples-other-sets.tex}
\newpage
\subsection*{Cardinality categories}
\input{../activity-snippets/cardinality-categories.tex}
\input{../activity-snippets/cardinality-countability-lemmas.tex}

\subsection*{Are there always *bigger* sets?}
\input{../activity-snippets/cardinality-power-sets.tex}
\newpage
\subsection*{Countable vs.\ uncountable: sets of numbers}
\input{../activity-snippets/cardinality-rationals-reals.tex}
\subsection*{Other examples of uncountable sets}
\input{../activity-snippets/cardinality-uncountable-examples.tex}
\newpage

\section*{Review Quiz}
\begin{enumerate}
    \item Sets of numbers. \hspace{1in}\\ \input{../activity-snippets/quiz-rationals-proofs.tex}
    \item Finite vs. infinite. \hspace{1in}\\ \input{../activity-snippets/quiz-finite-sets.tex}
    \item Functions. \hspace{1in}\\ \input{../activity-snippets/quiz-injective-surjective.tex}
    \item Functions. \hspace{1in}\\ \input{../activity-snippets/quiz-cardinality-witnessing-functions-q1.tex} 
    \item Functions. \hspace{1in}\\ \input{../activity-snippets/quiz-cardinality-witnessing-functions-q2.tex} 
    \item Diagonalization. \hspace{1in}\\ \input{../activity-snippets/quiz-diagonalization.tex}
    \item Classifying cardinality. \hspace{1in}\\ \input{../activity-snippets/quiz-cardinality-classifying.tex}
\end{enumerate}
\newpage

\subsection*{Week 9 at a glance}

\subsubsection*{We will be learning and practicing to:}
%classify cardinality
%important sets
%function and relation definitions
%functions for cardinality
%div and mod
%divisibility and primes
%congruence mod n
%binary relations properties
%special binary relations
%graph representations of relations
\begin{itemize}

\item Clearly and unambiguously communicate computational ideas using appropriate formalism. Translate across levels of abstraction.
\begin{itemize}
    \item Defining important sets of numbers, e.g. set of integers, set of rational numbers
    \item Classifying sets into: finite sets, countably infinite sets, uncountable sets
    \item Defining functions, predicates, and binary relations using multiple representations
    \item Determining whether a given binary relation is symmetric, antisymmetric, reflexive, and/or transitive
    \item Determining whether a given binary relation is an equivalence relation and/or a partial order
\end{itemize}

\item Know, select and apply appropriate computing knowledge and problem-solving techniques. Reason about computation and systems. Use mathematical techniques to solve problems. Determine appropriate conceptual tools to apply to new situations. Know when tools do not apply and try different approaches. Critically analyze and evaluate candidate solutions.
\begin{itemize}
    \item Using the definitions of the div and mod operators on integers
    \item Using divisibility and primality predicates
    \item Applying the definition of congruence modulo n and modular arithmetic
\end{itemize}

\item Apply proof strategies, including direct proofs and proofs by contradiction, and determine whether a proposed argument is valid or not.
\begin{itemize}
    \item Using proofs as knowledge discovery tools to decide whether a statement is true or false
\end{itemize}
\end{itemize}

\subsubsection*{TODO:}
\begin{list}
   {\itemsep2pt}
   \item Review for Test 2. The test is in class on Friday May 31, 2024.
   \item Homework assignment 6 (due Thursday June 6, 2024).
\end{list}

\newpage
\section*{Week 9 Monday: No class in observance of Memorial Day}
\section*{Week 9 Wednesday: Binary relations}

\subsection*{Cardinality of sets: recap}
%! app: TODOapp
%! outcome: functions for cardinality, classify cardinality, important sets


The set of positive integers $\mathbb{Z}^{+}$ is countably infinite.

The set of integers $\mathbb{Z}$ is countably infinite and is a proper superset of $\mathbb{Z}^{+}$. 
In fact, the set difference 
$$\mathbb{Z} \setminus \mathbb{Z}^{+} = \{ x \in \mathbb{Z} \mid x \notin \mathbb{Z}^+\} 
= \{ x \in \mathbb{Z} \mid x \leq 0 \}$$ is countably infinite.

The set of rationals 
$\mathbb{Q} = \left\{ \frac{p}{q} \mid p \in \mathbb{Z}  \text{ and  } 
q  \in \mathbb{Z} \text{ and } q \neq  0 \right\}$ is countably infinite.


The set of real numbers $\mathbb{R}$ is uncountable. In fact, the closed 
interval $\{x \in \mathbb{R} ~|~ 0 \leq x \leq 1\}$, 
any other nonempty closed interval of real numbers whose endpoints are 
unequal, as well as the related intervals that exclude one or both of the endpoints are
each uncountable.
The set of {\bf irrational} numbers $\overline{\mathbb{Q}} = \mathbb{R} - \mathbb{Q}
= \{ x \in \mathbb{R} \mid x \notin \mathbb{Q} \}$ is uncountable.

\vfill

We can classify any set as 
\begin{itemize}
\item {\bf Finite} size. Fact: For each positive number $n$, for any sets $X$ and $Y$ each 
size $n$, there is a bijection between $X$ and $Y$.
\item {\bf Countably Infinite}. Fact: for any countably infinite sets $X$ and $Y$, there is a bijection
between $X$ and $Y$.
\item {\bf Uncountable}. Examples: $\mathcal{P}(\mathbb{N})$, the power set of any infinite set, 
the set of real numbers, any nonempty interval of real numbers. Fact: there are (many) examples 
of uncountable sets that do not have a bijection between them.
\end{itemize}
\newpage
\subsection*{Binary relations}
\input{../activity-snippets/binary-relation-definition.tex}
\vfill
%! app: TODOapp
%! outcome: graph representations of relations

For relation $R$ on a set $A$, we can represent this relation as a
{\bf graph}: a collection of nodes (vertices) and edges (arrows). The 
nodes of the graph are the elements of $A$ and 
there is an edge from $a$ to $b$ exactly when $(a,b) \in R$.

\vfill
\newpage
\input{../activity-snippets/binary-relation-examples.tex}
\newpage
\subsection*{Properties of binary relations}
\input{../activity-snippets/reflexive-relation-definition.tex}
\input{../activity-snippets/reflexive-relation-informally.tex}
\vfill
\input{../activity-snippets/symmetric-relation-definition.tex}
\input{../activity-snippets/symmetric-relation-informally.tex}
\vfill
\input{../activity-snippets/transitive-relation-definition.tex}
\input{../activity-snippets/transitive-relation-informally.tex}
\vfill
\input{../activity-snippets/antisymmetric-relation-definition.tex}
\input{../activity-snippets/antisymmetric-relation-informally.tex}
\vfill
\newpage
\input{../activity-snippets/binary-relation-properties-examples.tex}
\vfill

{\it Summary}: binary relations can be useful for organizing elements in a domain. 
Some binary relations have special properties that make them act like some familiar relations.
Equivalence relations (reflexive, symmetric, transitive binary relations) ``act like'' equals.
Partial orders (reflexive, antisymmetric, transitive binary relations) ``act like'' less than or equals to.


\newpage
\end{document}