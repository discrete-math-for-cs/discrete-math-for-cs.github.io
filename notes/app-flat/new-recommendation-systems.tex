\documentclass[12pt, oneside]{article}

\usepackage[letterpaper, scale=0.89, centering]{geometry}
\usepackage{fancyhdr}
\setlength{\parindent}{0em}
\setlength{\parskip}{1em}

\pagestyle{fancy}
\fancyhf{}
\renewcommand{\headrulewidth}{0pt}
\rfoot{\href{https://creativecommons.org/licenses/by-nc-sa/2.0/}{CC BY-NC-SA 2.0} Version \today~(\thepage)}

\usepackage{amssymb,amsmath,pifont,amsfonts,comment,enumerate,enumitem}
\usepackage{currfile,xstring,hyperref,tabularx,graphicx,wasysym}
\usepackage[labelformat=empty]{caption}
\usepackage{xcolor}
\usepackage{multicol,multirow,array,listings,tabularx,lastpage,textcomp,booktabs}

\lstnewenvironment{algorithm}[1][] {   
    \lstset{ mathescape=true,
        frame=tB,
        numbers=left, 
        numberstyle=\tiny,
        basicstyle=\rmfamily\scriptsize, 
        keywordstyle=\color{black}\bfseries,
        keywords={,procedure, div, for, to, input, output, return, datatype, function, in, if, else, foreach, while, begin, end, }
        numbers=left,
        xleftmargin=.04\textwidth,
        #1
    }
}
{}
\lstnewenvironment{java}[1][]
{   
    \lstset{
        language=java,
        mathescape=true,
        frame=tB,
        numbers=left, 
        numberstyle=\tiny,
        basicstyle=\ttfamily\scriptsize, 
        keywordstyle=\color{black}\bfseries,
        keywords={, int, double, for, return, if, else, while, }
        numbers=left,
        xleftmargin=.04\textwidth,
        #1
    }
}
{}

\newcommand\abs[1]{\lvert~#1~\rvert}
\newcommand{\st}{\mid}

\newcommand{\A}[0]{\texttt{A}}
\newcommand{\C}[0]{\texttt{C}}
\newcommand{\G}[0]{\texttt{G}}
\newcommand{\U}[0]{\texttt{U}}

\newcommand{\cmark}{\ding{51}}
\newcommand{\xmark}{\ding{55}}

 
\begin{document}
\begin{flushright}
    \StrBefore{\currfilename}{.}
\end{flushright} \section*{Ratings encoding}


In the table  below,  each row represents a user's ratings of movies: 
\cmark~(check) indicates the person liked the movie, \xmark~(x)
that they didn't, and $\bullet$ (dot) that they didn't rate it one way or 
another (neutral rating or didn't watch). Can encode
these ratings numerically with $1$ for \cmark~(check), $-1$ for \xmark~(x), 
and $0$ for $\bullet$ (dot).

\vfill

\begin{center}
\begin{tabular}{c|cccc||c}
Person & Dune & Oppenheimer & Barbie & Nimona & Ratings written as a $4$-tuple\\
\hline
$P_1$     & \xmark & $\bullet$ & \cmark & \phantom{$(-1, 0, 1, 1)$} \\
&&&& \\
$P_2$     & \cmark & \cmark & \xmark & \phantom{$(1, 1, -1, 1)$} \\
&&&& \\
$P_3$     & \cmark & \cmark & \cmark & \phantom{$(1, 1, 1, 1)$} \\
&&&& \\
$P_4$     & $\bullet$ & \xmark & \cmark &  \\
&&&& \\
$You$     &  &  &  &  \\
&&&& \\
\end{tabular}
\end{center} \vfill
\section*{Defining sets}


{\it To define sets:}

To define a set using {\bf roster method}, explicitly list its elements. That is,
start with $\{$ then list elements of 
the set separated by commas and close with $\}$.

\vfill

To define a set using {\bf set builder definition}, either form 
``The set of all $x$ from the universe $U$ such that $x$ is ..." by writing
\[\{x \in U \mid ...x... \}\]
or form ``the collection of all outputs of some operation when the input ranges over the universe $U$"
by writing
\[\{ ...x... \mid x\in U \}\]

\vfill

We use the symbol $\in$ as ``is an element of'' to indicate membership in a set.\\

\newpage 

{\bf Example sets}: For each of the following, identify whether it's defined using the roster method
or set builder notation and give an example element.

Can we infer the data type of the example element from the notation?

\begin{itemize}
    \item[]$\{ -1, 1\}$
    \vfill
    \item[]$\{0, 0 \}$
    \vfill
    \item[]$\{-1, 0, 1 \}$
    \vfill
    \item[]$\{(x,x,x) \mid x \in \{-1,0,1\} \}$
    \vfill
    \item[]$\{ \}$
    \vfill
    \item[]$\{ x \in \mathbb{Z} \mid x \geq 0 \}$
    \vfill
    \item[]$\{ x \in \mathbb{Z}  \mid x > 0 \}$
    \vfill
    \item[]$\{ \smile, \sun \}$
    \vfill
    \item[]$\{\A,\C,\U,\G\}$
    \vfill
    \item[]$\{\A\U\G, \U\A\G, \U\G\A, \U\A\A \}$
    \vfill
\end{itemize}
 \vfill
\section*{Defining functions ratings}


Recall our representation of Netflix users' ratings of movies as $n$-tuples, where
$n$ is the number of movies in the database. 
Each component of the $n$-tuple is $-1$ (didn't like the movie), $0$ 
(neutral rating or didn't watch the movie), or $1$ (liked the movie).

Consider the ratings $P_1 = (-1, 0, 1, 0)$, $P_2 = (1, 1, -1, 0)$, $P_3 = (1, 1, 1, 0)$,
$P_4 = (0,-1,1, 0)$


Which of $P_1$, $P_2$, $P_3$ has movie preferences most similar to $P_4$?

One approach to answer this question: use {\bf functions} to quantify difference among user preferences.

For example, consider the function 
$d_0: \{-1,0,1\}^4 \times \{-1,0,1\}^4 \to \mathbb{R}$
given by
\[
d_0 (~(~ (x_1, x_2, x_3, x_4), (y_1, y_2, y_3, y_4) ~) ~) = \sqrt{ (x_1 - y_1)^2 + (x_2 - y_2)^2 + (x_3 -y_3)^2 + (x_4 -y_4)^2}
\]


\vfill
\vfill

\begin{comment}
    

{\it Extra example:} A new movie is released, and $P_1$ and $P_2$ watch it before $P_3$, and give it
ratings; $P_1$ gives \cmark~and $P_2$ gives \xmark.
Should this movie be recommended to $P_3$? Why or why not?

{\it Extra example:} Define a new function that could be used to compare the $4$-tuples of ratings encoding
movie preferences now that there are four movies in the database.

\vfill
\end{comment}
\newpage \vfill
\end{document}