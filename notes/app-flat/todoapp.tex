\documentclass[12pt, oneside]{article}

\usepackage[letterpaper, scale=0.89, centering]{geometry}
\usepackage{fancyhdr}
\setlength{\parindent}{0em}
\setlength{\parskip}{1em}

\pagestyle{fancy}
\fancyhf{}
\renewcommand{\headrulewidth}{0pt}
\rfoot{\href{https://creativecommons.org/licenses/by-nc-sa/2.0/}{CC BY-NC-SA 2.0} Version \today~(\thepage)}

\usepackage{amssymb,amsmath,pifont,amsfonts,comment,enumerate,enumitem}
\usepackage{currfile,xstring,hyperref,tabularx,graphicx,wasysym}
\usepackage[labelformat=empty]{caption}
\usepackage{xcolor}
\usepackage{multicol,multirow,array,listings,tabularx,lastpage,textcomp,booktabs}

\lstnewenvironment{algorithm}[1][] {   
    \lstset{ mathescape=true,
        frame=tB,
        numbers=left, 
        numberstyle=\tiny,
        basicstyle=\rmfamily\scriptsize, 
        keywordstyle=\color{black}\bfseries,
        keywords={,procedure, div, for, to, input, output, return, datatype, function, in, if, else, foreach, while, begin, end, }
        numbers=left,
        xleftmargin=.04\textwidth,
        #1
    }
}
{}
\lstnewenvironment{java}[1][]
{   
    \lstset{
        language=java,
        mathescape=true,
        frame=tB,
        numbers=left, 
        numberstyle=\tiny,
        basicstyle=\ttfamily\scriptsize, 
        keywordstyle=\color{black}\bfseries,
        keywords={, int, double, for, return, if, else, while, }
        numbers=left,
        xleftmargin=.04\textwidth,
        #1
    }
}
{}

\newcommand\abs[1]{\lvert~#1~\rvert}
\newcommand{\st}{\mid}

\newcommand{\A}[0]{\texttt{A}}
\newcommand{\C}[0]{\texttt{C}}
\newcommand{\G}[0]{\texttt{G}}
\newcommand{\U}[0]{\texttt{U}}

\newcommand{\cmark}{\ding{51}}
\newcommand{\xmark}{\ding{55}}

 
\begin{document}
\begin{flushright}
    \StrBefore{\currfilename}{.}
\end{flushright} \section*{Algorithm definition}


\fbox{\parbox{\textwidth}{{\bf New!} An algorithm is a finite sequence of precise instructions for solving a problem.
\hfill
}}

Algorithms can be expressed in English or in more formalized descriptions like pseudocode or fully executable programs.


Sometimes, we can define algorithms whose output matches the 
rule for a function we already care about. Consider the (integer) logarithm function
\[
logb  : \{b \in \mathbb{Z} \mid b >1 \}  \times \mathbb{Z}^+ ~~\to~~ \mathbb{N}
\]
defined by 
\[
logb (~ (b,n)~) =  \text{greatest integer } y \text{ so that } b^y  \text{ is less than or equal to } n 
\]



\begin{algorithm}[caption={Calculating integer part of base $b$ logarithm}]
   procedure $logb$($b$,$n$: positive integers with $b > 1$)
   $i$ := $0$
   while $n$ > $b-1$
     $i$ := $i + 1$
     $n$ := $n$ div $b$
   return $i$ $\{ i$ holds the integer part of the base $b$ logarithm of $n\}$
\end{algorithm} 
Trace this algorithm with inputs $b=3$ and $n=17$

\vfill
  \begin{tabular}{l|c|c|c|c}
  & $b$ & $n$ & $i$  & $n > b-1$?\\
  \hline 
  Initial value & ~$3$~ & $17$ & \phantom{$0$} & \phantom{~Yes~}\\
  After 1 iteration  & \phantom{$3$} & \phantom{$5$} & \phantom{$1$} & \phantom{T}\\
  After 2 iterations & \phantom{$3$} & \phantom{$1$} & \phantom{$2$} & \phantom{F}\\
  After 3 iterations &  &  & & \\
  &&&&\\
  \end{tabular}

\vfill

\vfill

Compare: does the output match the rule for the (integer) logarithm function?


\newpage \vfill
\section*{Fixed width definition}


{\bf Definition} For $b$ an integer greater than $1$, $w$ a positive integer, 
and $n$ a nonnegative integer
$\underline{\phantom{\hspace{1in}}}$, ~
the {\bf base $b$ fixed-width $w$ expansion of $n$}  is
\[
(a_{w-1} \cdots a_1 a_0)_{b,w}
\]
where  $a_0, a_1, \ldots, a_{w-1}$ are nonnegative integers less than $b$ and
\[
n =  \sum_{i=0}^{w-1} a_{i} b^{i}
\]
 \vfill
\section*{Fixed width example}


\begin{center}
    \begin{tabular}{|c|c|c|c|c|}
    \hline
    Decimal &  Binary  & Binary fixed-width $10$& Binary fixed-width $7$ & Binary fixed-width $4$\\
    $b=10$ & $b=2$ & $b=2$, $w =  10$& $b=2$, $w =  7$& $b=2$, $w =  4$ \\
    \hline 
    &&&&  \\
    $(20)_{10}$&\phantom{$(10100)_{2}$\qquad\qquad}&&  &\\
    &&&&  \\
\hline
    \end{tabular}
    \end{center}
 \vfill
\section*{Fixed width fractional definition}


{\bf Definition} For $b$ an integer greater than $1$, $w$ a positive integer, 
$w'$ a positive  integer, and $x$ a real number the {\bf base $b$ fixed-width 
expansion of $x$ with integer part width $w$  and fractional part width $w'$} is
$(a_{w-1} \cdots a_1 a_0 .  c_{1} \cdots c_{w'})_{b,w,w'}$
where  $a_0, a_1, \ldots, a_{w-1}, c_1, \ldots, c_{w'}$ are nonnegative integers less than $b$ and
$$x \geq \sum_{i=0}^{w-1} a_{i} b^{i} + \sum_{j=1}^{w'}  c_{j} b^{-j} \hfill
\textrm{\qquad and \qquad}
\hfill x < \sum_{i=0}^{w-1} a_{i} b^{i} + \sum_{j=1}^{w'} c_{j} b^{-j} + b^{-w'}$$

\begin{center}
\begin{tabular}{|c|p{5in}|}
\hline
& \\
$3.75$  in fixed-width binary,& \\
integer part width $2$,&\\
 fractional part width $8$ & \\
& \\
& \\
& \\
& \\
\hline
& \\
$0.1$  in fixed-width binary, & \\
integer part width $2$, &\\
 fractional part width $8$ & \\
 & \\
 & \\
 & \\
 & \\
 \hline
\end{tabular}
\end{center}

\vfill

\includegraphics[width=2in]{../../resources/images/ArithmeticDemo.png}

Note: Java uses floating point, not fixed width representation, but similar rounding errors appear in both.
 \vfill
\section*{Negative int expansions}


{\bf Representing negative integers in binary}: Fix a positive integer  width for the representation  $w$, $w >1$.

\begin{tabular}{|cc|p{3.4in}|p{3.7in}|}
\hline
& & To  represent a positive integer $n$ & To represent a negative integer $-n$\\
\hline
&& &  \\
&\parbox[t]{2mm}{\multirow{4}{*}{\rotatebox[origin=c]{90}{Sign-magnitude}}} &
$[ 0a_{w-2} \cdots a_0]_{s,w}$, where $n =  (a_{w-2} \cdots a_0)_{2,w-1}$& 
$[1a_{w-2} \cdots a_0]_{s,w}$
, where $n =  (a_{w-2} \cdots a_0)_{2,w-1}$\\
&& & \\
&& Example $n=17$, $w=7$:  & Example $-n=-17$, $w=7$: \\
&& & \\
&& & \\
&& & \\
&& & \\
&& & \\
&& & \\
&& & \\
\hline
&&  &  \\
&\parbox[t]{2mm}{\multirow{4}{*}{\rotatebox[origin=c]{90}{2s complement}}} &
$[0a_{w-2} \cdots a_0]_{2c,w}$, where $n =  (a_{w-2} \cdots a_0)_{2,w-1}$& $[1a_{w-2} \cdots a_0]_{2c,w}$, where $2^{w-1} - n =  (a_{w-2} \cdots a_0)_{2,w-1}$\\
&& & \\
&& Example $n=17$, $w=7$:  & Example $-n=-17$, $w=7$: \\
&& & \\
&& & \\
&& & \\
&& & \\
&& & \\
&& & \\
&& & \\
\hline
\end{tabular} \vfill
\section*{Calculating 2s complement}


For positive integer $n$, to represent $-n$ in 
$2$s complement with width $w$,
\begin{itemize}
    \item Calculate $2^{w-1} - n$, convert 
    result to binary fixed-width $w-1$, pad 
    with leading $1$, or
    \item Express $-n$ as a sum of powers of $2$, 
    where the leftmost $2^{w-1}$ is negative weight, or
    \item Convert $n$ to binary fixed-width $w$, 
    flip bits, add 1 (ignore overflow)
\end{itemize}

{\it Challenge: use definitions to explain why
each of these approaches works.} \vfill
\section*{Representing zero}


{\bf Representing $0$}:

So far, we have representations for
positive and negative integers. What about $0$?

\begin{tabular}{|cc|p{3.4in}|p{3.7in}|}
   \hline
   & & To  represent a {\bf non-negative} integer $n$ & To represent a {\bf non-positive} integer $-n$\\
   \hline
   && &  \\
   &\parbox[t]{2mm}{\multirow{4}{*}{\rotatebox[origin=c]{90}{Sign-magnitude}}} &
   $[ 0a_{w-2} \cdots a_0]_{s,w}$, where $n =  (a_{w-2} \cdots a_0)_{2,w-1}$& 
   $[1a_{w-2} \cdots a_0]_{s,w}$
   , where $n =  (a_{w-2} \cdots a_0)_{2,w-1}$\\
   && & \\
   && Example $n=0$, $w=7$:  & Example $-n=0$, $w=7$: \\
   && & \\
   && & \\
   && & \\
   && & \\
   && & \\
   && & \\
   && & \\
   && & \\
   && & \\
   && & \\
\hline
   &&  &  \\
   &\parbox[t]{2mm}{\multirow{4}{*}{\rotatebox[origin=c]{90}{2s complement}}} &
   $[0a_{w-2} \cdots a_0]_{2c,w}$, where $n =  (a_{w-2} \cdots a_0)_{2,w-1}$& $[1a_{w-2} \cdots a_0]_{2c,w}$, where $2^{w-1} - n =  (a_{w-2} \cdots a_0)_{2,w-1}$\\
   && & \\
   && Example $n=0$, $w=7$:  & Example $-n=0$, $w=7$: \\
   && & \\
   && & \\
   && & \\
   && & \\
   && & \\
   && & \\
   && & \\
   && & \\
   && & \\
   && & \\
\hline
\end{tabular} \vfill
\section*{Netflix intro}


What data should we encode about each Netflix account holder to help us make effective recommendations?

\vfill
\vfill

In machine learning, clustering can be used to group similar data for prediction and recommendation.  For example,
each Netflix user's viewing history can be represented as a $n$-tuple indicating their preferences about
movies in the database, where $n$ is the number of movies in the database.  People with similar tastes in movies can then be clustered to provide recommendations
of movies for one another.  Mathematically, clustering is based on a notion of distance between pairs of $n$-tuples.
 \vfill
\section*{Data types}


\begin{center}
    \begin{tabular}{p{4.6in}p{2.6in}}
    {\bf  Term} & {\bf Examples}:\\
    &  (add additional examples from class)\\
    \hline 
    {\bf set} \newline
    unordered collection of elements & $7 \in \{43, 7, 9 \}$ \qquad $2 \notin \{43, 7, 9 \}$ \\
    {\it repetition doesn't matter} & \\
    {\it Equal sets agree on membership of all elements}& \\
    \hline
    {\bf $n$-tuple} \newline
    ordered sequence of elements with $n$ ``slots" ($n >0$) & \\
    {\it repetition matters, fixed length} &\\
    {\it Equal $n$-tuples have corresponding components equal}& \\
    \hline
    {\bf string} \newline
    ordered finite sequence of elements each from specified
    set (called the alphabet over which the string is defined)& \\
    {\it repetition matters, arbitrary finite length} &\\
    {\it Equal strings have same length and corresponding characters equal}
    \end{tabular}
\end{center}

{\it Special cases}: 

When $n=2$, the 2-tuple is called an {\bf ordered pair}.

A string of length $0$ is called the {\bf empty string} and is denoted $\lambda$.

A set with no elements is called the {\bf empty set} and is denoted $\{\}$ or $\emptyset$. \vfill
\section*{Set operations}


\fbox{\parbox{\textwidth}{To define a set we can use the roster method, set builder notation, a recursive definition, 
and also we can apply a set operation to other sets. \\

{\bf New! Cartesian product of sets} and {\bf set-wise concatenation of sets of strings}\\


{\bf Definition}: Let $X$ and $Y$ be sets.  The {\bf Cartesian product} of $X$ and $Y$, denoted
$X \times Y$, is the set of all ordered pairs $(x,y)$ where $x \in X$ and $y \in Y$
\[
X \times Y = \{ (x,y) \mid x \in X \text{ and } y \in Y \}
\]

Conventions: (1) Cartesian products can be chained together to result in sets of $n$-tuples and 
(2) When we form the Cartesian product of a set with itself $X \times X$ we can denote that set as 
$X^2$, or $X^n$ for the Cartesian product of a set with itself $n$ times for a positive integer $n$.\\

{\bf Definition}: Let $X$ and $Y$ be sets of strings over the same alphabet. The {\bf set-wise concatenation} 
of $X$ and $Y$, denoted $X \circ Y$, is the set of all results of string concatenation $xy$ where $x \in X$ 
and $y \in Y$
\[
X \circ Y = \{ xy \mid x \in X \text{ and } y \in Y \}
\]
}}

{\bf Pro-tip}: the meaning of writing one element next to another like $xy$ depends on the data-types of $x$ and 
$y$. When $x$ and $y$ are strings, the convention is that $xy$ is the result of string concatenation. 
When $x$ and $y$ are numbers, the convention is that $xy$ is the result of multiplication. This is 
(one of the many reasons) why is it very important to declare the data-type of variables before we use them.

{\it Fill in the missing entries in the table}:

\begin{center}
\begin{tabular}{cc}
{\bf  Set} & {\bf Example elements in this set and their data type}:\\
\hline 
& \\
$B$ &\A \qquad \C \qquad \G \qquad \U \\
& \\
\hline
& \\
\phantom{$B \times B$} & $(\A, \C)$ \qquad $(\U, \U)$\\
& \\
\hline
& \\
$B \times \{-1,0,1\}$ & \\
& \\
\hline
& \\
$\{-1,0,1\} \times B$ & \\
& \\
\hline
& \\
\phantom{$\{-1,0,1\} \times \{-1,0,1\}  \times \{-1,0,1\} $} & \qquad $(0,0,0)$ \\
& \\
\hline
& \\
$ \{\A, \C, \G, \U \} \circ  \{\A, \C, \G, \U \}$& \\
& \\
\hline
& \\
\phantom{$\{G\} \circ \{G\} \circ \{G\}$} & \qquad $\G\G\G\G$ \\
& \\
\hline

\end{tabular}
\end{center}

\vfill \vfill
\section*{Defining functions}


\fbox{\parbox{\textwidth}{{\bf New! Defining functions} A function is defined by its (1) domain, 
(2) codomain, and (3) rule assigning each 
element in the domain exactly one element in the codomain.\\

The domain and codomain are nonempty sets.

The rule can be depicted as a table, formula, piecewise definition, or English description.

The notation is 
\begin{center}
    ``Let the function FUNCTION-NAME: DOMAIN $\to$ CODOMAIN be given by \\
FUNCTION-NAME(x) = \ldots for every $x \in DOMAIN$''.
\end{center}

or 
\begin{center}
    ``Consider the function FUNCTION-NAME: DOMAIN $\to$ CODOMAIN defined as  \\
FUNCTION-NAME(x) = \ldots for every $x \in DOMAIN$''.
\end{center}
}}

\vfill
\newpage

Example: The absolute value function 

{\bf Domain}

{\bf Codomain}

{\bf Rule}

\vfill 
 \vfill
\section*{Defining functions recursively}


When the domain of a function is a {\it recursively defined set}, the rule assigning 
images to domain elements (outputs) can also be defined recursively.

Recall: The set of RNA strands $S$ is defined (recursively) by:
\[
\begin{array}{ll}
\textrm{Basis Step: } & \A \in S, \C \in S, \U \in S, \G \in S \\
\textrm{Recursive Step: } & \textrm{If } s \in S\textrm{ and }b \in B \textrm{, then }sb \in S
\end{array}
\]
where $sb$ is string concatenation.

{\bf Definition} (Of a function, recursively) A function \textit{rnalen} that computes the length of RNA strands in $S$ is defined by:
\[
\begin{array}{llll}
& & \textit{rnalen} : S & \to \mathbb{Z}^+ \\
\textrm{Basis Step:} & \textrm{If } b \in B\textrm{ then } & \textit{rnalen}(b) & = 1 \\
\textrm{Recursive Step:} & \textrm{If } s \in S\textrm{ and }b \in B\textrm{, then  } & \textit{rnalen}(sb) & = 1 + \textit{rnalen}(s)
\end{array}
\]

The domain of \textit{rnalen} is \phantom{$S$}\\

The codomain of \textit{rnalen} is \phantom{$\mathbb{Z}^+$}\\

Example function application:
\[
rnalen(\A\C\U) = \phantom{1+ rnalen(\A\C) = 1 + (1 + rnalen(\A) ) = 1 + ( 1 + 1) = 3}
\]

\vfill

{\it Example}: A function \textit{basecount} that computes the number of a given base 
$b$ appearing in a RNA strand $s$ is defined recursively:
    
\[
\begin{array}{llll}
& & \textit{basecount} : S \times B & \to \mathbb{N} \\
\textrm{Basis Step:} &  \textrm{If } b_1 \in B, b_2 \in B & \textit{basecount}(~(b_1, b_2)~) & =
        \begin{cases}
            1 & \textrm{when } b_1 = b_2 \\
            0 & \textrm{when } b_1 \neq b_2 \\
        \end{cases} \\
\textrm{Recursive Step:} & \textrm{If } s \in S, b_1 \in B, b_2 \in B &\textit{basecount}(~(s b_1, b_2)~) & =
        \begin{cases}
            1 + \textit{basecount}(~(s, b_2)~) & \textrm{when } b_1 = b_2 \\
            \textit{basecount}(~(s, b_2)~) & \textrm{when } b_1 \neq b_2 \\
        \end{cases}
\end{array}
\]

\begin{comment}
$basecount(~(\A\C\U,\A)~) = basecount( ~(\A\C, \A)~) = basecount(~(\A, \A)~) = 1$\\


$basecount(~(\A\C\U,\G)~) = basecount( ~(\A\C, \G)~) = basecount(~(\A, \G)~) = 0$\\


\vfill
{\it Extra example}: The function which outputs $2^n$ when given a nonnegative integer $n$ can be defined recursively, 
because its domain is the set of nonnegative integers.

\vfill
\end{comment}
 \vfill
\end{document}