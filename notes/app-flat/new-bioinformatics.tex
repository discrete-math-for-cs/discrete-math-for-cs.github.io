\documentclass[12pt, oneside]{article}

\usepackage[letterpaper, scale=0.89, centering]{geometry}
\usepackage{fancyhdr}
\setlength{\parindent}{0em}
\setlength{\parskip}{1em}

\pagestyle{fancy}
\fancyhf{}
\renewcommand{\headrulewidth}{0pt}
\rfoot{\href{https://creativecommons.org/licenses/by-nc-sa/2.0/}{CC BY-NC-SA 2.0} Version \today~(\thepage)}

\usepackage{amssymb,amsmath,pifont,amsfonts,comment,enumerate,enumitem}
\usepackage{currfile,xstring,hyperref,tabularx,graphicx,wasysym}
\usepackage[labelformat=empty]{caption}
\usepackage{xcolor}
\usepackage{multicol,multirow,array,listings,tabularx,lastpage,textcomp,booktabs}

\lstnewenvironment{algorithm}[1][] {   
    \lstset{ mathescape=true,
        frame=tB,
        numbers=left, 
        numberstyle=\tiny,
        basicstyle=\rmfamily\scriptsize, 
        keywordstyle=\color{black}\bfseries,
        keywords={,procedure, div, for, to, input, output, return, datatype, function, in, if, else, foreach, while, begin, end, }
        numbers=left,
        xleftmargin=.04\textwidth,
        #1
    }
}
{}
\lstnewenvironment{java}[1][]
{   
    \lstset{
        language=java,
        mathescape=true,
        frame=tB,
        numbers=left, 
        numberstyle=\tiny,
        basicstyle=\ttfamily\scriptsize, 
        keywordstyle=\color{black}\bfseries,
        keywords={, int, double, for, return, if, else, while, }
        numbers=left,
        xleftmargin=.04\textwidth,
        #1
    }
}
{}

\newcommand\abs[1]{\lvert~#1~\rvert}
\newcommand{\st}{\mid}

\newcommand{\A}[0]{\texttt{A}}
\newcommand{\C}[0]{\texttt{C}}
\newcommand{\G}[0]{\texttt{G}}
\newcommand{\U}[0]{\texttt{U}}

\newcommand{\cmark}{\ding{51}}
\newcommand{\xmark}{\ding{55}}

 
\begin{document}
\begin{flushright}
    \StrBefore{\currfilename}{.}
\end{flushright} \section*{Algorithm rna mutation insertion deletion}


Recall that $S$ is defined as the set of all RNA strands, nonempty strings made of the bases in 
$B = \{\A,\U,\G,\C\}$. 
We define the functions 
\[
  \textit{mutation}: S \times \mathbb{Z}^+ \times B \to S
\qquad \qquad
  \textit{insertion}: S \times \mathbb{Z}^+ \times B \to S
\]
\[
  \textit{deletion}: \{ s\in S \mid rnalen(s) > 1\} \times \mathbb{Z}^+ \to S
  \qquad \qquad \textrm{with rules}
\]

\begin{algorithm}
procedure $\textit{mutation}$($b_1\cdots b_n$: $\textrm{a RNA strand}$, $k$: $\textrm{a  positive integer}$, $b$: $\textrm{an  element of } B$)
for $i$ := $1$ to $n$
  if $i$ = $k$
    $c_i$ := $b$
  else
    $c_i$ := $b_i$
return $c_1\cdots c_n$ $\{ \textrm{The return value is a RNA strand made of the } c_i \textrm{ values}\}$
\end{algorithm}

\begin{algorithm}
procedure $\textit{insertion}$($b_1\cdots b_n$: $\textrm{a RNA strand}$, $k$: $\textrm{a  positive integer}$, $b$: $\textrm{an  element of } B$)
if $k > n$
  for $i$ := $1$ to $n$
    $c_i$ := $b_i$
  $c_{n+1}$ := $b$
else 
  for $i$ := $1$ to $k-1$
    $c_i$ := $b_i$
  $c_k$ := $b$
  for $i$ := $k+1$ to $n+1$
    $c_i$ := $b_{i-1}$
return $c_1\cdots c_{n+1}$ $\{ \textrm{The return value is a RNA strand made of the } c_i \textrm{ values}\}$
\end{algorithm}

\begin{algorithm}
procedure $\textit{deletion}$($b_1\cdots b_n$: $\textrm{a RNA strand with } n>1$, $k$: $\textrm{a  positive integer}$)
if $k > n$
  $m$ := $n$
  for $i$ := $1$ to $n$
    $c_i$ := $b_i$
else
  $m$ := $n-1$
  for $i$ := $1$ to $k-1$ 
    $c_i$ := $b_i$
  for $i$ := $k$ to $n-1$
    $c_i$ := $b_{i+1}$
return $c_1\cdots c_m$ $\{ \textrm{The return value is a RNA strand made of the } c_i \textrm{ values}\}$
\end{algorithm}
 \vfill
\section*{Alternating quantifiers order rna examples}


{\bf Alternating nested quantifiers}



$$\forall s \in S ~\exists n \in \mathbb{N} ~(~basecount(~(s,\U)~) = n~)$$

In English: For each strand, there is a nonnnegative integer that counts the number of occurrences of $\U$ in that 
strand.\\

$$\exists n \in ~\forall s \in S ~\mathbb{N} ~(~basecount(~(s,\U)~) = n~)$$

In English: There is a nonnnegative integer that counts the number of occurrences of $\U$ in every 
strand.\\

\vfill

Are these statements true or false?

\newpage

$$\forall s \in S ~\exists b\in B ~(~basecount(~(s,b)~) = 3~)$$

In English: For each RNA strand there is a base that occurs 3 times in this strand.\\

Write the negation and use De Morgan's law to find a 
logically equivalent version where the negation is applied only to the 
$BC$ predicate (not next to a quantifier).

\vspace{60pt}


Is the original statement {\bf True} or {\bf False}?

\vfill
 \vfill
\section*{Alternating quantifiers strategies rna examples}


{\bf Alternating nested quantifiers}

\fbox{\parbox{\textwidth}{
$$\forall s \in S ~\exists b\in B ~(~basecount(~(s,b)~) = 3~)$$

In English: For each RNA strand there is a base that occurs 3 times in this strand.\\

Write the negation and use De Morgan's law to find a 
logically equivalent version where the negation is applied only to the 
$BC$ predicate (not next to a quantifier).

\vspace{60pt}


Is the original statement {\bf True} or {\bf False}?

}}

\vfill

\fbox{\parbox{\textwidth}{

$$\exists s \in S ~\forall b \in B ~\exists n \in \mathbb{N} ~(~basecount(~(s,b)~) = n~)$$

In English: There is an RNA strand so that for each base there is some nonnegative
integer that counts the number of occurrences of that base in this strand.\\

Write the negation and use De Morgan's law to find a 
logically equivalent version where the negation is applied only to the 
$BC$ predicate (not next to a quantifier).

\vspace{60pt}


Is the original statement {\bf True} or {\bf False}?

}}

\vfill
 \vfill
\section*{Defining functions more examples}


Let's practice with functions related to some of our applications so far.

Recall: We model the collection of user ratings of the four movies Dune, Oppenheimer, Barbie, Nimona as the set
$\{-1,0,1\}^4$ . One function that compares pairs of ratings is
$$d_0: \{-1,0,1\}^4 \times \{-1,0,1\}^4 \to \mathbb{R}$$
given by
\[
d_0 (~(~ (x_1, x_2, x_3, x_4), (y_1, y_2, y_3, y_4) ~) ~) = \sqrt{ (x_1 - y_1)^2 + (x_2 - y_2)^2 + (x_3 -y_3)^2 + (x_4 -y_4)^2}
\]

Notice: any ordered pair of ratings is an okay input to $d_0$.

Notice: there are (at most) 
\[
(3 \cdot 3 \cdot 3 \cdot 3)\cdot (3 \cdot 3 \cdot 3 \cdot 3) = 3^8 = 6561
\]
many pairs of ratings. There are therefore lots and lots of real numbers that are not the output of $d_0$.

\vfill

Recall: RNA is made up of strands of four different bases that encode genomic information
in specific ways.\\
The bases are elements of the set 
$B  = \{\A, \C, \U, \G \}$. The set of RNA strands $S$ is defined (recursively) by:
\[
\begin{array}{ll}
\textrm{Basis Step: } & \A \in S, \C \in S, \U \in S, \G \in S \\
\textrm{Recursive Step: } & \textrm{If } s \in S\textrm{ and }b \in B \textrm{, then }sb \in S
\end{array}
\]
where $sb$ is string concatenation.

\vfill
\newpage
{\bf Pro-tip}: informal definitions sometime use $\cdots$ to indicate ``continue the pattern''. Often, 
to make this pattern precise we use recursive definitions.

\vspace{-20pt}

\begin{center}
\begin{tabular}{p{0.65in}ccp{2.4in}p{2.4in}}
{\scriptsize {\bf Name}} & {\scriptsize {\bf  Domain}} & {\scriptsize {\bf Codomain}} & {\scriptsize {\bf Rule}} &{\scriptsize {\bf Example}}\\
\hline 
$rnalen$ & $S$ & $\mathbb{Z}^+$ & 
    {\begin{align*}    
    &\textrm{Basis Step:} \\
    &\textrm{If } b \in B\textrm{ then } \textit{rnalen}(b) = 1 \\
    &\textrm{Recursive Step:}\\
    &\textrm{If } s \in S\textrm{ and } b \in B\textrm{, then  }\\
    &\textit{rnalen}(sb) = 1 + \textit{rnalen}(s)
    \end{align*}} & 
    {\begin{align*}
        rnalen(\A\C) &\overset{\text{rec step}}{=} 1 +rnalen(\A) \\ 
        &\overset{\text{basis step}}{=} 1 + 1 = 2
    \end{align*}}\\
\hline
$basecount$ & $S \times B$ & $\mathbb{N}$ & 
{\begin{align*}    
    &\textrm{Basis Step:} \\
    &\textrm{If } b_1 \in B, b_2 \in B \textrm{ then} \\
    &basecount(~(b_1, b_2)~) = \\
    &\begin{cases}
        1 & \textrm{when } b_1 = b_2 \\
        0 & \textrm{when } b_1 \neq b_2 \\
    \end{cases}\\
    &\textrm{Recursive Step:}\\
    &\textrm{If } s \in S, b_1 \in B, b_2 \in B\\
    &basecount(~(sb_1, b_2)~) = \\
    &\begin{cases}
        1 + \textit{basecount}(~(s, b_2)~) & \textrm{when } b_1 = b_2 \\
        \textit{basecount}(~(s, b_2)~) & \textrm{when } b_1 \neq b_2 \\
    \end{cases}
    \end{align*}} & 
    {\begin{align*}
        basecount(~(\A\C\U, \C)~) = 
    \end{align*}}\\
\hline
``$2$ to the power of''& $\mathbb{N}$ & $\mathbb{N}$ & 
{\begin{align*}    
&\textrm{Basis Step:} \\
&2^0= 1 \\
&\textrm{Recursive Step:}\\
&\textrm{If } n \in \mathbb{N}, 2^{n+1} = \phantom{2 \cdot 2^n}
\end{align*}}\\
\hline
``$b$ to the power of $i$''& $\mathbb{Z}^+ \times \mathbb{N}$ & $\mathbb{N}$ & 
{\begin{align*}    
&\textrm{Basis Step:} \\
&b^0 = 1 \\
&\textrm{Recursive Step:}\\
&\textrm{If } i \in \mathbb{N}, b^{i+1} = b \cdot b^i
\end{align*}}
\end{tabular}
\end{center}

\fbox{\parbox{\textwidth}{
    $2^0 = 1$~~\hfill $2^1=2$~~\hfill $2^2=4$~~\hfill $2^3=8$~~
    \hfill $2^4=16$~~\hfill $2^5=32$~~
    \hfill $2^6=64$~~\hfill $2^7=128$~~
    \hfill $2^8=256$~~\hfill $2^9=512$~~
    \hfill $2^{10}=1024$}}
\newpage \vfill
\section*{Defining sets}


{\it To define sets:}

To define a set using {\bf roster method}, explicitly list its elements. That is,
start with $\{$ then list elements of 
the set separated by commas and close with $\}$.

\vfill

To define a set using {\bf set builder definition}, either form 
``The set of all $x$ from the universe $U$ such that $x$ is ..." by writing
\[\{x \in U \mid ...x... \}\]
or form ``the collection of all outputs of some operation when the input ranges over the universe $U$"
by writing
\[\{ ...x... \mid x\in U \}\]

\vfill

We use the symbol $\in$ as ``is an element of'' to indicate membership in a set.\\

\newpage 

{\bf Example sets}: For each of the following, identify whether it's defined using the roster method
or set builder notation and give an example element.

Can we infer the data type of the example element from the notation?

\begin{itemize}
    \item[]$\{ -1, 1\}$
    \vfill
    \item[]$\{0, 0 \}$
    \vfill
    \item[]$\{-1, 0, 1 \}$
    \vfill
    \item[]$\{(x,x,x) \mid x \in \{-1,0,1\} \}$
    \vfill
    \item[]$\{ \}$
    \vfill
    \item[]$\{ x \in \mathbb{Z} \mid x \geq 0 \}$
    \vfill
    \item[]$\{ x \in \mathbb{Z}  \mid x > 0 \}$
    \vfill
    \item[]$\{ \smile, \sun \}$
    \vfill
    \item[]$\{\A,\C,\U,\G\}$
    \vfill
    \item[]$\{\A\U\G, \U\A\G, \U\G\A, \U\A\A \}$
    \vfill
\end{itemize}
 \vfill
\section*{Rna motivation}


RNA is made up of strands of four different bases that encode genomic information
in specific ways.\\
The bases are elements of the set 
$B  = \{\A, \C, \U, \G \}$.
Strands are ordered nonempty finite sequences of bases.

Formally, to define the set of all RNA strands, we need more than roster
method or set builder descriptions. 

 \vfill
\section*{Set recursive examples}


{\bf Definition} The set of nonnegative integers $\mathbb{N}$ is defined (recursively) by: 
\[
\begin{array}{ll}
\textrm{Basis Step: } & \phantom{0 \in \mathbb{N}} \\
\textrm{Recursive Step: } & \phantom{\textrm{If } n \in \mathbb{N} \textrm{, then } n+1 \in \mathbb{N}}
\end{array}
\]

Examples: 

{\bf Definition} The set of all integers $\mathbb{Z}$ is defined (recursively) by: 
\[
\begin{array}{ll}
\textrm{Basis Step: } & \phantom{0 \in \mathbb{Z}} \\
\textrm{Recursive Step: } & \phantom{\textrm{If } x \in \mathbb{Z} \textrm{, then } x+1 \in \mathbb{Z}
\textrm{ and } x-1 \in \mathbb{Z}}
\end{array}
\]

Examples: 

\vfill

{\bf Definition} The set of RNA strands $S$ is defined (recursively) by:
\[
\begin{array}{ll}
\textrm{Basis Step: } & \A \in S, \C \in S, \U \in S, \G \in S \\
\textrm{Recursive Step: } & \textrm{If } s \in S\textrm{ and }b \in B \textrm{, then }sb \in S
\end{array}
\]
where $sb$ is string concatenation.

Examples: 

\vfill

{\bf Definition} The set of bitstrings (strings of 0s and 1s) is defined (recursively) by:
\[
\begin{array}{ll}
\textrm{Basis Step: } & \phantom{\lambda \in X} \\
\textrm{Recursive Step: } & \phantom{\textrm{If } s \in X \textrm{, then } s0 \in X \text{ and } s1 \in X}
\end{array}
\]

{\it Notation:} We call the set of bitstrings $\{0,1\}^*$ and we say 
this is the set of all strings over $\{0,1\}$.

Examples: 

\vfill \vfill
\end{document}