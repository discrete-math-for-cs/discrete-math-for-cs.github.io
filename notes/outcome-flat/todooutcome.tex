\documentclass[12pt, oneside]{article}

\usepackage[letterpaper, scale=0.89, centering]{geometry}
\usepackage{fancyhdr}
\setlength{\parindent}{0em}
\setlength{\parskip}{1em}

\pagestyle{fancy}
\fancyhf{}
\renewcommand{\headrulewidth}{0pt}
\rfoot{\href{https://creativecommons.org/licenses/by-nc-sa/2.0/}{CC BY-NC-SA 2.0} Version \today~(\thepage)}

\usepackage{amssymb,amsmath,pifont,amsfonts,comment,enumerate,enumitem}
\usepackage{currfile,xstring,hyperref,tabularx,graphicx,wasysym}
\usepackage[labelformat=empty]{caption}
\usepackage{xcolor}
\usepackage{multicol,multirow,array,listings,tabularx,lastpage,textcomp,booktabs}

\lstnewenvironment{algorithm}[1][] {   
    \lstset{ mathescape=true,
        frame=tB,
        numbers=left, 
        numberstyle=\tiny,
        basicstyle=\rmfamily\scriptsize, 
        keywordstyle=\color{black}\bfseries,
        keywords={,procedure, div, for, to, input, output, return, datatype, function, in, if, else, foreach, while, begin, end, }
        numbers=left,
        xleftmargin=.04\textwidth,
        #1
    }
}
{}
\lstnewenvironment{java}[1][]
{   
    \lstset{
        language=java,
        mathescape=true,
        frame=tB,
        numbers=left, 
        numberstyle=\tiny,
        basicstyle=\ttfamily\scriptsize, 
        keywordstyle=\color{black}\bfseries,
        keywords={, int, double, for, return, if, else, while, }
        numbers=left,
        xleftmargin=.04\textwidth,
        #1
    }
}
{}

\newcommand\abs[1]{\lvert~#1~\rvert}
\newcommand{\st}{\mid}

\newcommand{\A}[0]{\texttt{A}}
\newcommand{\C}[0]{\texttt{C}}
\newcommand{\G}[0]{\texttt{G}}
\newcommand{\U}[0]{\texttt{U}}

\newcommand{\cmark}{\ding{51}}
\newcommand{\xmark}{\ding{55}}

 
\begin{document}
\begin{flushright}
    \StrBefore{\currfilename}{.}
\end{flushright} \section*{Cartesian product definition}


{\bf Definition}: The {\bf Cartesian product} of the sets $A$ and $B$, 
$A \times B$, is the set of all ordered pairs $(a, b)$, where $a \in A$ and $b \in B$. 
That is: $A \times B = \{(a, b) \mid (a \in A) \land (b \in B)\}$.
The Cartesian product of the sets $A_1, A_2, \ldots ,A_n$, denoted by 
$A_1 \times A_2 \times \cdots \times A_n$, is the
set of ordered n-tuples $(a_1, a_2,...,a_n)$, where $a_i$ belongs to 
$A_i$ for $i = 1, 2,\ldots,n$. That is,
\[
    A_1 \times A_2 \times \cdots \times A_n = \{(a_1, a_2,\ldots,a_n) \mid a_i \in A_i \textrm{ for } i = 1, 2,\ldots,n\}
\] \vfill
\section*{Rna mutation insertion deletion example}


Trace the pseudocode to find the output of $\textit{mutation}(~ (\A\U\C, 3, \G) ~)$

\vspace{50pt}

Fill in the blanks so that $\textit{insertion}(~(\A\U\C, \underline{\phantom{3}}, \underline{\phantom{\G}})~) = \A\U\C\G$

\vspace{50pt}

Fill in the blanks so that $\textit{deletion}(~(\underline{\phantom{\G\G}}, \underline{\phantom{1}})~) =  \G$

\vspace{50pt}
 \vfill
\section*{Rna rnalen basecount definitions}


{\it Recall the definitions}: The set of RNA strands $S$ is defined (recursively) by:
\[
\begin{array}{ll}
\textrm{Basis Step: } & \A \in S, \C \in S, \U \in S, \G \in S \\
\textrm{Recursive Step: } & \textrm{If } s \in S\textrm{ and }b \in B \textrm{, then }sb \in S
\end{array}
\]
where $sb$ is string concatenation.

The function \textit{rnalen} that computes the length of RNA strands in $S$ is defined recursively by:
\[
\begin{array}{llll}
& & \textit{rnalen} : S & \to \mathbb{Z}^+ \\
\textrm{Basis Step:} & \textrm{If } b \in B\textrm{ then } & \textit{rnalen}(b) & = 1 \\
\textrm{Recursive Step:} & \textrm{If } s \in S\textrm{ and }b \in B\textrm{, then  } & \textit{rnalen}(sb) & = 1 + \textit{rnalen}(s)
\end{array}
\]

The function \textit{basecount} that computes the number of a given base 
$b$ appearing in a RNA strand $s$ is defined recursively by:
\[
\begin{array}{llll}
& & \textit{basecount} : S \times B & \to \mathbb{N} \\
\textrm{Basis Step:} &  \textrm{If } b_1 \in B, b_2 \in B & \textit{basecount}(~(b_1, b_2)~) & =
        \begin{cases}
            1 & \textrm{when } b_1 = b_2 \\
            0 & \textrm{when } b_1 \neq b_2 \\
        \end{cases} \\
\textrm{Recursive Step:} & \textrm{If } s \in S, b_1 \in B, b_2 \in B &\textit{basecount}(~(s b_1, b_2)~) & =
        \begin{cases}
            1 + \textit{basecount}(~(s, b_2)~) & \textrm{when } b_1 = b_2 \\
            \textit{basecount}(~(s, b_2)~) & \textrm{when } b_1 \neq b_2 \\
        \end{cases}
\end{array}
\] \vfill
\section*{Alternating quantifiers order rna examples}


{\bf Alternating nested quantifiers}



$$\forall s \in S ~\exists n \in \mathbb{N} ~(~basecount(~(s,\U)~) = n~)$$

In English: For each strand, there is a nonnnegative integer that counts the number of occurrences of $\U$ in that 
strand.\\

$$\exists n \in ~\forall s \in S ~\mathbb{N} ~(~basecount(~(s,\U)~) = n~)$$

In English: There is a nonnnegative integer that counts the number of occurrences of $\U$ in every 
strand.\\

\vfill

Are these statements true or false?

\newpage

$$\forall s \in S ~\exists b\in B ~(~basecount(~(s,b)~) = 3~)$$

In English: For each RNA strand there is a base that occurs 3 times in this strand.\\

Write the negation and use De Morgan's law to find a 
logically equivalent version where the negation is applied only to the 
$BC$ predicate (not next to a quantifier).

\vspace{60pt}


Is the original statement {\bf True} or {\bf False}?

\vfill
 \vfill
\section*{Proof strategies quantification finite domain}


When a predicate $P(x)$ is over a {\bf finite} domain:
\begin{itemize}
\item To show that $\forall x  P(x)$ is true: check that $P(x)$ evaluates to $T$ at each domain element by evaluating over and over. 
This is called ``Proof of universal by {\bf exhaustion}".
\item To show that $\forall x  P(x)$ is false: find a {\bf counterexample}, a domain element where $P(x)$~evaluates~to~$F$.
\item To show that $\exists x  P(x)$ is true: find a {\bf witness}, a domain element where $P(x)$ evaluates to $T$.
\item To show that $\exists x  P(x)$ is false: check that $P(x)$ evaluates to $F$ at each domain element by evaluating over and over.
DeMorgan's Law gives that $\lnot \exists x P(x) ~~\equiv~~ \forall x \lnot P(x)$ so this amounts to a proof of universal by exhaustion.
\end{itemize} \vfill
\section*{Proof strategy universal generalization}


\fbox{\parbox{\linewidth}{

{\bf New! Proof by universal generalization}: To prove that $\forall x \, P(x)$
is true, we can take an arbitrary element $e$ from the domain of 
quantification and show that $P(e)$ is true, without making any assumptions about $e$ 
other than that it comes from the domain.


An {\bf arbitrary} element of a set or domain is a fixed but unknown element from that set. 
}}
 \vfill
\section*{Quiz translating counting quantifiers}


Suppose $P(x)$ is  a predicate over a domain $D$.
\begin{enumerate}
    \item True or False: To translate the statement
    ``There are at least two  elements in $D$
    where the predicate $P$ evaluates to true", we
    could  write
    \[
    \exists  x_1 \in D \, \exists x_2 \in D  \, (P(x_1) \wedge P(x_2))
    \]
    \vfill
    \item True or False: To translate the statement
    ``There are at most two  elements in $D$
    where the predicate $P$ evaluates to true", we
    could write
    \[
    \forall  x_1 \in D \, \forall x_2 \in D \, \forall x_3 \in  D \, \left(~ (~P(x_1) \wedge P(x_2)  \wedge P(x_3) ~) \to (~ x_1 = x_2 \vee x_2 = x_3 \vee x_1 = x_3~)~\right)
    \]
    \vfill
\end{enumerate} \vfill
\section*{Proof strategies conditionals}


\fbox{\parbox{\linewidth}{

{\bf New! Proof of conditional by direct proof}: To prove that the conditional statement $p \to q$ is true, 
we can assume $p$ is true and use that assumption to show $q$ is true.
}}

\fbox{\parbox{\linewidth}{

{\bf New! Proof of conditional by contrapositive proof}: To prove that the implication $p \to q$ is true, 
we can assume $q$ is false and use that assumption to show $p$ is also false.
}}

\fbox{\parbox{\linewidth}{

{\bf New! Proof of disjuction using equivalent conditional}: To prove that the 
disjunction $p \lor q$ is true, we can rewrite it equivalently as $\lnot p \to q$ and
then use direct proof or contrapositive proof.
}} \vfill
\section*{Proof strategies proof by cases}


\fbox{\parbox{\linewidth}{{\bf New! Proof by Cases}: To prove $q$, we can 
work by cases by first describing all possible cases we might be in
and then showing that each one guarantees $q$.
Formally, if we know that $p_1 \lor p_2$ is true, 
and we can show that $(p_1 \to q)$ is true and we can show that $(p_2 \to q)$, 
then we can conclude $q$ is true.
}} \vfill
\section*{Proof strategies ands}


\fbox{\parbox{\linewidth}{
{\bf New! Proof of conjunctions with subgoals}:
To show that $p \land q$ is true, we have two subgoals: subgoal (1) prove $p$ 
is  true; and, subgoal (2) prove $q$ is true.

\vspace{1em}

 To show that $p \land q$ is false, it's enough to prove that $\lnot p$.
 
 To show that $p \land q$ is false, it's enough to prove that $\lnot q$.
}} \vfill
\section*{Sets proof strategies}


To prove that one set is a subset of another, e.g. to show $A \subseteq B$:

\vspace{50pt}

To prove that two sets are equal, e.g. to show $A = B$:

\vspace{50pt}
 \vfill
\section*{Sets equality example}


Example: $\{ 43, 7, 9 \} = \{ 7, 43, 9, 7\}$

\vspace{50pt}
 \vfill
\section*{Sets basic proofs}


{\bf Prove} or {\bf  disprove}: $\{ \A,  \C,  \U,  \G\} \subseteq \{ \A\A, \A\C, \A\U, \A\G \}$ 

\vspace{150pt}

{\bf Prove} or {\bf  disprove}: For some set $B$, $\emptyset \in B$.

\vspace{150pt}

{\bf Prove} or {\bf  disprove}: For every set $B$, $\emptyset \in B$.

\vspace{150pt}

{\bf Prove} or {\bf  disprove}: The empty set is a subset of every set.

\vspace{150pt}

{\bf Prove} or {\bf  disprove}: The empty set is a proper subset of every set.

\vspace{150pt}

{\bf Prove} or {\bf  disprove}: $\{ 4, 6 \} \subseteq \{ n \mid  \exists c \in \mathbb{Z} ( n = 4c) \} $

\vspace{150pt}

{\bf Prove} or {\bf  disprove}: $\{ 4, 6 \} \subseteq \{ n ~\textbf{mod}~10 \mid  \exists c \in \mathbb{Z} ( n = 4c) \} $

\vspace{150pt}

 \vfill
\section*{Proofs signposting}


\fbox{\parbox{\textwidth}{

\vspace{10pt}

Consider \ldots, an {\bf arbitrary} \ldots.
{\bf Assume} \ldots, we {\bf want to show} that \ldots. Which is what was needed,
so the proof is complete $\square$.

\vspace{20pt} {\it or, in other words:} \vspace{20pt}

Let \ldots be an {\bf arbitrary} \ldots. {\bf Assume} \ldots, {\bf WTS} that \ldots {\bf QED}.

\vspace{10pt}

}} \vfill
\section*{Set operations union intersection powerset}


{\bf Cartesian product}: When $A$ and  $B$ are sets, 
\[
    A \times  B = \{ (a,b) \mid a \in A  \wedge b  \in B \}
\]

Example: $\{43, 9\} \times  \{9, \mathbb{Z}\}  = $
    
Example: $\mathbb{Z} \times \emptyset  = $

{\bf Union}: When $A$ and  $B$ are sets,
\[
    A \cup  B = \{ x \mid x \in A  \vee x \in B \}
\]    
    
Example: $\{43, 9\} \cup \{9, \mathbb{Z}\}  = $

Example: $\mathbb{Z} \cup \emptyset  = $ 

{\bf Intersection}: When $A$ and  $B$ are sets,
\[
    A \cap  B = \{ x \mid x \in A  \wedge x \in B \}
\]    
Example: $\{43, 9\} \cap \{9,\mathbb{Z}\}  = $

Example: $\mathbb{Z} \cap \emptyset  = $


{\bf Set  difference}: When $A$ and  $B$ are sets,

\[
    A -  B = \{ x \mid x \in A  \wedge x \notin B \}
\]

Example: $\{43, 9\} - \{9, \mathbb{Z}\}  = $

Example: $\mathbb{Z} - \emptyset  = $

    
{\bf Disjoint sets}: sets $A$ and  $B$ are disjoint means $A \cap  B  = \emptyset$

Example: $\{43, 9\}, \{9, \mathbb{Z}\}$ are not  disjoint 

Example: The sets $\mathbb{Z}$ and $\emptyset$ are disjoint

    

{\bf Power set}: When $U$ is a set, $\mathcal{P}(U) = \{ X \mid X \subseteq U\}$

Example: $\mathcal{P}(\{43, 9\}) = $

Example: $\mathcal{P}(\emptyset) = $
 \vfill
\section*{Quantification definition}


The {\bf universal quantification} of predicate $P(x)$ over
domain $U$ is the statement ``$P(x)$ for all values of $x$ in the domain $U$''
and is written $\forall x P(x)$ or $\forall x \in U ~P(x)$. 
When the domain is finite, universal quantification over the domain 
is equivalent to iterated {\it conjunction} (ands).

The {\bf existential quantification} of predicate $P(x)$ 
over domain $U$ is the statement ``There exists an element $x$ 
in the domain $U$ such that $P(x)$'' and is written $\exists x P(x)$
for $\exists x \in U ~P(x)$. 
When the domain is finite, existential quantification over the domain 
is equivalent to iterated {\it disjunction} (ors).

An element for which $P(x) = F$ is called a {\bf counterexample} of $\forall x P(x)$.

An element for which $P(x) = T$ is called a {\bf witness} of $\exists x P(x)$.
 \vfill
\section*{Quantification logical equivalence}


Statements involving predicates and quantifiers are {\bf logically equivalent} 
means they have the same truth value no matter which predicates (domains and functions) 
are substituted in. 

{\bf Quantifier version of De Morgan's laws}: 
$\boxed{\neg \forall x P(x) ~\equiv~ \exists x \left( \neg P(x) \right)}$
\qquad
\qquad
$\boxed{\neg \exists x Q(x) ~\equiv~ \forall x \left( \neg Q(x) \right)}$
 \vfill
\section*{Quantification examples finite domain}


Examples of quantifications using $V(x), N(x), Mystery(x)$:

{\bf True} or {\bf False}: $\exists x~ (~V(x) \land N(x)~)$

\vfill

{\bf True} or {\bf False}: $\forall x~ (~V(x) \to N(x)~)$

\vfill

{\bf True} or {\bf False}: $\exists x~ (~N(x) \leftrightarrow Mystery(x)~)$

\vfill

Rewrite $\lnot \forall x~(~V(x) \oplus Mystery(x)~)$ into a logical equivalent statement.

\vspace{50pt}


Notice that these are examples where the predicates have {\it finite} domain.
How would we evaluate quantifications where the domain may be infinite?

\vfill

 \vfill
\section*{Rna rnalen basecount definitions}


{\it Recall the definitions}: The set of RNA strands $S$ is defined (recursively) by:
\[
\begin{array}{ll}
\textrm{Basis Step: } & \A \in S, \C \in S, \U \in S, \G \in S \\
\textrm{Recursive Step: } & \textrm{If } s \in S\textrm{ and }b \in B \textrm{, then }sb \in S
\end{array}
\]
where $sb$ is string concatenation.

The function \textit{rnalen} that computes the length of RNA strands in $S$ is defined recursively by:
\[
\begin{array}{llll}
& & \textit{rnalen} : S & \to \mathbb{Z}^+ \\
\textrm{Basis Step:} & \textrm{If } b \in B\textrm{ then } & \textit{rnalen}(b) & = 1 \\
\textrm{Recursive Step:} & \textrm{If } s \in S\textrm{ and }b \in B\textrm{, then  } & \textit{rnalen}(sb) & = 1 + \textit{rnalen}(s)
\end{array}
\]

The function \textit{basecount} that computes the number of a given base 
$b$ appearing in a RNA strand $s$ is defined recursively by:
\[
\begin{array}{llll}
& & \textit{basecount} : S \times B & \to \mathbb{N} \\
\textrm{Basis Step:} &  \textrm{If } b_1 \in B, b_2 \in B & \textit{basecount}(~(b_1, b_2)~) & =
        \begin{cases}
            1 & \textrm{when } b_1 = b_2 \\
            0 & \textrm{when } b_1 \neq b_2 \\
        \end{cases} \\
\textrm{Recursive Step:} & \textrm{If } s \in S, b_1 \in B, b_2 \in B &\textit{basecount}(~(s b_1, b_2)~) & =
        \begin{cases}
            1 + \textit{basecount}(~(s, b_2)~) & \textrm{when } b_1 = b_2 \\
            \textit{basecount}(~(s, b_2)~) & \textrm{when } b_1 \neq b_2 \\
        \end{cases}
\end{array}
\] \vfill
\section*{Predicates example rnalen basecount}


{\bf Using functions to define predicates}:

\fbox{\parbox{\textwidth}{
$L$ with domain $S \times \mathbb{Z}^+$ is defined by, for $s \in S$ and $n \in \mathbb{Z}^+$,
\[
L( ~(s, n)~) = \begin{cases}
T &\qquad\text{if $rnalen(s) = n$}\\
F &\qquad\text{otherwise}\\
\end{cases}
\]
In other words, $L(~(s,n)~)$ means $rnalen(s) = n$
}}

\vfill

\fbox{\parbox{\textwidth}{
$BC$ with domain $S \times B \times \mathbb{N}$ is defined by, 
for $s \in S$ and $b \in B$ and $n \in \mathbb{N}$,
\[
BC(~(s, b, n)~) = \begin{cases}
T &\qquad\text{if $basecount(~(s,b)~) = n$}\\
F &\qquad\text{otherwise}\\
\end{cases}
\]
In other words, $BC(~(s,b,n)~)$ means $basecount(~(s,b)~) = n$
}}


\vfill


Example where $L$ evaluates to $T$: $\underline{\phantom{(\A, 1)\hspace{1in}}}$  Why?

\vfill


Example where $BC$ evaluates to $T$: $\underline{\phantom{(\A, \A1)\hspace{1in}}}$  Why?

\vfill


Example where $L$ evaluates to $F$: $\underline{\phantom{(\A, 2)\hspace{1in}}}$ Why?

\vfill


Example where $BC$ evaluates to $F$: $\underline{\phantom{(\A, \C, 1)\hspace{1in}}}$ Why? 

\vfill


\fbox{\parbox{\textwidth}{
\[\exists t ~BC(t) \qquad \qquad 
\exists (s,b,n) \in S \times B \times \mathbb{N}~ (basecount(~(s,b)~) = n)\]

In English: \phantom{There exists an ordered $3$-tuple 
at which the predicate $BC$ evaluates to $T$.}

\vspace{30pt}

Witness that proves this existential quantification is true:\phantom{$(\G\G, \G, 2)$ or $(\G\A\U\G, \G, 2)$)}
}}

\fbox{\parbox{\textwidth}{
\[\forall t ~BC(t) \qquad \qquad 
\forall(s,b,n) \in S \times B \times \mathbb{N} ~(basecount(~(s,b)~) = n)\]

In English:\phantom{For all ordered $3$-tuples
the predicate $BC$ evaluates to $T$.}

\vspace{30pt}

Counterexample that proves this universal quantification is false: \phantom{$(\G\G, \A, 2)$ or $(\G\A\U\G, \G, 3)$)}
}}
 \vfill
\section*{Predicates projecting example rna basecount}


{\bf New predicates from old}
\begin{enumerate}
\item Define the {\bf new} predicate with domain $S \times B$ and rule
\[
basecount(~(s,b)~) = 3
\]
Example domain element where predicate is $T$: \phantom{$(\A\U\A\A, \A)$}\\

\vfill

\item Define the {\bf new} predicate with domain $S \times \mathbb{N}$ and rule
\[
basecount(~(s,\A)~) = n
\]
Example domain element where predicate is $T$: \phantom{$(\A\U\A,2)$}\\

\vfill


\item Define the {\bf new} predicate with domain $S \times B$ and rule
\[
\exists n \in \mathbb{N} ~(basecount(~(s,b)~) = n)
\]
Example domain element where predicate is $T$: \phantom{$(\A\U\A,\A)$}\\

\vfill


\item Define the {\bf new} predicate with domain $S$ and rule
\[
\forall b \in B ~(basecount(~(s,b)~) = 1)
\]
Example domain element where predicate is $T$: \phantom{$\A\C\G\U$}\\

\vfill


\end{enumerate} \vfill
\section*{Nested quantifiers}


{\bf Nested quantifiers}

\fbox{\parbox{\textwidth}{
\[
    \forall s \in S ~\forall b \in B ~\forall n \in \mathbb{N} ~(basecount(~(s,b)~) = n)
\]

In English: \phantom{There exists an ordered $3$-tuple 
at which the predicate $BC$ evaluates to $T$.}

\vspace{30pt}

Counterexample that proves this universal quantification is false:
\phantom{$(\G\G, \G, 3)$ or $(\G\A\U\G, \G, 2)$)}

\vspace{30pt}

}}

\vfill

\fbox{\parbox{\textwidth}{
\[
    ~\forall n \in \mathbb{N} ~\forall s \in S ~\forall b \in B  ~(basecount(~(s,b)~) = n)
\]

In English: \phantom{There exists an ordered $3$-tuple 
at which the predicate $BC$ evaluates to $T$.}

\vspace{30pt}

Counterexample that proves this universal quantification is false:
\phantom{$(\G\G, \G, 3)$ or $(\G\A\U\G, \G, 2)$)}

\vspace{30pt}

}}

\vfill \vfill
\section*{Alternating quantifiers strategies rna examples}


{\bf Alternating nested quantifiers}

\fbox{\parbox{\textwidth}{
$$\forall s \in S ~\exists b\in B ~(~basecount(~(s,b)~) = 3~)$$

In English: For each RNA strand there is a base that occurs 3 times in this strand.\\

Write the negation and use De Morgan's law to find a 
logically equivalent version where the negation is applied only to the 
$BC$ predicate (not next to a quantifier).

\vspace{60pt}


Is the original statement {\bf True} or {\bf False}?

}}

\vfill

\fbox{\parbox{\textwidth}{

$$\exists s \in S ~\forall b \in B ~\exists n \in \mathbb{N} ~(~basecount(~(s,b)~) = n~)$$

In English: There is an RNA strand so that for each base there is some nonnegative
integer that counts the number of occurrences of that base in this strand.\\

Write the negation and use De Morgan's law to find a 
logically equivalent version where the negation is applied only to the 
$BC$ predicate (not next to a quantifier).

\vspace{60pt}


Is the original statement {\bf True} or {\bf False}?

}}

\vfill
 \vfill
\section*{Tautology contradiction contingency examples}


Label each of the following as a tautology, contradiction, or contingency.

$p \land p$

\vfill

$p \oplus p$

\vfill

$p \lor p$

\vfill

$p \lor \lnot p$

\vfill

$p \land \lnot p$

\vfill

 \vfill
\section*{Why represent numbers}


Modeling uses data-types that are encoded in a computer.
The details of the encoding impact the efficiency of algorithms
we use to understand the systems we are modeling and the 
impacts of these algorithms on the people using the systems.
Case study: how to encode numbers?

\phantom{
Positional representation with familiar (decimal) number encodings
\vspace{30pt}
}
\vfill \vfill
\section*{Fixed width definition}


{\bf Definition} For $b$ an integer greater than $1$, $w$ a positive integer, 
and $n$ a nonnegative integer
$\underline{\phantom{\hspace{1in}}}$, ~
the {\bf base $b$ fixed-width $w$ expansion of $n$}  is
\[
(a_{w-1} \cdots a_1 a_0)_{b,w}
\]
where  $a_0, a_1, \ldots, a_{w-1}$ are nonnegative integers less than $b$ and
\[
n =  \sum_{i=0}^{w-1} a_{i} b^{i}
\]
 \vfill
\section*{Fixed width example}


\begin{center}
    \begin{tabular}{|c|c|c|c|c|}
    \hline
    Decimal &  Binary  & Binary fixed-width $10$& Binary fixed-width $7$ & Binary fixed-width $4$\\
    $b=10$ & $b=2$ & $b=2$, $w =  10$& $b=2$, $w =  7$& $b=2$, $w =  4$ \\
    \hline 
    &&&&  \\
    $(20)_{10}$&\phantom{$(10100)_{2}$\qquad\qquad}&&  &\\
    &&&&  \\
\hline
    \end{tabular}
    \end{center}
 \vfill
\section*{Fixed width fractional definition}


{\bf Definition} For $b$ an integer greater than $1$, $w$ a positive integer, 
$w'$ a positive  integer, and $x$ a real number the {\bf base $b$ fixed-width 
expansion of $x$ with integer part width $w$  and fractional part width $w'$} is
$(a_{w-1} \cdots a_1 a_0 .  c_{1} \cdots c_{w'})_{b,w,w'}$
where  $a_0, a_1, \ldots, a_{w-1}, c_1, \ldots, c_{w'}$ are nonnegative integers less than $b$ and
$$x \geq \sum_{i=0}^{w-1} a_{i} b^{i} + \sum_{j=1}^{w'}  c_{j} b^{-j} \hfill
\textrm{\qquad and \qquad}
\hfill x < \sum_{i=0}^{w-1} a_{i} b^{i} + \sum_{j=1}^{w'} c_{j} b^{-j} + b^{-w'}$$

\begin{center}
\begin{tabular}{|c|p{5in}|}
\hline
& \\
$3.75$  in fixed-width binary,& \\
integer part width $2$,&\\
 fractional part width $8$ & \\
& \\
& \\
& \\
& \\
\hline
& \\
$0.1$  in fixed-width binary, & \\
integer part width $2$, &\\
 fractional part width $8$ & \\
 & \\
 & \\
 & \\
 & \\
 \hline
\end{tabular}
\end{center}

\vfill

\includegraphics[width=2in]{../../resources/images/ArithmeticDemo.png}

Note: Java uses floating point, not fixed width representation, but similar rounding errors appear in both.
 \vfill
\section*{Negative int expansions}


{\bf Representing negative integers in binary}: Fix a positive integer  width for the representation  $w$, $w >1$.

\begin{tabular}{|cc|p{3.4in}|p{3.7in}|}
\hline
& & To  represent a positive integer $n$ & To represent a negative integer $-n$\\
\hline
&& &  \\
&\parbox[t]{2mm}{\multirow{4}{*}{\rotatebox[origin=c]{90}{Sign-magnitude}}} &
$[ 0a_{w-2} \cdots a_0]_{s,w}$, where $n =  (a_{w-2} \cdots a_0)_{2,w-1}$& 
$[1a_{w-2} \cdots a_0]_{s,w}$
, where $n =  (a_{w-2} \cdots a_0)_{2,w-1}$\\
&& & \\
&& Example $n=17$, $w=7$:  & Example $-n=-17$, $w=7$: \\
&& & \\
&& & \\
&& & \\
&& & \\
&& & \\
&& & \\
&& & \\
\hline
&&  &  \\
&\parbox[t]{2mm}{\multirow{4}{*}{\rotatebox[origin=c]{90}{2s complement}}} &
$[0a_{w-2} \cdots a_0]_{2c,w}$, where $n =  (a_{w-2} \cdots a_0)_{2,w-1}$& $[1a_{w-2} \cdots a_0]_{2c,w}$, where $2^{w-1} - n =  (a_{w-2} \cdots a_0)_{2,w-1}$\\
&& & \\
&& Example $n=17$, $w=7$:  & Example $-n=-17$, $w=7$: \\
&& & \\
&& & \\
&& & \\
&& & \\
&& & \\
&& & \\
&& & \\
\hline
\end{tabular} \vfill
\section*{Calculating 2s complement}


For positive integer $n$, to represent $-n$ in 
$2$s complement with width $w$,
\begin{itemize}
    \item Calculate $2^{w-1} - n$, convert 
    result to binary fixed-width $w-1$, pad 
    with leading $1$, or
    \item Express $-n$ as a sum of powers of $2$, 
    where the leftmost $2^{w-1}$ is negative weight, or
    \item Convert $n$ to binary fixed-width $w$, 
    flip bits, add 1 (ignore overflow)
\end{itemize}

{\it Challenge: use definitions to explain why
each of these approaches works.} \vfill
\section*{Representing zero}


{\bf Representing $0$}:

So far, we have representations for
positive and negative integers. What about $0$?

\begin{tabular}{|cc|p{3.4in}|p{3.7in}|}
   \hline
   & & To  represent a {\bf non-negative} integer $n$ & To represent a {\bf non-positive} integer $-n$\\
   \hline
   && &  \\
   &\parbox[t]{2mm}{\multirow{4}{*}{\rotatebox[origin=c]{90}{Sign-magnitude}}} &
   $[ 0a_{w-2} \cdots a_0]_{s,w}$, where $n =  (a_{w-2} \cdots a_0)_{2,w-1}$& 
   $[1a_{w-2} \cdots a_0]_{s,w}$
   , where $n =  (a_{w-2} \cdots a_0)_{2,w-1}$\\
   && & \\
   && Example $n=0$, $w=7$:  & Example $-n=0$, $w=7$: \\
   && & \\
   && & \\
   && & \\
   && & \\
   && & \\
   && & \\
   && & \\
   && & \\
   && & \\
   && & \\
\hline
   &&  &  \\
   &\parbox[t]{2mm}{\multirow{4}{*}{\rotatebox[origin=c]{90}{2s complement}}} &
   $[0a_{w-2} \cdots a_0]_{2c,w}$, where $n =  (a_{w-2} \cdots a_0)_{2,w-1}$& $[1a_{w-2} \cdots a_0]_{2c,w}$, where $2^{w-1} - n =  (a_{w-2} \cdots a_0)_{2,w-1}$\\
   && & \\
   && Example $n=0$, $w=7$:  & Example $-n=0$, $w=7$: \\
   && & \\
   && & \\
   && & \\
   && & \\
   && & \\
   && & \\
   && & \\
   && & \\
   && & \\
   && & \\
\hline
\end{tabular} \vfill
\end{document}