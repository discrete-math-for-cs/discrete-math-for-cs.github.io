\documentclass[12pt, oneside]{article}

\usepackage[letterpaper, scale=0.89, centering]{geometry}
\usepackage{fancyhdr}
\setlength{\parindent}{0em}
\setlength{\parskip}{1em}

\pagestyle{fancy}
\fancyhf{}
\renewcommand{\headrulewidth}{0pt}
\rfoot{\href{https://creativecommons.org/licenses/by-nc-sa/2.0/}{CC BY-NC-SA 2.0} Version \today~(\thepage)}

\usepackage{amssymb,amsmath,pifont,amsfonts,comment,enumerate,enumitem}
\usepackage{currfile,xstring,hyperref,tabularx,graphicx,wasysym}
\usepackage[labelformat=empty]{caption}
\usepackage{xcolor}
\usepackage{multicol,multirow,array,listings,tabularx,lastpage,textcomp,booktabs}

\lstnewenvironment{algorithm}[1][] {   
    \lstset{ mathescape=true,
        frame=tB,
        numbers=left, 
        numberstyle=\tiny,
        basicstyle=\rmfamily\scriptsize, 
        keywordstyle=\color{black}\bfseries,
        keywords={,procedure, div, for, to, input, output, return, datatype, function, in, if, else, foreach, while, begin, end, }
        numbers=left,
        xleftmargin=.04\textwidth,
        #1
    }
}
{}
\lstnewenvironment{java}[1][]
{   
    \lstset{
        language=java,
        mathescape=true,
        frame=tB,
        numbers=left, 
        numberstyle=\tiny,
        basicstyle=\ttfamily\scriptsize, 
        keywordstyle=\color{black}\bfseries,
        keywords={, int, double, for, return, if, else, while, }
        numbers=left,
        xleftmargin=.04\textwidth,
        #1
    }
}
{}

\newcommand\abs[1]{\lvert~#1~\rvert}
\newcommand{\st}{\mid}

\newcommand{\A}[0]{\texttt{A}}
\newcommand{\C}[0]{\texttt{C}}
\newcommand{\G}[0]{\texttt{G}}
\newcommand{\U}[0]{\texttt{U}}

\newcommand{\cmark}{\ding{51}}
\newcommand{\xmark}{\ding{55}}

 
\begin{document}
\begin{flushright}
    \StrBefore{\currfilename}{.}
\end{flushright} \section*{Division algorithm}


{\bf Integer division and remainders} (aka The Division Algorithm) Let $n$ be an integer 
and $d$ a positive integer. There are unique integers $q$ and $r$, with $0 \leq r < d$, such that 
$n = dq + r$. In this case, $d$ is called the divisor, $n$ is called the dividend, 
$q$ is called the quotient, 
and $r$ is called the remainder. 

Because these numbers are guaranteed to exist, the following functions are well-defined: 
\begin{itemize}\setlength{\leftmargin}{-0.25in}
\item $\textbf{ div } : \mathbb{Z} \times \mathbb{Z}^+ \to \mathbb{Z}$ given by $\textbf{ div } ( ~(n,d)~)$ 
is the quotient when $n$ is the dividend and $d$ is the divisor.
\item $\textbf{ mod } : \mathbb{Z} \times \mathbb{Z}^+ \to \mathbb{Z}$ given by $\textbf{ mod } ( ~(n,d)~)$ 
is the remainder when $n$ is the dividend and $d$ is the divisor.
\end{itemize}
Because these functions are so important, we sometimes use the notation
$n \textbf{ div } d = \textbf{ div } ( ~(n,d)~)$ and $n \textbf{ mod } d = \textbf{ mod } (~(n,d)~)$.


{\bf Pro-tip}: The functions $\textbf{ div }$ and $\textbf{ mod }$ are similar to (but not exactly the same as) 
the operators $/$ and $\%$ in Java and python.

\vfill

{\it Example calculations}:

$20 \textbf{ div } 4$

\vspace{20pt}

$20 \textbf{ mod } 4$

\vspace{20pt}

$20 \textbf{ div } 3$

\vspace{20pt}

$20 \textbf{ mod } 3$

\vspace{20pt}

$-20 \textbf{ div } 3$

\vspace{20pt}

$-20 \textbf{ mod } 3$

\vfill \vfill
\end{document}