\documentclass[12pt, oneside]{article}

\usepackage[letterpaper, scale=0.89, centering]{geometry}
\usepackage{fancyhdr}
\setlength{\parindent}{0em}
\setlength{\parskip}{1em}

\pagestyle{fancy}
\fancyhf{}
\renewcommand{\headrulewidth}{0pt}
\rfoot{\href{https://creativecommons.org/licenses/by-nc-sa/2.0/}{CC BY-NC-SA 2.0} Version \today~(\thepage)}

\usepackage{amssymb,amsmath,pifont,amsfonts,comment,enumerate,enumitem}
\usepackage{currfile,xstring,hyperref,tabularx,graphicx,wasysym}
\usepackage[labelformat=empty]{caption}
\usepackage{xcolor}
\usepackage{multicol,multirow,array,listings,tabularx,lastpage,textcomp,booktabs}

\lstnewenvironment{algorithm}[1][] {   
    \lstset{ mathescape=true,
        frame=tB,
        numbers=left, 
        numberstyle=\tiny,
        basicstyle=\rmfamily\scriptsize, 
        keywordstyle=\color{black}\bfseries,
        keywords={,procedure, div, for, to, input, output, return, datatype, function, in, if, else, foreach, while, begin, end, }
        numbers=left,
        xleftmargin=.04\textwidth,
        #1
    }
}
{}
\lstnewenvironment{java}[1][]
{   
    \lstset{
        language=java,
        mathescape=true,
        frame=tB,
        numbers=left, 
        numberstyle=\tiny,
        basicstyle=\ttfamily\scriptsize, 
        keywordstyle=\color{black}\bfseries,
        keywords={, int, double, for, return, if, else, while, }
        numbers=left,
        xleftmargin=.04\textwidth,
        #1
    }
}
{}

\newcommand\abs[1]{\lvert~#1~\rvert}
\newcommand{\st}{\mid}

\newcommand{\A}[0]{\texttt{A}}
\newcommand{\C}[0]{\texttt{C}}
\newcommand{\G}[0]{\texttt{G}}
\newcommand{\U}[0]{\texttt{U}}

\newcommand{\cmark}{\ding{51}}
\newcommand{\xmark}{\ding{55}}

 
\begin{document}
\begin{flushright}
    \StrBefore{\currfilename}{.}
\end{flushright} \section*{Base expansion definition}


{\bf Definition} For $b$ an integer greater than $1$ and $n$ a positive integer, 
the {\bf base $b$ expansion of $n$}  is
\[
(a_{k-1} \cdots a_1 a_0)_b
\]
where $k$ is a positive integer, $a_0, a_1, \ldots, a_{k-1}$ 
are (symbols for) nonnegative integers less than $b$, $a_{k-1} \neq  0$, and
\[
n =  \sum_{i=0}^{k-1} a_{i} b^{i}
\]

Notice: {\it The base $b$ expansion of a positive integer $n$ is a string over the alphabet 
$\{x \in \mathbb{N} \st x < b\}$
whose leftmost character is nonzero.}

\begin{center}
\begin{tabular}{|c|c|}
\hline
Base $b$ & Collection of possible coefficients in base $b$ expansion of  a positive integer \\
\hline
& \\
Binary ($b=2$) & $\{0,1\}$ \\
\hline
& \\
Ternary ($b=3$) & $\{0,1, 2\}$ \\
\hline
& \\
Octal ($b=8$) & $\{0,1, 2, 3, 4, 5, 6, 7\}$\\
\hline
& \\
Decimal ($b=10$) & $\{0,1, 2, 3, 4, 5, 6, 7, 8, 9\}$\\
\hline
& \\
Hexadecimal ($b=16$) &  $\{0,1, 2, 3, 4, 5, 6, 7, 8, 9, A, B, C, D, E, F\}$\\
& letter coefficient symbols represent numerical values $(A)_{16} = (10)_{10}$\\
&$(B)_{16} = (11)_{10} ~~(C)_{16} = (12)_{10} ~~
 (D)_{16} = (13)_{10} ~~ (E)_{16} = (14)_{10} ~~ (F)_{16} = (15)_{10} $\\
\hline
\end{tabular}
\end{center}

 \vfill
\section*{Base expansion examples}




{\it Examples}:

$(1401)_{2}$

\vfill

$(1401)_{10}$

\vfill
\vfill
\vfill


$(1401)_{16}$

\vfill
\vfill
\vfill
 \vfill
\section*{Base expansion review}


Find and fix any and all mistakes with the following:
\begin{itemize}
\item[(a)] $(1)_2 = (1)_8$
\item[(b)] $(142)_{10} = (142)_{16}$
\item[(c)] $(20)_{10} = (10100)_2$
\item[(d)] $(35)_8 = (1D)_{16}$
\end{itemize} \vfill
\end{document}