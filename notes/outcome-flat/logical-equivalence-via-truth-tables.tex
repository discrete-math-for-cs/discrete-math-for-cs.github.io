\documentclass[12pt, oneside]{article}

\usepackage[letterpaper, scale=0.89, centering]{geometry}
\usepackage{fancyhdr}
\setlength{\parindent}{0em}
\setlength{\parskip}{1em}

\pagestyle{fancy}
\fancyhf{}
\renewcommand{\headrulewidth}{0pt}
\rfoot{\href{https://creativecommons.org/licenses/by-nc-sa/2.0/}{CC BY-NC-SA 2.0} Version \today~(\thepage)}

\usepackage{amssymb,amsmath,pifont,amsfonts,comment,enumerate,enumitem}
\usepackage{currfile,xstring,hyperref,tabularx,graphicx,wasysym}
\usepackage[labelformat=empty]{caption}
\usepackage{xcolor}
\usepackage{multicol,multirow,array,listings,tabularx,lastpage,textcomp,booktabs}

\lstnewenvironment{algorithm}[1][] {   
    \lstset{ mathescape=true,
        frame=tB,
        numbers=left, 
        numberstyle=\tiny,
        basicstyle=\rmfamily\scriptsize, 
        keywordstyle=\color{black}\bfseries,
        keywords={,procedure, div, for, to, input, output, return, datatype, function, in, if, else, foreach, while, begin, end, }
        numbers=left,
        xleftmargin=.04\textwidth,
        #1
    }
}
{}
\lstnewenvironment{java}[1][]
{   
    \lstset{
        language=java,
        mathescape=true,
        frame=tB,
        numbers=left, 
        numberstyle=\tiny,
        basicstyle=\ttfamily\scriptsize, 
        keywordstyle=\color{black}\bfseries,
        keywords={, int, double, for, return, if, else, while, }
        numbers=left,
        xleftmargin=.04\textwidth,
        #1
    }
}
{}

\newcommand\abs[1]{\lvert~#1~\rvert}
\newcommand{\st}{\mid}

\newcommand{\A}[0]{\texttt{A}}
\newcommand{\C}[0]{\texttt{C}}
\newcommand{\G}[0]{\texttt{G}}
\newcommand{\U}[0]{\texttt{U}}

\newcommand{\cmark}{\ding{51}}
\newcommand{\xmark}{\ding{55}}

 
\begin{document}
\begin{flushright}
    \StrBefore{\currfilename}{.}
\end{flushright} \section*{Logical equivalence identities}


{\bf (Some) logical equivalences}

{\it Can replace $p$ and $q$ with any compound proposition}

\begin{tabular}{llp{3in}}
$\lnot ( \lnot p) \equiv p$ & & {\bf Double negation}\\
&& \\
&& \\
$p \lor q \equiv q \lor p$ & $p \land q \equiv q \land p$ & {\bf Commutativity} Ordering of terms\\
&& \\
&& \\
$(p \lor q) \lor r  \equiv p \lor (q \lor r)$ & $(p \land q) \land r  \equiv p \land (q \land r)$ & {\bf Associativity} Grouping of terms\\
&& \\
&& \\
$p \land F \equiv F$ \qquad $p \lor T \equiv T$ & $p \land T \equiv p$ \qquad $p \lor F \equiv p$ & {\bf Domination} aka 
short circuit evaluation\\
&& \\
&& \\
$\lnot (p \land q) \equiv \lnot p \lor \lnot q$ & $\lnot (p \lor q) \equiv \lnot p \land\lnot q$  & {\bf DeMorgan's Laws}\\
&& \\
\end{tabular}
\vfill

\begin{tabular}{llp{3in}}
$p \to q \equiv \lnot p \lor q$ & & \\
&& \\
&& \\
$p \to q \equiv \lnot q \to \lnot p$ & &{\bf Contrapositive} \\
&& \\
&& \\
$\lnot (p \to q) \equiv p\land \lnot q$  & &\\
&& \\
&& \\
$\lnot( p \leftrightarrow q) \equiv p \oplus q$ && \\
&& \\
&& \\
$p \leftrightarrow q \equiv q \leftrightarrow p$ &&\\
&& \\
\end{tabular}

\vfill

{\it Extra examples}:

$p \leftrightarrow q$ is not logically equivalent to $p \land q$ because \underline{\phantom{\hspace{4in}}} 

$p \to q$ is not logically equivalent to $q \to p$ because \underline{\phantom{\hspace{4in}}} 
\vfill
 \vfill
\section*{Logical operators example truth table}


\begin{center}
    \begin{tabular}{ccc||p{3in}|c|c}
    \multicolumn{3}{c||}{Input}  & \multicolumn{3}{c}{Output} \\
    $p$ & $q$ & $r$  &  &  $(p \land q) \oplus (~ ( p \oplus q) \land r~)$ & $(p \land q) \vee (~ ( p \oplus q) \land r~)$ \\
    \hline
    $T$ & $T$  & $T$ &   && \\
    $T$ & $T$  & $F$ &   && \\
    $T$ & $F$  & $T$ &   && \\
    $T$ & $F$  & $F$ &   && \\
    $F$ & $T$  & $T$ &   && \\
    $F$ & $T$  & $F$ &   && \\
    $F$ & $F$  & $T$ &   && \\
    $F$ & $F$  & $F$ &   && \\
    \end{tabular}
\end{center}
    \vfill \vfill
\section*{Logical equivalence}


{\bf Logical equivalence }: Two compound  propositions are {\bf logically  equivalent} means that  they 
have the  same  truth  values for all settings of truth  values to their propositional  variables.

{\bf Tautology}:  A compound proposition that evaluates to true
for all settings of truth  values to its propositional  variables; it is  abbreviated $T$.

{\bf Contradiction}: A compound proposition that  evaluates  to  false 
for  all settings of truth  values to its propositional  variables; it  is abbreviated $F$.

{\bf Contingency}: A compound proposition that is neither a tautology nor a contradiction.
 \vfill
\section*{Logical equivalence extra example}


{\it Extra Example}: Which of the  compound propositions in the table below are logically equivalent?
\begin{center}
\begin{tabular}{cc||c|c|c|c|c}
\multicolumn{2}{c||}{Input}  & \multicolumn{5}{c}{Output} \\
$p$ & $q$ & $\lnot (p \land \lnot q)$ & $\lnot (\lnot p  \lor \lnot q)$ &  $(\lnot p \lor  q)$
& $(\lnot q \lor \lnot p)$ & $(p \land q)$  \\
\hline
$T$ & $T$ & &&&&\\
$T$ & $F$ & &&&&\\
$F$ & $T$ & &&&&\\
$F$ & $F$ & &&&&\\
\end{tabular}
\end{center} \vfill
\end{document}