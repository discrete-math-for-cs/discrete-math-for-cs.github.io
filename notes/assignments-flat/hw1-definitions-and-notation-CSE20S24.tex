\documentclass[12pt, oneside]{article}

\usepackage[letterpaper, scale=0.8, centering]{geometry}
\usepackage{fancyhdr}
\setlength{\parindent}{0em}
\setlength{\parskip}{1em}

\pagestyle{fancy}
\fancyhf{}
\renewcommand{\headrulewidth}{0pt}
\rfoot{{\footnotesize Copyright Mia Minnes, 2024, Version \today~(\thepage)}}

\usepackage{titlesec}

\author{CSE20S24}

\newcommand{\instructions}{{\bf For all HW assignments:} 
These homework assignments may be done individually or in groups of up to 3 students.
Please ensure your name(s) and PID(s)
are clearly visible on the first page of your homework
submission, start each question on a new page, and upload the PDF to Gradescope.
If you're working in a group, {\it submit only one submission per group}: one partner uploads the
submission through their Gradescope account and then adds the other group member(s) to the Gradescope submission
by selecting their name(s) in the ``Add Group Members'' dialog box. You will need to re-add your group member(s)
every time you resubmit a new version of your assignment.

Each homework question will be graded either for
{\bf correctness} (including clear and precise explanations and justifications of all answers) or
{\bf fair effort completeness}. You may collaborate on ``graded for correctness''
questions only with CSE 20 students in your group; if your
 group has questions about a problem, you may ask in drop-in help hours or post a private
post (visible only to the Instructors) on Piazza.  
 For ``graded for completeness''
 questions: collaboration is allowed with any CSE 20 students this quarter; 
 if your group has questions about a problem, you may ask in drop-in 
 help hours or post a public post on Piazza.

All submitted homework for this class must be typed. 
You can use a word processing editor if you like (Microsoft Word, Open Office, Notepad, Vim, Google Docs, etc.) 
but you might find it useful to take this opportunity to learn LaTeX. 
LaTeX is a markup language used widely in computer science and mathematics. 
The homework assignments are typed using LaTeX and you can use the source files 
as templates for typesetting your solutions.

{\bf Integrity reminders}
\begin{itemize}
\item Problems should be solved together, not divided up between the partners. The homework is
designed to give you practice with the main concepts and techniques of the course, 
while getting to know and learn from your classmates.
\item You may not collaborate on homework questions graded for correctness with anyone other than your group members.
You may ask questions about the homework in office hours (of the instructor, TAs, and/or tutors) and 
on Piazza (as private notes viewable only to the Instructors).  
You \emph{cannot} use any online resources about the course content other than the class material 
from this quarter -- this is primarily to ensure that we all use consistent notation and
definitions (aligned with the textbook) and also to protect the learning experience you will have when
the `aha' moments of solving the problem authentically happen.
\item Do not share written solutions or partial solutions for homework with 
other students in the class who are not in your group. Doing so would dilute their learning 
experience and detract from their success in the class.
\end{itemize}

}

\newcommand{\gradeCorrect}{({\it Graded for correctness}) }
\newcommand{\gradeCorrectFirst}{\gradeCorrect\footnote{This means your solution 
will be evaluated not only on the correctness of your answers, but on your ability
to present your ideas clearly and logically. You should explain how you 
arrived at your conclusions, using
mathematically sound reasoning. Whether you use formal proof techniques or 
write a more informal argument
for why something is true, your answers should always be well-supported. 
Your goal should be to convince the
reader that your results and methods are sound.} }
\newcommand{\gradeComplete}{({\it Graded for completeness}) }
\newcommand{\gradeCompleteFirst}{\gradeComplete\footnote{This means you will 
get full credit so long as your submission demonstrates honest effort to 
answer the question. You will not be penalized for incorrect answers. 
To demonstrate your honest effort in answering the question, we 
expect you to include your attempt to answer *each* part of the question. 
If you get stuck with your attempt, you can still demonstrate 
your effort by explaining where you got stuck and what 
you did to try to get unstuck.} }

\usepackage{tikz}
\usetikzlibrary{circuits.logic.US,circuits.logic.IEC}

\usepackage{amssymb,amsmath,pifont,amsfonts,comment,enumerate,enumitem}
\usepackage{currfile,xstring,hyperref,tabularx,graphicx,wasysym}
\usepackage[labelformat=empty]{caption}
\usepackage{xcolor}
\usepackage{multicol,multirow,array,listings,tabularx,lastpage,textcomp,booktabs}

\lstnewenvironment{algorithm}[1][] {   
    \lstset{ mathescape=true,
        frame=tB,
        numbers=left, 
        numberstyle=\tiny,
        basicstyle=\rmfamily\scriptsize, 
        keywordstyle=\color{black}\bfseries,
        keywords={,procedure, div, for, to, input, output, return, datatype, function, in, if, else, foreach, while, begin, end, }
        numbers=left,
        xleftmargin=.04\textwidth,
        #1
    }
}
{}
\lstnewenvironment{java}[1][]
{   
    \lstset{
        language=java,
        mathescape=true,
        frame=tB,
        numbers=left, 
        numberstyle=\tiny,
        basicstyle=\ttfamily\scriptsize, 
        keywordstyle=\color{black}\bfseries,
        keywords={, int, double, for, return, if, else, while, }
        numbers=left,
        xleftmargin=.04\textwidth,
        #1
    }
}
{}

\newcommand\abs[1]{\lvert~#1~\rvert}
\newcommand{\st}{\mid}

\newcommand{\A}[0]{\texttt{A}}
\newcommand{\C}[0]{\texttt{C}}
\newcommand{\G}[0]{\texttt{G}}
\newcommand{\U}[0]{\texttt{U}}

\newcommand{\cmark}{\ding{51}}
\newcommand{\xmark}{\ding{55}}

  
\title{hw1-definitions-and-notation}
\date{Due: 4/9/24 at 5pm (no penalty late submission until 8am next morning)}
\begin{document}
\maketitle
\thispagestyle{fancy}


{\bf In this assignment,}

You will practice reading and
applying definitions to get comfortable working with mathematical language.

{\bf Relevant class material}: Week 1.

You will submit this assignment via Gradescope
(\href{https://www.gradescope.com}{https://www.gradescope.com}) 
in the assignment called ``hw1-definitions-and-notation''.

\instructions


{\bf Assigned questions}

\begin{enumerate}[labelindent=0pt, leftmargin=0pt]
    \item Modeling
    \begin{enumerate}[labelindent=0pt, leftmargin=0pt]
        \item\gradeCompleteFirst In class, we used $4$-tuples to represent the user ratings for four movies. 
        This representation is memory-efficient because we use the order of the components in the $4$-tuples to 
        represent which move is being rated. However, it is not easily extended when we want to add new movies to the database.
        Define a new model that would allow us to represent the user ratings of movie databases where we allow for new movies to be added.
        Use only the data types we have talked about in class: sets, $n$-tuples, and strings. Explain the design choices
        that you used to define your model by referencing properties of the data-type(s) you choose.
        Demonstrate your model by showing how the rating of the user who dislikes Dune and Oppenheimer and likes Barbie and Nimona
        is represented.
        
        \item\gradeComplete 
        Colors can be described as amounts of red, green, and blue mixed together
        \footnote{This RGB representation is common in web applications.  Many online tools are available to play around with mixing these colors,  e.g. \url{https://www.w3schools.com/colors/colors_rgb.asp}. }
        Mathematically, a color can be represented as a $3$-tuple $(r, g, b)$ where $r$ represents the red component, $g$ the green component, $b$ the blue component and where each of $r$, $g$, $b$ must be a value from this collection of numbers:
        \begin{quote}
        $\{$0, 1, 2, 3, 4, 5, 6, 7, 8, 9, 10, 11, 12, 13, 14, 15, 16, 17, 18, 19, 20, 21, 22, 23, 24, 25, 26, 27, 28, 29, 30, 31, 32, 33, 34, 35, 36, 37, 38, 39, 40, 41, 42, 43, 44, 45, 46, 47, 48, 49, 50, 51, 52, 53, 54, 55, 56, 57, 58, 59, 60, 61, 62, 63, 64, 65, 66, 67, 68, 69, 70, 71, 72, 73, 74, 75, 76, 77, 78, 79, 80, 81, 82, 83, 84, 85, 86, 87, 88, 89, 90, 91, 92, 93, 94, 95, 96, 97, 98, 99, 100, 101, 102, 103, 104, 105, 106, 107, 108, 109, 110, 111, 112, 113, 114, 115, 116, 117, 118, 119, 120, 121, 122, 123, 124, 125, 126, 127, 128, 129, 130, 131, 132, 133, 134, 135, 136, 137, 138, 139, 140, 141, 142, 143, 144, 145, 146, 147, 148, 149, 150, 151, 152, 153, 154, 155, 156, 157, 158, 159, 160, 161, 162, 163, 164, 165, 166, 167, 168, 169, 170, 171, 172, 173, 174, 175, 176, 177, 178, 179, 180, 181, 182, 183, 184, 185, 186, 187, 188, 189, 190, 191, 192, 193, 194, 195, 196, 197, 198, 199, 200, 201, 202, 203, 204, 205, 206, 207, 208, 209, 210, 211, 212, 213, 214, 215, 216, 217, 218, 219, 220, 221, 222, 223, 224, 225, 226, 227, 228, 229, 230, 231, 232, 233, 234, 235, 236, 237, 238, 239, 240, 241, 242, 243, 244, 245, 246, 247, 248, 249, 250, 251, 252, 253, 254, 255$\}$
        \end{quote}
        (This is the same definition as in the Week 1 Review quiz.)
        
        Can you find two different $3$-tuples that represent colors that are indistinguishable to your eye? You can use the website in the footnote to play around
        with different choices of red, green, and blue levels to see if you can distinguish between the resulting colors.
        Why or why not? 

        A complete answer will include the specific example $3$-tuples that work, along with a description of the colors that they represent and why they
        are indistinguishable, or an explanation of why there can't be such an example.
    \end{enumerate}

    \item Sets and functions
    \begin{enumerate}
        \item \gradeCorrectFirst Each of the sets below is described 
using set builder notation or recursion or as a result of set operations
applied to other known sets.  Rewrite each of the sets using the roster method.

Remember our discussions of data-types: use clear notation that 
is consistent with our class notes and definitions 
to communicate the data-types of the elements in each set.


\rule{0.5\textwidth}{.4pt}

{\it Sample response that can be used as reference for the detail expected 
in your answer:} 

The set $\{ \A \} \circ \{ \A\U, \A\C, \A\G\}$ can be written using
the roster method as 
\[
\{ \A\A\U, \A\A\C, \A\A\G \}
\]
because set-wise concatenation gives a set whose elements are 
all possible results of concatenating an element of the 
left set with an 
element of the right set. Since the left set in this example only
has one element, namely $\A$, each of the elements of the set we 
described starts with $\A$. There are three elements of this set, 
one for each of the distinct elements of the right set.

\rule{0.5\textwidth}{.4pt}

\begin{enumerate}
\item $$\{ n \in \mathbb{Z}^+ \mid n \leq 3 \} \times \{ m \in \mathbb{N} \mid m \leq 3\}$$ (Note: typo fixed Apr 3)
\item The set $X$ defined recursively as 
    \[
    \begin{array}{ll}
    \textrm{Basis Step: } & 1 \in X, 3 \in X, 5 \in X \\
    \textrm{Recursive Step: } & \textrm{If the integer } n \in X \textrm{, then the result of multiplying } n \textrm{ by } -1 \textrm{ is in }X
    \end{array}
    \]
\item $$\{ x \in S \mid rnalen(x) = 2 \} \circ \{x \in S \mid rnalen(x) = 0 \}$$
where $S$ is the set of RNA strands and $rnalen$ is the recursively defined
function that we discussed in class,
\[
\begin{array}{llll}
& & \textit{rnalen} : S & \to \mathbb{Z}^+ \\
\textrm{Basis Step:} & \textrm{If } b \in B\textrm{ then } & \textit{rnalen}(b) & = 1 \\
\textrm{Recursive Step:} & \textrm{If } s \in S\textrm{ and }b \in B\textrm{, then  } & \textit{rnalen}(sb) & = 1 + \textit{rnalen}(s)
\end{array}
\]
\item $$\{ (r,g,b) \in C \mid r+g+b = 2 \textrm{ and } g=1\}$$ where 
$C = \{ (r,g,b) \mid 0 \leq r \leq 255, 0 \leq g \leq 255, 0 \leq b \leq 255, r \in \mathbb{N}, g \in \mathbb{N}, b \in \mathbb{N} \}$
is the set that you worked with in Monday's review quiz.
\end{enumerate}

\item\gradeCorrect Recall the function which takes an ordered pair of ratings $4$-tuples and returns a measure of the difference between them
\[
    d_0: \{-1,0,1\}^4 \times \{-1,0,1\}^4 \to \mathbb{R}
\]
given by
\[
d_0 (~(~ (x_1, x_2, x_3, x_4), (y_1, y_2, y_3, y_4) ~) ~) = \sqrt{ (x_1 - y_1)^2 + (x_2 - y_2)^2 + (x_3 -y_3)^2 + (x_4 -y_4)^2}
\]

Define a {\bf new} function which we'll call $d_{new}$ with the same domain $\{-1,0,1\}^4 \times \{-1,0,1\}^4$
and codomain $\mathbb{R}$ but where there is some example pair of ratings $4$-tuples 
\[(~ (x_1, x_2, x_3, x_4), (y_1, y_2, y_3, y_4)~)\]
where
\[
d_0 (~(~ (x_1, x_2, x_3, x_4), (y_1, y_2, y_3, y_4) ~) ~) \neq d_{new} (~(~ (x_1, x_2, x_3, x_4), (y_1, y_2, y_3, y_4) ~) ~)
\]

    Your answer should include  {\bf both} a precise and clear definition for the rule defining
    $d_{new}$ which unambiguously specifies output for each input of the function {\bf and}
    the example ordered pair of ratings $4$-tuples that demonstrate that the functions are not equal.
    Also include a justification 
    of your answers with (clear, correct, complete) calculations for each of the function applications 
    and/or references to definitions and connecting them with
    the desired conclusion.

\item\gradeCorrect A function \textit{basecount} that computes the number of a given base 
$b$ appearing in a RNA strand $s$ is defined recursively:
    
\[
\begin{array}{llll}
& \textit{basecount} : S \times B & \to \mathbb{N} &\\
\textrm{Basis Step:} &  \\
\textrm{If } b_1 \in B, b_2 \in B & \textit{basecount}(~(b_1, b_2)~) & = 
        \begin{cases}
            1 & \textrm{when } b_1 = b_2 \\
            0 & \textrm{when } b_1 \neq b_2 \\
        \end{cases}& \\
\textrm{Recursive Step:} & \\
\textrm{If } s \in S, b_1 \in B, b_2 \in B &\textit{basecount}(~(s b_1, b_2)~) & =
        \begin{cases}
            1 + \textit{basecount}(~(s, b_2)~) & \textrm{when } b_1 = b_2 \\
            \textit{basecount}(~(s, b_2)~) & \textrm{when } b_1 \neq b_2 \\
        \end{cases} &
\end{array}
\]
Consider the function application 
\[
  basecount( ~(\A\C\A\U, \A)~)
\]
What is the input?  What is the output? Give an example of a different choice of input that 
gives the same output.

Your answer should include clearly labeled answers to each of the three parts of the question, 
along with a justification for the values of the applications that makes specific reference to 
the parts of the recursive definition of the $basecount$ function used to calculate it.
\end{enumerate}
\end{enumerate}
\end{document}