\documentclass[12pt, oneside]{article}

\usepackage[letterpaper, scale=0.8, centering]{geometry}
\usepackage{fancyhdr}
\setlength{\parindent}{0em}
\setlength{\parskip}{1em}

\pagestyle{fancy}
\fancyhf{}
\renewcommand{\headrulewidth}{0pt}
\rfoot{{\footnotesize Copyright Mia Minnes, 2024, Version \today~(\thepage)}}

\usepackage{titlesec}

\author{CSE20S24}

\newcommand{\instructions}{{\bf For all HW assignments:} 
These homework assignments may be done individually or in groups of up to 3 students.
Please ensure your name(s) and PID(s)
are clearly visible on the first page of your homework
submission, start each question on a new page, and upload the PDF to Gradescope.
If you're working in a group, {\it submit only one submission per group}: one partner uploads the
submission through their Gradescope account and then adds the other group member(s) to the Gradescope submission
by selecting their name(s) in the ``Add Group Members'' dialog box. You will need to re-add your group member(s)
every time you resubmit a new version of your assignment.

Each homework question will be graded either for
{\bf correctness} (including clear and precise explanations and justifications of all answers) or
{\bf fair effort completeness}. You may collaborate on ``graded for correctness''
questions only with CSE 20 students in your group; if your
 group has questions about a problem, you may ask in drop-in help hours or post a private
post (visible only to the Instructors) on Piazza.  
 For ``graded for completeness''
 questions: collaboration is allowed with any CSE 20 students this quarter; 
 if your group has questions about a problem, you may ask in drop-in 
 help hours or post a public post on Piazza.

All submitted homework for this class must be typed. 
You can use a word processing editor if you like (Microsoft Word, Open Office, Notepad, Vim, Google Docs, etc.) 
but you might find it useful to take this opportunity to learn LaTeX. 
LaTeX is a markup language used widely in computer science and mathematics. 
The homework assignments are typed using LaTeX and you can use the source files 
as templates for typesetting your solutions.

{\bf Integrity reminders}
\begin{itemize}
\item Problems should be solved together, not divided up between the partners. The homework is
designed to give you practice with the main concepts and techniques of the course, 
while getting to know and learn from your classmates.
\item You may not collaborate on homework questions graded for correctness with anyone other than your group members.
You may ask questions about the homework in office hours (of the instructor, TAs, and/or tutors) and 
on Piazza (as private notes viewable only to the Instructors).  
You \emph{cannot} use any online resources about the course content other than the class material 
from this quarter -- this is primarily to ensure that we all use consistent notation and
definitions (aligned with the textbook) and also to protect the learning experience you will have when
the `aha' moments of solving the problem authentically happen.
\item Do not share written solutions or partial solutions for homework with 
other students in the class who are not in your group. Doing so would dilute their learning 
experience and detract from their success in the class.
\end{itemize}

}

\newcommand{\gradeCorrect}{({\it Graded for correctness}) }
\newcommand{\gradeCorrectFirst}{\gradeCorrect\footnote{This means your solution 
will be evaluated not only on the correctness of your answers, but on your ability
to present your ideas clearly and logically. You should explain how you 
arrived at your conclusions, using
mathematically sound reasoning. Whether you use formal proof techniques or 
write a more informal argument
for why something is true, your answers should always be well-supported. 
Your goal should be to convince the
reader that your results and methods are sound.} }
\newcommand{\gradeComplete}{({\it Graded for completeness}) }
\newcommand{\gradeCompleteFirst}{\gradeComplete\footnote{This means you will 
get full credit so long as your submission demonstrates honest effort to 
answer the question. You will not be penalized for incorrect answers. 
To demonstrate your honest effort in answering the question, we 
expect you to include your attempt to answer *each* part of the question. 
If you get stuck with your attempt, you can still demonstrate 
your effort by explaining where you got stuck and what 
you did to try to get unstuck.} }



\usepackage{amssymb,amsmath,pifont,amsfonts,comment,enumerate,enumitem}
\usepackage{currfile,xstring,hyperref,tabularx,graphicx,wasysym}
\usepackage[labelformat=empty]{caption}
\usepackage{xcolor}
\usepackage{multicol,multirow,array,listings,tabularx,lastpage,textcomp,booktabs}

\lstnewenvironment{algorithm}[1][] {   
    \lstset{ mathescape=true,
        frame=tB,
        numbers=left, 
        numberstyle=\tiny,
        basicstyle=\rmfamily\scriptsize, 
        keywordstyle=\color{black}\bfseries,
        keywords={,procedure, div, for, to, input, output, return, datatype, function, in, if, else, foreach, while, begin, end, }
        numbers=left,
        xleftmargin=.04\textwidth,
        #1
    }
}
{}
\lstnewenvironment{java}[1][]
{   
    \lstset{
        language=java,
        mathescape=true,
        frame=tB,
        numbers=left, 
        numberstyle=\tiny,
        basicstyle=\ttfamily\scriptsize, 
        keywordstyle=\color{black}\bfseries,
        keywords={, int, double, for, return, if, else, while, }
        numbers=left,
        xleftmargin=.04\textwidth,
        #1
    }
}
{}

\newcommand\abs[1]{\lvert~#1~\rvert}
\newcommand{\st}{\mid}

\newcommand{\A}[0]{\texttt{A}}
\newcommand{\C}[0]{\texttt{C}}
\newcommand{\G}[0]{\texttt{G}}
\newcommand{\U}[0]{\texttt{U}}

\newcommand{\cmark}{\ding{51}}
\newcommand{\xmark}{\ding{55}}

  
\title{hw1-definitions-and-notation}
\date{Due: 4/9/24 at 5pm (no penalty late submission until 8am next morning)}
\begin{document}
\maketitle
\thispagestyle{fancy}


{\bf In this assignment,}

You will practice reading and
applying definitions to get comfortable working with mathematical language.

{\bf Relevant class material}: Week 1.

You will submit this assignment via Gradescope
(\href{https://www.gradescope.com}{https://www.gradescope.com}) 
in the assignment called ``hw1-definitions-and-notation''.

\instructions


{\bf Assigned questions}

\begin{enumerate}[labelindent=0pt, leftmargin=0pt]
    \item Modeling
    \begin{enumerate}[labelindent=0pt, leftmargin=0pt]
        \item\gradeCompleteFirst In class, we used $4$-tuples to represent the user ratings for four movies. 
        This representation is memory-efficient because we use the order of the components in the $4$-tuples to 
        represent which move is being rated. However, it is not easily extended when we want to add new movies to the database.
        Define a new model that would allow us to represent the user ratings of movie databases where we allow for new movies to be added.
        Use only the data types we have talked about in class: sets, $n$-tuples, and strings. Explain the design choices
        that you used to define your model by referencing properties of the data-type(s) you choose.
        Demonstrate your model by showing how the rating of the user who dislikes Dune and Oppenheimer and likes Barbie and Nimona
        is represented.
        
        \item\gradeComplete 
        Colors can be described as amounts of red, green, and blue mixed together
        \footnote{This RGB representation is common in web applications.  Many online tools are available to play around with mixing these colors,  e.g. \url{https://www.w3schools.com/colors/colors_rgb.asp}. }
        Mathematically, a color can be represented as a $3$-tuple $(r, g, b)$ where $r$ represents the red component, $g$ the green component, $b$ the blue component and where each of $r$, $g$, $b$ must be a value from this collection of numbers:
        \begin{quote}
        $\{$0, 1, 2, 3, 4, 5, 6, 7, 8, 9, 10, 11, 12, 13, 14, 15, 16, 17, 18, 19, 20, 21, 22, 23, 24, 25, 26, 27, 28, 29, 30, 31, 32, 33, 34, 35, 36, 37, 38, 39, 40, 41, 42, 43, 44, 45, 46, 47, 48, 49, 50, 51, 52, 53, 54, 55, 56, 57, 58, 59, 60, 61, 62, 63, 64, 65, 66, 67, 68, 69, 70, 71, 72, 73, 74, 75, 76, 77, 78, 79, 80, 81, 82, 83, 84, 85, 86, 87, 88, 89, 90, 91, 92, 93, 94, 95, 96, 97, 98, 99, 100, 101, 102, 103, 104, 105, 106, 107, 108, 109, 110, 111, 112, 113, 114, 115, 116, 117, 118, 119, 120, 121, 122, 123, 124, 125, 126, 127, 128, 129, 130, 131, 132, 133, 134, 135, 136, 137, 138, 139, 140, 141, 142, 143, 144, 145, 146, 147, 148, 149, 150, 151, 152, 153, 154, 155, 156, 157, 158, 159, 160, 161, 162, 163, 164, 165, 166, 167, 168, 169, 170, 171, 172, 173, 174, 175, 176, 177, 178, 179, 180, 181, 182, 183, 184, 185, 186, 187, 188, 189, 190, 191, 192, 193, 194, 195, 196, 197, 198, 199, 200, 201, 202, 203, 204, 205, 206, 207, 208, 209, 210, 211, 212, 213, 214, 215, 216, 217, 218, 219, 220, 221, 222, 223, 224, 225, 226, 227, 228, 229, 230, 231, 232, 233, 234, 235, 236, 237, 238, 239, 240, 241, 242, 243, 244, 245, 246, 247, 248, 249, 250, 251, 252, 253, 254, 255$\}$
        \end{quote}
        (This is the same definition as in the Week 1 Review quiz.)
        
        Can you find two different $3$-tuples that represent colors that are indistinguishable to your eye? You can use the website in the footnote to play around
        with different choices of red, green, and blue levels to see if you can distinguish between the resulting colors.
        Why or why not? 

        A complete answer will include the specific example $3$-tuples that work, along with a description of the colors that they represent and why they
        are indistinguishable, or an explanation of why there can't be such an example.
    \end{enumerate}

    \item Sets and functions
    \begin{enumerate}
        \item \gradeCorrectFirst Each of the sets below is described 
using set builder notation or recursion or as a result of set operations
applied to other known sets.  Rewrite each of the sets using the roster method.

Remember our discussions of data-types: use clear notation that 
is consistent with our class notes and definitions 
to communicate the data-types of the elements in each set.


\rule{0.5\textwidth}{.4pt}

{\it Sample response that can be used as reference for the detail expected 
in your answer:} 

The set $\{ \A \} \circ \{ \A\U, \A\C, \A\G\}$ can be written using
the roster method as 
\[
\{ \A\A\U, \A\A\C, \A\A\G \}
\]
because set-wise concatenation gives a set whose elements are 
all possible results of concatenating an element of the 
left set with an 
element of the right set. Since the left set in this example only
has one element, namely $\A$, each of the elements of the set we 
described starts with $\A$. There are three elements of this set, 
one for each of the distinct elements of the right set.

\rule{0.5\textwidth}{.4pt}

\begin{enumerate}
\item $$\{ n \in \mathbb{Z}^+ \mid n \leq 3 \} \times \{ m \in \mathbb{N} \mid m \leq 3\}$$ (Note: typo fixed Apr 3)
\item The set $X$ defined recursively as 
    \[
    \begin{array}{ll}
    \textrm{Basis Step: } & 1 \in X, 3 \in X, 5 \in X \\
    \textrm{Recursive Step: } & \textrm{If the integer } n \in X \textrm{, then the result of multiplying } n \textrm{ by } -1 \textrm{ is in }X
    \end{array}
    \]
\item $$\{ x \in S \mid rnalen(x) = 2 \} \circ \{x \in S \mid rnalen(x) = 0 \}$$
where $S$ is the set of RNA strands and $rnalen$ is the recursively defined
function that we discussed in class,
\[
\begin{array}{llll}
& & \textit{rnalen} : S & \to \mathbb{Z}^+ \\
\textrm{Basis Step:} & \textrm{If } b \in B\textrm{ then } & \textit{rnalen}(b) & = 1 \\
\textrm{Recursive Step:} & \textrm{If } s \in S\textrm{ and }b \in B\textrm{, then  } & \textit{rnalen}(sb) & = 1 + \textit{rnalen}(s)
\end{array}
\]
\item $$\{ (r,g,b) \in C \mid r+g+b = 2 \textrm{ and } g=1\}$$ where 
$C = \{ (r,g,b) \mid 0 \leq r \leq 255, 0 \leq g \leq 255, 0 \leq b \leq 255, r \in \mathbb{N}, g \in \mathbb{N}, b \in \mathbb{N} \}$
is the set that you worked with in Monday's review quiz.
\end{enumerate}

\item\gradeCorrect Recall the function which takes an ordered pair of ratings $4$-tuples and returns a measure of the difference between them
\[
    d_0: \{-1,0,1\}^4 \times \{-1,0,1\}^4 \to \mathbb{R}
\]
given by
\[
d_0 (~(~ (x_1, x_2, x_3, x_4), (y_1, y_2, y_3, y_4) ~) ~) = \sqrt{ (x_1 - y_1)^2 + (x_2 - y_2)^2 + (x_3 -y_3)^2 + (x_4 -y_4)^2}
\]

Define a {\bf new} function which we'll call $d_{new}$ with the same domain $\{-1,0,1\}^4 \times \{-1,0,1\}^4$
and codomain $\mathbb{R}$ but where there is some example pair of ratings $4$-tuples 
\[(~ (x_1, x_2, x_3, x_4), (y_1, y_2, y_3, y_4)~)\]
where
\[
d_0 (~(~ (x_1, x_2, x_3, x_4), (y_1, y_2, y_3, y_4) ~) ~) \neq d_{new} (~(~ (x_1, x_2, x_3, x_4), (y_1, y_2, y_3, y_4) ~) ~)
\]

    Your answer should include  {\bf both} a precise and clear definition for the rule defining
    $d_{new}$ which unambiguously specifies output for each input of the function {\bf and}
    the example ordered pair of ratings $4$-tuples that demonstrate that the functions are not equal.
    Also include a justification 
    of your answers with (clear, correct, complete) calculations for each of the function applications 
    and/or references to definitions and connecting them with
    the desired conclusion.

\item\gradeCorrect A function \textit{basecount} that computes the number of a given base 
$b$ appearing in a RNA strand $s$ is defined recursively:
    
\[
\begin{array}{llll}
& \textit{basecount} : S \times B & \to \mathbb{N} &\\
\textrm{Basis Step:} &  \\
\textrm{If } b_1 \in B, b_2 \in B & \textit{basecount}(~(b_1, b_2)~) & = 
        \begin{cases}
            1 & \textrm{when } b_1 = b_2 \\
            0 & \textrm{when } b_1 \neq b_2 \\
        \end{cases}& \\
\textrm{Recursive Step:} & \\
\textrm{If } s \in S, b_1 \in B, b_2 \in B &\textit{basecount}(~(s b_1, b_2)~) & =
        \begin{cases}
            1 + \textit{basecount}(~(s, b_2)~) & \textrm{when } b_1 = b_2 \\
            \textit{basecount}(~(s, b_2)~) & \textrm{when } b_1 \neq b_2 \\
        \end{cases} &
\end{array}
\]
Consider the function application 
\[
  basecount( ~(\A\C\A\U, \A)~)
\]
What is the input?  What is the output? Give an example of a different choice of input that 
gives the same output.

Your answer should include clearly labeled answers to each of the three parts of the question, 
along with a justification for the values of the applications that makes specific reference to 
the parts of the recursive definition of the $basecount$ function used to calculate it.
\end{enumerate}
\end{enumerate}
\newpage

\title{hw2-numbers}
\date{Due: 4/16/24 at 5pm (no penalty late submission until 8am next morning)}

\maketitle
\thispagestyle{fancy}


{\bf In this assignment,}

You will applying functions and tracing algorithms in multiple contexts, 
and exploring properties of positional number representations.

{\bf Relevant class material}: Week 2.

You will submit this assignment via Gradescope
(\href{https://www.gradescope.com}{https://www.gradescope.com}) 
in the assignment called ``hw2-numbers''.

\instructions

{\bf Assigned questions}

\begin{enumerate}[labelindent=0pt, leftmargin=0pt]
\item Functions and algorithms: in this question you'll explore the properties of the (integer) power and logarithm
functions. We'll use some definitions we introduced in class this week, namely the function
$b^i$ with domain $\mathbb{Z}^+ \times \mathbb{N}$ and codomain $\mathbb{N}$ defined recursively by
\begin{align*}    
&\textrm{Basis Step:} \\
&b^0 = 1 \\
&\textrm{Recursive Step:}\\
&\textrm{If } i \in \mathbb{N}, b^{i+1} = b \cdot b^i
\end{align*}
and the algorithm:


\begin{algorithm}[caption={Calculating integer part of base $b$ logarithm}]
   procedure $logb$($b$,$n$: positive integers with $b > 1$)
   $i$ := $0$
   while $n$ > $b-1$
     $i$ := $i + 1$
     $n$ := $n$ div $b$
   return $i$ $\{ i$ holds the integer part of the base $b$ logarithm of $n\}$
\end{algorithm} 

    \begin{enumerate}
    \item\gradeCorrectFirst Choose a positive integer $b$ between $3$ and $6$ (inclusive) and choose a 
    nonnegative integer $i$ between $2$ and $5$ (inclusive). Demonstrate how to calculate 
    the result of the $logb$ algorithm (procedure) when its input is the base $b$ you chose and
    $n = b^i$ (for the $i$ you choose.) 
    A complete answer will include the specific choice of $b$ and $i$, along with 
    trace of the calculations of $b^i$ and the computation of the algorithm, including 
    (clear, correct, complete) calculations for each of the function applications 
    and/or references to definitions and a trace table for algorithm that includes the values of all relevant 
    variables at each iteration (See the annotated Week 2 notes for the level of detail expected
    in a trace of a function application and a trace table).
    \item\gradeCorrect Now we'll go in other order: Demonstrate how to calculate the result of 
    $7^y$ where $y$ is the result of the $logb$ algorithm (procedure) when its input is $b=7$ and $n=30$.
    A complete answer will include clearly labelled 
    traces of the calculations of $b^i$ and the computation of the algorithm, including 
    (clear, correct, complete) calculations for each of the function applications 
    and/or references to definitions and a trace table for algorithm that includes the values of all relevant 
    variables at each iteration (See the annotated Week 2 notes for the level of detail expected
    in a trace of a function application and a trace table).    
    \item\gradeCompleteFirst Logarithms and powers are supposed to ``undo" one another. Explain whether your work in parts (a) and (b)
    supports that idea,  whether you saw anything confusing or surprising about these calculations, and how to explain what you saw.
    \end{enumerate}

\item Base expansions
    \begin{enumerate}
        \item\gradeComplete Pick an integer between $50$ and $1000$ (inclusive) that you (or one of your
        group members) came across at some point this week. 
        In a sentence or two, give some context for why you're choosing this number). 
        Write the base expansion of your chosen number in base $2$ (binary), base $3$ (ternary), base $4$, and base $16$ (hexadecimal).
        \item\gradeCorrect What is the {\bf smallest} width $w$ in which they could write your chosen number in base $8$ (octal) 
        fixed-width $w$? Justify your answer with reference to the definitions of fixed-width expansions and relevant calculations.
        \item\gradeCorrect Express in roster method the set of numbers between $1$ and $2$ (exclusive) that be written without error 
        (full precision)
        in binary fixed width expansion with integer part width $3$ and fractional part width $2$. Justify your answer with 
        reference to the definitions of fixed-width expansions and relevant calculations.
        \item\gradeCorrect Consider the strings of $1$s that have length $3$, $5$, $7$, and $10$.
        Calculate the numbers 
        \begin{itemize}
            \item[] $[111]_{s,3}$
            \item[] $[11111]_{s,5}$
            \item[] $[1111111]_{s,7}$
            \item[] $[1111111111]_{s,10}$
            \item[] $[111]_{2c,3}$
            \item[] $[11111]_{2c,5}$
            \item[] $[1111111]_{2c,7}$
            \item[] $[1111111111]_{2c,10}$
        \end{itemize}
        Justify your answers with specific reference to the definitions of sign-magnitude and $2$s complement expansions and relevant calculations.
        \item \gradeComplete What patterns do you notice in your calculations in part (d)?
    \end{enumerate}

\item Multiple representations

Recall that, mathematically, a color can be represented as a $3$-tuple $(r, g, b)$ 
where $r$ represents the red component, $g$ the green component, $b$ the blue component and where each of $r$, $g$, $b$ must be from the collection $\{x \in \mathbb{N}\mid 0 \leq x \leq 255 \}$.
As an alternative representation, in this assignment
we'll use base $b$ fixed-width expansions to represent colors
as individual numbers (this definition was introduced in this week's Review quiz).

{\bf Definition}: A {\bf hex color} is a nonnegative
integer, $n$, that has a base $16$ fixed-width $6$  expansion
$$n = (r_1r_2g_1g_2b_1b_2)_{16,6}$$ 
where $(r_1r_2)_{16,2}$ is the red
component, $(g_1g_2)_{16,2}$ is the green component, 
and $(b_1b_2)_{16,2}$ is the
blue. 
For this question, let's call the set of hex colors $H$. In the Week 2 Review quiz (Question 3d), 
we explored a few different set builder definitions for $H$.

\begin{enumerate}
\item \gradeComplete Rewrite the set builder definition of a set below so that it 
refers to colors rather than numbers: $\{ c \in H \mid c \mod 256 = 0\}$. A complete answer will 
justify the new set builder definition by connection with the definition of hex colors and how it impacts
the colors that satisfy the specific property for this set.
\item \gradeCorrect Rewrite the set builder definition of a set below so that it 
refers to colors rather than numbers: $\{ c \in H \mid c < 65536\}$. A complete answer will 
justify the new set builder definition by connection with the definition of hex colors and how it impacts
the colors that satisfy the specific property for this set.
\item \gradeCorrect In art, we can mix two colors to get a new color. For example, red and blue make 
purple, blue and yellow make green, red and yellow make orange. 
A mathematical definition of {\bf hex color mixing} would be a function with domain $H \times H$
and codomain $H$ where the result of applying this function to an input $(n_1, n_2)$ is the hex color
that results from mixing the hex colors $n_1$ and $n_2$.


\rule{0.5\textwidth}{.4pt}

{\it Sample work for a related question that can be used as reference for the detail expected 
in your answer:} Consider the attempted mathematical definition  $f_1: H\times H \to H$ given by
\[
f_1 (~(n_1, n_2) ~) = n_1 + n_2
\]
We will show that this function is not well-defined so it cannot give a hex color mixing definition. 
The reason it is not well-defined
is that the application of the rule sometimes gives values that are not in the stated codomain. 
To see this, we need an example input to $f_1$ for which the output of the rule is not in $H$.
Here is one such example: $(n_1, n_2) = (~(FFFFFF)_{16,6}, (FFFFFF)_{16,6}~)$. 
This example is in $H \times H$
because $(FFFFFF)_{16,6}$ is a nonnegative integer that has a base $16$ fixed-width $6$ expansion 
so it is a hex color.
We now apply the definition of the rule in $f_1$:
\begin{align*}
f_1 (~(n_1, n_2) ~) &=  f_1 ( (~(FFFFFF)_{16,6}, (FFFFFF)_{16,6}~) ) = (FFFFFF)_{16,6} + (FFFFFF)_{16,6} \\
&= 2 \left( 15\cdot 16^5 + 15\cdot 16^4 + 15 \cdot 16^3 + 15 \cdot 16^2 + 15 \cdot 16^1 + 15 \cdot 16^0 \right) \\
&= 2 \cdot 15 \cdot (1048576 + 65536 + 4096 + 256 + 16 + 1) = 2 \cdot 15 \cdot 1118481 = 33554430
\end{align*}
This is not a hex color because it is greater than or equal to $16^6 = 16777216$, 
so (using the Week 2 notes on which numbers can be represented
with fixed-width expansions) it does not have a hexadecimal {\bf fixed width $6$} expansion.

\rule{0.5\textwidth}{.4pt}

In this question, we'll look at another attempted mathematical definition for hex color mixing. 
We define $f_2: H\times H \to H$ given by
\[
f_2 (~(n_1, n_2) ~) = (n_1 + n_2) \textbf{ div } 2
\]

This is a well-defined function. You do not need to prove this or hand it in, but it's good practice to make 
sure you understand why it's well-defined. Notice that the function application 
\[
f_2 ( ~(~ (FF0000)_{16,6} , (FFFE00)_{16,6} ~)~)
\]
can be calculated as:
\begin{align*}
f_2&(((FF0000)_{16,6}, (FFFE00)_{16,6}))=
 ((FF0000)_{16,6} + (FFFE00)_{16,6})) \textbf{ div } 2\\
&= \big(~ \left(15\cdot16^5 + 15\cdot16^4 + 0\cdot16^3 + 0\cdot16^2 + 0\cdot16^1 + 0\cdot16^0 \right) \\
&~~~~+~
    \left((15\cdot16^5 + 15\cdot16^4 + 15\cdot16^3 + 14\cdot16^2 + 0\cdot16^1 + 0\cdot16^0\right )~\big )\textbf{ div } 2\\
&= (16711680 + 16776704) \textbf{ div } 2 = 33488384 \textbf{ div } 2 = 16744192
\end{align*}
Since this is less than $16^6 = 16777216$, we can represent it using base $16$ fixed-width $6$:.
$$(16744192)_{10} = 15\cdot16^5 + 15\cdot16^4 + 7\cdot16^3 + 15\cdot16^2 + 0\cdot16^1 + 0\cdot16^0 = (FF7F00)_{16,6}$$

Using the web tool \url{https://www.w3schools.com/colors/colors_rgb.asp}, we can verify that $(FF0000)_{16,6}$ is red, 
$(FFFE00)_{16,6}$ is yellow, and $(FF7F00)_{16,6}$ is orange.

Show that $f_2$ does not work as a hex color mixing definition by finding
another ordered pair of hex colors for which the result of applying $f_2$ does not give the expected hex color.
Include the numerical description of each colors you mention, alongside a description of them in English. 
Describe how you know what these colors are
(if you use a web color tool, include its URL in your submission writeup; if not, describe your reasoning).
Justify your 
example with (clear, correct, complete) calculations and/or references to definitions, and connecting them with
the desired conclusion.
\end{enumerate}
\end{enumerate}
\newpage

\title{Project}
\date{Part 1 due 10/14/21; Part 2 due 11/4/21; Part 3 due 12/2/21}


\maketitle
\thispagestyle{fancy}
The project component of this class will be an opportunity for you to extend your 
work on assignments and explore applications of your choosing. 

{\it Why?}
To go deeper and explore the material from discrete math and how it relates to Computer Science.

{\it How?} During emergency remote instruction last academic year, we discovered
that video assessement and some open-ended personalized projects help ensure fairness
and can be less stressful for students than in-person midterm exams. Asynchronous project
submission also gives flexibility and allows more physical distancing.

Your videos: We will delete all the videos we receive from you after assigning final grades for the course, 
and they will be stored in a university-controlled Google Drive directory 
only accessible to the course staff during the quarter. 
Please send an email to the instructor (minnes@eng.ucsd.edu) if you have 
concerns about 
the video / screencast components of this project or cannot complete projects in this style for some reason.

You may produce screencasts with any software you choose. 
One option is to record yourself with Zoom; a tutorial on how to use Zoom to record a 
screencast (courtesy of Prof. Joe Politz)  is here: 

\url{https://drive.google.com/open?id=1KROMAQuTCk40zwrEFotlYSJJQdcG_GUU}.

The video that was produced from that recording session in Zoom is here:

\url{https://drive.google.com/open?id=1MxJN6CQcXqIbOekDYMxjh7mTt1TyRVMl}

\subsection*{What resources can you use?}
This project must be completed individually, without any help from other people, 
including the course staff (other than logistics support if you get stuck with screencast). 

You can use any of this quarter's CSE 20 offering (notes, readings, class videos, homework feedback). 
These resources should be more than enough. If you are struggling to get started and want to 
look elsewhere online, you must acknowledge this by listing and citing any resources you consult 
(even if you do not explicitly quote them). Link directly to them and include the name of the 
author / video creator and the reason you consulted this reference. The work you submit for 
the project needs to be your own. Again, you shouldn't need to look anywhere other 
than this quarter's material and doing so may result in definitions or notations 
that conflict with our norms in this class so think carefully before you go down this path.

The project has three parts. 
\begin{itemize}
    \item Part 1 of Project: due Thursday October 14
    \item Part 2 of Project: due Thursday November 4
    \item Part 3 of Project: due Thursday December 2
\end{itemize}

\newpage
\subsection*{Part 1: due Thursday October 14}
\subsubsection*{Written component}
\begin{enumerate}
\item 

In Week 1, we discussed the mathematical definition of a function, namely
that a {\bf function} is defined by its (1) domain, (2) codomain, and (3) rule assigning each 
element in the domain exactly one element in the codomain.

\begin{enumerate}
    \item Write out the definition for an example function you make up.
    You can choose any example you like, so long as it is your own independent 
    effort and it is not a function from a class example, homework, or Review Quiz in this class so far.
    Use the notation defined in class. Label and define the domain, codomain, and rule clearly.
    \item Calculate the result of applying your function to an element in its domain.
    Just like in homework, include (clear, correct, complete) calculations and/or references to definitions
    to present your function application.
\end{enumerate}

\item In CSE 20 this quarter, we will be exploring the applications of discrete mathematics for core CS topics. 
The following videos introduce some of these topics and the 
work happening here at UCSD to explore them. Pick one of the following videos, watch it, and 
then write a few sentences answering the following:
\begin{enumerate}
\item Which video did you watch? Why did you choose the video you watched?
\item What followup question(s) would you like to ask the person in the video about their work?
At least one of your followup questions should be about a technical aspect of the work that you
would like to learn more about.
\end{enumerate}

({\it Click on the video titles below for the links})

\href{https://www.youtube.com/watch?v=PrAoks7OhE8}{Bioinformatics and virology: 
{\it Niema Moshiri - Genome Sequence Alignment}}

\href{https://www.uctv.tv/computer-science/search-details.aspx?showID=33425}{Human robotics interaction: 
{\it Angelique Taylor - Improving Human-Robot Interaction}}

\href{https://www.uctv.tv/computer-science/search-details.aspx?showID=33421}{Computer vision: 
{\it Manmohan Chandraker: Giving Computers the Gift of Vision}}

\href{https://www.uctv.tv/computer-science/search-details.aspx?showID=33423}{Data centers and energy efficiency:
{\it  Max Mellette: Improving Data Centers with Photonics}}

\href{https://www.uctv.tv/computer-science/search-details.aspx?showID=33420}{Programming languages 
and data structures: {\it Nadia Polikarpova: Creating New Languages for Programming}}

\href{https://www.uctv.tv/computer-science/search-details.aspx?showID=34350}{Machine learning (and surfing) 
for climate science: {\it Studying Climate Change Through Surfing with Smartfin - We Are CSE: Jasmine Simmons}}
\end{enumerate}
\subsubsection*{Video component}
Presenting your reasoning and demonstrating it via screenshare are important 
skills that also show us a lot of your learning. Getting practice with this style of 
presentation is a good thing for you to learn in general and a rich way for us to assess your skills. 

Prepare a 3-5 minute screencast video that starts with 
your face and your student ID for a few seconds at the beginning, and introduce yourself audibly while on screen. 
You don't have to be on camera for the rest of the video, though it's fine if you are. 
We are looking for a brief confirmation that it's you creating the video and doing the work 
submitted for the project.

Then, explain your work in question 1 of the written component.
Discuss at least one potential mistake that someone solving 
a similar question should avoid (this could be a mistake you made while thinking about this 
problem or something you anticipate a classmate might struggle with); explain why the 
mistake is wrong and how to fix it.

Finally, explain any differences between your pre-survey description 
of ``what a function is" and the mathematical definition of what a function is. 
You should have an email copy of your responses to the pre-survey, 
and you can refer to what you wrote in your explanation.

Use this Google form

\url{https://forms.gle/SLd8SrJdXR5HCLQr7}  (click to follow link) 

to directly upload a video file for this assignment.
It should be a file that you can easily play on your system. 
One way you can determine if this is true is if you can store it on your Google Drive and play it from there,
since that's how we will watch it.

\subsubsection*{Checklist (this is how we will grade Part 1 of the project)}
\begin{itemize}
\item Question 1
    \begin{itemize}
        \item The function definition is complete and uses correct notation, 
        and is different from class, homework, and quiz examples.
        \item The calculation of the function application is correct and is 
        supported by clear, correct, complete justification.
    \end{itemize}
\item Question 2
    \begin{itemize}
        \item (At least) one of the videos is mentioned and a reason for selecting it is included.
        \item At least one technical question in described that is connected to the video selected.
    \end{itemize}
\item Video
    \begin{itemize}
        \item Video loads correctly and is between 3 and 5 minutes. It includes your face and your student ID, 
        and you introduce yourself audibly while on screen.
        \item Video presents your solution for Question 1.
        \item A potential mistake is presented and discussed.
        \item The mathematical definition of function is compared to your response in the pre-survey.
    \end{itemize}
\end{itemize}

\newpage
\subsection*{Part 2: due Thursday November 4}
\subsubsection*{Written component}
\begin{enumerate}
\item In this part of the project, you will select one question from one of the review quizzes 
from 10/4/2021 (Monday of Week 1) to 10/29/2021 (Friday of Week 5) to revisit. 
Include the problem statement, why you picked this question (e.g. what is interesting about it, 
what is hard about it, or why you wanted to take a second look at it), and your solution. 
    \begin{itemize}
        \item Question selection: you can pick any {\bf one question} listed in the Review 
        sections of the relevant notes documents, and you must address all of its parts.
        \item For each part of your chosen question: prepare a complete solution 
        (you can use the homework solutions we post for guidance about the style). 
        Your submission will be evaluated not only on the correctness of your answers, 
        but on your ability to present your ideas clearly and logically. 
        You should explain how you arrived at your conclusions, using mathematically 
        sound reasoning. Your goal should be to convince the reader that your results 
        and methods are sound. Imagine you are preparing these solutions for someone else 
        taking CSE 20 who missed that week and is ``catching up".
    \end{itemize}
\item In this part of the project, you'll consider the importance of data types
and precision in Computer Science. 
Read three articles

(1) This discussion of the causes of a wide-spread problem in published genomics papers

\url{https://www.nature.com/articles/d41586-021-02211-4} (Click to follow link)

from the journal Nature; 

(2) this IEEE profile of Katherine Johnson, 

{\tiny \url{https://spectrum.ieee.org/the-institute/ieee-history/katherine-johnson-the-hidden-figures-mathematician-who-got-astronaut-john-glenn-into-space}}

(Click to follow link) a NASA ``computer" who calculated trajectories for 
early space exploration and who passed away in 2020; and 

(3) this NASA report about the unsuccessful 1999 Mars Climate Orbiter mission

\url{https://solarsystem.nasa.gov/missions/mars-climate-orbiter/in-depth/}

In one or two sentences, summarize the main lesson you draw from each article.

Thinking back to your own experiences, give an example of when you used computers or Computer Science
to help you *avoid* an error. Also, give an example when your use of computers or Computer Science
*caused* an error.

What measures do you take to increase your confidence in the results of your own human and digital 
(i.e. machine) computation? Why do you think these are sufficient?


\end{enumerate}

\subsubsection*{Video component}

Presenting your reasoning and demonstrating it via screenshare are important skills that 
also show us a lot of your learning. Getting practice with this style of presentation 
is a good thing for you to learn in general and a rich way for us to assess your skills. 

Prepare a 3-5 minute screencast video explaining your work in question 1 of the written component.
During your solution presentation, point out at least one potential mistake that someone 
solving a similar question should avoid (this could be a mistake you made while thinking 
about this problem or something you anticipate a classmate might struggle with); 
explain why the mistake is wrong and how to fix it. 

You do not need to include complete details of every part of your solution. 
It is up to you to choose what is most important so that you can stick to the 
timing guidelines and still have time to include discussing potential mistakes.

Include your face and your student ID (we'd like a photo ID that includes your name 
and picture if possible) for a few seconds at the beginning, and introduce yourself 
audibly while on screen. You don't have to be on camera the whole time, though it's fine 
if you are. We are looking for a brief confirmation that it's you creating the 
video/doing the work attached to the video.

Use this Google form

\url{https://forms.gle/SLd8SrJdXR5HCLQr7}  (click to follow link) 

to directly upload a video file for this assignment.
It should be a file that you can easily play on your system. 
One way you can determine if this is true is if you can store it on your Google Drive and play it from there,
since that's how we will watch it.

\subsubsection*{Checklist (this is how we will grade Part 2 of the project)}
\begin{itemize}
\item Question 1
    \begin{itemize}
        \item Selected review quiz question is labelled clearly, including the day 
        it belongs to and the statement of the question.
        \item Solution is complete: it addresses each part of the review quiz question selected.
        \item Solution is correct: it clearly and correctly justifies the correct answer 
        for each part of the question. You are welcome to check your answers with the 
        Gradescope autograder (we will be reopening the review quizzes for this purpose). 
        We will evaluate your submissions for the quality of your justification.
    \end{itemize}
\item Question 2
    \begin{itemize}
        \item A key lesson from each of the three references is stated clearly and 
        is relevant to the message of the articles. Supporting explanations are included.
        \item A specific example of an instance where using computers/ CS *caused* an error is described.
        \item A specific example of an instance where using computers/ CS helped *avoid* an error is described.
        \item Lesson(s) are drawn from the previous experiences.
        \item Specific strategies for increasing confidence in computation are described and justified.
    \end{itemize}
    \item Video
    \begin{itemize}
        \item Video loads correctly and is between 3 and 5 minutes. It includes your face and your student ID, 
        and you introduce yourself audibly while on screen.
        \item Video presents your solution for Question 1.
        \item A potential mistake is presented and discussed.
    \end{itemize}
\end{itemize}

\newpage
\subsection*{Part 3: due Thursday December 2}
\subsubsection*{Written component}
\begin{enumerate}
    \item In this part of the project, you will analyze a quantified statement about RNA strands. 
    The definitions for RNA strands are available in the class notes. 
    Example quantified statements about RNA strands are in the homework. 
    Complete the following:
    \begin{enumerate}
        \item Write a quantified statement symbolically. Your quantified statement should satisfy {\bf all} of the following requirements:
            \begin{itemize}
                \item Have a nesting of quantifiers with at least one forall quantification 
                and at least one existential quantification.
                \item Have at least one negation and at least one binary logic operation (and, or, xor, if, iff).
                \item Negations appear only within predicates (that is, so that no negation is outside 
                a quantifier or an expression involving logical connectives).
                \item Use {\bf exactly one} of the predicates Mut, Ins, Del.
                \item Not be a statement we have analyzed already in class materials.
            \end{itemize}
        \item Translate your statement from part a. to English.
        \item Negate the whole statement from part a. and rewrite this negated statement so 
        that negations appear only within predicates (that is, so that no negation is outside a 
        quantifier or an expression involving logical connectives).
        \item Prove or disprove your statement from part a.
    \end{enumerate}
    \item In this part of the project, you'll consider the impact Computer Science has on society.
    Read two articles:

    (1) This policy piece about facial recognition software

    {\tiny \url{https://thehill.com/policy/technology/569543-federal-agencies-planning-to-expand-use-of-facial-recognition}}
    (Click to follow link); and 

    (2) this exploration of accessibility for visually impaired website users
    
    {\tiny \url{https://www.wsj.com/articles/colorblind-users-push-technology-designers-to-use-signals-beyond-color-11591351201}}
    (Click to follow link).

    Update (November 18): the Wall Street Journal about web design is now behind a pay wall. If you 
    do not have access to it, please read the following two articles instead:

    {\tiny \url{https://webaim.org/articles/visual/colorblind}}
    (Click to follow link) and
    {\tiny \url{https://webaim.org/resources/designers/}}
    (Click to follow link)


    In one or two sentences, summarize the main lesson you draw from each article.

    Give an example of an algorithm or computer system that you use or have been used by others to 
    make some decision that affects you. The algorithms may be institutional or personal, formal or 
    heuristic, and should output a 
    specific result or decision. Explain your answer, either by reference to your knowledge of 
    the algorithm itself or by observations you make 
    about outputs of the algorithm.

    Give an example of how someone different from you might have a different experience 
    with this algorithm. Support your example with lessons you learned in the 
    readings, citing what you learned from the articles.
\end{enumerate}

\subsubsection*{Video component}
Presenting your reasoning and demonstrating it via screenshare are important skills that 
also show us a lot of your learning. Getting practice with this style of presentation 
is a good thing for you to learn in general and a rich way for us to assess your skills. 

Prepare a 3-5 minute screencast video explaining your work in question 1 parts (c) and (d)
of the written component (i.e. the negation and proof).
During your solution presentation, point out at least one potential mistake that someone 
solving a similar question should avoid (this could be a mistake you made while thinking 
about this problem or something you anticipate a classmate might struggle with); 
explain why the mistake is wrong and how to fix it. 

You do not need to include complete details of every part of your solution to these parts. 
It is up to you to choose what is most important so that you can stick to the 
timing guidelines and still have time to include discussing potential mistakes.

Include your face and your student ID (we'd like a photo ID that includes your name 
and picture if possible) for a few seconds at the beginning, and introduce yourself 
audibly while on screen. You don't have to be on camera the whole time, though it's fine 
if you are. We are looking for a brief confirmation that it's you creating the 
video/doing the work attached to the video.

Use this Google form

\url{https://forms.gle/SLd8SrJdXR5HCLQr7}  (click to follow link) 

to directly upload a video file for this assignment.
It should be a file that you can easily play on your system. 
One way you can determine if this is true is if you can store it on your Google Drive and play it from there,
since that's how we will watch it.

\subsubsection*{Checklist (this is how we will grade Part 3 of the project)}
\begin{itemize}
    \item Question 1
        \begin{itemize}
            \item Quantified statement is clearly stated, is well-defined, syntactically correct, and meets all requirements.
            \item Translation to English is clear, correct, and complete.
            \item The negation of the quantified statement is clearly stated, is well-defined, syntactically correct, and meets all requirements.
            \item The proof or disproof of the original statement is clear, correct, and complete. 
        \end{itemize}
    \item Question 2
        \begin{itemize}
            \item A key lesson from each of the two references is stated clearly and 
            is relevant to the message of the articles. Supporting explanations are included.
            \item A specific example of an algorithm or computer systems that impacts you is described.
            \item A description of how this algorithm or computer system might impact someone different from you
            differently is included, and is supported with references to one or both of the articles.
            \end{itemize}
\item Video
    \begin{itemize}
        \item Video loads correctly and is between 3 and 5 minutes. It includes your face and your student ID, 
        and you introduce yourself audibly while on screen.
        \item Video presents your solution for Question 1 parts (c) and (d).
        \item A potential mistake is presented and discussed.
    \end{itemize}
\end{itemize}
\newpage

\end{document}