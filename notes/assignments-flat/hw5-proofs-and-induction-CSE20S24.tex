\documentclass[12pt, oneside]{article}

\usepackage[letterpaper, scale=0.8, centering]{geometry}
\usepackage{fancyhdr}
\setlength{\parindent}{0em}
\setlength{\parskip}{1em}

\pagestyle{fancy}
\fancyhf{}
\renewcommand{\headrulewidth}{0pt}
\rfoot{{\footnotesize Copyright Mia Minnes, 2024, Version \today~(\thepage)}}

\usepackage{titlesec}

\author{CSE20S24}

\newcommand{\instructions}{{\bf For all HW assignments:} 
These homework assignments may be done individually or in groups of up to 3 students.
Please ensure your name(s) and PID(s)
are clearly visible on the first page of your homework
submission, start each question on a new page, and upload the PDF to Gradescope.
If you're working in a group, {\it submit only one submission per group}: one partner uploads the
submission through their Gradescope account and then adds the other group member(s) to the Gradescope submission
by selecting their name(s) in the ``Add Group Members'' dialog box. You will need to re-add your group member(s)
every time you resubmit a new version of your assignment.

Each homework question will be graded either for
{\bf correctness} (including clear and precise explanations and justifications of all answers) or
{\bf fair effort completeness}. You may collaborate on ``graded for correctness''
questions only with CSE 20 students in your group; if your
 group has questions about a problem, you may ask in drop-in help hours or post a private
post (visible only to the Instructors) on Piazza.  
 For ``graded for completeness''
 questions: collaboration is allowed with any CSE 20 students this quarter; 
 if your group has questions about a problem, you may ask in drop-in 
 help hours or post a public post on Piazza.

All submitted homework for this class must be typed. 
You can use a word processing editor if you like (Microsoft Word, Open Office, Notepad, Vim, Google Docs, etc.) 
but you might find it useful to take this opportunity to learn LaTeX. 
LaTeX is a markup language used widely in computer science and mathematics. 
The homework assignments are typed using LaTeX and you can use the source files 
as templates for typesetting your solutions.

{\bf Integrity reminders}
\begin{itemize}
\item Problems should be solved together, not divided up between the partners. The homework is
designed to give you practice with the main concepts and techniques of the course, 
while getting to know and learn from your classmates.
\item You may not collaborate on homework questions graded for correctness with anyone other than your group members.
You may ask questions about the homework in office hours (of the instructor, TAs, and/or tutors) and 
on Piazza (as private notes viewable only to the Instructors).  
You \emph{cannot} use any online resources about the course content other than the class material 
from this quarter -- this is primarily to ensure that we all use consistent notation and
definitions (aligned with the textbook) and also to protect the learning experience you will have when
the `aha' moments of solving the problem authentically happen.
\item Do not share written solutions or partial solutions for homework with 
other students in the class who are not in your group. Doing so would dilute their learning 
experience and detract from their success in the class.
\end{itemize}

}

\newcommand{\gradeCorrect}{({\it Graded for correctness}) }
\newcommand{\gradeCorrectFirst}{\gradeCorrect\footnote{This means your solution 
will be evaluated not only on the correctness of your answers, but on your ability
to present your ideas clearly and logically. You should explain how you 
arrived at your conclusions, using
mathematically sound reasoning. Whether you use formal proof techniques or 
write a more informal argument
for why something is true, your answers should always be well-supported. 
Your goal should be to convince the
reader that your results and methods are sound.} }
\newcommand{\gradeComplete}{({\it Graded for completeness}) }
\newcommand{\gradeCompleteFirst}{\gradeComplete\footnote{This means you will 
get full credit so long as your submission demonstrates honest effort to 
answer the question. You will not be penalized for incorrect answers. 
To demonstrate your honest effort in answering the question, we 
expect you to include your attempt to answer *each* part of the question. 
If you get stuck with your attempt, you can still demonstrate 
your effort by explaining where you got stuck and what 
you did to try to get unstuck.} }



\usepackage{amssymb,amsmath,pifont,amsfonts,comment,enumerate,enumitem}
\usepackage{currfile,xstring,hyperref,tabularx,graphicx,wasysym}
\usepackage[labelformat=empty]{caption}
\usepackage{xcolor}
\usepackage{multicol,multirow,array,listings,tabularx,lastpage,textcomp,booktabs}

\lstnewenvironment{algorithm}[1][] {   
    \lstset{ mathescape=true,
        frame=tB,
        numbers=left, 
        numberstyle=\tiny,
        basicstyle=\rmfamily\scriptsize, 
        keywordstyle=\color{black}\bfseries,
        keywords={,procedure, div, for, to, input, output, return, datatype, function, in, if, else, foreach, while, begin, end, }
        numbers=left,
        xleftmargin=.04\textwidth,
        #1
    }
}
{}
\lstnewenvironment{java}[1][]
{   
    \lstset{
        language=java,
        mathescape=true,
        frame=tB,
        numbers=left, 
        numberstyle=\tiny,
        basicstyle=\ttfamily\scriptsize, 
        keywordstyle=\color{black}\bfseries,
        keywords={, int, double, for, return, if, else, while, }
        numbers=left,
        xleftmargin=.04\textwidth,
        #1
    }
}
{}

\newcommand\abs[1]{\lvert~#1~\rvert}
\newcommand{\st}{\mid}

\newcommand{\A}[0]{\texttt{A}}
\newcommand{\C}[0]{\texttt{C}}
\newcommand{\G}[0]{\texttt{G}}
\newcommand{\U}[0]{\texttt{U}}

\newcommand{\cmark}{\ding{51}}
\newcommand{\xmark}{\ding{55}}

  
\title{hw5-proofs-and-induction}
\date{Due: 5/21/24 at 5pm (no penalty late submission until 8am next morning)}
\begin{document}
\maketitle
\thispagestyle{fancy}

{\bf In this assignment}, you will work with recursively defined sets and functions and prove 
properties about them, practicing induction and other proof strategies.

{\bf Relevant class material}: Weeks 5,6,7.

You will submit this assignment via Gradescope
(\href{https://www.gradescope.com}{https://www.gradescope.com}) 
in the assignment called ``hw5-proofs-and-induction''.

\instructions

In your proofs and disproofs of statements below, justify each  step
by reference to  a component of the  following proof  strategies
we  have discussed so far, and/or to relevant definitions and calculations.

\vspace{-10pt}

\begin{itemize}
    \item A counterexample can be used to prove that  $\forall x P(x)$ is {\bf false}.
    \item  A witness can be used  to  prove that  $\exists x P(x)$ is {\bf true}.
    \item {\bf Proof of universal by exhaustion}: To prove that $\forall x \, P(x)$
is true when $P$ has a finite domain, evaluate the predicate at {\bf each} domain element to confirm that it is always T.
    \item  {\bf Proof by universal generalization}: To prove that $\forall x \, P(x)$
is true, we can take an arbitrary element $e$ from the domain and show that $P(e)$ is true, without making any assumptions about $e$ other than that it comes from the domain.
    \item To  prove  that $\exists x P(x)$ is {\bf false}, write the universal statement that is logically equivalent to its negation and then prove it true using universal generalization.
    \item {\bf Strategies for conjunction}: To prove that $p \land q$ is true, have two subgoals: subgoal (1) prove $p$ 
is  true; and, subgoal (2) prove $q$ is true. To prove that $p \land q$ is false, it's enough to prove that $p$ is false.
 To prove that $p \land q$ is false, it's enough to prove that $q$ is false.
    \item {\bf Proof of Conditional by Direct Proof}: To prove that the implication $p \to q$ is true, we can assume $p$ is true and use that assumption to show $q$ is true.
    \item {\bf Proof of Conditional by Contrapositive Proof}: To prove that the implication $p \to q$ is true, we can assume $\neg q$ is true and use that assumption to show $\neg p$ is true.
    \item {\bf Proof by Cases}: To prove $q$ when we know $p_1 \lor p_2$, show that $p_1 \to q$ and $p_2 \to q$.
    \item
    {\bf Proof by Structural Induction}: To prove that $\forall x \in X \, P(x)$ where $X$ is a recursively defined set, prove two cases:
        
        \begin{tabularx}{\textwidth}{l X}
        Basis Step: & Show the statement holds for elements specified in the basis step of the definition. \\
        Recursive Step: & Show that if the statement is true for each of the elements used to construct
    new elements in the recursive step of the definition, the result holds for these new elements.
    \end{tabularx}
    
    \item {\bf Proof by Mathematical Induction}: To prove a universal quantification over the set of  all integers greater than  or  equal to some base integer $b$:
    
    \begin{tabularx}{\textwidth}{l X}
        Basis Step: & Show the statement holds for $b$. \\
        Recursive Step: & Consider an arbitrary integer $n$ greater than or  equal to  $b$, assume
        (as the {\bf induction hypothesis})  that the property holds  for $n$, and use  this and
        other facts to  prove that  the property holds for $n+1$.
    \end{tabularx}
    
    \item {\bf Proof by Strong Induction} To prove that a universal quantification over the set of all integers greater than or equal to some  base integer $b$ holds,  pick a  fixed nonnegative integer  $j$ and then: \hfill 
    
    \begin{tabularx}{\textwidth}{l X}
        Basis Step: & Show the statement holds for $b$, $b+1$, \ldots, $b+j$. \\
        Recursive Step: & Consider an arbitrary integer $n$ greater than or  equal to  $b+j$, assume
        (as the {\bf strong  induction hypothesis})  that the property holds  for {\bf each of} $b$, $b+1$, \ldots, $n$, 	
        and use  this and
        other facts to  prove that  the property holds for $n+1$.
    \end{tabularx}

    \item {\bf Proof by Contradiction} 

    To prove that a statement $p$ is true, pick another statement $r$ and once we show
    that $\neg p  \to (r \wedge  \neg r)$ then  we can conclude that  $p$ is  true.
    
    {\it Informally} The statement we care about can't possibly be false, so it must be true.
\end{itemize}


{\bf Assigned questions}

\begin{enumerate}[labelindent=0pt, leftmargin=0pt]
    \item Mathematical and strong induction for properties of numbers.

    \begin{enumerate}
    \item \gradeCompleteFirst    Consider each of the following statements and attempted proofs below. Pretend you are the TA/tutor for CSE 20 and grade each of these attempts. 
    In particular, you should give each one a score out of $5$ points and justify your decisions with {\bf specific}, 
    {\bf actionable}, and {\bf justified} feedback;
    your explanations should be convincing but brief.
    Here is the rubric\footnote{According to the Merriam Webster definition, a {\bf rubric} is ``a guide listing specific criteria for grading or scoring academic papers, projects, or tests''. For CSE 20, you can see the rubrics 
    we use to grade assignments and exams on Gradescope: next to your submission for each question you will
    find the rubric items and associated point values. The highlighted items are the ones we select
    to describe your work; these correspond to the score assigned for the question.} you should use: 
    \begin{description}
    \item[5 points] Induction proof includes correctly executed base case
     and recursive step (with induction hypothesis, IH, clearly and correctly defined and used). Uses clear and correct calculations and references to definitions in both steps, including using the IH, to conclude both.
    \item[4 points] Induction proof includes base case and recursive step (with induction hypothesis clearly and mostly correctly defined and used),
    where base case attempted but incomplete OR, incorrect base element chosen but valid proof made for chosen
    element, OR induction hypothesis incorrectly stated but correctly used.
    \item[3 points] Any one of the following: (1) Induction proof with correctly executed base case 
    and recursive step that demonstrates
    connection between induction hypothesis and property being true for $n+1$, some missing or 
    incorrect logical glue; (2) Induction proof with missing base case and correctly executed recursive step.
    \item[2 points] Any one of the following: (1) Induction proof includes attempts at correct steps of an 
    induction proof (base case, recursive  step), with significant logical gaps and/or errors; (2) correctly executed base case 
    with missing / incorrect recursive step,
    \item[1 point] Demonstrates knowledge of proof techniques (e.g.\ attempts some proof type other than 
    induction, but uses some proof technique correctly) and/or structure of induction argument.
    \end{description}
    
    \begin{enumerate}
\item Statement: ``the sum of the first 
$n$ positive odd integers is $n^2$"

{\bf Attempted Proof: }
\begin{quote}

Base Case: $n = 1$\\
First odd number is $1$; $1^2 = 1$. True.\\
$(n + 1)^2 = n^2 + 2n + 1$\\
$n^2$ is the sum of the first $n$ odd numbers, and $2n + 1$ is the next odd number in the sequence, therefore $(n + 1)^2 =$ the sum of the first $n + 1$ odd numbers.
\end{quote}


\item Statement: ``For every nonnegative integer $n$, $3 | n$." (Recall that the $\mid$ symbol is used to mean ``divides'' or ``is a factor of''.)

{\bf Attempted Proof: }
  \begin{quote}
   Attempted Proof: We proceed by strong mathematical induction.

{\it Basis step:} Indeed, $3 | 0$ because there is an integer, namely $0$ such that $0 = 3\cdot 0$.


{\it Induction step:} Let $k$ be arbitrary.  Assume, as the strong induction hypothesis, 
    that for all nonnegative integers $j$ with $0 \leq j \leq k$, that $3 | j$.  Write $k+1 = m + n$, where
    $m,n$ are integers less than $k+1$.  By the induction hypothesis, $3 | m$ and $3 | n$.  That is, there
    are integers $a,b$ such that $m = 3a$ and $n=3b$.  Therefore, $k+1 = (3a) + (3b) = 3 (a+b)$.
    We can choose $a + b$, where we know $a + b \in \mathbb{Z}$, to show that $3 | (k+1)$, as
     required.
     \end{quote}

    \end{enumerate}

    \item \gradeComplete Decide whether each statement above is true or false, 
    give correct and complete induction proofs for the true statement(s) and disprove 
    by counterexample for the false statement(s).
    \end{enumerate}

    \item \gradeCorrect Games and induction.  The game of Nim-Var is a two-player game (which is a variant of Nim).
    At the start of the game, there are two piles, each containing $n$ 
    jelly beans
    ($n$ is a positive integer).  On a player's turn, that player picks one 
    of the two piles and does {\bf one} of the following: either
    \begin{itemize}
    \item removes some positive number of jelly beans from that pile, {\bf or}
    \item moves some positive number of jelly beans (that is less than the
    current total number of jelly beans in the pile) from that pile to the other.  
    {\it Note: if there is only one jelly bean left in a pile, the player cannot move
    this jelly bean to the other pile.}
    \end{itemize}
    The player to take the last jelly bean wins.  
    Use strong induction to prove that the second player always has a winning strategy in Nim-Move.
    A complete and correct solution will first identify what the strategy is, and then prove that 
    following this strategy will lead the second player to win the game (no matter what the first 
    player chooses to do at each turn).
    
    
    
    \item Linked Lists. Recall the recursive definition of the set of linked lists of 
    natural numbers (from class)
    \[
    \begin{array}{ll}
        \textrm{Basis Step: } & [] \in L \\
        \textrm{Recursive Step: } & \textrm{If } l \in L\textrm{ and }n \in \mathbb{N} \textrm{, then } (n, l) \in L
    \end{array}
    \]
    and the definitions of the function which gives the length of a linked list of natural numbers 
    $length: L \to \mathbb{N}$
        \[
        \begin{array}{llll}
            \textrm{Basis Step:} &  & length(~[]~) &= 0 \\
            \textrm{Recursive Step:} & \textrm{If } l \in L\textrm{ and }n \in \mathbb{N}\textrm{, then  } & length(~(n, l)~)  &= 1+ length(l)
        \end{array}
    \]
    the function $append : L \times \mathbb{N} \to L$ that 
    adds an element at the end of a linked list
    \[
        \begin{array}{llll}
        \textrm{Basis Step:} & \textrm{If } m \in \mathbb{N}\textrm{ then } & append(~([], m)~) & = (m, [])\\
        \textrm{Recursive Step:} & \textrm{If } l \in L\textrm{ and }n \in \mathbb{N}\textrm{ and }m \in \mathbb{N}\textrm{, then  } & append(~(~(n, l), m~)~)  &= (n, append(~(l, m)~)~)
        \end{array}
    \]
    and the function $prepend : L \times \mathbb{N} \to L$ that adds an element at the 
    front of a linked list 
    \[
    prepend(~(l, n)~) = (n, l)
    \]

    \rule{0.5\textwidth}{.4pt}
    
    {\it Sample response that can be used as reference for the detail expected 
    in your answer:} 
    To evaluate the result of $length(~(4,(2,(7,[])))~)$ we calculate: 
    \begin{align*}
        length(~(4,(2,(7,[])))~) &= 1 + length ( ~(2,(7,[]))~) \\
        & \qquad \qquad \text{using recursive step of definition of $length$, with $n=4, l=(2,(7,[]))$} \\
        &= 1 + 1 + length ( ~(7,[])~) \\
        & \qquad \qquad \text{using recursive step of definition of $length$, with $n=2, l=(7,[])$} \\
        &= 1 + 1 + 1+ length ( ~[]~) \\
        & \qquad \qquad \text{using recursive step of definition of $length$, with $n=7, l=[]$} \\
        &= 1 + 1 + 1 + 0\\
        & \qquad \qquad \text{using basis step of definition of $length$, since input to $length$ is $[]$} \\
    \end{align*}
    \rule{0.5\textwidth}{.4pt}
    

    In this question, we'll consider the combination of these functions with a new function, one 
    that removes the element at the front of the list (if there is any). We define $remove: L \to L$
    by
    \[
        \begin{array}{ll}
            \textrm{Basis Step: } & remove([]) = [] \\
            \textrm{Recursive Step: } & \textrm{If } l \in L\textrm{ and }n \in \mathbb{N} \textrm{, then } remove(~(n, l)~) = l
        \end{array}        
    \]
    \begin{enumerate}
        \item\gradeCorrect What is the result of $remove(~append(~(prepend~(~(~(10,[]), 5~)~), 20)~)~)$? 
        For full credit, include all intermediate steps with brief justifications for each.
        \item\gradeCorrect Prove the statement
        \[
            \forall l \in L \left(~ remove( prepend(~(l,0)~)) = l ~\right) 
        \]
        \item\gradeCorrect Disprove the statement
        \[
            \forall l \in L \left(~ remove( append(~(l,0)~)) = l ~\right) 
        \]
    \end{enumerate}

    \item Primes, divisors, and proof strategies. For each statement below, identify the {\bf main logical structure or connective} of the 
    statement, list the proof strategies that could be used to prove and to disprove a statement with that structure, 
    then identify whether the statement is true or false and justify with a proof of the statement or its negation. 
    ({\it Graded for correctness of identification of logical structure and proof strategies and evaluation of statement (is it true or false?) 
    and fair effort completeness of the proof}) 


    \rule{0.5\textwidth}{.4pt}
    
    {\it Sample response that can be used as reference for the detail expected 
    in your answer:} 

    Consider the statement: There is a greatest negative integer.

    The main logical structure for this statement is that it is an {\bf existential} statement, as we can 
    see by translating it to symbols: 
    \[
        \exists g \in \mathbb{Z}^{-} \forall x \in \mathbb{Z}^{-} ( g \geq x)
    \]
    To prove an existential statement, the main proof strategy we could use is to find a witness. 
    Proof by cases and proof by contradiction could also be used, because 
    both can be used to prove any statement (no matter its logical structure).

    To disprove an existential statement, we would need to prove its negation, which (using DeMorgan's Laws) can be 
    written as a universal statement. Therefore, to disprove this statement the strategies we could use are 
    universal generalization or structural induction (because $\mathbb{Z}^{-}$ is a recursively defined set) or 
    proof by cases or proof by contradiction. Notice that proof by exhaustion is not possible because the domain is not finite.

    The statement is true, as we can see from the witness $g = -1$, since it is in the domain $\mathbb{Z}^{-}$ and when we evaluate
    \[
        \forall x \in \mathbb{Z}^{-} ( -1 \geq x)
    \]
    we can proceed by universal generalization and take an arbitrary negative integer $x$, which by definition means $x < 0$, 
    and since $x$ is an integer, guarantees $x \leq -1 = g$, as required.
    \rule{0.5\textwidth}{.4pt}

        \begin{enumerate}
            \item The quotient of any even number with any nonzero even number is even.
            \item There are two odd numbers (not necessarily distinct) whose sum is even.
            \item The greatest common divisor of 5 and 23 is 1 and the greatest common divisior of 7 and 19 is 1.
            \item There are two positive integers greater than 20 that have the same greatest common divisor with 20. {\it Note: typo fixed, 
            originally said ``greatest common division'' instead of ``greatest common divisor''.}
        \end{enumerate}
    \end{enumerate}
\end{document}
