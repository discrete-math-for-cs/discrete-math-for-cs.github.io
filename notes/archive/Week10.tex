\input{../../resources/lesson-head.tex}

\section*{Monday November 29}
\input{../activity-snippets/equivalence-relations-partitions.tex}

Last time, we saw that partitions associated to equivalence relations
were useful in the context of modular arithmetic.
Today we'll look at a different application.

\input{../activity-snippets/equivalence-relations-examples-ratings.tex}
\newpage
\section*{Review}
\begin{enumerate}
    \item Select all and only the partitions of $\{1,2,3,4,5\}$ from the sets below.
    \begin{enumerate}
    \item $\{1,2,3,4,5\}$
    \item $\{\{1,2,3,4,5\}\}$
    \item $\{\{1\},\{2\},\{3\},\{4\},\{5\}\}$
    \item $\{ \{1\}, \{2,3\}, \{4\} \}$
    \item $\{ \{\emptyset, 1, 2\}, \{3,4,5\}\}$
    \end{enumerate}
    \item \hspace{1in}\\ \input{../activity-snippets/quiz-equivalence-class-types.tex}
\end{enumerate}
\newpage

\section*{Wednesday December 1}
%! app: TODOapp
%! outcome: TODOoutcome

{\bf Scenario}: Good morning! You're a user experience engineer at Netflix. A
product goal is to design customized home pages for groups of users who have
similar interests. Your manager tasks you with designing an algorithm for
producing a clustering of users based on their movie interests,
so that customized homepages can be engineered for each group.

%{\bf Conventions for today}: 
%We will use $U = \{r_1, r_2, \cdots, r_t\}$ to 
%refer to an arbitrary set of user ratings (we'll pick some 
%specific examples to explore) that are a subset of $Rt_5$. 
%We will be interested in creating partitions $C_1, \cdots, C_m$ of 
%$U$. We'll assume that each user represented by an element of $U$ 
%has a unique ratings tuple.

Your idea: equivalence relations! 
%You offer your manager three great options: 

\[
    E_{id} = \{ ( ~(x_1, x_2, x_3, x_4, x_5), (x_1, x_2, x_3, x_4, x_5)~) \mid 
    (x_1, x_2, x_3, x_4, x_5) \in Rt_5  \}
\]

{\it Describe how each homepage should be designed \ldots }

\vspace{100pt}



\[
    E_{proj} =  \{ ( ~(x_1, x_2, x_3, x_4, x_5), (y_1, y_2, y_3, y_4, y_5)~) \in
         Rt_5 \times Rt_5 ~\mid~(x_1 = y_1) \land  (x_2 = y_2) \land (x_3 = y_3) \}
\]


{\it Describe how each homepage should be designed \ldots }

\vspace{100pt}

\[
E_{circ} =  \{ (u,v) \in Rt_5 \times Rt_5 ~\mid~ d(~ ( ~(0,0,0,0,0)~, u)~ ) =  d( ~(~(0,0,0,0,0),v~)~) \}
\]

{\it Describe how each homepage should be designed \ldots }


\vspace{100pt}


%{\bf Scenario}: Good morning! You're a user experience engineer at Netflix. A
%product goal is to design customized home pages for groups of users who have
%similar interests. You task your team with designing an algorithm for
%producing a clustering of users based on their movie interests. Your team
%implements two algorithms that produce different clusterings. How do you
%decide which one to use? What feedback do you give the team in order to help
%them improve? Clearly, you will need to use math.


%Your idea: find a way to {\bf score} clusterings (partitions) 


%{\bf Definition}: For a cluster of ratings $C = \{r_1, r_2, \cdots, r_n \} 
%\subseteq U$, the {\bf diameter} of the cluster is defined by:

%$$\textit{diameter}(C) = \max_{1 \leq i, j \leq n} (d(~(r_i, r_j)~))$$ 

%Consider $x = (1, 0, 1, 0, 1)$, $y = (1, 1, 1, 0, 1)$, $z = (-1, -1, 0, 0, 0)$, $w = (0, 0, 0, 1, 0)$.

%What is $\textit{diameter}(\{x, y, z\})$? $\textit{diameter}(\{x, y\})$? $\textit{diameter}(\{x, z, w\})$?

%\vspace{100pt}

%\textit{diameter} works on single clusters. One way to aggregate across a
%clustering $C_1, \cdots, C_m$ is $\sum_{k=1}^m diameter(C_k)$


%Is this a good score?

%\vspace{100pt}

%How can we express the idea of {\bf many elements within a small area}? Key idea: ``give credit'' to small diameter clusters with many elements.

%{\bf Definition}: For a cluster of ratings $C = \{r_1, r_2, \cdots, r_n \} 
%\subseteq U$, the {\bf density} of the cluster is defined by:
%\[
%    \frac{n}{1+ diameter(C)}
%\]

%\newpage

%Can you use density to decide whether the partition given by 
%the equivalence classes of $E_{proj}$ or $E_{circ}$ for this task?
\newpage

\section*{Looking forward}

\subsection*{Tips for future classes from the CSE 20 TAs and tutors}
\begin{itemize}
\item In class
\begin{itemize}
\item Go to class
\item Show up to class early because sometimes seats get taken/ the classroom gets full and then you have to sit on the floor
\item There's usually a space for skateboards/longboards/eboards to go at the front or rear of the lecture hall 
\item If you have a flask water bottle please ensure that its secured during a lecture and it cannot fall - putting on the floor often leads to it falling since people sometimes cross your seats.
\item Take notes - it's much faster and more effective to note-take in class than watch recordings after, particularly if you do so longhand
\item Resist the urge to sit in the back. You will be able to focus much better sitting near the front, where there are fewer screens in front of you to distract from the lecture content
\item If you bring your laptop to class to take notes / access class materials, sit towards the back of the room to minimize distractions for people sitting behind you!
\item On zoom it's easy to just type a question out in chat, it might be a little more awkward to do so in person, but it is definitely worth it. Don't feel like you should already know what's being covered
\item Always check you have your iclicker\footnote{iclickers are used in many classes to encourage active participation in class. They're remotes that allow you to respond to multiple choice questions and the instructor can show a histogram of responses in real time.} on you. Just keep it in your backpack permanently. That way you can never forget it. 
\item Don't be afraid to talk to the people next to you during group discussions. Odds are they're as nervous as you are, and you can all benefit from sharing your thoughts and understanding of the material 
\item Certain classes will podcast the lectures, just like Zoom archives lecture recordings, at podcast.ucsd.edu
\item If they aren't podcasted, and you want to record lectures, ask your professor for consent first
\end{itemize}
\item Office hours, tutor hours, and the CSE building
\begin{itemize}
\item Office hours are a good place to hang out and get work done while being able to ask questions as they come up 
\item Office hour attendance is typically much busier in person (and confined to the space in the room)
\item Get to know the CSE building: CSE B260, basement labs, office hours rooms
\item Know how to get in to the building after-hours
\end{itemize}
\item Libraries and on-campus resources
\begin{itemize}
\item Look up what library floors are for what, how to book rooms: East wing of Geisel is open 24/7 (they might ask to see an ID if you stay late), East Wing of Geisel has chess boards and jigsaw puzzles, study pods on the 8th floor, 
free computers/wifi
\item Know Biomed exists and is usually less crowded
\item Most libraries allow you to borrow whiteboards and markers (also laptops, tablets, microphones, and other cool stuff) for 24 hours
\item Take advantage of Dine with a prof / Coffee with a prof program. It's legit free food / coffee once per quarter. 
\item When planning out your daily schedule, think about where classes are, how much time will they take, are their places to eat nearby and how you can schedule social time with friends to nearby areas 
\item Take into account the distances between classes if they are back to back
\end{itemize} 
\item Final exams
\begin{itemize}
\item What are 8am finals? Basically in-person exams are different
\item Don't forget your university card during exams
\item Blue books for exams (what they are, where to get them) 
\item Seating assignments for exams and go early to make sure you're in the right place (check the exits to make sure you're reading it the right way) 
\item Know where your exam is being held (find it on a map at least a day beforehand). Finals are often in strange places that take a while to find 
\end{itemize}
\end{itemize}

\subsection*{CSE department course numbering system}

{\bf Lower division}

\begin{itemize}
\setlength\itemsep{-2pt}
\item CSE 12, Basic Data Structures and Object-Oriented Programming 
\item CSE 15L (2 units), Software Tools and Technique Laboratory 
\item CSE 20 or Math 15A, Introduction to Discrete Mathematics
\item CSE 21 Mathematics for Algorithms and Systems
\item CSE 30, Computer Organization \& Systems Programming
\end{itemize}

{\bf Upper division}

\begin{itemize}
\setlength\itemsep{-2pt}
\item Advanced Data Structures and Programming: CSE 100
\item Theory and Algorithms: CSE 101, CSE 105
\item Software Engineering: CSE 110, CSE 112
\item Systems/Networks: CSE 120 or CSE 123 or CSE 124
\item Programming Languages /Databases: CSE 130 or CSE 132A
\item Security/Cryptography: CSE 107 or CSE 127
\item AI / Machine Learning/ Vision/ Graphics:
CSE 150A or CSE 151A or CSE 151B or CSE 152A or CSE 158 or CSE 167
\item Hardware / Architecture: CSE 140/ CSE 140L Components and Design Techniques for Digital Systems Architecture, CSE 141 / CSE 141L Introduction to Computer Architecture and CSE 141L Project in Computer Architecture (2 units), CSE 142 / CSE 142L Comp Arch Software Perspective
\end{itemize}

\subsection*{Review}
\begin{enumerate}
    \item \hspace{1in}\\ \input{../activity-snippets/quiz-binary-relation-ratings.tex}
    \item \hspace{1in}\\ \input{../activity-snippets/quiz-clustering-ratings.tex} 
\end{enumerate}

\newpage
\section*{Friday December 3}
%! app: Numbers
%! outcome: representing numbers

Convert $(2A)_{16}$ to 
\begin{itemize}
\item binary (base \underline{\phantom{~~~2~~}})

\vspace{50pt}

\item decimal (base \underline{\phantom{~~10~~}})

\vspace{50pt}

\item octal (base \underline{\phantom{~~~8~~}})

\vspace{50pt}

\item ternary (base \underline{\phantom{~~~3~~}})

\vspace{50pt}

\end{itemize}
\newpage
\input{../activity-snippets/set-construction-final-review.tex}
\newpage
\input{../activity-snippets/set-operations-final-review.tex}
\newpage
\input{../activity-snippets/function-properties-final-review.tex}
\newpage
\input{../activity-snippets/cardinality-final-review.tex}
\newpage
%! app: TODOapp
%! outcome: TODOoutcome

Compute the last digit of 
\[
    (42)^{2024}
\]

\vfill

{\it Extra} Describe the pattern that helps you perform this computation 
and prove it using mathematical induction.
\newpage
\subsection*{Review}
\begin{enumerate}
    \item \hspace{1in}\\  Please complete the CAPE and TA evaluations.  Once 
    you have done so complete the custom feedback form for this quarter: 
    \url{https://forms.gle/pbYWSRDP2znkciM46}
    
    Then, (we're using the honor system here), write out the statement
    ``I have completed the end  of quarter evaluations"  and you'll receive credit 
    for  this  question.
\end{enumerate}
\end{document}