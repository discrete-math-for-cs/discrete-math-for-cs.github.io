\input{../../resources/assignment-head.tex}

\title{hw1-definitions-and-notation}
\date{Due: 4/9/24 at 5pm (no penalty late submission until 8am next morning)}
\begin{document}
\maketitle
\thispagestyle{fancy}


{\bf In this assignment,}

You will practice reading and
applying definitions to get comfortable working with mathematical language.

{\bf Relevant class material}: Week 1.

You will submit this assignment via Gradescope
(\href{https://www.gradescope.com}{https://www.gradescope.com}) 
in the assignment called ``hw1-definitions-and-notation''.

\instructions


{\bf Assigned questions}

\begin{enumerate}[labelindent=0pt, leftmargin=0pt]
    \item Modeling
    \begin{enumerate}[labelindent=0pt, leftmargin=0pt]
        \item\gradeCompleteFirst In class, we used $4$-tuples to represent the user ratings for four movies. 
        This representation is memory-efficient because we use the order of the components in the $4$-tuples to 
        represent which move is being rated. However, it is not easily extended when we want to add new movies to the database.
        Define a new model that would allow us to represent the user ratings of movie databases where we allow for new movies to be added.
        Use only the data types we have talked about in class: sets, $n$-tuples, and strings. Explain the design choices
        that you used to define your model by referencing properties of the data-type(s) you choose.
        Demonstrate your model by showing how the rating of the user who dislikes Dune and Oppenheimer and likes Barbie and Nimona
        is represented.
        
        \item\gradeComplete 
        Colors can be described as amounts of red, green, and blue mixed together
        \footnote{This RGB representation is common in web applications.  Many online tools are available to play around with mixing these colors,  e.g. \url{https://www.w3schools.com/colors/colors_rgb.asp}. }
        Mathematically, a color can be represented as a $3$-tuple $(r, g, b)$ where $r$ represents the red component, $g$ the green component, $b$ the blue component and where each of $r$, $g$, $b$ must be a value from this collection of numbers:
        \begin{quote}
        $\{$0, 1, 2, 3, 4, 5, 6, 7, 8, 9, 10, 11, 12, 13, 14, 15, 16, 17, 18, 19, 20, 21, 22, 23, 24, 25, 26, 27, 28, 29, 30, 31, 32, 33, 34, 35, 36, 37, 38, 39, 40, 41, 42, 43, 44, 45, 46, 47, 48, 49, 50, 51, 52, 53, 54, 55, 56, 57, 58, 59, 60, 61, 62, 63, 64, 65, 66, 67, 68, 69, 70, 71, 72, 73, 74, 75, 76, 77, 78, 79, 80, 81, 82, 83, 84, 85, 86, 87, 88, 89, 90, 91, 92, 93, 94, 95, 96, 97, 98, 99, 100, 101, 102, 103, 104, 105, 106, 107, 108, 109, 110, 111, 112, 113, 114, 115, 116, 117, 118, 119, 120, 121, 122, 123, 124, 125, 126, 127, 128, 129, 130, 131, 132, 133, 134, 135, 136, 137, 138, 139, 140, 141, 142, 143, 144, 145, 146, 147, 148, 149, 150, 151, 152, 153, 154, 155, 156, 157, 158, 159, 160, 161, 162, 163, 164, 165, 166, 167, 168, 169, 170, 171, 172, 173, 174, 175, 176, 177, 178, 179, 180, 181, 182, 183, 184, 185, 186, 187, 188, 189, 190, 191, 192, 193, 194, 195, 196, 197, 198, 199, 200, 201, 202, 203, 204, 205, 206, 207, 208, 209, 210, 211, 212, 213, 214, 215, 216, 217, 218, 219, 220, 221, 222, 223, 224, 225, 226, 227, 228, 229, 230, 231, 232, 233, 234, 235, 236, 237, 238, 239, 240, 241, 242, 243, 244, 245, 246, 247, 248, 249, 250, 251, 252, 253, 254, 255$\}$
        \end{quote}
        (This is the same definition as in the Week 1 Review quiz.)
        
        Can you find two different $3$-tuples that represent colors that are indistinguishable to your eye? You can use the website in the footnote to play around
        with different choices of red, green, and blue levels to see if you can distinguish between the resulting colors.
        Why or why not? 

        A complete answer will include the specific example $3$-tuples that work, along with a description of the colors that they represent and why they
        are indistinguishable, or an explanation of why there can't be such an example.
    \end{enumerate}

    \item Sets and functions
    \begin{enumerate}
        \item \gradeCorrectFirst Each of the sets below is described 
using set builder notation or recursion or as a result of set operations
applied to other known sets.  Rewrite each of the sets using the roster method.

Remember our discussions of data-types: use clear notation that 
is consistent with our class notes and definitions 
to communicate the data-types of the elements in each set.


\rule{0.5\textwidth}{.4pt}

{\it Sample response that can be used as reference for the detail expected 
in your answer:} 

The set $\{ \A \} \circ \{ \A\U, \A\C, \A\G\}$ can be written using
the roster method as 
\[
\{ \A\A\U, \A\A\C, \A\A\G \}
\]
because set-wise concatenation gives a set whose elements are 
all possible results of concatenating an element of the 
left set with an 
element of the right set. Since the left set in this example only
has one element, namely $\A$, each of the elements of the set we 
described starts with $\A$. There are three elements of this set, 
one for each of the distinct elements of the right set.

\rule{0.5\textwidth}{.4pt}

\begin{enumerate}
\item $$\{ n \in \mathbb{Z}^+ \mid n \leq 3 \} \times \{ m \in \mathbb{N} \mid m \leq 3\}$$ (Note: typo fixed Apr 3)
\item The set $X$ defined recursively as 
    \[
    \begin{array}{ll}
    \textrm{Basis Step: } & 1 \in X, 3 \in X, 5 \in X \\
    \textrm{Recursive Step: } & \textrm{If the integer } n \in X \textrm{, then the result of multiplying } n \textrm{ by } -1 \textrm{ is in }X
    \end{array}
    \]
\item $$\{ x \in S \mid rnalen(x) = 2 \} \circ \{x \in S \mid rnalen(x) = 0 \}$$
where $S$ is the set of RNA strands and $rnalen$ is the recursively defined
function that we discussed in class,
\[
\begin{array}{llll}
& & \textit{rnalen} : S & \to \mathbb{Z}^+ \\
\textrm{Basis Step:} & \textrm{If } b \in B\textrm{ then } & \textit{rnalen}(b) & = 1 \\
\textrm{Recursive Step:} & \textrm{If } s \in S\textrm{ and }b \in B\textrm{, then  } & \textit{rnalen}(sb) & = 1 + \textit{rnalen}(s)
\end{array}
\]
\item $$\{ (r,g,b) \in C \mid r+g+b = 2 \textrm{ and } g=1\}$$ where 
$C = \{ (r,g,b) \mid 0 \leq r \leq 255, 0 \leq g \leq 255, 0 \leq b \leq 255, r \in \mathbb{N}, g \in \mathbb{N}, b \in \mathbb{N} \}$
is the set that you worked with in Monday's review quiz.
\end{enumerate}

\item\gradeCorrect Recall the function which takes an ordered pair of ratings $4$-tuples and returns a measure of the difference between them
\[
    d_0: \{-1,0,1\}^4 \times \{-1,0,1\}^4 \to \mathbb{R}
\]
given by
\[
d_0 (~(~ (x_1, x_2, x_3, x_4), (y_1, y_2, y_3, y_4) ~) ~) = \sqrt{ (x_1 - y_1)^2 + (x_2 - y_2)^2 + (x_3 -y_3)^2 + (x_4 -y_4)^2}
\]

Define a {\bf new} function which we'll call $d_{new}$ with the same domain $\{-1,0,1\}^4 \times \{-1,0,1\}^4$
and codomain $\mathbb{R}$ but where there is some example pair of ratings $4$-tuples 
\[(~ (x_1, x_2, x_3, x_4), (y_1, y_2, y_3, y_4)~)\]
where
\[
d_0 (~(~ (x_1, x_2, x_3, x_4), (y_1, y_2, y_3, y_4) ~) ~) \neq d_{new} (~(~ (x_1, x_2, x_3, x_4), (y_1, y_2, y_3, y_4) ~) ~)
\]

    Your answer should include  {\bf both} a precise and clear definition for the rule defining
    $d_{new}$ which unambiguously specifies output for each input of the function {\bf and}
    the example ordered pair of ratings $4$-tuples that demonstrate that the functions are not equal.
    Also include a justification 
    of your answers with (clear, correct, complete) calculations for each of the function applications 
    and/or references to definitions and connecting them with
    the desired conclusion.

\item\gradeCorrect A function \textit{basecount} that computes the number of a given base 
$b$ appearing in a RNA strand $s$ is defined recursively:
    
\[
\begin{array}{llll}
& \textit{basecount} : S \times B & \to \mathbb{N} &\\
\textrm{Basis Step:} &  \\
\textrm{If } b_1 \in B, b_2 \in B & \textit{basecount}(~(b_1, b_2)~) & = 
        \begin{cases}
            1 & \textrm{when } b_1 = b_2 \\
            0 & \textrm{when } b_1 \neq b_2 \\
        \end{cases}& \\
\textrm{Recursive Step:} & \\
\textrm{If } s \in S, b_1 \in B, b_2 \in B &\textit{basecount}(~(s b_1, b_2)~) & =
        \begin{cases}
            1 + \textit{basecount}(~(s, b_2)~) & \textrm{when } b_1 = b_2 \\
            \textit{basecount}(~(s, b_2)~) & \textrm{when } b_1 \neq b_2 \\
        \end{cases} &
\end{array}
\]
Consider the function application 
\[
  basecount( ~(\A\C\A\U, \A)~)
\]
What is the input?  What is the output? Give an example of a different choice of input that 
gives the same output.

Your answer should include clearly labeled answers to each of the three parts of the question, 
along with a justification for the values of the applications that makes specific reference to 
the parts of the recursive definition of the $basecount$ function used to calculate it.
\end{enumerate}
\end{enumerate}
\newpage

\title{hw2-numbers}
\date{Due: 4/16/24 at 5pm (no penalty late submission until 8am next morning)}

\maketitle
\thispagestyle{fancy}


{\bf In this assignment,}

You will practice applying functions and tracing algorithms in multiple contexts, 
and exploring properties of positional number representations.

{\bf Relevant class material}: Week 2.

You will submit this assignment via Gradescope
(\href{https://www.gradescope.com}{https://www.gradescope.com}) 
in the assignment called ``hw2-numbers''.

\instructions

{\bf Assigned questions}

\begin{enumerate}[labelindent=0pt, leftmargin=0pt]
\item Functions and algorithms: in this question you'll explore the properties of the (integer) power and logarithm
functions. We'll use some definitions we introduced in class this week, namely the function
$b^i$ with domain $\mathbb{Z}^+ \times \mathbb{N}$ and codomain $\mathbb{N}$ defined recursively by
\begin{align*}    
&\textrm{Basis Step:} \\
&b^0 = 1 \\
&\textrm{Recursive Step:}\\
&\textrm{If } i \in \mathbb{N}, b^{i+1} = b \cdot b^i
\end{align*}
and the algorithm:
\input{../activity-snippets/algorithm-log.tex}


    \begin{enumerate}
    \item\gradeCorrectFirst Choose a positive integer $b$ between $3$ and $6$ (inclusive) and choose a 
    nonnegative integer $i$ between $2$ and $5$ (inclusive). Demonstrate how to calculate 
    the result of the $logb$ algorithm (procedure) when its input is the base $b$ you chose and
    $n = b^i$ (for the $i$ you choose.) 
    A complete answer will include the specific choice of $b$ and $i$, along with 
    trace of the calculations of $b^i$ and the computation of the algorithm, including 
    (clear, correct, complete) calculations for each of the function applications 
    and/or references to definitions and a trace table for algorithm that includes the values of all relevant 
    variables at each iteration (See the annotated Week 2 notes for the level of detail expected
    in a trace of a function application and a trace table).
    \item\gradeCorrect Now we'll go in other order: Demonstrate how to calculate the result of 
    $7^y$ where $y$ is the result of the $logb$ algorithm (procedure) when its input is $b=7$ and $n=30$.
    A complete answer will include clearly labelled 
    traces of the calculations of $b^i$ and the computation of the algorithm, including 
    (clear, correct, complete) calculations for each of the function applications 
    and/or references to definitions and a trace table for algorithm that includes the values of all relevant 
    variables at each iteration (See the annotated Week 2 notes for the level of detail expected
    in a trace of a function application and a trace table).    
    \item\gradeCompleteFirst Logarithms and powers are supposed to ``undo" one another. Explain whether your work in parts (a) and (b)
    supports that idea,  whether you saw anything confusing or surprising about these calculations, and how to explain what you saw.
    \end{enumerate}

\item Base expansions
    \begin{enumerate}
        \item\gradeComplete Pick an integer between $50$ and $1000$ (inclusive) that you (or one of your
        group members) came across at some point this week. 
        In a sentence or two, give some context for why you're choosing this number). 
        Write the base expansion of your chosen number in base $2$ (binary), base $3$ (ternary), base $4$, and base $16$ (hexadecimal).
        \item\gradeCorrect What is the {\bf smallest} width $w$ in which they could write your chosen number in base $8$ (octal) 
        fixed-width $w$? Justify your answer with reference to the definitions of fixed-width expansions and relevant calculations.
        \item\gradeCorrect Express in roster method the set of numbers between $1$ and $2$ (exclusive) that be written without error 
        (full precision)
        in binary fixed width expansion with integer part width $3$ and fractional part width $2$. Justify your answer with 
        reference to the definitions of fixed-width expansions and relevant calculations.
        \item\gradeCorrect Consider the strings of $1$s that have length $3$, $5$, $7$, and $10$.
        Calculate the numbers 
        \begin{itemize}
            \item[] $[111]_{s,3}$
            \item[] $[11111]_{s,5}$
            \item[] $[1111111]_{s,7}$
            \item[] $[1111111111]_{s,10}$
            \item[] $[111]_{2c,3}$
            \item[] $[11111]_{2c,5}$
            \item[] $[1111111]_{2c,7}$
            \item[] $[1111111111]_{2c,10}$
        \end{itemize}
        Justify your answers with specific reference to the definitions of sign-magnitude and $2$s complement expansions and relevant calculations.
        \item \gradeComplete What patterns do you notice in your calculations in part (d)?
    \end{enumerate}

\item Multiple representations

Recall that, mathematically, a color can be represented as a $3$-tuple $(r, g, b)$ 
where $r$ represents the red component, $g$ the green component, $b$ the blue component and where each of $r$, $g$, $b$ must be from the collection $\{x \in \mathbb{N}\mid 0 \leq x \leq 255 \}$.
As an alternative representation, in this assignment
we'll use base $b$ fixed-width expansions to represent colors
as individual numbers (this definition was introduced in this week's Review quiz).

\input{../activity-snippets/quiz-color-hexcolor-definition.tex}

For this question, let's call the set of hex colors $H$. In the Week 2 Review quiz (Question 3d), 
we explored a few different set builder definitions for $H$.

\begin{enumerate}
\item \gradeComplete Rewrite the set builder definition of a set below so that it 
refers to colors rather than numbers: $\{ c \in H \mid c \mod 256 = 0\}$. A complete answer will 
justify the new set builder definition by connection with the definition of hex colors and how it impacts
the colors that satisfy the specific property for this set.
\item \gradeCorrect Rewrite the set builder definition of a set below so that it 
refers to colors rather than numbers: $\{ c \in H \mid c < 65536\}$. A complete answer will 
justify the new set builder definition by connection with the definition of hex colors and how it impacts
the colors that satisfy the specific property for this set.
\item \gradeCorrect In art, we can mix two colors to get a new color. For example, red and blue make 
purple, blue and yellow make green, red and yellow make orange. 
A mathematical definition of {\bf hex color mixing} would be a function with domain $H \times H$
and codomain $H$ where the result of applying this function to an input $(n_1, n_2)$ is the hex color
that results from mixing the hex colors $n_1$ and $n_2$.


\rule{0.5\textwidth}{.4pt}

{\it Sample work for a related question that can be used as reference for the detail expected 
in your answer:} Consider the attempted mathematical definition  $f_1: H\times H \to H$ given by
\[
f_1 (~(n_1, n_2) ~) = n_1 + n_2
\]
We will show that this function is not well-defined so it cannot give a hex color mixing definition. 
The reason it is not well-defined
is that the application of the rule sometimes gives values that are not in the stated codomain. 
To see this, we need an example input to $f_1$ for which the output of the rule is not in $H$.
Here is one such example: $(n_1, n_2) = (~(FFFFFF)_{16,6}, (FFFFFF)_{16,6}~)$. 
This example is in $H \times H$
because $(FFFFFF)_{16,6}$ is a nonnegative integer that has a base $16$ fixed-width $6$ expansion 
so it is a hex color.
We now apply the definition of the rule in $f_1$:
\begin{align*}
f_1 (~(n_1, n_2) ~) &=  f_1 ( (~(FFFFFF)_{16,6}, (FFFFFF)_{16,6}~) ) = (FFFFFF)_{16,6} + (FFFFFF)_{16,6} \\
&= 2 \left( 15\cdot 16^5 + 15\cdot 16^4 + 15 \cdot 16^3 + 15 \cdot 16^2 + 15 \cdot 16^1 + 15 \cdot 16^0 \right) \\
&= 2 \cdot 15 \cdot (1048576 + 65536 + 4096 + 256 + 16 + 1) = 2 \cdot 15 \cdot 1118481 = 33554430
\end{align*}
This is not a hex color because it is greater than or equal to $16^6 = 16777216$, 
so (using the Week 2 notes on which numbers can be represented
with fixed-width expansions) it does not have a hexadecimal {\bf fixed width $6$} expansion.

\rule{0.5\textwidth}{.4pt}

In this question, we'll look at another attempted mathematical definition for hex color mixing. 
We define $f_2: H\times H \to H$ given by
\[
f_2 (~(n_1, n_2) ~) = (n_1 + n_2) \textbf{ div } 2
\]

This is a well-defined function. You do not need to prove this or hand it in, but it's good practice to make 
sure you understand why it's well-defined. Notice that the function application 
\[
f_2 ( ~(~ (FF0000)_{16,6} , (FFFE00)_{16,6} ~)~)
\]
can be calculated as:
\begin{align*}
f_2&(((FF0000)_{16,6}, (FFFE00)_{16,6}))=
 ((FF0000)_{16,6} + (FFFE00)_{16,6})) \textbf{ div } 2\\
&= \big(~ \left(15\cdot16^5 + 15\cdot16^4 + 0\cdot16^3 + 0\cdot16^2 + 0\cdot16^1 + 0\cdot16^0 \right) \\
&~~~~+~
    \left((15\cdot16^5 + 15\cdot16^4 + 15\cdot16^3 + 14\cdot16^2 + 0\cdot16^1 + 0\cdot16^0\right )~\big )\textbf{ div } 2\\
&= (16711680 + 16776704) \textbf{ div } 2 = 33488384 \textbf{ div } 2 = 16744192
\end{align*}
Since this is less than $16^6 = 16777216$, we can represent it using base $16$ fixed-width $6$:.
$$(16744192)_{10} = 15\cdot16^5 + 15\cdot16^4 + 7\cdot16^3 + 15\cdot16^2 + 0\cdot16^1 + 0\cdot16^0 = (FF7F00)_{16,6}$$

Using the web tool \url{https://www.w3schools.com/colors/colors_rgb.asp}, we can verify that $(FF0000)_{16,6}$ is red, 
$(FFFE00)_{16,6}$ is yellow, and $(FF7F00)_{16,6}$ is orange.

Show that $f_2$ does not work as a hex color mixing definition by finding
another ordered pair of hex colors for which the result of applying $f_2$ does not give the expected hex color.
Include the numerical description of each colors you mention, alongside a description of them in English. 
Describe how you know what these colors are
(if you use a web color tool, include its URL in your submission writeup; if not, describe your reasoning).
Justify your 
example with (clear, correct, complete) calculations and/or references to definitions, and connecting them with
the desired conclusion.
\end{enumerate}
\end{enumerate}
\newpage


\title{hw3-circuits-and-logic}
\date{Due: 4/23/24 at 5pm (no penalty late submission until 8am next morning)}

\maketitle
\thispagestyle{fancy}

{\bf In this assignment}, you will consider how circuits and logic can be used to represent
mathematical claims. You will use propositional operators
to express and evaluate these claims.

{\bf Relevant class material}: Week 2 and Week 3.

You will submit this assignment via Gradescope
(\href{https://www.gradescope.com}{https://www.gradescope.com}) 
in the assignment called ``hw3-circuits-and-logic''.

\instructions

{\bf Assigned questions}

\begin{enumerate}[labelindent=0pt, leftmargin=0pt]
        \item Fixed-width addition.  
        \begin{enumerate}
            \item\gradeCompleteFirst Choose example width $5$ first summand and width $5$ second summand so that 
            in the  binary fixed-width addition (adding one bit at time, using 
            the usual column-by-column and carry arithmetic, and ignoring the carry 
            from the  leftmost column), the example satisfies all three conditions below simultaneously
            \begin{itemize}
            \item[(1)] When interpreting each of the summands and the result in binary fixed-width 5, 
            the result represents the actual value of the sum of the summands {\bf and}
            \item[(2)] when interpreting each of the summands and the sum in sign-magnitude width 5, the result  
            represents the actual value of the sum of the summands {\bf and}
            \item[(3)] when interpreting each of the summands and the sum in 2s complement width 5, the result 
            represents the actual value of the sum of the summands.
            \end{itemize}
            \item\gradeCorrectFirst Choose an example width $5$ first summand and second summand so that 
            in the binary fixed-width addition (adding one bit at time, using 
            the usual column-by-column and carry arithmetic, and ignoring the carry 
            from the  leftmost column),  the example satisfies all three conditions below simultaneously
            \begin{itemize}
            \item[(1)] When interpreting each of the summands and the result in binary fixed-width 5, 
            the result {\bf does not} represent the actual value of the sum of the summands {\bf and}
            \item[(2)] when interpreting each of the summands and the sum in sign-magnitude width 5, the result  
            {\bf does not} represents the actual value of the sum of the summands {\bf and}
            \item[(3)] when interpreting each of the summands and the sum in 2s complement width 5, the result 
            represents the actual value of the sum of the summands.
            \end{itemize}
            A complete solution will clearly specify each summand and the result of binary fixed-width addition 
            with this choice of summands; will specify  the value of each summand and the result for 
            binary fixed-width 5, 
            sign-magnitude width 5, and 2s complement width 5 (and include calculations connecting with the 
            definitions of these representations to explain these values); and a conclusion connecting
            the calculations to the properties laid out in the question.
        \end{enumerate}    
        \item Circuits. 
        \begin{enumerate}
            \item\gradeComplete Consider the circuit below with inputs $x$ and $y$. 
            Identify a pair of gates that could 
            be switched without changing the input-output table of the circuit.
            If you do, write 
            out the input-output table that results, and briefly explain why 
            this choice of gates works. If there is no such pair of gates, 
            explain why not with reference to the definitions
            of the logic gates.            
        \begin{center}
            \includegraphics[width=2in]{../../files/circuits-hw3.png}
        \end{center}

        \item\gradeCorrect  Is there a way to fill in the blank portion of 
        the two logic circuits below *with the same gates connected in the same way*
        so that the resulting circuits have the same 
        input-output value *even though* one uses an OR gate at the end and the other 
        uses an XOR gate? If so, design the circuit that would be used, write 
        out the input-output table that results, and briefly explain why your 
        design works. If not, explain why not with reference to the definitions
        of the logic gates.
        \begin{multicols}{2}
            \begin{center}
                \includegraphics[width=2in]{../../files/circuit-blank-or-hw3.png}
            \end{center}
     
            \begin{center}
                \includegraphics[width=2in]{../../files/circuit-blank-xor-hw3.png}
            \end{center}
        \end{multicols}

    \end{enumerate}
    \newpage
    \item Compound propositions. The set of strings of length $4$ whose characters are $0$s or $1$s is
        the result of four successive set-wise concatenations: $\{0,1\} \circ \{0,1\} \circ \{0,1\}\circ \{0,1\}$. Let's call
        this set $X_4$. Consider the function $f: X_4 \to X_4$ defined by
        $$ f(x) = 
        \begin{cases} 
            y &\text{when $(x)_{2,4} < 15$ and $(y)_{2,4} = (x)_{2,4} + 1$}\\
            1111 &\text{when $x = 1111$}
        \end{cases}$$
        for each $x \in X_4$. In other words, we can describe the function as: $f$ takes a string, interprets it as the binary fixed-width $4$
        expansion of an integer, and then adds $1$ to that integer (unless $x$ is already representing
        the greatest integer that can be represented in binary fixed-width $4$) and outputs the binary 
        fixed-width $4$ expansion of the result.
        \begin{enumerate} 
        \item\gradeComplete Fill in the blanks in the following input-output definition table 
        with four inputs $x_3$, $x_2$, $x_1$, $x_0$ and four outputs $y_3$, $y_2$, $y_1$, $y_0$
        so that $f(x_3x_2x_1x_0) = y_3y_2y_1y_0$.
        
        \begin{center}
        \begin{tabular}{cccc|cccc}
        $x_3$ & $x_2$ & $x_1$ & $x_0$ & $y_3$ & $y_2$ & $y_1$ & $y_0$\\
        \hline
        $1$ & $1$ & $1$ & $1$ & $1$ & $1$ & BLANK1 & $1$\\
        $1$ & $1$ & $1$ & $0$ & $1$ & $1$ & $1$ & $1$\\
        $1$ & $1$ & $0$ & $1$ & $1$ & BLANK2 & $1$ & $0$\\
        $1$ & $1$ & $0$ & $0$ & $1$ & $1$ & $0$ & $1$\\
        $1$ & $0$ & $1$ & $1$ & $1$ & $1$ & $0$ & $0$\\
        $1$ & $0$ & $1$ & $0$ & $1$ & $0$ & $1$ & $1$\\
        $1$ & $0$ & $0$ & $1$ & $1$ & $0$ & $1$ & $0$\\
        $1$ & $0$ & $0$ & $0$ & BLANK3& $0$ & $0$ & $1$\\
        $0$ & $1$ & $1$ & $1$ & $1$ & $0$ & $0$ & $0$\\
        $0$ & $1$ & $1$ & $0$ & $0$ & $1$ & $1$ & BLANK4\\
        $0$ & $1$ & $0$ & $1$ & $0$ & $1$ & $1$ & BLANK5\\
        $0$ & $1$ & $0$ & $0$ & $0$ & $1$ & $0$ & $1$\\
        $0$ & $0$ & $1$ & $1$ & $0$ & $1$ & $0$ & $0$\\
        $0$ & $0$ & $1$ & $0$ & $0$ & $0$ & $1$ & $1$\\
        $0$ & $0$ & $0$ & $1$ & $0$ & $0$ & $1$ & $0$\\
        $0$ & $0$ & $0$ & $0$ & BLANK6 & $0$ & $0$ & $1$\\
        \end{tabular}
        \end{center}
  
        \item\gradeCorrect
        Construct an expression (as a compound proposition) for $y_0$ in terms of the inputs 
        $x_3, x_2, x_1, x_0$. Justify your expression by referring to the definition of the logic 
        gates XOR, AND, OR, NOT and the definition of the function $f$. Hint: our work on the half-adder might 
        be helpful.
        \item\gradeCorrect
        Construct an expression (as a compound proposition) for $y_1$ in terms of the inputs 
        $x_3, x_2, x_1, x_0$. Justify your expression by referring to the definition of the logic 
        gates XOR, AND, OR, NOT and the definition of the function $f$. Hint: our work on the half-adder might 
        be helpful.
        \item\gradeComplete Draw a combinatorial circuit corresponding to these compound propositions.
        Remember that the symbols for the inputs will be on the left-hand-side and 
        the symbol for the outputs $y_0$ and $y_1$ will be on the right-hand side. Use gates (draw the appropriate
        shapes and add labels for clarity) and wires to connect the inputs appropriately to give the output.
        \item\gradeComplete Construct expressions (as a compound propositions) for $y_2$ and $y_3$ 
        in terms of the inputs  $x_3, x_2, x_1, x_0$. Are these similar to the expressions for $y_0$ and $y_1$?
        \end{enumerate}
        \item Logical Equivalence. Imagine a friend suggests the following argument to you: ``The compound proposition
        \[
        (x \lor y) \land z
        \]
        is logically equivalent to 
        \[
        x \lor (y \land z)
        \]
        because I can transform one to the other using the following sequence of logical equivalences: 
        \[
           (x \lor y) \land z \equiv
           (x \lor (y \land y)) \land z \equiv
           x \lor ( (y \lor y) \land z) \equiv x \lor (y \land z) 
        \]
        because $y$ is logically equivalent to both $y \land y$ and to $y \lor y$".
        
        \begin{enumerate}
        \item\gradeCorrect Prove to your friend that they made a mistake by giving a truth
        assignment to the propositional variables $x,y,z$ so that 
        the two compound propositions 
        $ (x \lor y) \land z$ and $ x \lor (y \land z)$ have different truth values.
        Justify your choice by evaluating these compound propositions using the definitions of the logical connectives 
        and include enough intermediate steps so that a student in CSE 20 who may be 
        struggling with the material can still follow along with your reasoning.
        
        \item\gradeComplete 
        Help your friend find the problem in their argument by pointing out which step(s) were incorrect.
        
        \item\gradeComplete Give {\bf three} different compound propositions
        that are actually logically equivalent to (and not the same as)
        \[
            (x \lor y) \land z
        \]
        Justify each one of these logical equivalences either by applying a sequence of logical equivalences
        or using a truth table.  Notice that you can use other logical operators (e.g. $\lnot, \lor, \land, \oplus, \to, 
        \leftrightarrow$) 
        when constructing your compound propositions.
     
        {\it Bonus; not for credit (do not hand in)}: How would you translate each of the equivalent compound
        propositions in English? Does doing so help illustrate why they are equivalent?
        \end{enumerate}
        
     
\end{enumerate}

\newpage

\setlength{\parindent}{0em}
\setlength{\parskip}{0em}

\title{Project}
\date{Due: 5/8/24 at 5pm (late submission until 8am next morning)}


\maketitle
\thispagestyle{fancy}

\vspace{-20pt}

In the project component of this class, you will extend your 
work on assignments and explore applications of your choosing. 
{\it Why?}
To go deeper and explore the material from discrete math and how it relates to Computer Science.
You will watch some videos and read some articles, and then connect them to our work in CSE 20. There 
are two tasks in the project, and for each one you will submit a short video and a PDF document, each 
addressing specific questions.


As you work on the project, keep in mind our three high-level goals for CSE 20:
\begin{itemize}
\item Model systems with tools from discrete mathematics and reason about implications 
of modelling choices. Explore applications in CS through multiple perspectives, including software, hardware, and theory.
\item Know, select and apply appropriate computing knowledge and problem-solving techniques. Reason about computation and systems. Use mathematical techniques to solve problems. Determine appropriate conceptual tools to apply to new situations. Know when tools do not apply and try different approaches. Critically analyze and evaluate candidate solutions.
\item Clearly and, unambiguously communicate computational ideas using appropriate formalism. Translate across levels of abstraction.
\end{itemize}


\subsubsection*{What resources can you use?} This project must be completed individually, 
without any help from other people, including the course staff (other than logistics support if 
you get stuck with screencast).
You can use any of this quarter's CSE 20 offering (notes, readings, class videos, homework feedback)
and videos and articles explicitly referenced in the project description. 
These resources should be more than enough.
If you are struggling to get started and want to look elsewhere online, 
you must acknowledge this by listing and citing any resources you consult 
(even if you do not explicitly quote them), including any large-language model style resources (ChatGPT, CoPilot, etc.). 
Link directly to them and include the name of the author / video creator, 
any search strings or prompts you used, and the reason you consulted this reference.

If you get stuck on any part of the project, we encourage you to focus on communicating what you think 
the question might mean, including referring to an example from class or homework you think might be relevant, 
and include in your submission a discussion of any aspect where you're unsure. Clear communication about these
theoretical ideas and their applications is one of the main goals of the project.

\subsubsection*{Submitting the project} You will submit a PDF plus a video file for each of the 
two tasks. All file submissions will be in Gradescope. 
One way to record the video is to record your screen (this is sometimes called screencast).
You can use any software you choose. 
One option is to record yourself with Zoom; a tutorial on how to use Zoom to record a 
screencast (courtesy of Prof. Joe Politz)  is here: 
\url{https://drive.google.com/open?id=1KROMAQuTCk40zwrEFotlYSJJQdcG_GUU}.
The video that was produced from that recording session in Zoom is here:
\url{https://drive.google.com/open?id=1MxJN6CQcXqIbOekDYMxjh7mTt1TyRVMl}
Please send an email to the instructor (minnes@ucsd.edu) if you have 
concerns about  the video / screencast components of this project or cannot 
complete projects in this style for some reason.

\subsubsection*{Task 1: Exploring an application}
In CSE 20 this quarter, we will be exploring the applications of discrete mathematics for core Computer
Science and Engineering topics. {\bf Pick one} of the following videos about work done 
here at UC San Diego (or by people associated with UC San Diego) and complete both steps of 
the task described below.


[{\bf Video}] Bioinformatics and virology {\it Niema Moshiri}
{\small \url{https://www.youtube.com/watch?v=PrAoks7OhE8}}

[{\bf Video}] Human robotics interaction {\it Healthcare Robotics Lab at UCSD}
{\small \url{https://www.youtube.com/watch?v=bS0-asHDXPc}}

[{\bf Video}] Natural Language Processing {\it Taylor Berg-Kirkpatrick}
{\small \url{https://www.youtube.com/watch?v=8zMfAdPZKnk}}

[{\bf Video}] Cryptography and Complexity {\it Russell Impagliazzo}
{\small \url{https://youtu.be/RjzSFa03i2U}}

[{\bf Video}] Machine learning (and surfing) for climate science {\it Jasmine Simmons and Engineers for Exploration}
{\small \url{https://www.uctv.tv/computer-science/search-details.aspx?showID=34350}}



\begin{enumerate}
\item Watch the video from above that you selected. {\bf Record a new video} where you present 
your answers to the following three questions:
\begin{itemize}
    \item Which video did you watch?
    \item Why did you choose the video you watched?
    \item What are three kinds of data or information that are related to the project described in the video?
\end{itemize}
Your video for this task should be 1-3 minutes. Start with 
your face and your student ID visible for a few seconds at the beginning, and introduce yourself audibly while on screen. 
You don't have to be on camera for the rest of the video, though it's fine if you are. 
We are looking for a brief confirmation that it's you creating the video and doing the work 
submitted for the project. When you are explaining the three kinds of data or information (third part of question 1), 
we recommend you show them in the video in some form.

\item In the document part of this task, you will explore how to use CSE 20 techniques to 
model each of the three kinds of data or 
information that you identified in 
your video. In particular, answer the following questions in a document 
that you will submit to Gradescope:
\begin{itemize}
    \item Describe how each of the three kinds of data or information 
    can be modelled using the data types that we discussed in class (sets, ordered $n$-tuples, 
    strings, or functions). Explain why you are choosing to model this data or information with this data type: what are the benefits 
    and what are the limitations of this model?
    \item Write a set using roster method or set builder notation or recursive description that has at least three elements
    and that demonstrates
    your model. Include a description of the data or information in English and also include how it is represented 
    in your model.
\end{itemize}
{\bf Type out your answers} to these questions for all of the three kinds of data you identified, 
how you could model each one, and example sets, and upload your PDF 
to Gradescope.
\end{enumerate}


\subsubsection*{Task 2: Errors and multiple representations}

Sometimes, the way we represent data leads to imprecision or outright mistakes. 
Watch the video and read the articles below and then complete the task described.


[{\bf Video}] Minecraft mysteries: {\small \url{https://youtu.be/ei58gGM9Z8k?si=oWZQtM_9-7WTGuRO}}

[{\bf Article}]  Excel bug causes a wide-spread problem in published genomics papers.
{\small \url{https://www.nature.com/articles/d41586-021-02211-4}  (You may need to be on the UCSD network to access this article.)}

[{\bf Article}] IEEE profile of Katherine Johnson, a NASA ``computer" who calculated trajectories for 
early space exploration and who passed away in 2020

{\small \url{https://spectrum.ieee.org/the-institute/ieee-history/katherine-johnson-the-hidden-figures-mathematician-who-got-astronaut-john-glenn-into-space}}

[{\bf Article}] NASA report about the unsuccessful 1999 Mars Climate Orbiter mission
{\small \url{https://solarsystem.nasa.gov/missions/mars-climate-orbiter/in-depth/}}

[{\bf Article}] Article about NASA Voyager 1 data corruption

{\small \url{https://www.sciencealert.com/nasa-has-finally-identified-the-reason-behind-voyager-1s-gibberish}}


\begin{enumerate}
    \item {\bf Record a video} where you discuss your answers to these questions:
    \begin{itemize}
        \item What are examples in the video or articles above where computers or Computer Science were used
        to help *avoid* an error?
        \item Give an example where *you* used computers or Computer Science 
        to help you *avoid* an error?
        \item What are examples in the video or articles above where the use of computers or Computer Science
         *caused* an error.
        \item Give an example where *your* use of computers or Computer Science 
        *caused* an error.
        \item What do you do to increase your confidence in the results of your own human and digital 
        (i.e. machine) computation? Why do you think these are sufficient?
     \end{itemize}
    Focus on your communication clarity in the video for this task. Imagine that your audience is a high school student
    who is exploring the benefits and drawbacks of using computers to solve problems.
    Your video for this task should be 1-3 minutes. Start with 
    your face and your student ID visible for a few seconds at the beginning, and introduce yourself audibly while on screen. 
    You don't have to be on camera for the rest of the video, though it's fine if you are. 
    When you are giving examples, you should speak about them as well as having them 
    displayed on the screen (written or typed using clear and correct notation if relevant, or screen shots from 
    the video if relevant).
    
    \item In the document part of this task, you will explore what mistakes our choice of representations can cause by doing the following:
    \begin{itemize}
        \item Pick one of the definitions we've used in CSE 20 for representing {\bf numbers}. 
        \item Copy the definition for this representation into your writeup, and cite 
        which page of which week's notes you're using.
        \item Describe, using roster method or set builder notation, the set of numbers that can represented 
        with this number representation definition.
        \item Give an example of a limitation of this number representation by showing 
        what error could be introduced when using this number representation for an application of your choosing.
    \end{itemize}
    {\bf Type out your work} above and upload your PDF to Gradescope.
    \end{enumerate}

\subsubsection*{Grading}
Your work on the project will be assigned a letter grade. 
\begin{itemize}
\item To earn at least a C on the project, most parts of the project should be attempted
and your submission should correctly demonstrate some of the tools, techniques, and formalisms we 
used in class. 
\item To earn at least a B on the project, almost all of the parts of the project should be 
substantially complete, and a significant amount of detail and correct notation should be 
used throughout your examples and explanations.
\item To earn at least an A on the project, all of the parts of the project should be substantially complete,
with correct and appropriate use of CSE 20 concepts and notation and clear and detailed explanations.
\end{itemize}

Since the project is also used to add +/- modifiers at the end of the quarter, you can consider going beyond
the requirements as well.
\begin{itemize}
    \item For example: in your PDF writeup for task 1, you could propose several alternate modelling choices and 
discuss the tradeoffs (advantages and disadvantages) between them. 
    \item Another exmaple: in your PDF writeup for 
task 2, you could discuss what could be done to detect or correct possible errors resulting from choices of
data representations. 
    \item You are welcome to explore other extensions too.  Keep in mind, however, that resource limitations
    means we will be limited to grading no more than 4 minutes of video and 2 pages for each task.
\end{itemize}
\newpage%\documentclass{article}


%\usepackage{amsfonts,amssymb,amsmath,amsthm,color,comment,enumerate, graphicx,euscript,hyperref}
%\usepackage[makeroom]{cancel}
%\usepackage{paralist,listings}

\setlength{\evensidemargin}{0in}
\setlength{\oddsidemargin}{0in}
\setlength{\textwidth}{6.6in}
\setlength{\textheight}{8.8in}
\setlength{\topmargin}{0in}
\setlength{\footskip}{0.45in}
\renewcommand{\baselinestretch}{1}
\newlength{\saveparindent}
\setlength{\saveparindent}{\parindent}

% % NOTE(joe): This environment is credit @pnpo (https://tex.stackexchange.com/a/218450)
% \lstnewenvironment{algorithm}[1][] %defines the algorithm listing environment
% {   
%     \lstset{ %this is the stype
%         mathescape=true,
%         frame=tB,
%         numbers=left, 
%         numberstyle=\tiny,
%         basicstyle=\rmfamily\scriptsize, 
%         keywordstyle=\color{black}\bfseries,
%         keywords={,procedure, div, mod, for, to, input, output, return, datatype, function, in, if, else, foreach, while, begin, end, } %add the keywords you want, or load a language as Rubens explains in his comment above.
%         numbers=left,
%         xleftmargin=.04\textwidth,
%         #1 % this is to add specific settings to an usage of this environment (for instnce, the caption and referable label)
%     }
% }
% {}

% \newcommand{\A}[0]{\texttt{A}}
% \newcommand{\U}[0]{\texttt{U}}
% \newcommand{\G}[0]{\texttt{G}}
% \newcommand{\C}[0]{\texttt{C}}

\newif \ifsolution
\solutiontrue
\solutionfalse

\newcommand{\sol}[1]{\medskip\fbox{\begin{minipage}{5.5in}{#1}\end{minipage}}\medskip}


\begin{center}
{\Large
CSE 20 Spring 2024\\ 
Practice for Test 1 \ifsolution{\qquad Solutions}\fi}
\end{center}

\thispagestyle{empty}

\ifsolution{}
\else{}
Below are the instructions that will be on the first page of the test package:

\begin{center}
  \begin{minipage}[t]{7in}
  \rule{\linewidth}{2pt}
  \textbf{INSTRUCTIONS --- READ THIS NOW}
  \begin{itemize}
  
  \setlength{\itemsep}{0.025in}
  
  \item  Write your name, PID, current seat number, exam time, 
  and the academic integrity pledge in the indicated space above and 
  on the designated  {\bf answer sheet}.
  We will check for {\bf all} of this identifying information before grading.
  Write your answers in the specified areas, or your work will not be graded. 
  
  \item We will not be answering questions about the exam during the exam period. 
  If any bugs are found in the exam after the exam period, the affected question part(s) will be addressed.
  
  \item  You may use one 8.5"x11", doublesided sheet of notes that you create and bring to the exam room, but no other books, notes, or aids.
  
  \item You may not speak to any other student in the exam room while the exam 
  is in progress (including after you hand in your own exam).  You may not share
  {\bf any information} about the exam with anyone who has not taken it.
  
  \item Turn off and put away all cellphones, calculators, and other electronic devices.
  You may not access any electronic device during the exam period. If you need to leave 
  the room during the exam period, you must leave all electronic devices with an exam proctor.
  
  \item  To receive full credit, your answers must
  be written neatly, legibly, and sufficiently darkly to scan well in the indicated answer box. Your solution will be evaluated both for correctness and clarity.
  Read the instructions for each part carefully to determine what is required for full credit.
  This test has $??$ problems worth a total of $??$ points.
  
  \item This exam is {\bf 45 minutes} long. Read all the problems first before you start 
  working on any of them, so you can manage your time wisely.
  
  \item Please stay seated until the end of the exam period.
  We will collect all exams and note sheets at the {\bf end} of the exam period, to minimize disruption 
  for students who wish to use the full time for the exam. Please show your ID to a proctor when you hand 
  in your exam. 
  
  
  \end{itemize}
  \end{minipage} \hfill
  \end{center}
  \newpage
\fi

\begin{description}

\item[1. Modeling, Sets and Functions, Algorithms] 
In machine learning, clustering can be used to group similar data for prediction and recommendation.  For example,
each Netflix user's viewing history can be represented as a $n$-tuple indicating their preferences about
movies in the database, where $n$ is the number of movies in the database.  Each element in the $n$-tuple indicate
the user's rating of the corresponding movie: 
$1$ indicates the person liked the movie, $-1$
that they didn't, and $0$ that they didn't rate it one way or another.

Consider the following algorithm for determining if a user's viewing history represents
strong opinions on many movies.
\begin{algorithm}[caption={Determine if user's ratings tuple encode strong opinions}]
procedure $\textit{opinion}$($(r_1, \ldots, r_n)$: a $n$-tuple of ratings; $c$: a nonnegative integer)
$sum$ := $0$
for $i$ := $1$ to $n$
  if $r_i \neq 0$
    $sum$ := $sum + 1$
return $sum \geq c$
\end{algorithm}

  \begin{enumerate}%[(a)]
  \item When $n = 3$, describe the set of all $n$-tuples representing user ratings
  \begin{enumerate}%[i.]
  \item By the roster method.
  
  \ifsolution{
  \sol{There are 27 $3$-tuples whose elements are each $-1$, $0$, or $1$.  We can list them as follows.
  \begin{align*}
  \{ & (-1, -1, -1) , (-1, -1, 0) , (-1, -1, 1) , (-1, 0, -1), (-1, 0, 0), (-1, 0, 1),\\
     & (-1, 1, -1), (-1, 1, 0), (-1, 1, 1), (0, -1, -1) , (0, -1, 0) , (0, -1, 1) , \\
     & (0, 0, -1), (0, 0, 0), (0, 0, 1), (0, 1, -1), (0, 1, 0), (0, 1, 1), (1, -1, -1) , \\
     & (1, -1, 0) , (1, -1, 1) , (1, 0, -1), (1, 0, 0), (1, 0, 1), (1, 1, -1), (1, 1, 0), (1, 1, 1) \}  
  \end{align*}
  }}\fi
  
  \item With set builder notation.
  
  \ifsolution{
  \sol{Using set builder notation, we can specify the constraints on each component of the $3$-tuple:
\[
	\{ (x,y,z) ~\mid~ x \in \{-1,0,1\}~\text{and}~ y\in \{-1,0,1\}~\text{and}~ z \in \{-1,0,1\} \}
\]  
}}\fi
  
  \item Using a recursive definition.
  
  \ifsolution{
  \sol{  Let's call the set of ratings of three movies $R_3$. We can define this set recursively 
  by listing each of the finitely many tuples in $R_3$ in the basis step and then 
  defining the recursive step to be: when $u \in R_3$, so is $u$.

  Note that this recursive definition is somewhat contrived: the recursive step isn't doing anything helpful.
  }}\fi
  \end{enumerate}
 \item What are the possible return (output) values of this algorithm?
  
  \ifsolution{
  \sol{ Since the algorithm returns the result of a comparison, the possible values are 
  True or False.}
  }\fi
  
  \item {\bf Trace} the computation of $opinion( (-1, 0, 0, 0, 1, 1,-1), 2)$.

\ifsolution{\sol{
The computation of $opinion( (-1, 0, 0, 0, 1, 1,-1), 2)$  first (in line 2) initializes
the variable $sum$ to $0$ and then steps through each component of the tuple in the for 
loop in line 3, checking (in line 4) if the value is zero or not. The first component
of $(-1, 0, 0, 0, 1, 1,-1)$ is nonzero so line 5 increments $sum$, which now has value $1$.
When the for loop checks the components at positions 2,3,4, the $sum$ variable is not updated.
When the for loop 
gets to each of positions 5,6,7, the nonzero value of the elements at these positions means the 
if branch of the condition is followed and the $sum$ variable is incremented three times, to 
get to the value $4$.  In line $6$, the comparison $sum \geq c$ evaluates to $4 \geq 2$, which is True.
Thus, $opinion( (-1, 0, 0, 0, 1, 1,-1), 2)$ returns True.
}}\fi

  \item Give an example of a nonnegative integer $c$ so that, no matter which $n$-tuple
  $(r_1, \ldots, r_n)$ we consider, $opinion( (r_1, \ldots, r_n), c)$ will have the same value.
  What value is it?

\ifsolution{\sol{The nonnegative integer $c = 0$ is an example.  The value of $sum$ is initialized to $0$ in line 2. 
The values of the components $n$-tuple $(r_1, \ldots, r_n)$ determine whether the value of $sum$ ever increases in the form loop lines 3-5; the value can never decrease. Thus, no matter the choice of $n$-tuple,
 the comparison $sum \geq 0$ will evaluate to True.
}}\fi


\end{enumerate}

\item[2. Sets and Functions] \qquad \\
\begin{enumerate}%[(a)]
\item Give a recursive definition for the set $\mathbb{N}$.

\ifsolution{\sol{The set of nonnegative integers can be defined recursively as follows:

Basis step: $0 \in \mathbb{N}$

Recursive step: If $x \in \mathbb{N}$ then $x+1 \in \mathbb{N}$.
}}\fi

\item Use the recursive definition from part (a) to give a recursive definition for the function with domain $\mathbb{N}$, codomain
$\mathbb{N}$ which computes, for input $i$, the sum of the first $i$ powers of $2$.  For example, on input $0$, the function 
evaluates to $2^0$, namely to $1$.  On input $1$, the function evaluates to $2^0 + 2^1$, namely $3$. 
On input $2$, the function evaluates to $2^0 + 2^1 + 2^2$, namely $7$. 

\ifsolution{\sol{Let's call this function $sumPow$.  Its recursive definition will mirror the recursive definition of its domain, from part (a).

$sumPow: \mathbb{N} \to \mathbb{N}$

Basis step: $sumPow(0) = 1$.

Recursive step: If $x \in \mathbb{N}$ then $sumPow(x+1) = sumPow(x) + 2^{x+1}$.

In this function definition, we assumed that we have access to a definition of the power function
to compute powers of two. In fact, we can define this
function recursively as well:

$powTwo: \mathbb{N} \to \mathbb{N}$

Basis step: $powTwo(0) = 1$.

Recursive step: If $x \in \mathbb{N}$ then $powTwo(x+1) = 2 powTwo(x) = powTwo(x) + powTwo(x)$.

}}\fi
\end{enumerate}

\item[3. Base expansions, Multiple representations] \qquad \\
\begin{enumerate}%[(a)]
\item Compute the ternary (base 3) expansion of $28$.

\ifsolution
\sol{
We can express $28$ in terms of powers of $3$ as 
\[
28 = 1 \cdot 27 + 1  = 1 \cdot 3^3 + 0 \cdot 3^2 + 0 \cdot 3^1 + 1 \cdot 3^0 = (1001)_3.
\]
Therefore, by definition of base $3$ expansion, $28 = (1001)_3$.
}
\else{}
\fi
\item Compute the product of $(6A)_{16}$ and $(11)_{16}$, without converting
either number to another base.

\ifsolution
\sol{
We use the usual algorithm for multiplication, except using hexadecimal symbols.

\begin{align*}
   &6A \\
\underline{\times~ }&\underline{11} \\
& 6A \\
\underline{+6~}&\underline{A0} \\
7&0A
\end{align*}

Thus, the product of $(6A)_{16}$ and $(11)_{16}$ is $(70A)_{16}$.
}
\else{}
\fi

\item Confirm your answer for part (b) by converting $(6A)_{16}$ and $(11)_{16}$ to
decimal, multiplying them, and converting the product to base $16$.

\ifsolution
\sol{
Converting:
\[
(6A)_{16} = 6  \cdot  16^1 +  10 \cdot 16^0  = 96 +  10 =  106
\]
\[
(11)_{16} = 1 \cdot 16^1 +  1 \cdot 16^0  = 16+  1 =  17
\]
Multiplying in decimal:

\begin{align*}
   &106 \\
\underline{\times~ }&\underline{~17} \\
& 742 \\
\underline{+1}&\underline{060} \\
1&802
\end{align*}

Converting back to hexadecimal:  by long  division, 
\begin{align*}
1802 &= 112\cdot 16  + 10 \\
112  &= 7 \cdot 16 +  0\\
7 &= 0 \cdot 16 + 7
\end{align*}
Thus, $1802  = (70A)_{16}$, agreeing  with part (b).
}

\else{}
\fi

\item How many bits will there be in the binary (base $2$) expansion of $2020$?  Can you compute this
without fully converting $2020$ to base $2$?


\ifsolution
\sol{
The  leading coefficient will need to be in the column  corresponding to the highest power  of $2$ less
than  or equal to   $2020$. Listing the powers of $2$:
\begin{center}
\begin{tabular}{|l|c|c|c|c|c|c|c|c|c|c|c|c|c|}
\hline
Exponent & $0$ & $1$ & $2$ &  $3$  &  $4$  & $5$ & $6$ & $7$ & $8$ & $9$ &  $10$ & $11$\\
\hline
Power &$1$ & $2$ & $4$ &  $8$  &  $16$  & $32$ & $64$ & $128$ & $256$ & $512$ &  $1024$ & $2048$\\
\hline
\end{tabular}
\end{center}
we see  that the binary expansion of $2020$  will  be of the  form  $(1a_{9} \cdots a_0)_{2}$ and will therefore
have $11$ bits.
}
\else{}
\fi

\item Give an example of a number that can be represented in base 2 fixed-width 3,
but not in base 2 expansion.

\ifsolution
\sol{
The number  $0$ can't be  represented in binary (base 2)  but is represented in base $2$ fixed-width $3$
$(000)_{2,3}$.
}
\else{}
\fi

\item Give an example of a number that can be represented in
base 2 expansion, but not in base 2, fixed-width 3 expansion.

\ifsolution
\sol{
The number $8$ can be  represented in base  $2$ as $(1000)_2$ but can't be represented in 
binary fixed-width $3$ because it requires four bits.}
\else{}
\fi

\item Give the representation of -7 in sign-magnitude width 3 and 2s complement
width 3. Then do the same for width 4.

\ifsolution
\sol{
The  number $-7$ can't be represented in sign magnitude width 3 or in 2s complement width 3:
\begin{itemize}
\item The least number that can be represented in sign magnitude width 3 has representation $[111]_{s,3}$ 
and is  $-3$.
\item The least number that can be represented in 2s complement width 3 has representation $[100]_{2c,3}$ 
and is  $-4$.
\end{itemize}
With width 4 we can use the following representations:
\begin{itemize}
\item In sign magnitude width 4, $-7$ has representation $[1111]_{s,4}$  because the leftmost bit  is the 
sign bit and the magnitude  of $-7$ has binary fixed width 3 representation $7 = (111)_{2,3}$.
\item In 2s complement width 4, $-7$ has representation $[1001]_{2c,4}$  because the leftmost bit  is the 
sign bit and rightmost bits need to  represent $2^{4-1}-7 = 1$ in binary fixed width 3, $(001)_{2,3}$.
Alternatively, we calculate that $[1001]_{2c,4} = 1 \cdot (-2)^3 + 0 \cdot 2^2 + 0 \cdot 2^1 + 1 \cdot 2^0 = -8 + 1 = -7$.
\end{itemize}
}
\else{}
\fi

\item Give the representation of 10.5625 in fixed-width base-2 expansion with
integer width 4 and fractional width 9.

\ifsolution
\sol{
We need  to write $10.5625$ as 
\[
(a_3 a_2 a_1 a_0.c_1  c_2 c_3  c_4 c_5  c_6 c_7 c_8 c_9)_{2,4,9}
\]
For the integer part:  $10 = 1\cdot 2^3 +  0 \cdot 2^2 + 1 \cdot 2^1 + 0 \cdot  2^0 = (1010)_2$

For the fractional part: $0.5625 =  1\cdot 2^{-1}  +  0 \cdot  2^{-2} + 0 \cdot 2^{-3} + 1 \cdot 2^{-4} $
where
\begin{center}
\begin{tabular}{|l|c|c|c|c|c|c|c|c|}
\hline
Exponent & $-1$ & $-2$ & $-3$ &  $-4$  &  $-5$ & $-6$ & $-7$ \\
\hline
Power &$0.5$ & $0.25$ & $0.125$ &  $0.0625$  &  $0.03125$  & $0.015625$ & $0.0078125$\\
\hline
\end{tabular}
\end{center}
Thus, 
\[
10.5625 = (1010.100100000)_{2,4,9}
\]
}
\else{}
\fi
\end{enumerate}


\item[4. Circuits]   
A triangular number (or triangle number) counts the objects that can form an equilateral
triangle, as in the diagram below. The $n^{\text{th}}$ triangular number is the sum of the first $n$
 integers,
as shown in the following figure illustrating the first four triangular numbers (what is the
fifth one?):
\begin{center}
\includegraphics[width=2.7in]{../../resources/images/triangle.png}
\end{center}
Design a circuit that takes four inputs $x_0, x_1, x_2, x_3$ and outputs
True (T or 1) if the integer value 
$(x_3x_2x_1x_0)_{2,4} $ is a triangular number, and False (F or 0) otherwise. You may assume that 0 is
not a triangular number. ({\it Credit: UBC Department of Computer Science})

\ifsolution{
\sol{
Using the definition, we can write a table of values for the circuit we need to build, since the output will be 
one when $(x_3 x_2 x_1 x_0)_{2,4}$ will be one of $1, 3, 6, 10, 15$.
\begin{center}
\begin{tabular}{cccc||c}
$x_3$& $x_2$ & $x_1$ & $x_0$ & Output \\
\hline
$1$ & $1$ & $1$ & $1$ & $1$ \\
$1$ & $1$ & $1$ & $0$ & $0$ \\
$1$ & $1$ & $0$ & $1$ & $0$ \\
$1$ & $1$ & $0$ & $0$ & $0$ \\
$1$ & $0$ & $1$ & $1$ & $0$ \\
$1$ & $0$ & $1$ & $0$ & $1$ \\
$1$ & $0$ & $0$ & $1$ & $0$ \\
$1$ & $0$ & $0$ & $0$ & $0$ \\
$0$ & $1$ & $1$ & $1$ & $0$ \\
$0$ & $1$ & $1$ & $0$ & $1$ \\
$0$ & $1$ & $0$ & $1$ & $0$ \\
$0$ & $1$ & $0$ & $0$ & $0$ \\
$0$ & $0$ & $1$ & $1$ & $1$ \\
$0$ & $0$ & $1$ & $0$ & $0$ \\
$0$ & $0$ & $0$ & $1$ & $1$ \\
$0$ & $0$ & $0$ & $0$ & $0$ \\
\end{tabular}
\end{center}
The output can be described using a compound proposition in DNF with five clauses (because there 
are five rows with output $1$): 
\begin{align*}
&(x_3 \land x_2 \land x_1 \land x_0) \vee (x_3 \land \neg x_2 \land x_1 \land \neg x_0) \vee
(\neg x_3 \land x_2 \land x_1 \land \neg x_0) \\
&\vee (\neg x_3 \land \neg x_2 \land x_1 \land x_0) \vee
(\neg x_3 \land \neg x_2 \land \neg x_1 \land x_0)
\end{align*}
We can use this DNF to draw a circuit that takes the four inputs and whose output is described.

Alternatively, a  logically equivalent compound proposition is
\[
output = \left [ x_1 \wedge ( x_0 \oplus x_2) \oplus x_3 \right] \vee \left [ \neg x_3 \wedge \neg x_2 \wedge \neg x_1 \wedge x_0 \right]
\]

Using this compound proposition, we can draw a circuit with fewer gates than the one described above.

Note: we didn't include picture of circuits in these sample solutions; 
please come to office hours or post to Piazza if you'd like to get feedback 
on a circuit you draw.
}}
\fi


\item[5. Circuits, Compound Propositions, Logical Equivalence] \qquad \\
\begin{enumerate}%[(a)]
\item Draw a logic circuit that uses 
{\bf exactly three} gates and is logically equivalent to
\[
q \leftrightarrow (p \wedge r)
\]
You may (only) use AND, OR, NOT, and XOR gates.

\ifsolution{
\sol{
Since XOR gates are available, we can use a convenient relationship between XOR and $\leftrightarrow$:
\[
q \leftrightarrow (p \wedge r) \equiv \neg ( q \oplus (p \land r) )
\]
Implementing this compound proposition as a circuit, we will use one XOR gate, one AND gate, and
one NOT gate, as required.

Note: we didn't include picture of circuits in these sample solutions; 
please come to office hours or post to Piazza if you'd like to get feedback 
on a circuit you draw.
}}
\fi


\item Write a compound proposition which is logically equivalent to
\[
(p \oplus q) \leftrightarrow r
\] 
You may only use the logical operators negation ($\neg$),
conjunction $(\wedge)$, and disjunction $(\vee)$.

\ifsolution{
\sol{ We apply several logical equivalences in turn: first, we can rewrite $\leftrightarrow$ in terms of 
$\land$ and $\to$:
\[
(p \oplus q) \leftrightarrow r \equiv (~ (p \oplus q) \to r~) \land (~r \to (p \oplus q)~)
\]
Now, we replace $\to$ by its equivalent disjunctive form: $HYP \to CONC \equiv (\neg HYP) \vee CONC$.
Thus, 
\[
(~ (p \oplus q) \to r~) \land (~r \to (p \oplus q)~) \equiv (~ \neg (p \oplus q) \vee r~) \land (~\neg r \vee (p \oplus q)~)
\]
Last, we replace the XOR with equivalent compound propositions in terms of the other connectives:
the XOR is true means one of its inputs is T and the other F; the XOR is false means its inputs are both T or both F.
\begin{align*}
&  (~ \neg (p \oplus q) \vee r~) \land (~\neg r \vee (p \oplus q)~) \equiv \\
&(~ ( (p \land q) \vee (\neg p \land \neg q) ) \vee r~) \land (~\neg r \vee ( ( p \land \neg q) \vee (\neg p \land q)) ~)
\end{align*}
}}
\fi

\item Find a compound proposition that is in DNF (disjunctive normal form) and is logically equivalent
to 
\[
(p \vee q \vee \neg r) \wedge (p \vee \neg q \vee r ) \wedge (\neg p \vee q \vee r)
\]

\ifsolution{
\sol{
The given compound proposition is in CNF and indicates that the rows in the truth table 
with $p=F, q=F, r = T$, $p=F, q= T, r= F$, and $p =T, q=F, r=F$ are set to F.  Thus, the truth table is 

\begin{center}
\begin{tabular}{ccc||c}
$p$ & $q$ & $r$ & Output \\
\hline 
$T$ & $T$ & $T$& $T$ \\
$T$ & $T$ & $F$& $T$ \\
$T$ & $F$ & $T$& $T$ \\
$T$ & $F$ & $F$& $F$ \\
$F$ & $T$ & $T$& $T$ \\
$F$ & $T$ & $F$& $F$ \\
$F$ & $F$ & $T$& $F$ \\
$F$ & $F$ & $F$& $T$ \\
\end{tabular}
\end{center}

To find a logically equivalence compound proposition in DNF, we represent the disjunction
of landing in each of the rows that evaluate to $T$: the rows where
$p =T, q=T, r=T$, or $p=T, q=T, r=F$, or $p=T, q=F, r=T$, or $p=F, q=T, r=T$, or $p=F, q=F, r=F$.
\begin{align*}
(p \land &q \land r)  \vee ( p \land q \land \neg r) \vee (p \land \neg q \land r) \\
&\vee (\neg p \land q \land r) \vee (\neg p \land \neg q \land \neg r)
\end{align*}
}}
\fi

\end{enumerate}

\item[6. Translating Propositional Logic]  \qquad \\

\begin{multicols}{2}
  \begin{itemize}
  \item[] $p$ is ``The display is 13.3-inch"
  \item[] $r$ is ``There is at least 128GB of flash storage"
  \item[] $u$ is ``There is at least 512GB of flash storage"
  \columnbreak
  \item[] $q$ is ``The processor is 2.2 GHz"
  \item[] $s$ is ``There is at least 256GB of flash storage"
  \end{itemize}
\end{multicols}
  
\begin{enumerate}%[(a)]
\item Are the statements 
\[
p \to (r \vee s \vee u) \qquad, \qquad
q \to (s \vee u) \qquad, \qquad
p \leftrightarrow q
\qquad, \qquad \neg u
\]
consistent?  If so, translate to English a possible assignment of truth values to the input propositions
that makes all four statements true simultaneously.

\ifsolution
\sol{
Yes, these statements are consistent.  Consider the assignment of truth values
\[
p = T \qquad q = T \qquad r = T \qquad s = T \qquad u = F
\]
We evaluate the four compound propositions at this assignment to confirm that it makes 
all of them true simulatneously:
\begin{align*}
p \to (r \vee s \vee u) & = T \to (T \vee T \vee F) = T \to T = T \\
q \to (s \vee u) &= T \to (T \vee F) = T \to T = T \\ 
p \leftrightarrow q &= T \leftrightarrow T = T \\
\neg u & = \neg F = T
\end{align*}
Translating to English, we get ``The display is 13.3-inch, the processor is 2.2GHz, 
there is at least 
128GB of flash storage, there is at least 256GB of flash storage, but there is not at 
least 512GB of flash storage."
}
\else{}
\fi

\item Consider this statement in English: 
\begin{quote}
It's not the case that both the display is 13.3-inch and the processor is 2.2 GHz.
\end{quote}
Determine whether each of the compound propositions below is equivalent to the {\bf negation} of that statement, and justify your answers using either truth tables or other equivalences.

Possible compound propositions:
\begin{itemize}
%\begin{inparaenum}[(I)]
\item $\neg p \vee \neg q$ \qquad
\item $\neg (p \to \neg q)$ \qquad
\item $\neg ( p \wedge q)$ \qquad
\item $(\neg p \leftrightarrow \neg q) \wedge p$
%\end{inparaenum}
\end{itemize}
\ifsolution
\sol{
The sentence can be translated to $\neg ( p \wedge q)$.  Its negation is then
\[
\neg \neg (p \wedge q) ~\equiv~  ( p \wedge q) 
~\equiv~ \neg ( p \to \neg q)
\]
because $A \wedge \neg B \equiv \neg (A \to B)$
so $p \wedge q \equiv \neg (A \to \neg B)$.

This is also equivalent to  $(\neg p \leftrightarrow \neg q) \wedge p$, as we can see from
the last two columns of the the truth 
table below

\begin{center}
\begin{tabular}{cc||c|c|c}
$p$ & $q$ & $(\neg p \leftrightarrow \neg q)$ & $(\neg p \leftrightarrow \neg q) \land p$ & $p \land q$ \\
\hline
$T$ & $T$ & $T$ & $T$ & $T$ \\
$T$ & $F$ & $F$ & $F$ & $F$ \\
$F$ & $T$ & $F$ & $F$ & $F$ \\
$F$ & $F$ & $T$ & $F$ & $F$ \\
\end{tabular}
\end{center}
}
\else{}
\fi

\item Consider the compound proposition 
\[
(p \wedge q) \to (r \vee s \vee u)
\]
Express the {\bf contrapositive} of this conditional as a compound proposition.

Then, give an assignment of truth values to each of the input propositional variables
for which the original compound proposition is True but its {\bf converse} is False.

\ifsolution
\sol{
The contrapositive of this conditional 
is 
\[
\neg (r \vee s \vee u) \to \neg (p \land q)
\]
The converse of the original conditional is
\[
(r \vee s \vee u) \to (p \land q)
\]
Consider the assignment of truth values
\[
p = T \qquad q = F \qquad r = T \qquad s = F \qquad u = F
\]
We evaluate the original conditional and its converse with this assignment of truth values:
\begin{align*}
\textrm{Original}: (p \wedge q) \to (r \vee s \vee u) &= (T \wedge F) \to (T \vee F \vee F) = F \to T = T \\
\textrm{Converse}: (r \vee s \vee u) \to (p \wedge q) &= (T \vee F \vee F) \to (T \wedge F) = T \to F = F 
\end{align*}
}
\else{}
\fi
\end{enumerate}


\item[7. Evaluating predicates, Evaluating nested predicates] \qquad \\
\begin{enumerate}%[(a)]
\item Over the domain $\{ 1,2,3,4,5\}$ give an example of predicates $P(x), Q(x)$
which demonstrate that 
\[
(\forall x P(x)) \vee (\forall x Q(x))
\qquad \text{is not logically equivalent to} \qquad 
 \forall x  (~ P(x) \vee Q(x) ~)
\]

\ifsolution
\sol{
We will define  predicates which  make one of  the  statements False and the other  True.
Consider
\begin{center}
\begin{tabular}{c||c||c}
$x$ & $P(x)$  & $Q(x)$ \\
\hline
$1$ & T &  F \\
$2$ & F &  T \\
$3$ & T &  F \\
$4$ & F &  T \\
$5$ & T &  F \\
\end{tabular}
\end{center}
Evaluating the statements: $\forall x (P(x) \vee Q(x))$ is True because each row in the table
has  at  least one  of $P(x)$, $Q(x)$  evaluate  to True.  On the other hand,  neither $\forall x P(x)$
nor $\forall x Q(x)$ is True, because each predicate  at least one  domain element where it evaluates
to  False (for counterexamples: $P(2)$ is  False and $Q(1)$  is False).  Since $F \vee  F = F$, 
$(\forall x P(x)) \vee (\forall x Q(x))$ is False. Since there is  some definition of the predicates $P$ and
$Q$ where the two  statements  
\[
(\forall x P(x)) \vee (\forall x Q(x))~~,~~
 \forall x  (~ P(x) \vee Q(x) ~)
\]
evaluate to different  truth values,  they  are  not logically  equivalent
}
\else{}
\fi


\item Over the domain $\{0,1,2\}$, give an example of predicates $P(x), Q(x)$
for which all of these statements are true:
$$\forall x (P(x) \to Q(x))$$
$$\exists x \, P(x)$$
$$\exists x \, \neg P(x)$$
$$\exists x \, \neg Q(x)$$

\ifsolution
\sol{
We will define  predicates using their  tables  of  values:
\begin{center}
\begin{tabular}{c||c||c}
$x$ & $P(x)$  & $Q(x)$ \\
\hline
$0$ & T &  T \\
$1$ & F &  T \\
$2$ & F &  F \\
\end{tabular}
\end{center}
We  now prove  that each  of the statements is True:
\begin{itemize}
\item To prove  the universal  statement $\forall x (P(x) \to Q(x))$, we  evaluate  the conditional  
at each element in the domain: At  $0$,  $P(0)  \to Q(0) = T \to T = T$;  at $1$,  $P(1) \to Q(1) = F \to T =  T$; 
at $2$,  $P(2) \to Q(2) = F \to F =  T$.   Since the body of the quantification is True at  all element 
in the domain,  $\forall x  (P (x) \to Q(x))$ is True.
\item  To prove the  existential  statement $\exists x P(x)$, we need a witness  in  the  domain where
$P(x)$  evaluates to  True.  The witness is $0$, since the first row in the definition table gives $P(0) = T$.
\item  To prove the  existential  statement $\exists x \neg P(x)$, we need a witness  in  the  domain where
$\neg P(x)$  evaluates to  True.  The witness is $1$, since the second row in the definition table gives $P(1) = F$ so $\neg P(1) = T$.
\item  To prove the  existential  statement $\exists x \neg Q(x)$, we need a witness  in  the  domain where
$\neg Q(x)$  evaluates to  True.  The witness is $2$, since the third row in the definition table gives $Q(2) = F$ so $\neg Q(2) = T$.

\end{itemize}
}
\else{}
\fi

\end{enumerate}

\item[8. Evaluating predicates, Evaluating nested predicates]  Recall that $S$ is defined as the set of all RNA strands, 
where each strand is a nonempty string made of the bases in 
 $B = \{\A,\U,\G,\C\}$. Recall the definition of the following predicates: $F_{\A}$ with domain $S$ is defined recursively by: 
\begin{itemize}
\item[]Basis step: $F_{\A}(\A) = T$, $F_{\A}(\C) = F_{\A}(\G) = F_{\A}(\U) = F$
\item[]Recursive step: If $s \in S$ and $b \in B$, then $F_{\A}(sb) = F_{\A}(s)$
\end{itemize}

$P_{\A\U\C}$ with domain $S$ is defined as the predicate whose truth set
is the collection of RNA strands where the string $\A\U\C$
is a substring (appears inside $s$, in order and consecutively)

$L$ with domain $S \times \mathbb{Z}^+$ is defined by, for $s \in S$ and $n \in \mathbb{Z}^+$,
\[
L( s, n) = \begin{cases}
T &\qquad\text{if $rnalen(s) = n$}\\
F &\qquad\text{otherwise}\\
\end{cases}
\]
\begin{enumerate}%[(a)]
\item Give a witness for $\exists s_1  \, \exists s_2 \,(L(s_1, 4) \land L(s_2,4) \land 
\neg ( s_1 = s_2) )$ where $S$ is
the set of RNA strands, $L(s, n)$ is a predicate with domain $S
\times \mathbb{Z}^+$ that is true when $s$ has length $n$

\ifsolution
\sol{
In English, the statement  can be translated to ``There  are strands $s_1, s_2$ where each has length
$4$ and the strands are not equal."
A witness  is  $s_1 = \C\C\C\C$, $s_2 = \G\G\G\G$.  Since  each of  $s_1, s_2$ is  an element of $S$, 
$s_1$ has  length $4$, $s_2$ has length $4$ and these two strings are not the same.
} 
\else{}
\fi

\item Give a counterexample that disproves  
$\forall i  \,(\neg L(\A\C\U, i)) $ 

\ifsolution
\sol{
In English, the statement  can be translated to ``No positive integer is the length of  the strand \A\C\U."

A counterexample that helps us  disprove this statement is $3$.  This is a positive integer and
is the length of $\A\C\U$ so  $L(\A\C\U, 3)$ is True, hence $\neg L(\A\C\U, 3)$
is False.}
\else{}
\fi

\item Determine which of the following statements is True or False (you do not need to prove your answer).
\begin{enumerate}%[i.]
\item $\exists n \in \mathbb{Z}^+~ \forall s \in S~ ( P_{\A\U\C}(s) \to \lnot L(s,n) )$

\ifsolution{\sol{In English, this statement is saying that there is a positive integer that is not the length of every
RNA strand with $\A\U\C$.  This is {\bf true}, as we can see with the witness $n = 1$.
}}\fi

\item $\exists n \in \mathbb{Z}^+~ \forall s \in S~ ( P_{\A\U\C}(s) \land \lnot L(s,n) )$

\ifsolution{\sol{In English, this statement is saying that there is a positive integer that is not the length of any
RNA strand and that each RNA strand has $\A\U\C$ as a substring.  This is {\bf false}, because
when we consider any positive integer, we'll be able to find a counterexample strand that does not have $\A\U\C$
as a substring and so makes the conjunction false.
}}\fi

\item $\forall s \in S~\exists n \in \mathbb{Z}^+~ ( P_{\A\U\C}(s) \lor \lnot L(s,n) )$

\ifsolution{\sol{In English, this statement is saying that for each RNA strand we can find
some positive integer so that at least one of ``the strand has $\A\U\C$ as a substring" and ``this number
is not the length of this strand" is true. This is {\bf true}, because given any RNA strand, we can pick a witness
positive integer by choosing a number that is not its length.
}}\fi

\item $\forall s \in S~\exists n \in \mathbb{Z}^+~ (L(s,n) \to \lnot P_{\A\U\C}(s))$

\ifsolution{\sol{In English, this statement is saying that for each RNA strand, there is a witness positive integer
which makes the conditional statement ``if the RNA strand has this length then it does not have \A\U\C~as a substring" 
true. This is {\bf true}, because given any RNA strand, we can pick a witness
positive integer by choosing a number that is not its length which will make the hypothesis of the conditional statement
false and therefor the conditional will be true.
}}\fi\end{enumerate}
\end{enumerate}



\end{description}



\newpage

\end{document}