%\documentclass{article}
\documentclass[12pt, oneside]{article}

\usepackage[letterpaper, scale=0.8, centering]{geometry}
\usepackage{fancyhdr}
\setlength{\parindent}{0em}
\setlength{\parskip}{1em}

\pagestyle{fancy}
\fancyhf{}
\renewcommand{\headrulewidth}{0pt}
\rfoot{{\footnotesize Copyright Mia Minnes, 2024, Version \today~(\thepage)}}

\usepackage{titlesec}

\author{CSE20S24}

\newcommand{\instructions}{{\bf For all HW assignments:} 
These homework assignments may be done individually or in groups of up to 3 students.
Please ensure your name(s) and PID(s)
are clearly visible on the first page of your homework
submission, start each question on a new page, and upload the PDF to Gradescope.
If you're working in a group, {\it submit only one submission per group}: one partner uploads the
submission through their Gradescope account and then adds the other group member(s) to the Gradescope submission
by selecting their name(s) in the ``Add Group Members'' dialog box. You will need to re-add your group member(s)
every time you resubmit a new version of your assignment.

Each homework question will be graded either for
{\bf correctness} (including clear and precise explanations and justifications of all answers) or
{\bf fair effort completeness}. You may collaborate on ``graded for correctness''
questions only with CSE 20 students in your group; if your
 group has questions about a problem, you may ask in drop-in help hours or post a private
post (visible only to the Instructors) on Piazza.  
 For ``graded for completeness''
 questions: collaboration is allowed with any CSE 20 students this quarter; 
 if your group has questions about a problem, you may ask in drop-in 
 help hours or post a public post on Piazza.

All submitted homework for this class must be typed. 
You can use a word processing editor if you like (Microsoft Word, Open Office, Notepad, Vim, Google Docs, etc.) 
but you might find it useful to take this opportunity to learn LaTeX. 
LaTeX is a markup language used widely in computer science and mathematics. 
The homework assignments are typed using LaTeX and you can use the source files 
as templates for typesetting your solutions.

{\bf Integrity reminders}
\begin{itemize}
\item Problems should be solved together, not divided up between the partners. The homework is
designed to give you practice with the main concepts and techniques of the course, 
while getting to know and learn from your classmates.
\item You may not collaborate on homework questions graded for correctness with anyone other than your group members.
You may ask questions about the homework in office hours (of the instructor, TAs, and/or tutors) and 
on Piazza (as private notes viewable only to the Instructors).  
You \emph{cannot} use any online resources about the course content other than the class material 
from this quarter -- this is primarily to ensure that we all use consistent notation and
definitions (aligned with the textbook) and also to protect the learning experience you will have when
the `aha' moments of solving the problem authentically happen.
\item Do not share written solutions or partial solutions for homework with 
other students in the class who are not in your group. Doing so would dilute their learning 
experience and detract from their success in the class.
\end{itemize}

}

\newcommand{\gradeCorrect}{({\it Graded for correctness}) }
\newcommand{\gradeCorrectFirst}{\gradeCorrect\footnote{This means your solution 
will be evaluated not only on the correctness of your answers, but on your ability
to present your ideas clearly and logically. You should explain how you 
arrived at your conclusions, using
mathematically sound reasoning. Whether you use formal proof techniques or 
write a more informal argument
for why something is true, your answers should always be well-supported. 
Your goal should be to convince the
reader that your results and methods are sound.} }
\newcommand{\gradeComplete}{({\it Graded for completeness}) }
\newcommand{\gradeCompleteFirst}{\gradeComplete\footnote{This means you will 
get full credit so long as your submission demonstrates honest effort to 
answer the question. You will not be penalized for incorrect answers. 
To demonstrate your honest effort in answering the question, we 
expect you to include your attempt to answer *each* part of the question. 
If you get stuck with your attempt, you can still demonstrate 
your effort by explaining where you got stuck and what 
you did to try to get unstuck.} }

%\usepackage{tikz}
%\usetikzlibrary{circuits.logic.US,circuits.logic.IEC}

\usepackage{amssymb,amsmath,pifont,amsfonts,comment,enumerate,enumitem}
\usepackage{currfile,xstring,hyperref,tabularx,graphicx,wasysym}
\usepackage[labelformat=empty]{caption}
\usepackage{xcolor}
\usepackage{multicol,multirow,array,listings,tabularx,lastpage,textcomp,booktabs}

% NOTE(joe): This environment is credit @pnpo (https://tex.stackexchange.com/a/218450)
\lstnewenvironment{algorithm}[1][] %defines the algorithm listing environment
{   
    \lstset{ %this is the stype
        mathescape=true,
        frame=tB,
        numbers=left, 
        numberstyle=\tiny,
        basicstyle=\rmfamily\scriptsize, 
        keywordstyle=\color{black}\bfseries,
        keywords={,procedure, div, for, to, input, output, return, datatype, function, in, if, else, foreach, while, begin, end, }
        numbers=left,
        xleftmargin=.04\textwidth,
        #1
    }
}
{}
\lstnewenvironment{java}[1][]
{   
    \lstset{
        language=java,
        mathescape=true,
        frame=tB,
        numbers=left, 
        numberstyle=\tiny,
        basicstyle=\ttfamily\scriptsize, 
        keywordstyle=\color{black}\bfseries,
        keywords={, int, double, for, return, if, else, while, }
        numbers=left,
        xleftmargin=.04\textwidth,
        #1
    }
}
{}

\newcommand\abs[1]{\lvert~#1~\rvert}
\newcommand{\st}{\mid}

\newcommand{\A}[0]{\texttt{A}}
\newcommand{\C}[0]{\texttt{C}}
\newcommand{\G}[0]{\texttt{G}}
\newcommand{\U}[0]{\texttt{U}}

\newcommand{\cmark}{\ding{51}}
\newcommand{\xmark}{\ding{55}}




%\usepackage{amsfonts,amssymb,amsmath,amsthm,color,comment,enumerate, graphicx,euscript,hyperref}
%\usepackage[makeroom]{cancel}
%\usepackage{paralist,listings}

%\setlength{\evensidemargin}{0in}
%\setlength{\oddsidemargin}{0in}
%\setlength{\textwidth}{6.6in}
%\setlength{\textheight}{8.8in}
%\setlength{\topmargin}{0in}
%\setlength{\footskip}{0.45in}
\renewcommand{\baselinestretch}{1}
%\newlength{\saveparindent}
%\setlength{\saveparindent}{\parindent}

% % NOTE(joe): This environment is credit @pnpo (https://tex.stackexchange.com/a/218450)
% \lstnewenvironment{algorithm}[1][] %defines the algorithm listing environment
% {   
%     \lstset{ %this is the stype
%         mathescape=true,
%         frame=tB,
%         numbers=left, 
%         numberstyle=\tiny,
%         basicstyle=\rmfamily\scriptsize, 
%         keywordstyle=\color{black}\bfseries,
%         keywords={,procedure, div, mod, for, to, input, output, return, datatype, function, in, if, else, foreach, while, begin, end, } %add the keywords you want, or load a language as Rubens explains in his comment above.
%         numbers=left,
%         xleftmargin=.04\textwidth,
%         #1 % this is to add specific settings to an usage of this environment (for instnce, the caption and referable label)
%     }
% }
% {}

% \newcommand{\A}[0]{\texttt{A}}
% \newcommand{\U}[0]{\texttt{U}}
% \newcommand{\G}[0]{\texttt{G}}
% \newcommand{\C}[0]{\texttt{C}}

\newif \ifsolution
\solutiontrue
\solutionfalse

\newcommand{\soltwo}[1]{\medskip\fbox{\begin{minipage}{6.5in}{#1}\end{minipage}}\medskip}

\begin{document}
\begin{center}
{\Large
CSE 20 Spring 2024\\ 
Practice for Test 2 \ifsolution{\qquad Solutions}\fi}
\end{center}

\thispagestyle{empty}

\ifsolution{}
\else{}
Below are the instructions that will be on the first page of the test package:

\begin{center}
  \begin{minipage}[t]{7in}
  \rule{\linewidth}{2pt}
  \textbf{INSTRUCTIONS --- READ THIS NOW}
  \begin{itemize}
  
  \setlength{\itemsep}{0.025in}
  
  \item  Write your name, PID, current seat number, exam time, 
  and the academic integrity pledge in the indicated space above and 
  on the designated  {\bf answer sheet}.
  We will check for {\bf all} of this identifying information before grading.
  Write your answers in the specified areas, or your work will not be graded. 
  
  \item We will not be answering questions about the exam during the exam period. 
  If any bugs are found in the exam after the exam period, the affected question part(s) will be addressed.
  
  \item  You may use one 8.5"x11", doublesided sheet of notes that you create and bring to the exam room, but no other books, notes, or aids.
  
  \item You may not speak to any other student in the exam room while the exam 
  is in progress (including after you hand in your own exam).  You may not share
  {\bf any information} about the exam with anyone who has not taken it.
  
  \item Turn off and put away all cellphones, calculators, and other electronic devices.
  You may not access any electronic device during the exam period. If you need to leave 
  the room during the exam period, you must leave all electronic devices with an exam proctor.
  
  \item  To receive full credit, your answers must
  be written neatly, legibly, and sufficiently darkly to scan well in the indicated answer box. Your solution will be evaluated both for correctness and clarity.
  Read the instructions for each part carefully to determine what is required for full credit.
  This test has $??$ problems worth a total of $??$ points.
  
  \item This exam is {\bf 45 minutes} long. Read all the problems first before you start 
  working on any of them, so you can manage your time wisely.
  
  \item Please stay seated until the end of the exam period.
  We will collect all exams and note sheets at the {\bf end} of the exam period, to minimize disruption 
  for students who wish to use the full time for the exam. Please show your ID to a proctor when 
  asked.
  
  
  \end{itemize}
  \end{minipage} \hfill
  \end{center}
  \newpage
\fi

\begin{description}

  \item[1. Predicates and Quantifiers] 
  Express each of the following statements symbolically using quantifiers, variables, propositional connectives, 
  and predicates (make sure to define the domain and the meaning of any predicates 
  you introduce). Then, express its negation as a logically equivalent compound 
  proposition without a $\neg$ in front.  Decide whether the statement or its negation is true,
  and prove it.
  
  \begin{enumerate}
  \item[(a)] If $m$ is an integer, $m^2 + m +1$ is odd.
  
  \ifsolution
  \soltwo{
  Define the predicate $D(x)$ to mean ``$x$ is odd" with domain $\mathbb{Z}$.
  
  
  The sentence is then $\forall m \in \mathbb{Z} ( D(m^2 + m + 1))$.
  
  Its negation is $\neg \forall m \in \mathbb{Z} ( D(m^2 + m + 1)) \equiv \exists m ( \neg D(m^2 + m + 1))$.
  
  The original statement is true.
  
  {\bf  Proof} Towards universal generalization, let $m$ be an arbitrary integer.  
  By   definition of odd, {\bf to show}: 
  there is an integer $k$ such 
  that $m^2 + m + 1 = 2k + 1$. Factoring the LHS, $m^2 + m + 1 = m(m+1) + 1$.  
  We proceed  in proof by cases, since $m$ is either even or odd.
  \begin{itemize}
  \item Case 1  {\bf to  show}: $(m$ is even $) \to  D(m^2+m+1)$.  In a direct proof of the conditional, 
   assume $m$ is even and choose $c$ be an integer such that $m = 2c$. 
   {\bf To show}: there is an integer such that $m^2+m+1$  is twice  that integer plus $1$.
  To find a witness  for this existential, we substitute  $m=2c$ and calculate, $m(m+1) = 2c(2c+1) = 2( 2c^2 + c) $.
  Since $2c^2 +c \in \mathbb{Z}$ (by properties of integer addition and multiplication), 
  it is the witness we need, because $2(2c^2+c) +1 =  2c(2c+1) +1 = m(m+1)+1 =  m^2+m+1$.
  \item Case 2 {\bf  to  show}: $(m$ is odd $) \to  D(m^2+m+1)$.  In a direct proof of the conditional, 
   assume $m$ is odd and choose $c$ be an integer such that $m = 2c+1$. 
   {\bf To show}: there is an integer such that $m^2+m+1$  is twice  that integer plus $1$.
  To find a witness  for this existential, we substitute  $m=2c+1$ and calculate, $m(m+1) = (2c+1)(2c+2) = 2( 2c+1)(c+1) $.
  Since $(2c+1)(c+1)\in \mathbb{Z}$ (by properties of integer addition and multiplication), 
  it is the witness we need, because $2(2c+1)(c+1)  + 1=  m(m+1) + 1=  m^2+m+1$.
  \end{itemize}
  
  {\it Alternate proof} Let $m$ be an integer.  WTS that $m^2+m+1 \mod 2 = 1$. 
  Case 1: $m \mod 2 = 0$.  Then by modular arithmetic (Corollary 2 on page 242), 
  \begin{align*}
  (m^2 + m + 1) \mod 2 &= \left( (m \mod 2 ) (m \mod 2) + (m \mod 2) + 1 \right) \mod 2 \\
  & = (0 \cdot 0 + 0 +1 ) \mod 2 = 1
  \end{align*}
  as required.  Case 2: $m \mod 2 = 1$.  Then 
  by modular arithmetic (Corollary 2 on page 242), 
  \begin{align*}
  (m^2 + m + 1) \mod 2 &= \left( (m \mod 2 ) (m \mod 2) + (m \mod 2) + 1 \right) \mod 2 \\
  & = (1 \cdot 1 + 1 +1 ) \mod 2 = 3 \mod 2 = 1
  \end{align*}
  as required.
  }
  \else{}
  \fi
  
  \item[(b)] For all integers $n$ greater than $5$, $2^n-1$ is not prime.
  
  \ifsolution
  \soltwo{
  
  Define the predicate: $P(x)$ to mean ``$x$ is prime" with domain: $\mathbb{Z}$. 
  
  The sentence translates to $\forall n ( n > 5 \to \neg P(2^n - 1) )$.
  
  Its negation is 
  $\neg \forall n  \in \mathbb{Z} ( n > 5 \to \neg P(2^n - 1) ) \equiv \exists n \in  \mathbb{Z} (n > 5 \wedge P(2^n -1))$.
  
  The negation is true: to prove it, we need a  witness integer greater than  $5$ where 
  $2^n-1$ is prime.  Consider $n=7$,  an integer so  in the domain.  Calculating, $2^7 -1 =128-1 =127$.   This is a prime
  number because all integers greater than $1$ and less than $\sqrt{127}$  (namely   $2, 3, 4, 5, 6, 7, 8, 9, 10, 11$)
  are not factors of $127$.
  }
  \else{}
  \fi
  
  
  \item[(c)] If $n$ and $m$ are even integers, then $n - m$ is even.
  
  \ifsolution
  \soltwo{
  Define the predicate $E(x)$ to mean 	``$x$ is even" with domain $\mathbb{Z}$.  Note: this 
  predicate could be expressed symbolically as $\exists k \in  \mathbb{Z} (x=2k)$.
  
  The sentence translates to  $\forall x \in \mathbb{Z} ~\forall y \in \mathbb{Z}~( ( E(x) \wedge E(y) ) \to E(x-y) )$.
  
  Its negation is $$\neg \forall x \in \mathbb{Z} ~\forall y \in \mathbb{Z}~( ( E(x) \wedge E(y) ) \to E(x-y) ) 
  \equiv \exists x\in \mathbb{Z} \exists y \in \mathbb{Z} ( E(x) \wedge E(y) \wedge \neg E(x-y))$$
  
  The original statement is true: Towards universal generalization let $x,y$ be arbitrary integers and
  to show is $( E(x) \wedge E(y) ) \to E(x-y)$. Assume, towards
  a direct proof of the conditional that $x$ and $y$ are both even. We WTS that $x-y$ is even.  By definition of even, 
  there are integers $j, h$ such that $x=2j$ and $y=2h$.  Then $x-y = (2j) - (2h) = 2(j-h)$.
  Choosing the  witness $k=j-h$ (an integer), we have proved that $x - y$ is even.
  }
  \else{}
  \fi
  \end{enumerate}
  
  
  \item[2. Proof strategies] Prove each of the following claims.
  Use only basic definitions and general proof strategies in your proofs; 
  do not use any results proved in class / the textbook as lemmas. You may use basic 
  properties of real numbers that we stated (without proof) in class, e.g. that the difference of two
  integers is an integer.
  \begin{enumerate}
  \item[(a)] The difference of any rational number and any irrational number is irrational.
  
  \ifsolution
  \soltwo{
  {\bf Proof}: Symbolically, to show is:
  \[
  \forall x \in \mathbb{R} \forall y \in \mathbb{R}  (  (x \in \mathbb{Q}  \land  y  \notin \mathbb{Q}) \to  (x-y  \notin  \mathbb{Q}))
  \]
  . Towards universal  generalization,  let $x,y$ be arbitrary  real numbers.
  Proceed in a  proof by  contradiction with  
  \[
  p  = (x \in \mathbb{Q}  \land  y  \notin \mathbb{Q}) \to  (x-y  \notin  \mathbb{Q})  \qquad
  r = y  \in  \mathbb{Q}
  \]
  . To show: $\neg p \to (r  \land \neg  r)$.  
  Assume $\neg p$, that is  $x \in \mathbb{Q}  \land  y  \notin \mathbb{Q} \land  x-y  \in  \mathbb{Q}$.  
  To show: $r \land \neg r$. By definition of conjunction,  each conjunct  is  true  and  $y  \notin \mathbb{Q}$  being true means 
  $\neg (y \in  \mathbb{Q}$ is  true.  Thus,  $\neg  r$ is true, and  to  show is:  $r$.
  By definition of $\mathbb{Q}$, there are integers $a,b,p, q$ with $b \neq 0$ and $q \neq 0$ such that $x = \frac{a}{b}, x-y = \frac{p}{q}$.  Then 
  \[
  y = x- (x-y) = \frac{a}{b} - \frac{p}{q} = \frac{aq - bp}{bq}.
  \]
  By  properties of integer multiplication and subtraction, $aq-bp \in \mathbb{Z}$
  and $bq\in \mathbb{Z}$, and since $b,q$ are both nonzero $bq \neq 0$.  Thus, by definition
  of $\mathbb{Q}$, $y \in \mathbb{Q}$, namely,  $r$.
  Thus, we proved $\neg p \to  (r \land  \neg r)$  and so 
  we conclude that $p$ holds.
  
  }
  \else{}
  \fi
  
  \item[(b)] If $a, b, c$ are integers and $a^2 + b^2 = c^2$, then at least one of $a$ and $b$ is even.
  
  \ifsolution
  \soltwo{
  We prove two helper claims (lemmas) before the proof.
  
  {\bf Lemma 1}: For any integer $x$, if $x^2$ is even , then so is $x$.
  
  {\it Proof of Lemma 1}:  Let $x$ be an arbitrary integer and assume (towards proof by contrapositive)
  that $x$ is not even (hence, odd).  WTS that $x^2$ is also not even.
  By assumption, $x = 2k+1$ for some integer $k$.  Squaring: 
  $x^2 = 4k^2+4k+1 = 2(2k^2+2k) + 1$, and by properties of integer addition
  and multiplication $2k^2 + 2k \in \mathbb{Z}$, so $x^2$ is odd, hence not even. \\
  
  
  {\bf Lemma 2}: For any odd integer $x$, $x^2~\text{\bf mod}~4 = 1$.
  
  {\it Proof of Lemma 2}: Let $x$ be an arbitrary odd integer.  By definition, this means there 
  is an integer, call it $k$, such that $x = 2k+1$.  Squaring: $x^2 = 4k^2 + 4k + 1 = 4( k^2 + k) + 1$.
  Since $k^2 + k \in \mathbb{Z}$, $x^2~\text{\bf mod}~4 = 1$.\\
  
  
  {\bf Proof}: Let $a,b,c$ be arbitrary integers.  To show: $a^2 +  b^2 = c^2  \to (E(a) \vee E(b) )$.
  Towards a contradiction, consider
  \[
  p  =   ( a^2 +  b^2 = c^2)  \to (E(a) \vee E(b) ) \qquad  r= (c^2~\text{\bf mod}~4 = 0)
  \]
  To show:  $\neg p \to  (r \land \neg r)$.  Assume $\neg p$, that is 
  $a^2 + b^2 =c^2$ and $a$ and $b$ are both not even.  
  By properties of even/odd, $a$ and $b$ are both odd.  Thus, by Lemma 2,
  $a^2~\text{\bf mod}~4 = b^2 ~\text{\bf mod}~4 = 1$.  By modular arithmetic,
  $c^2 ~\text{\bf mod}~4 = (a^2  + b^2)~\text{\bf mod}~4 = (1 + 1) ~\text{\bf mod}~4 = 2$.
  Since $2 \neq 0$,  this proves $\neg r$.  To show:  $r$.
  Since $c^2~\text{\bf mod}~4 =  2$, there is an integer, call it $q$, such that $c^2 = 4q + 2 = 2(2q+1)$. Thus,
  $c^2$ is even.  Therefore, applying Lemma 1, $c$ is even.
  By definition, this gives an integer $k$ such that $c = 2k$.  Thus, $c^2 = 4k^2 = 4k^2  +0$
  so $c^2~\text{\bf mod}~4 = 0$, namely $r$. Thus, we proved $\neg p \to  (r \land  \neg r)$  and so 
  we conclude that $p$ holds.
  }
  \else{}
  \fi
  
  \item[(c)] For any integer $n$, $n^2 + 5$ is not divisible by $4$.
  
  \ifsolution
  \soltwo{
  Let $n$ be an arbitrary integer.  Towards proof by cases, notice that 
  $(n~\text{\bf mod}~4 = 0) \lor (n~\text{\bf mod}~4 = 1) \lor (n~\text{\bf mod}~4 = 2) \lor (n~\text{\bf mod}~4 = 3)$
  \begin{itemize}
  \item {\bf Case 1 to show}: $(n~\text{\bf mod}~4 = 0) \to (n^2 + 5 \text{ is not divisible by } 4)$.  
  Assume $(n~\text{\bf mod}~4 = 0)$. Then, by modular arithmetic,
  \[
  (n^2 + 5)~\text{\bf mod}~4 = ((n~\text{\bf mod}~4)^2 + 5~\text{\bf mod}~4)~\text{\bf mod}~ 4= ( 0^2 + 1)~\text{\bf mod}~4 = 1~\text{\bf mod}~4.
  \]
  Since $(n^2 +5) ~\text{\bf mod}~4 \neq 0$, $n^2 + 5$ is not divisible by $4$, as required.
  \item {\bf Case 2 to show}: $(n~\text{\bf mod}~4 = 1) \to (n^2 + 5 \text{ is not divisible by } 4)$.  
  Assume $(n~\text{\bf mod}~4 = 1)$. Then, by modular arithmetic,
  \[
  (n^2 + 5)~\text{\bf mod}~4 = ((n~\text{\bf mod}~4)^2 + 5~\text{\bf mod}~4)~\text{\bf mod}~ 4= ( 1^2 + 1)~\text{\bf mod}~4 = 2~\text{\bf mod}~4.
  \]
  Since $(n^2 +5) ~\text{\bf mod}~4 \neq 0$, $n^2 + 5$ is not divisible by $4$, as required.
  \item {\bf Case 3 to show}: $(n~\text{\bf mod}~4 = 2) \to (n^2 + 5 \text{ is not divisible by } 4)$.  
  Assume $(n~\text{\bf mod}~4 = 2)$. Then, by modular arithmetic,
  \[
  (n^2 + 5)~\text{\bf mod}~4 = ((n~\text{\bf mod}~4)^2 + 5~\text{\bf mod}~4)~\text{\bf mod}~ 4= ( 2^2 + 1)~\text{\bf mod}~4 = 5~\text{\bf mod}~4 = 1
  \]
  Since $(n^2 +5) ~\text{\bf mod}~4 \neq 0$, $n^2 + 5$ is not divisible by $4$, as required.
  \item {\bf Case 4 to show}: $(n~\text{\bf mod}~4 = 3) \to (n^2 + 5 \text{ is not divisible by } 4)$.  
  \[
  (n^2 + 5)~\text{\bf mod}~4 = ((n~\text{\bf mod}~4)^2 + 5~\text{\bf mod}~4)~\text{\bf mod}~ 4= ( 3^2 + 1)~\text{\bf mod}~4 = 10~\text{\bf mod}~4 = 2
  \]
  Since $(n^2 +5) ~\text{\bf mod}~4 \neq 0$, $n^2 + 5$ is not divisible by $4$, as required.
  \end{itemize}
  The proof by cases  is now complete for the arbitrary integer $n$.
  }
  \else{}
  \fi
  \item[(d)] For all positive integers $a,b,c$, if $a \not | bc$ then $a \not | b$. {\scriptsize (The notation
  $a \not | b$ means ``a does not divide b'').}
  
  \ifsolution
  \soltwo{
  Let $a,b,c$ be arbitrary  positive integers.  Assume, towards a proof by contrapositive, that 
  $a |b$.  To show: $a | bc$.  By definition of divisibility, there is an integer $k$ such that
  \[
  b = ak.
  \]
  Multiplying both sides by $c$,
  \[
  bc = (ak) c = a (kc).
  \]
  By properties of integer multiplication, $kc \in \mathbb{Z}$ and so it witnesses 
  $a | bc$, as required.
  }
  \else{}
  \fi
  
  \end{enumerate}
  
  {\bf Bonus}: Watch the video \url{https://www.youtube.com/watch?v=MhJN9sByRS0} and write
  down the statement and one or both of the proofs described using the notation, definitions, and 
  proof strategies from CSE 20. 

  \item[3. Sets]  Prove or disprove each of the following statements.
  \begin{enumerate}
  \item[(a)]  For all sets $A$ and $B$, $\mathcal{P}(A) \cup \mathcal{P}(B) = \mathcal{P}(A \cup B)$.
  
  \ifsolution
  \soltwo{
  False.  A counterexample would be $A = \{1\}, B = \{2\}$.  Then 
  \[
  \mathcal{P}(A) \cup \mathcal{P}(B) = \{ \emptyset, \{1\} \} \cup \{ \emptyset , \{2\} \} = 
  \{ \emptyset, \{1\}, \{2\} \}
  \]
  but
  \[
  \mathcal{P}(A \cup B) = \mathcal{P}(\{1,2\}) = \{ \emptyset, \{1\}, \{2\}, \{1,2\} \}.
  \]
  These sets are not equal because they disagree about membership of $\{1,2\}$.
  }
  \else{}
  \fi
  
  \item[(b)] For all sets $A$ and $B$, $\mathcal{P}(A) \cap \mathcal{P}(B) = \mathcal{P}(A \cap B)$.
  
  \ifsolution
  \soltwo{
  True.  Proof: let $A, B$ be arbitrary sets.  We WTS that 
  $\mathcal{P}(A) \cap \mathcal{P}(B) \subseteq \mathcal{P}(A \cap B)$
  and $\mathcal{P}(A \cap B) \subseteq \mathcal{P}(A) \cap \mathcal{P}(B)$.
  \begin{itemize}
  \item[(c)] For the first subset inclusion, consider arbitrary $X \in \mathcal{P}(A) \cap \mathcal{P}(B)$.
  By definition of intersection, $X\in \mathcal{P}(A)$ and $X \in \mathcal{P}(B)$.
  By definition of power set, this means that $X \subseteq A$ and $X \subseteq B$.
  WTS that $X \subseteq A \cap B$: let $x \in X$.  Since $X \subseteq A$, $x \in A$.
  Since $X \subseteq B$, $x \in B$.  Thus, $x \in A$ and $x \in B$ so $x \in A \cap B$.
  Since $x$ was arbitrary, $\forall x ( x \in X \to x \in A \cap B)$ so $X \subseteq A \cap B$.
  Therefore, by definition of power set $X \in \mathcal{P}(A \cap B)$, as required.
  
  \item[(d)] For the second subset inclusion, consider arbitrary $X \in \mathcal{P}(A \cap B)$.  
  By definition of power set, $X \subseteq (A \cap B)$.  WTS that $X \subseteq A$ and 
  $X \subseteq B$.  Let $x \in X$.  Then by definition of subsets, $x \in A \cap B$. 
  By definition of intersection, $x \in A$ and $x \in B$.  Thus $X \subseteq A$ and 
  $X \subseteq B$ and so by definition of power set, $X \in \mathcal{P}(A)$
  and $X \in \mathcal{P}(B)$.  By definition of intersection, $X \in \mathcal{P}(A) \cap 
  \mathcal{P}(B)$, 
  as required.
  \end{itemize}
  }
  \else{}
  \fi
  
  \item[(e)] For any sets $A, B, C, D$,  if the Cartesian products $A\times B$
  and $C \times D$ are disjoint then either $A$ and $C$ are disjoint 
  or $B$ and $D$ are disjoint (or both).
  
  \ifsolution
  \soltwo{
  True.  Proof:  Let $A, B, C, D$ be arbitrary sets.  Towards a proof by contrapositive, 
  we assume that $A \cap C \neq \emptyset$ and $B \cap D \neq \emptyset$ and 
  WTS that $(A \times B) \cap (C \times D) \neq \emptyset$. Since $A \cap C \neq \emptyset$,
  let $x \in A \cap C $.  Since $B \cap D \neq \emptyset$, let $y \in B \cap D$.  
  Let's consider $(x,y)$.  By definition of intersection, since $x \in A \cap  C$
  and $y \in B \cap D$, $x \in A$ and $y \in B$.  Thus, by definition of Cartesian product,
  $(x,y) \in A \times B$.   Similarly, $(x,y) \in C \times D$.  Thus, by definition of intersection, 
  $(x,y) \in (A \times B) \cap (C \times D)$.  In particular, this means that 
  $(A \times B) \cap (C \times D) \neq \emptyset$, as required.
  }
  \else{}
  \fi
  
  \item[(f)] There are sets $A, B$ such that $A \in B$ and $A \subseteq B$.
  
  
  \ifsolution
  \soltwo{
  True.  Proof:  Consider the example $A = \{ 1, 2 \} $ and $B = \{ 1, 2, \{1,2\} \}$. 
  Then $A \in B$ because it ($\{1,2\}$) shows up in the list of elements of $B$.
  Moreover, since each of the elements of $A$ (the numbers $1$ and $2$) are also
  elements of $B$ (they each show up in $B$'s list of elements), $A$ is a subset of $B$.
  }
  \else{}
  \fi
  
  
  \item[(g)] For all sets $A, B, C$: $A \cap B = \emptyset$ and $B \cap C = \emptyset$
  if and only if  $(A \cap B) \cap C = \emptyset$.
  
  \ifsolution
  \soltwo{
  False. Proof:  Consider the counterexample $A = \{ 1, 2 \} $ and $B = \{ 2,3\}$ and
  $C = \{ 3\}$.  Then it is not the case that ``$A \cap B = \emptyset$ and $B \cap C = \emptyset$"
  because $A \cap B = \{ 2\}$.  However, it is the case that $(A \cap B) \cap C = \emptyset$
  because
  \[
  (A \cap B) \cap C = (\{1,2 \} \cap \{ 2,3\} ) \cap \{3\} = \{2\} \cap \{3\} = \emptyset.
  \]
  The biconditional statement is false because one of its arguments is false while the
  other is true.
  }
  \else{}
  \fi
  
  
  \end{enumerate}
  
  
  \item[4. Induction and Recursion] 
  
  Prove the following statements:
  
  \begin{enumerate}
  
  \item[(a)] $\forall s \in S \, (~rnalen(s) = basecount(s,\A) + basecount(s,\C) + basecount(s,\U) + basecount(s,\G)~)$ where $S$ RNA strands and the functions $rnalen$ and $basecount$ are define recursively as: 
  \hspace{-1in}
  \[
  \begin{array}{llll}
  & & \textit{rnalen} : S & \to \mathbb{Z}^+ \\
  \textrm{Basis Step:} & \textrm{If } b \in B\textrm{ then } & \textit{rnalen}(b) & = 1 \\
  \textrm{Recursive Step:} & \textrm{If } s \in S\textrm{ and }b \in B\textrm{, then  } & \textit{rnalen}(sb) & = 1 + \textit{rnalen}(s)
  \end{array}
  \]
  \[
  \begin{array}{llll}
  & & \textit{basecount} : S \times B & \to \mathbb{N} \\
  \textrm{Basis Step:} &  \textrm{If } b_1 \in B, b_2 \in B & \textit{basecount}(b_1, b_2) & =
          \begin{cases}
              1 & \textrm{when } b_1 = b_2 \\
              0 & \textrm{when } b_1 \neq b_2 \\
          \end{cases} \\
  \textrm{Recursive Step:} & \textrm{If } s \in S, b_1 \in B, b_2 \in B &\textit{basecount}(s b_1, b_2) & =
          \begin{cases}
              1 + \textit{basecount}(s, b_2) & \\
              \qquad \textrm{when } b_1 = b_2 &\\
              \textit{basecount}(s, b_2) & \\
              \qquad \textrm{when } b_1 \neq b_2 &\\
          \end{cases}
  \end{array}
  \]
  
  \ifsolution
  \soltwo{
  We proceed by structural induction:
  \begin{itemize}
  \item Basis step: We consider the four cases of $b \in \{ \A, \C, \U, \G \}$, in each to 
  show is that $rnalen(b) = basecount(b,\A) + basecount(b,\C) + basecount(b,\U) + basecount(b,\G)$.
  \[
  rnalen(\A) = 1 \qquad\text{by the basis step in the recursive definition of $rnalen$}
  \]
  \[
  basecount(\A,\A) + basecount(\A,\C) + basecount(\A,\U) + basecount(\A,\G) = 1 + 0 + 0 + 0 = 1
  \]
  by the basis step in the recursive definition of $basecount$, where the first term is the case where 
  $b_1 = b_2 = \A$ and the other three terms are the case where $b_1 \neq b_2$. Thus, 
  $rnalen(\A)  = 1 = basecount(\A,\A) + basecount(\A,\C) + basecount(\A,\U) + basecount(\A,\G)$.
  
  Similarly, 
  \[
  rnalen(\C) = 1 \qquad\text{by the basis step in the recursive definition of $rnalen$}
  \]
  \[
  basecount(\C,\A) + basecount(\C,\C) + basecount(\C,\U) + basecount(\C,\G) = 0 + 1 + 0 + 0 = 1
  \]
  by the basis step in the recursive definition of $basecount$, where the second term is the case where 
  $b_1 = b_2 = \C$ and the other three terms are the case where $b_1 \neq b_2$. Thus, 
  $rnalen(\C)  = 1 = basecount(\C,\A) + basecount(\C,\C) + basecount(\C,\U) + basecount(\C,\G)$; 
  \[
  rnalen(\U) = 1 \qquad\text{by the basis step in the recursive definition of $rnalen$}
  \]
  \[
  basecount(\U,\A) + basecount(\U,\C) + basecount(\U,\U) + basecount(\U,\G) = 0 + 0 + 1+ 0 = 1
  \]
  by the basis step in the recursive definition of $basecount$, where the third term is the case where 
  $b_1 = b_2 = \U$ and the other three terms are the case where $b_1 \neq b_2$. Thus, 
  $rnalen(\U)  = 1 = basecount(\U,\A) + basecount(\U,\C) + basecount(\U,\U) + basecount(\U,\G)$; 
  \[
  rnalen(\G) = 1 \qquad\text{by the basis step in the recursive definition of $rnalen$}
  \]
  \[
  basecount(\G,\A) + basecount(\G,\C) + basecount(\G,\U) + basecount(\G,\G) = 0 + 0 + 0 +1= 1
  \]
  by the basis step in the recursive definition of $basecount$, where the last term is the case where 
  $b_1 = b_2 = \G$ and the other three terms are the case where $b_1 \neq b_2$. Thus, 
  $rnalen(\G)  = 1 = basecount(\G,\A) + basecount(\G,\C) + basecount(\G,\U) + basecount(\G,\G)$.
  The basis step is now complete.
  \item Recursive step: Consider an arbitrary strand $s$ and an arbitrary base $b$.  
  Assume, as the {\bf induction hypothesis} that 
  \[
  rnalen(s) = basecount(s,\A) + basecount(s,\C) + basecount(s,\U) + basecount(s,\G)
  \]
  (continued next page)
  \end{itemize}
  }
  
  \fi
  \ifsolution
  \soltwo{
  
  \begin{itemize}
  \item (Recursive step, continued).  
  We need to show that 
  \[
  rnalen(sb) = basecount(sb,\A) + basecount(sb,\C) + basecount(sb,\U) + basecount(sb,\G)
  \]
  
  We consider the four cases of $b \in \{ \A, \C, \U, \G \}$. In each case, the LHS of the to show is
  \[
  rnalen(sb) = 1 + rnalen(s) \qquad\text{by the recursive step in the  definition of $rnalen$}
  \]
  For the RHS, when $b = \A$:
  \begin{align*}
  RHS = &basecount(s\A,\A) + basecount(s\A,\C) + basecount(s\A,\U) + basecount(s\A,\G) \\
  = &(1 + basecount(s,\A)) + basecount(s,\C) +basecount(s,\U) + basecount(s,\G) \\
  = &1 + ( basecount(s,\A)) + basecount(s,\C) +basecount(s,\U) + basecount(s,\G) ) \\
  = &1 + rnalen(s)  \qquad \text{by the induction hypothesis}
  \end{align*}
  (where the first equation listed is by the recursive step in the definition of basecount, the first term 
  from the case $b_1 = b_2 = \A$  and the rest of the terms from the case $b_1 \neq b_2)$).
  Thus, 
  $LHS = rnalen(s\A)  = 1+ rnalen(s) = RHS$.
  
  Similarly, when $b = \C$:
  \begin{align*}
  RHS = &basecount(s\C,\A) + basecount(s\C,\C) + basecount(s\C,\U) + basecount(s\C,\G) \\
  = &basecount(s,\A) + ( 1+ basecount(s,\C) ) +basecount(s,\U) + basecount(s,\G) \\
  = &1 + rnalen(s)  \qquad \text{by the induction hypothesis}
  \end{align*}
  (where the first equation listed is by the recursive step in the definition of basecount, the second term 
  from the case $b_1 = b_2 = \C$  and the rest of the terms from the case $b_1 \neq b_2)$).
  Thus, 
  $LHS = rnalen(s\C)  = 1+ rnalen(s) = RHS$. When $b = \U$:
  \begin{align*}
  RHS = &basecount(s\U,\A) + basecount(s\U,\C) + basecount(s\U,\U) + basecount(s\U,\G) \\
  = &basecount(s,\A) + basecount(s,\C) + ( 1 + basecount(s,\U)) + basecount(s,\G) \\
  = &1 + rnalen(s)  \qquad \text{by the induction hypothesis}
  \end{align*}
  (where the first equation listed is by the recursive step in the definition of basecount, the third term 
  from the case $b_1 = b_2 = \U$  and the rest of the terms from the case $b_1 \neq b_2)$).
  Thus, 
  $LHS = rnalen(s\U)  = 1+ rnalen(s) = RHS$. Finally, for $b = \G$:
  \begin{align*}
  RHS = &basecount(s\G,\A) + basecount(s\G,\C) + basecount(s\G,\U) + basecount(s\G,\G) \\
  = &basecount(s,\A) + basecount(s,\C)  + basecount(s,\U) + ( 1+ basecount(s,\G) )\\
  = &1 + rnalen(s)  \qquad \text{by the induction hypothesis}
  \end{align*}
  (where the first equation listed is by the recursive step in the definition of basecount, the last term 
  from the case $b_1 = b_2 = \G$  and the rest of the terms from the case $b_1 \neq b_2)$).
  Thus, 
  $LHS = rnalen(s\G)  = 1+ rnalen(s) = RHS$.
  
  \end{itemize}
  
  Since the basis step and recursive steps are complete, we have proved that the equation holds
  for all RNA strands.
  }
\fi

\item[(b)] $\forall l_1 \in L \, \forall l_2 \in L \, (\textit{sum}(\textit{concat}(l_1, l_2)) = \textit{sum}(l_1) + \textit{sum}(l_2))$, where the function $\textit{concat}$ is defined as:
\[
\begin{array}{llll}
& & \textit{concat} : L \times L & \to L \\
\textrm{Basis Step:} & \textrm{If } l \in L & \textit{concat}([], l) & = l \\
\textrm{Recursive Step:} & \textrm{If } l, l' \in L\textrm{ and }n \in \mathbb{N}\textrm{, then  } & \textit{concat}((n, l), l') & = (n, \textit{concat}(l, l'))
\end{array}
\]
and the function $\textit{sum} : L \to \mathbb{N}$ that sums all the elements of a list and is defined by:
\[
\begin{array}{llll}
& & \textit{sum} : L & \to \mathbb{N} \\
\textrm{Basis Step:} & & \textit{sum}([]) & = 0 \\
\textrm{Recursive Step:} & \textrm{If } l \in L, n \in \mathbb{N} & \textit{sum}((n, l)) & = n + \textit{sum}(l)
\end{array}
\]

\ifsolution
\soltwo{
We proceed by structural induction on $l_1$ by definition of $L$.
\begin{itemize}
\item Basis step: To show is that $\forall l_2 \in L \, (\textit{sum}(\textit{concat}([], l_2)) = \textit{sum}([]) + \textit{sum}(l_2))$. We proceed by universal generalization and let $l_2$ be an arbitrary element of $L$. By definition of \textit{concat}, we can rewrite the To Show as $\textit{sum}(l_2) = \textit{sum}([]) + \textit{sum}(l_2)$. By definition of \textit{sum}, we can rewrite the right-hand side as $0 + \textit{sum}(l_2)$. This shows that our goal can be rewritten to $\textit{sum}(l_2) = \textit{sum}(l_2)$, which is true.
\item Recursive step: To show is that $\forall l_2 \in L \, (\textit{sum}(\textit{concat}((n, l_1'), l_2)) = \textit{sum}((n, l_1')) + \textit{sum}(l_2))$, where $n \in \mathbb{N}$ and $l_1' \in L$. We assume as the inductive hypothesis that $\forall l_2 \in L \, (\textit{sum}(\textit{concat}(l_1', l_2)) = \textit{sum}(l_1') + \textit{sum}(l_2))$.

We proceed by universal generalization and assume that $l_2$ is an arbitrary element of $L$. Applying the definition of \textit{concat}, we can rewrite To Show as $$\textit{sum}((n, \textit{concat}(l_1', l_2))) = \textit{sum}((n, l_1')) + \textit{sum}(l_2)$$. We can further apply the definition of \textit{sum} on both sides to rewrite To Show as: $$n + \textit{sum}(\textit{concat}(l_1', l_2))) = n + \textit{sum}(l_1') + \textit{sum}(l_2)$$. By the rules of arithmetic on $+$, this is true when $$\textit{sum}(\textit{concat}(l_1', l_2))) = \textit{sum}(l_1') + \textit{sum}(l_2)$$. Since $l_2 \in L$, we can apply the inductive hypothesis, whose body exactly matches this goal.

\end{itemize}
}
\fi

\end{enumerate}


\item[5. Induction and Recursion] 
At time 0, a particle resides at the point 0 on the real line.  
Within 1 second, it divides into 2 particles that fly in opposite directions and 
stop at distance 1 from the original particle. Within the next second, 
each of these particles again divides into 2 particles flying in opposite directions 
and stopping at distance 1 from the point of division, and so on. 
Whenever particles meet they annihilate (leaving nothing behind). 
How many particles will there be at time $2^{11} -1$? 
You do not need to justify your answer. 
Hint: Derive a formula for the number/ locations of particles at time $2^n-1$ for arbitrary 
positive integer n, prove the formula using induction, and apply it when $n=11$.

\ifsolution
\soltwo{\qquad  {\bf $2^{11}$ particles}  \qquad

  Proof: We will prove by mathematical induction that, 

  \begin{quote}
    For each positive integer $n$, a particle that 
    started at time $0$ in position $0$ has the 
    property that between time $0$ and time $2^n-1$, 
    it and all of its descendants have stayed in the 
    interval $[-2^n+1, 2^n-1]$ and at time $2^n-1$, its descendants are located exactly
    on each odd location in $[-2^{n}+1,2^n-1]$ (inclusive).
  \end{quote}
  
  {\it How to use this lemma?} 
  Substituting $n = 11$, this lemma says that at time $2^{11}-1$, the
  descendants of the original particle are located at odd each location in 
  the interval $[-2^{11}+1, 2^{11}-1]$ (inclusive).  Therefore, the number of 
  particles at this time is the number of odd integers in the interval.
  There are $\lceil \frac{2^{11}-1}{2} \rceil = \frac{2^{11}}{2} = 2^{10}$ many 
  such negative odd integers, 
  and $\lceil \frac{2^{11}-1}{2} \rceil =\frac{2^{11}}{2} = 2^{10}$   
  many such positive odd integers, so a total 
  of $2^{10}+2^{10} = 2^{11}$ particles.\\
  
  To prove the lemma, we proceed by mathematical induction.

  \begin{itemize}
    \item {\bf Basis step} For $n=1$, we WTS that between time $0$ and
      time $2^1-1$, the particle and all its descendants have stayed in the interval
      $[-2^1+1, 2^1-1]$, and at time $2^1-1$, there are particles at each odd location
      this interval.  Plugging in $n=1$, we see that the interval is $[-1,1]$ and we WTS that 
      at time $1$ there are particles exactly at the endpoints of this interval (these are 
      the odd locations).  Applying the definition of the process, at time $1$, the original 
      particle divides into two and the two new particles reach distance $1$ from the original.
      This means that the only locations at which particles resided in this interval are
      $0,-1,1$ (all within the required interval) and that at time $1$, the particles
      are at $-1$ and $1$, as required.
  \end{itemize}  
  (continued next page)
  }
\fi
\ifsolution {
\newpage
\soltwo{
  \begin{itemize}
  \item {\bf Induction step} Let $k$ be an arbitrary positive integer.  Assume,
    as the {\bf Induction Hypothesis (IH)}, that
    \begin{quote}
      between time $0$ and time $2^k-1$, 
      the particle and all of its descendants have stayed in the region on the number 
      line $[-2^k+1, 2^k-1]$ and at time $2^k-1$, its descendants are located exactly
      on each odd location in $[-2^{k}+1,2^k-1]$ (inclusive).
    \end{quote}
  
    We WTS that
    \begin{quote}
      between time $0$ and time $2^{k+1}-1$, 
      the particle and all of its descendants have stayed in the region on the number 
      line $[-2^{k+1}+1, 2^{k+1}-1]$ and at time $2^{k+1}-1$, 
      its descendants are located exactly
      on each odd location in $[-2^{k+1}+1,2^{k+1}-1]$ (inclusive).
    \end{quote}
    We'll consider the progress of the experiment through time.  At time step
    $2^k$, one unit of time has elapsed since the situation described in the IH.
    By the setup of the experiment, during this unit of time, each particle splits into
    two and these travel one unit in each axis direction.  Since (by the IH) 
    at time $2^k-1$, there are particles
    at each odd location in the interval $[-2^{k}+1,2^k-1]$, each of these particles
    are two units apart.  Thus, the particles originating from successive
    odd locations in time $2^{k}-1$ will meet and annihilate at time $2^k$. 
    The only particles that will remain at time $2^k$ are (1) the particle that originates
    at time $2^k-1$ from location $-2^k +1$ and goes left and (2) the particle that 
    originates at time $2^k-1$ from location $2^k-1$ and goes right.  Thus, 
    at $2^k$ there are two particles, one at $-2^k$ and the other at $2^k$.
    For definiteness, let's call the particle at position $-2^k$, $p_L$, and the one
    at position $2^k$, $p_R$.    Recalibrating the experiment with position $2^k$ as
    the new ``origin" and particle $p_R$ as our original particle, the IH guarantees that 
    between the current time ($2^k$) and $2^k-1$ steps in the future, $p_R$ will
    divide into particles such that all of its descendants 
    will stay in the interval $[\text{``origin"}-2^{k}+1, \text{``origin"}+2^{k}-1]$ 
    and  its descendants will end up exactly
    on each odd location in $[\text{``origin"}-2^{k}+1, \text{``origin"}+2^{k}-1]$
    (inclusive).  That is, at time $2^k+2^k-1 = 2^{k+1}-1$, the process 
    starting $p_R$ will have generated particles at all odd locations
    in $[0, 2^{k+1}-1]$.  Symmetrically, the IH guarantees that at time $2^{k+1}-1$,
    all descendants of the particle $p_L$  will be at the odd locations in
    $[-2^{k+1}+1, 0]$, and will never have gone beyond that interval at any time 
    since 
    time $2^{k}-1$.  Since the intervals in which the descendants of $p_L$ and 
    $p_R$ land between these timestamps only overlap at $0$, the particles
    generated by these two experiments never collide and so the two 
    experiments run independently.  Thus, at time $2^{k+1}-1$, there
    are $2^k + 2^k = 2^{k+1}$ many particles, occupying each odd location in the interval
    $[-2^{k+1}+1, 0] \cup [0, 2^{k+1}-1] = [-2^{k+1}+1, 2^{k+1}-1]$, and 
    these particles (and their predecessors) have not left this interval at any point 
    up to this time step.  In particular, this means that the lemma is proved.
  \end{itemize} }}
\else{}
\fi

\item[6. Induction and Recursion] 
Prove that every positive integer has a base $3$ expansion. {\it Hint: use strong induction.}

\ifsolution
\soltwo{
By definition of base expansion, we need to prove that for every positive integer $n$, 
there is a positive integer $k$ and nonnegative integers $a_0, a_1, \ldots, a_{k-1}$ 
such that each $a_i \in \{0,1,2\}$, $a_{k-1} \neq  0$, and
\[
n = \sum_{i=0}^{k-1} a_i 3^i
\]
We proceed by strong induction on $n \geq 0$.
\begin{itemize}
\item Basis steps (using the terminology for strong induction: we choose $b=1, j=1$): For the positive integer $1$, we have witnesses
$k=1, a_0 = 1$ and since $1 \in \{0,1,2\}$, $1 \neq 0$, and $1 = \sum_{i=0}^0 a_i 3^i$, the claim
is proved. For the positive integer $2$, we have witnesses
$k=1, a_0 = 2$ and since $2 \in \{0,1,2\}$, $2 \neq 0$, and $2 = \sum_{i=0}^0 a_i 3^i$, the claim
is proved.
\item Recursive step: Consider an arbitrary positive integer $n$ greater than or equal to $2$. 
Assume, as the strong induction induction hypothesis that each positive number less than or equal to $n$ 
has a base $3$ expansion. We need to show that $n+1$ has a base $3$ expansion. To build the required
witnesses, consider the integer $c = (n+1) \textbf{ div } 3$. By properties of integers, since $n+1 \geq 3$
(because $n \geq 2$), $1 \leq c \leq n$.  Thus, the strong induction hypothesis applies to $c$ and there
is a positive integer $k_c$ and nonnegative integers $x_0, x_1, \ldots, x_{k_c-1}$ 
such that each $x_i \in \{0,1,2\}$, $x_{k_c-1} \neq  0$, and
\[
c = \sum_{i=0}^{k_c-1} x_i 3^i
\]
Define $k = k_c + 1$, $a_i = x_{i-1}$ for each $i$ from $1$ to $k$, and $a_0 = (n+1) \textbf{ mod } 3$.
Then $k$ is positive (because $k_c$ is), each $a_i \in \{0,1,2\}$ (because all the $x_i$ are in this
set, and the remainder upon division by $3$ of any positive integer is also in this set), $a_{k-1} =
a_{k_c+1 - 1} = x_{k_c-1} \neq 0$ (by choice of $k$ and definition of $x_i$s). It remains to prove that the 
sum gives the right value: 
\begin{align*}
\sum_{i=0}^{k-1} a_i 3^i &= \left( \sum_{i=1}^{k-1} a_i 3^i \right) + a_0 3^0
=  3 \left( \sum_{i=1}^{(k_c+1)-1} x_{i-1} 3^{i-1} \right) + a_0 3^0
= 3 \left( \sum_{j=0}^{k_c - 1} x_j 3^j \right) + a_0\\
&= 3 c + a_0 = 3 ( (n+1) \textbf{ div } 3) + ( (n+1) \textbf{ mod } 3 ) = n+1
\end{align*}
as required.
\end{itemize}
Since the existence of base $3$ expansions was proved for all positive integers (for $1, 2$ in
the basis cases, and all integers greater than or equal to $3$ in the recursive step), the proof
by strong induction 
is complete.
}
\else{}
\fi



\item[7. Functions \& Cardinalities of sets] Prove each of the following claims.
\begin{enumerate}

\item[(a)] For the ``function" $f: \mathbb{Z} \to \{0, 1, 2, 3 \}$ given by $f(x) = x~\text{\bf mod}~5$, it is not the case that every element of the domain maps to exactly one element of the codomain (that is, it is not a well-defined function).


\ifsolution
\soltwo{
This function is not well defined because $f(4)$ is not in the codomain
even though $4$ is in the domain: 
$f(4) =4~\text{\bf mod}~5 = 4 \notin \{0,1,2,3\}$.
}
\else{}
\fi

\item[(b)] The ``function" $f: \mathcal{P}( \mathbb{Z}^+) \to \mathbb{Z}$ given by 
\[
f(A) = \text{the maximum element in 
A}
\] is not well-defined.

\ifsolution
\soltwo{
This function is not well defined because $f(\mathbb{Z}^+)$ is not well-defined
even though $\mathbb{Z}^+$ is in the domain: 
the set $\mathbb{Z}^+$ doesn't have a maximum element.
}
\else{}
\fi

\item[(c)] There is a one-to-one function with domain $\{a,b,c\}$ and codomain $\mathbb{R}$.

\ifsolution
\soltwo{
Define the function $f:\{a,b,c\} \to  \mathbb{R}$ 
by the piecewise definition
\begin{align*}
f(a) &= 1, f(b) = 2, f(c) = 3
\end{align*}
This function is well-defined because it maps every domain element to a unique output, and
all outputs are in the specified codomain.  Moreover, it is one-to-one: each of the three
distinct domain elements have different images from one another.
}
\else{}
\fi

\item[(d)] There is an onto function with domain $R$ and codomain $\{ \pi, \frac{1}{17} \}$, where $R$ is the set of 
user ratings in a database with 5 movies.

\ifsolution
\soltwo{
Define the function $f: R \to \{\pi, \frac{1}{17} \}$ 
by the piecewise definition
\begin{align*}
f(w) &=  \begin{cases}
\pi \qquad \text{if $w = (0,0,0,0,0)$} \\
\frac{1}{17} \qquad \text{otherwise}
\end{cases}
\end{align*}
This function is well-defined because it maps every ratings $5$-tuple to a unique output, and
all outputs are in the specified codomain.  Moreover, it is onto: we need to confirm
that each element in the codomain has a preimage. Consider, for example
\[
f( (0,0,0,0,0) ) = \pi \qquad f( (1,1,1,1,1) ) = \frac{1}{17}.
\]
}
\else{}
\fi



\item[(e)] The Cartesian product $\mathbb{Z}^+ \times \{ a,b,c \}$ is countable.

\ifsolution
\soltwo{
We rewrite
\[
\mathbb{Z}^+ \times \{ a,b,c \} = \left( \mathbb{Z}^+ \times\{a\} \right) \cup \left( \mathbb{Z}^+ \times\{b\} \right)
\cup \left( \mathbb{Z}^+ \times \{c\} \right) 
\]
Using the function $f: \mathbb{Z}^+ \to \mathbb{Z}^+ \times\{a\}$ given by $f(n) = (n,a)$, 
we can prove that $\mathbb{Z}^+ \times\{a\}$ is countably infinite.  Similarly, 
$\mathbb{Z}^+ \times\{b\}$  and $\mathbb{Z}^+ \times\{c\}$ are countably infinite.
By Theorem 1 on page 174 in the book, $ \left( \mathbb{Z}^+ \times\{a\} \right) \cup \left( \mathbb{Z}^+ \times\{b\} 
\right)$ is countable because it is the union of
two countably infinite (hence countable) sets.  
Applying this theorem again when taking the union of this resulting set
with the set $\left( \mathbb{Z}^+ \times\{c\} \right)$, we see that the set is countable.
}
\else{}
\fi

\item[(f)] The interval of real numbers $\{x \in \mathbb{R}  ~\mid~ 5  \leq x \leq 8\}$ is uncountable.  {\it Hint: you may use
the fact that the unit interval $\{  x \in \mathbb{R} ~\mid~ 0  \leq x \leq  1\}$ is uncountable.}

\ifsolution
{\soltwo{
For brevity, we use the interval notation  $[5,8] = \{x \in \mathbb{R}  ~\mid~ 5  \leq x \leq 8\}$
and  $[0,1] = \{  x \in \mathbb{R} ~\mid~ 0  \leq x \leq  1\}$.
Define the function $f: [0,1] \to [5,8]$ by $f(x) = 5 + 3x$.  Then $f$ is a bijection: 
\begin{itemize}
\item Well-defined? for each $x$ in $[0,1]$, $0 \leq x \leq 1$ so $0 \leq 3x \leq 3$
and $5 \leq 5 + 3x \leq 8$.  Thus, $f(x) \in [5,8]$, as required.
\item Invertible? Consider the function $g:[5,8] \to[0,1]$ defined by $g(x) = \frac{x-5}{3}$.
WTS that for each $x \in [0,1]$, $g(f(x)) = x$ and for each $x \in [5,8]$, $f(g(x)) = x$.
Let $x \in [0,1]$.  Then 
\[
g(f(x)) = g( 5+3x) = \frac{(5+3x) - 5}{3} = \frac{3x}{3} = x,
\]
as required.  Similarly, let $x \in[5,8]$.  Then
\[
f(g(x)) = f ( \frac{x-5}{3}) = 5 + 3 \cdot \frac{x-5}{3} = 5 + (x-5) = x,
\]
as required.  Thus, $f$ is invertible (with inverse $g$) and so is a bijection.
\end{itemize}
}
}
\else{}
\fi

\end{enumerate}

\end{description}

\end{document}

